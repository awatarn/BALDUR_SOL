%  This is a LaTeX ASCII file.  To typeset document type: latex theory
%  To extract the fortran source code, obtain the utility "xtverb" from
%  Glenn Bateman (bateman@plasma.physics.lehigh.edu) and type:
%  xtverb < theory.tex > theory.f
%
% The following lines control how the LaTeX document is typeset
 
\documentstyle{article}
\headheight 0pt \headsep 0pt  \topmargin 0pt  \oddsidemargin 0pt
\textheight 9.0in \textwidth 6.5in
\begin{document}           % End of preamble and beginning of text.

\begin{center}
\Large {\bf Multi-Mode Transport Model mmm98d} \\
\vspace{1pc} \normalsize
Glenn Bateman, Arnold H.~Kritz \\
 Lehigh University Physics Department \\
16 Memorial Drive East, Bethlehem PA 18015 \\
bateman@plasma.physics.lehigh.edu \\
kritz@plasma.physics.lehigh.edu
\end{center}

This file documents a subroutine called {\tt mmm98d}, which computes
plasma transport coefficients using the Multi-Mode transport model
which has been held fixed since 1995.  A complete derivation of the
mmm98d model is given in reference \cite{bate98a}.

\begin{verbatim}
c@mmm98d.tex  .../baldur/code/bald/mmm98d.tex
c  rgb 13-nov-98 set isdasw(1) = lsuper
c  rgb 11-oct-98 fig, frb, and fkb must be passed through argument list
c    zepm = max ( zep, 0.05 )
c  rgb 08-oct-98 started with mmm95 to write sbrtn mmm98d
c    added delta to argument list, implemented kbmodels routine
c--------1---------2---------3---------4---------5---------6---------7-c
c
      subroutine mmm98d (
     &   rminor,  rmajor,   elong,    delta
     & , dense,   densh,    densimp,  densfe
     & , xzeff,   tekev,    tikev,    q,       btor
     & , avezimp, amassimp, amasshyd, aimass,  wexbs
     & , grdne,   grdni,    grdnh,    grdnz,   grdte,   grdti,  grdq
     & , thiig,   thdig,    theig,    thzig
     & , thirb,   thdrb,    therb,    thzrb
     & , thikb,   thdkb,    thekb,    thzkb
     & , gamma,   omega,    difthi,   velthi,  vflux
     & , matdim,  npoints,  nprout,   lprint,  nerr
     & , lsuper,  lreset,   lswitch,  cswitch, fig,    frb,     fkb)
c
c
c    All the following 1-D arrays are assumed to be defined on flux
c    surfaces called zone boundaries where the transport fluxes are
c    to be computed.  The number of flux surfaces is given by npoints
c    (see below).  For example, if you want to compute the transport
c    on only one flux surface, set npoints = 1.
c
c  Input arrays:
c  -------------
c
c  rminor(jz)   = minor radius (half-width) of zone boundary [m]
c  rmajor(jz)   = major radius to geometric center of zone bndry [m]
c  elong(jz)    = local elongation of zone boundary
c  delta(jz)    = local triangularity of zone boundary
c
c  dense(jz)    = electron density [m^-3]
c  densh(jz)    = sum over thermal hydrogenic ion densities [m^-3]
c  densimp(jz)  = sum over impurity ion densities [m^-3]
c  densfe(jz)   = electron density from fast (non-thermal) ions [m^-3]
c
c  xzeff(jz)    = Z_eff
c  tekev(jz)    = T_e (electron temperature) [keV] 
c  tikev(jz)    = T_i (temperature of thermal ions) [keV]
c  q(jz)        = magnetic q-value
c  btor(jz)     = ( R B_tor ) / rmajor(jz)  [tesla]
c
c  avezimp(jz)  = average density weighted charge of impurities
c               = ( sum_imp n_imp Z_imp ) / ( sum_imp n_imp ) where
c                 sum_imp = sum over impurity ions with charge state Z_imp
c
c  amassimp(jz) = average density weighted atomic mass of impurities
c               = ( sum_imp n_imp M_imp ) / ( sum_imp n_imp ) where 
c                 sum_imp = sum over impurity ions, each with mass M_imp
c
c  amasshyd(jz) = average density weighted atomic mass of hydrogen ions
c               = ( sum_hyd n_hyd M_hyd ) / ( sum_hyd n_hyd ) where
c                 sum_hyd = sum over hydrogenic ions, each with mass M_hyd
c
c  aimass(jz)   = mean atomic mass of thermal ions [AMU]
c               = ( sum_i n_i M_i ) / ( sum_i n_i ) where
c                 sum_i = sum over all ions, each with mass M_i
c
c  wexbs(jz)    = ExB shearing rate in [rad/s]
c
c    All of the following normalized gradients are at zone boundaries.
c    r = half-width, R = major radius to center of flux surface
c
c  grdne(jz) = -R ( d n_e / d r ) / n_e
c  grdni(jz) = -R ( d n_i / d r ) / n_i
c  grdnh(jz) = -R ( d n_h / d r ) / n_h
c  grdnz(jz) = -R ( d Z n_Z / d r ) / ( Z n_Z )
c  grdte(jz) = -R ( d T_e / d r ) / T_e
c  grdti(jz) = -R ( d T_i / d r ) / T_i
c  grdq (jz) =  R ( d q   / d r ) / q    related to magnetic shear
c
c  where:
c    n_i     = thermal ion density (sum over hydrogenic and impurity)
c    n_h     = thermal hydrogenic density (sum over hydrogenic species)
c    n_Z     = thermal impurity density,  Z = average impurity charge
c                      sumed over all impurities
c
c  Output:
c  -------
c
c    The following effective diffusivities represent contributions
c    to the total diffusivity matrix (difthi and velthi given below)
c    from each of the models that contribute to the Multi-Mode model.
c    Generally, these arrays are used for diagnostic output only.
c
c  thiig(jz) = ion thermal diffusivity from the Weiland model
c  thdig(jz) = hydrogenic ion diffusivity from the Weiland model
c  theig(jz) = elelctron thermal diffusivity from the Weiland model
c  thzig(jz) = impurity ion diffusivity from the Weiland model
c	    
c  thirb(jz) = ion thermal diffusivity from resistive ballooning modes
c  thdrb(jz) = hydrogenic ion diffusivity from resistive ballooning modes
c  therb(jz) = elelctron thermal diffusivity from resistive ballooning modes
c  thzrb(jz) = impurity ion diffusivity from resistive ballooning modes
c	    
c  thikb(jz) = ion thermal diffusivity from kinetic ballooning modes
c  thdkb(jz) = hydrogenic ion diffusivity from kinetic ballooning modes
c  thekb(jz) = elelctron thermal diffusivity from kinetic ballooning modes
c  thzkb(jz) = impurity ion diffusivity from kinetic ballooning modes
c
c    The following are growth rates and mode frequencies from the
c    Weiland model for drift modes such as ITG and TEM.
c    These arrays are intended for diagnostic output.
c
c  gamma(jm,jz) = growth rate for mode jm at point jz ( 1/sec )
c  omega(jm,jz) = frequency for mode jm at point jz ( radians/sec )
c
c    All of the transport coefficients are given in the following two
c    matricies for diffusion difthi and convection velthi in MKS units.
c    See the LaTeX documentation for difthi and velthi just below.
c
c    NOTE:  difthi and velthi include all of the anomalous transport.
c    There are no additional contributions to the heat fluxs from
c    charged particle convection.
c
c  difthi(j1,j2,jz) = full matrix of anomalous transport diffusivities
c  velthi(j1,jz)    = convective velocities
c  vflux(j1,jz)     = flux matrix
\end{verbatim}

The full matrix form of anomalous transport has the form
$$ \frac{\partial}{\partial t}
 \left( \begin{array}{c} n_H T_H  \\ n_H \\ n_e T_e \\
    n_Z \\ n_Z T_Z \\ \vdots
    \end{array} \right)
 = - \nabla \cdot
\left( \begin{array}{l} {\rm vFlux}_1 \; n_H T_H \\
 {\rm vFlux}_2 \; n_H \\
 {\rm vFlux}_3 \; n_e T_e \\
 {\rm vFlux}_4 \; n_Z \\
 {\rm vFlux}_5 \; n_Z T_Z \\
 \vdots \end{array} \right) 
 + \left( \begin{array}{c} S_{T_H} \\ S_{n_H} \\ S_{T_e} \\
    S_{n_Z} \\ S_{T_Z} \\ \vdots
    \end{array} \right)
$$
$$
 = \nabla \cdot
\left( \begin{array}{llll}
D_{1,1} n_H & D_{1,2} T_H & D_{1,3} n_H T_H / T_e \\
D_{2,1} n_H / T_H & D_{2,2} & D_{2,3} n_H / T_e \\
D_{3,1} n_e T_e / T_H & D_{3,2} n_e T_e / n_H & D_{3,3} n_e & \vdots \\
D_{4,1} n_Z / T_H & D_{4,2} n_Z / n_H & D_{4,3} n_Z / T_e \\
D_{5,1} n_Z T_Z / T_H & D_{5,2} n_Z T_Z / n_H &
        D_{5,3} n_Z T_Z / T_e \\
 & \ldots & & \ddots
\end{array} \right)
 \nabla
 \left( \begin{array}{c}  T_H \\ n_H \\  T_e \\
   n_Z \\  T_Z \\ \vdots
    \end{array} \right)
$$
$$
 + \nabla \cdot
\left( \begin{array}{l} {\bf v}_1 \; n_H T_H \\ {\bf v}_2 \; n_H \\
   {\bf v}_3 \; n_e T_e \\
   {\bf v}_4 \; n_Z \\ {\bf v}_5 \; n_Z T_Z \\
    \vdots \end{array} \right) +
 \left( \begin{array}{c} S_{T_H} \\ S_{n_H} \\ S_{T_e} \\
    S_{n_Z} \\ S_{T_Z} \\ \vdots
    \end{array} \right) $$
Note that all the diffusivities are in units of m$^2$/sec while the
convective velocities and vfluxes are in units of m/sec.

WARNING:  Do not add separate convective transport terms to this
anomalous transport model.  All the anomalous transport 
predicted by this Multi-Mode model is contained
in the diffusion coefficients {\tt difthi} and {\tt velthi} given
above.

\begin{verbatim}
c
c  Input integers:
c  ---------------
c
c  matdim  = first and second dimension of transport matricies
c            difthi(j1,j2,jz) and velthi(j1,jz) and the first 
c            dimension of gamma and omega.  matdim must be at least 5
c
c  npoints = number of values of jz in all of the above arrays
c
c  nprout  = output unit number for long printout
c
c  nerr    = 0 on input; returning with value .ne. 0 indicates error
c
c  Input switches
c  --------------
c
c  lprint      controls the amount of printout (0 => no printout)
c              higher values yield more diagnostic output
c
c  lsuper  > 0 for supershot simulations
c          = 0 for simulations of all other discharges
c
c  lreset  = 0 to use only internal settings for lswitch, cswitch
c              and for the coefficients fig, frb, and fkb
c
c    Note that when lreset = 0, the values of the switches and
c    coefficients in the argument list are ignored and all the 
c    switches and coefficients are set internally.
c
c    WARNING:  use lreset > 0 only if you want to pass all the switches
c    lswitch, cswitch, fig, frb, and fkb through the argument list.
c
c  Internal control variables:
c  ---------------------------
c
c  lswitch(j), j=1,8   integer control variables: 
c
c  cswitch(j), j=1,25   general control variables:
c
c  lswitch(1)  controls which version of the Weiland model is used
c             = 2  2 eqn  Weiland model Hydrogen \eta_i mode only
c             = 4  4 eqn  Weiland model with Hydrogen and trapped electrons
c             = 5  5 eqn  Weiland model with trapped electrons, FLR effects, 
c                         and parallel ion motion
c             = 6  6 eqn  Weiland model Hydrogen, trapped electrons,
c                    and one impurity species
c             = 7  7 eqn   Weiland model Hydrogen, trapped electrons,
c                  one impurity species, and collisions
c             = 8  8 eqn  Weiland model Hydrogen, trapped electrons,
c                  one impurity species, collisions, and parallel
c                  ion (hydrogenic) motion
c             = 9  9 eqn  Weiland model Hydrogen, trapped electrons,
c                  one impurity species, collisions, and finite beta
c             = 10 10 eqn Weiland model Hydrogen, trapped electrons,
c                  one impurity species, collisions, parallel
c                  ion (hydrogenic) motion, and finite beta
c             = 11 11 eqn Weiland model Hydrogen, trapped electrons,
c                  one impurity species, collisions, parallel
c                  ion (hydrogenic, impurity) motion, and finite beta
c
c  lswitch(2) = 0  full matrix representation for difthi and velthi
c             = 1  set diagonal matrix elements difthi and velthi
c             = 2  set diagonal matrix elements = effective diffusivities
c
c  lswitch(3) = 1 use (1+\kappa^2)/2 instead of \kappa scaling
c                 otherwise use \kappa scaling
c                 raised to exponents (cthery(12) - ctheory(16))
c
c  lswitch(4) > 0 to replace negative diffusivity with velocity
c
c  lswitch(5) = 1 to limit magnitude of all normalized gradients
c                    to ( major radius ) / ( ion Larmor radius )
c
c  cswitch(1)   0.5  minimum shear
c  cswitch(3)  -4.0  exponent of local elongation multiplying drift waves
c  cswitch(4)  -4.0  exponent of local elongation multiplying resistive
c                     balllooning modes
c  cswitch(5)  -4.0  exponent of local elongation multiplying
c                     kinetic balllooning modes
c  cswitch(6)   0.0  k_y \rho_s (= 0.316 if abs(cswitch(6)) < zepslon)
c  cswitch(9)  0.15  alpha in diamagnetic stabilization in GD model
c  cswitch(10)  0.0  transfer from thigi(jz) to velthi(1,jz)
c  cswitch(11)  0.0  transfer from zddig(jz) to velthi(2,jz)
c  cswitch(12)  0.0  transfer from thige(jz) to velthi(3,jz)
c  cswitch(13)  0.0  transfer from zdzig(jz) to velthi(4,jz)
c
c  cswitch(14)  1.0  include effect of finite beta in weiland14 
c                    = cetain(20)
c  cswitch(15)  0.0  min value of impurity charge state zimpz
c  cswitch(16)  0.0  include superthermal ions
c  cswitch(17)  1.0  effect of parallel ion motion in weiland14 
c                    = cetain(10)
c  cswitch(18)  0.0  -> 1.0 for effect of collisions in weiland14 
c                    = cetain(15)
c  cswitch(19)  0.0  -> 1.0 for v_parallel in strong ballooning limit 
c                    = cetain(12)
c  cswitch(20)  0.0  trapping fraction used in weiland14 (when > 0.0)
c                     multiplies electron trapping fraction when < 0.0
c  cswitch(21)  1.0  multiplier for wexbs in weiland14
c  cswitch(22)  0.0  -> 1.0 adds impurity heat flow to total ionic heat 
c                     flow for the weiland model
c  cswitch(23)  0.0  controls finite diff to construct the zgm matrix 
c                    = cetain(30)
c
c     contributions to vfluxes and interchanges: 
c
c  fig(1)   hydrogen particle transport from ITG (eta_i) mode
c  fig(2)   electron thermal  transport from ITG (eta_i) mode
c  fig(3)   ion      thermal  transport from ITG (eta_i) mode
c  fig(4)   impurity particle transport from ITG (eta_i) mode
c
c  frb(1)   hydrogen particle transport from resistive ballooning mode
c  frb(2)   electron thermal  transport from resistive ballooning mode
c  frb(3)   ion      thermal  transport from resistive ballooning mode
c  frb(4)   impurity particle transport from resistive ballooning mode
c
c  fkb(1)   hydrogen particle transport from kinetic ballooning mode
c  fkb(2)   electron thermal  transport from kinetic ballooning mode
c  fkb(3)   ion      thermal  transport from kinetic ballooning mode
c  fkb(4)   impurity particle transport from kinetic ballooning mode
c
c
c***********************************************************************
c
c-----------------------------------------------------------------------
c
c  Compile this routine and routines that it calls with a compiler 
c  option, such as -r8, to convert real to double precision when used on 
c  workstations.
c
c-----------------------------------------------------------------------
c
c  External dependencies:
c
c  Call tree: mmm98d calls the following routines
c
c  WEILAND14       - Computes diffusion matrix and convect velocities
c                        for the Weiland transport model
c    WEILAND14FLUX - Calculates fluxes and effective diffusivities
c      TOMSQZ      - Wrapper for QZ algorithm solving Ax = lambda Bx
c         CQZHES   - First step in QZ algorithm 
c         CQZVAL   - Second and third step in QZ algorithm
c         CQZVEC   - Fourth step in QZ algorithm
c
c-----------------------------------------------------------------------

      implicit none
c
      integer km, klswitch, kcswitch
c
      parameter ( km = 12, klswitch = 8, kcswitch = 25 )
c
      integer  matdim,  npoints, nprout,  lprint,   nerr
c
      real
     &   rminor(*),  rmajor(*),   elong(*),    delta(*)
     & , dense(*),   densh(*),    densimp(*),  densfe(*)
     & , xzeff(*),   tekev(*),    tikev(*),    q(*),       btor(*)
     & , avezimp(*), amassimp(*), amasshyd(*), aimass(*),  wexbs(*)
     & , grdne(*),   grdni(*),    grdnh(*),    grdnz(*)
     & , grdte(*),   grdti(*),    grdq(*)
c
      real  
     &   thiig(*),   thdig(*),    theig(*),    thzig(*)
     & , thirb(*),   thdrb(*),    therb(*),    thzrb(*)
     & , thikb(*),   thdkb(*),    thekb(*),    thzkb(*)
     & , omega(matdim,*),         gamma(matdim,*)
     & , difthi(matdim,matdim,*), velthi(matdim,*)
     & , vflux(matdim,*)
c
      real     cswitch(*)
c
      integer  lsuper,  lreset,  lswitch(*)
c
      real     fig(*),  fkb(*),  frb(*)
c
c..physical constants
c
      real zpi,  zcc,  zcmu0,  zceps0,  zckb,  zcme,  zcmp,  zce
c
c  zpi     = pi
c  zcc     = speed of light                  [m/sec]
c  zcmu0   = vacuum magnetic permeability    [henrys/m]
c  zceps0  = vacuum electrical permittivity  [farads/m]
c  zckb    = energy conversion factor        [Joule/keV]
c  zcme    = electron mass                   [kg]
c  zcmp    = proton mass                     [kg]
c  zce     = electron charge                 [Coulomb]
c
c..computer constants
c
      real  zepslon, zlgeps
c
c  zepslon = machine epsilon [smallest number so that 1.0+zepslon>1.0]
c  zlgeps  = ln ( zepslon )
c
c
c..local variables
c
      integer  jz, j1, j2, jm

      real  zelong, zelonf,  zai,    zne,     zni,    zte,    zti
     & ,    zq,     zeff,    zgne,   zgni,    zgnh,   zgnz,   zgte
     & ,    zgth,   zgtz,    zshear, zrmin,   zrmaj,  zbtor,  zep
     & ,    zprth,  zgyrfi,  zbeta,  zvthe,   zvthi,  zsound, zlog
     & ,    zcf,    znuei,   znueff, zlari,   zlarpo, zrhos,  zwn
     & ,    znude,  znuhat,  zdprth, zsgdpr,  zdpdr,  zgpr,   zscyl
     & ,    zsmin,  zshat,   zgmax,  zdelta
c
c.. variables for Weiland model
c
c  iletai(j1) and cetain(j1) are control variables
c
      integer        iletai(32)
c
      real  cetain(32), zomega(km), zgamma(km), zchieff(km)
c
      real           zdfthi(km,km),    zvlthi(km),     zflux(km)
c
      integer        idim,    ieq,     imodes
c
      real  zthte,   zbetae,  ztz,     znz,    zmass,  zimpz
     & ,    ztzte,   zfnzne,  zmzmh,   zfnsne, zftrap, zkyrho
     & ,    zomegde, zwexb,   znormd,  znormv
c
c  zexb    = local copy of ExB shearing rate
c  znormd  = factor to convert normalized diffusivities
c  znormv  = factor to convert normalized convective velocities
c
        real zbetah, zbetaz, zkparl, zperf(km)
c
c..local variables for the drift Alfven model
c
      integer iswitchda, isdasw(klswitch)
c
      real zsdasw(klswitch), zchrgns 
     &   , zfldath, zfldanh, zfldate, zfldanz, zfldatz
     &   , zdfdath, zdfdanh, zdfdate, zdfdanz, zdfdatz
c
c..local variables for kinetic ballooning modes
c
      integer ikbmodels(25)
c
      real zkbmodels(25), zbetac1, zbetac2, zchifact
     &  , zchiikb, zchiekb, zdifhkb, zdifzkb, zepm
c
c-----------------------------------------------------------------------
c
c..physical constants
c
        zpi     = atan2 ( 0.0, -1.0 )
        zcc     = 2.997925e+8
        zcmu0   = 4.0e-7 * zpi
        zceps0  = 1.0 / ( zcc**2 * zcmu0 )
        zckb    = 1.60210e-16
        zcme    = 9.1091e-31
        zcmp    = 1.67252e-27
        zce     = 1.60210e-19
c
c..computer constants
c
        zepslon = 1.0e-34
        zlgeps  = log ( zepslon )
c
c
c..initialize arrays
c
      do jz = 1, npoints
        thiig(jz)  = 0.
        thdig(jz)  = 0.
        theig(jz)  = 0.
        thzig(jz)  = 0.
        therb(jz)  = 0.
        thirb(jz)  = 0.
        thdkb(jz)  = 0.
        thekb(jz)  = 0.
        thzkb(jz)  = 0.
        thikb(jz)  = 0.
        thdkb(jz)  = 0.
        thekb(jz)  = 0.
        thzkb(jz)  = 0.
      enddo
c
      do jz = 1, npoints
        do j1 = 1, matdim
          velthi(j1,jz) = 0.0
          vflux(j1,jz) = 0.0
          do j2 = 1, matdim
            difthi(j1,j2,jz) = 0.0
          enddo
        enddo
      enddo
c
c..if lreset < 1, use internal settings for switches and coefficients
c  otherwise, use values passed through the argument list above
c
      if ( lreset .lt. 1 ) then
c
c..initialize switches
c
      do j1=1,kcswitch
        cswitch(j1) = 0.0
      enddo
c
      do j1=1,klswitch
        lswitch(j1) = 0
      enddo
c
c
c  Multi Mode Model in sbrtn THEORY version mmm98d
c  for use in the BALDUR transport code
c
      lswitch(1) = 11 ! Weiland ITG model weiland14 (11 eqns, no collisions)
      lswitch(2) = 2  ! use effective diffusivities
      lswitch(3) = 0  ! use kappa instead of (1+\kappa^2)/2
      lswitch(4) = 1  ! replace -ve diffusivity with convective velocity
      lswitch(5) = 1  ! limit gradients by major radius / ion Larmor radius
c
c  misc. parameters for sub. theory
c
      cswitch(1)  =  0.05 ! minimum value of shear
      cswitch(3)  = -4.0  ! elongation scaling for Weiland model
      cswitch(4)  = -4.0  ! elongation scaling for RB model
      cswitch(5)  = -4.0  ! elongation scaling for KB mode
      cswitch(6)  =  0.0  ! k_y \rho_s (= 0.316 if abs(cswitch(6)) < zepslon)
      cswitch(9)  = 0.15  ! Diamagnetic stabilization in Guzdar-Drake model
      cswitch(10) =  0.0  ! difthi -> velthi for chi_i
      cswitch(11) =  0.0  ! difthi -> velthi for hydrogen
      cswitch(12) =  0.0  ! difthi -> velthi for chi_e
      cswitch(13) =  0.0  ! difthi -> velthi for impurity
      cswitch(14) =  1.0  ! coeff of finite beta in weiland14 = cetain(20)
      cswitch(15) =  0.0  ! min value of impurity charge state zimpz
      cswitch(16) =  1.0  ! set fast particle fraction for use in weiland14
      cswitch(17) =  1.0  ! coeff of k_\parallel in weiland14 = cetain(10)
      cswitch(18) =  0.0  ! coeff of nuhat in weiland14 = cetain(15)
      cswitch(19) =  0.0  ! 0.0 -> 1.0 for v_parallel in strong balloon limit = cetain(12)
      cswitch(20) =  0.0  ! trapping fraction used in weiland14 (when > 0.0)
                          ! multiplies electron trapping fraction when < 0.0
      cswitch(21) =  1.0  ! multiplier for wexbs 
      cswitch(22) =  1.0  ! multiplier to impurity heat flux
      cswitch(23) =  0.0  ! controls finite diff to construct the zgm matrix = cetain(30)

c  contributions to hydrogenic particle, elec-energy, ion-energy,
c    and impurity ion fluxes
c
c        fig(1) = 0.80
c        fig(2) = 0.80
c        fig(3) = 0.80
c        fig(4) = 0.80
c
c        fkb(1) = 0.01
c        fkb(2) = 0.01
c        fkb(3) = 0.01
c        fkb(4) = 0.01
c
c        if ( lsuper .eq. 1 ) then
c          fkb(1) = 1.e-6
c          fkb(2) = 1.e-6
c          fkb(3) = 1.e-6
c          fkb(4) = 1.e-6
c        endif
c
c        frb(1) = 1.00
c        frb(2) = 1.00
c        frb(3) = 1.00
c        frb(4) = 1.00
c
      endif

\end{verbatim}

We then enter a loop over the spatial zones, and set the following
BALDUR {\tt common} variables to variables local to subroutine {\tt
theory}: the mean atomic mass number of the thermal ions, $A_{i}$
({\tt zai}), the electron and ion density and temperature, $n_{e}$
({\tt zne}), $n_{i}=\sum_{a}n_{a}$ ({\tt zni}), $T_{e}$ ({\tt zte}),
and $T_{i}$ ({\tt zti}), the safety factor, $q$ ({\tt zq}), the
effective charge, $Z_{eff}$ ({\tt zeff}),the midplane halfwidth of a
flux surface, $r$ ({\tt zrmin}), the major radius, $R_{o}$ ({\tt
zrmaj}), and the toroidal field at $R_0$ major radius, $B_{0}$
({\tt zbtor}).  Therefore, the only variables not defined in
subroutine {\tt mmm98d} that are needed to complete the rest of the
calculation are $\pi$ ({\tt zpi}), the small overflow protection
variable {\tt zepslon}.  The points of defining so many local
variables are to compact the notation.  The relevant coding for the
calculations just described is:

\begin{verbatim}
c
c.. start the main do-loop over the radial index "jz"..........
c
c
      do 300 jz = 1, npoints
c
c  transfer common to local variables to compact the notation
c
      zelong = max (zepslon,elong(jz))
      if ( lswitch(3) .eq. 1 ) then
        zelonf = ( 1. + zelong**2 ) / 2.
      else
        zelonf = zelong
      endif
c
      zai    = aimass(jz)
      zne    = dense(jz)
      zni    = densh(jz) + densimp(jz)
      znz    = densimp(jz)
      zte    = tekev(jz)
      zti    = tikev(jz)
      zq     = q(jz)
      zeff   = xzeff(jz)
c
c  normalized gradients
c
      zgne   = grdne(jz)
      zgni   = grdni(jz)
      zgnh   = grdnh(jz)
      zgnz   = grdnz(jz)
      zgte   = grdte(jz)
      zgth   = grdti(jz)
      zgtz   = grdti(jz)

      zrmin  = max( rminor(jz), zepslon )
      zrmaj  = rmajor(jz)
      zdelta = delta(jz)
c
      zshear = grdq(jz) * zrmin / zrmaj
      zbtor  = btor(jz)
c
c  compute inverse aspect ratio
c
      zep    = max( zrmin/zrmaj, zepslon )
c
\end{verbatim}

To complete the rest of the calculation we then compute various
quantities needed for the transport flux formulas (as in Table 1 of
the Comments paper, from which
$\omega_{ce}$ was inadvertantly omitted).
To begin with, we compute only quantities
which do not involve scale heights.
In the order in which they are computed, algebraic notation for
these quantities is:
$$ p=n_e T_e + n_i T_i \ --- \ {\rm (thermal)}\eqno{\tt zprth} $$  
$$ \omega_{ci}=eB_{o}/(m_{p}A_{i}) \eqno{\tt zgyrfi} $$
$$ \beta=(2\mu_{o}k_{b}/B_{o}^{2})(n_{e}T_{e}+n_{i}T_{i})
 \eqno{\tt zbeta} $$
$$ v_{e}=(2k_{b}T_{e}/m_{e})^{1/2} \eqno{\tt zvthe} $$
$$ v_{i}=(2k_{b}T_{i}/m_{p}A_{i})^{1/2} \eqno{\tt zvthi} $$
$$ c_{s}=[k_{b}T_{e}/(m_{p}A_{i})]^{1/2} \eqno{\tt zsound} $$
$$ \ln (\lambda)=37.8 - \ln (n_{e}^{1/2}T_{e}^{-1}) \eqno{\tt zlog} $$
$$ \nu_{ei}=4(2\pi)^{1/2}n_{e}(\ln \lambda)e^{4}Z_{eff}
               /[3(4\pi \epsilon_{o})^{2}m_{e}^{1/2}(k_{b}T_{e})^{3/2}]
 \eqno{\tt znuei} $$
$$ \eta=\nu_{ei}/(2\epsilon_{o}\omega_{pe}^{2}) \eqno{\tt zresis} $$
$$ \nu_{eff}=\nu_{ei}/\epsilon \eqno{\tt znueff} $$
$$ \nu_{e}^{*}=\nu_{ei}qR_{o}/(\epsilon^{3/2}v_{e}) \eqno{\tt thnust} $$
$$ \hat{\nu}=\nu_{eff}/\omega_{De} \eqno{\tt znuhat} $$
$$ \rho_{\theta i}=\rho_{i}q/\epsilon \eqno{\tt zlari} $$
$$ \rho_{i}=v_{i}/\omega_{ci} \eqno{\tt zlarpo} $$
$$ \rho_{s}=c_{s}/\omega_{ci} \eqno{\tt zrhos} $$
$$ k_{\perp}=0.3/\rho_{s} \eqno{\tt zwn} $$

The corresponding coding is:

\begin{verbatim}
c
      zprth  = zne * zte + zni * zti
      zgyrfi = zce * zbtor / (zcmp * zai)
      zbeta  = (2. * zcmu0 * zckb / zbtor**2) * (zne * zte + zni * zti)
      zvthe  = sqrt(2. * zckb * zte / zcme)
      zvthi  = sqrt(2. * zckb * zti / (zcmp * zai))
      zsound = sqrt(zckb * zte / (zcmp * zai))
      zlog   = 37.8-log(sqrt(zne) / zte)
      zcf    = (4. * sqrt(zpi) / 3.)
      zcf    = zcf * (zce / (4. * zpi * zceps0))**2
      zcf    = zcf * (zce / zckb) * sqrt( (zce/zcme) * (zce/zckb) )
      znuei  = zcf * sqrt(2.) * zne * zlog * zeff / (zte * sqrt(zte))
c
      znueff = znuei / zep
      zlari  = zvthi / zgyrfi
      zlarpo = max(zlari * zq / zep, zepslon)
      zrhos  = zsound / zgyrfi
      zwn    = 0.3 / zrhos
      znude  = 2 * zwn * zrhos * zsound / zrmaj
      znuhat = znueff / znude
c
c..if lswitch(5) = 1, limit magnitude of normalized gradients
c                    to ( major radius ) / ( ion Larmor radius )
c
      zgmax = zrmaj / zlarpo
c
      if ( lswitch(5) .eq. 1 ) then
c
        zgne = sign ( min ( abs ( zgne ), zgmax ), zgne )
        zgni = sign ( min ( abs ( zgni ), zgmax ), zgni )
        zgnh = sign ( min ( abs ( zgnh ), zgmax ), zgnh )
        zgnz = sign ( min ( abs ( zgnz ), zgmax ), zgnz )
        zgte = sign ( min ( abs ( zgte ), zgmax ), zgte )
        zgth = sign ( min ( abs ( zgth ), zgmax ), zgth )
        zgtz = sign ( min ( abs ( zgtz ), zgmax ), zgtz )
c
      endif
c
c  zgpr = -R ( d p   / d r ) / p    for thermal pressure
c
c  Compute the pressure scale length using smoothed and bounded
c  density and temperature
c
      zgpr = ( zne * zte * ( zgne + zgte )
     &         + zni * zti * ( zgni + zgth ) )
     &         / ( zne * zte + zni * zti )
c
      if ( lswitch(5) .eq. 1 )
     &  zgpr = sign ( min ( abs ( zgpr ), zgmax ), zgpr )
c
c
\end{verbatim}

Our formulas for the shear begin with
$$ {\hat s}_{cyl}=|(r/q)(\partial q/\partial r)| \eqno{\tt zscyl} $$
computed earlier in this subroutine.
The minimum prescribed shear is 
$$ {\hat s}_{min}=max(c_{1},0) \eqno{\tt zsmin} $$
where $c_1=$ = {\tt cswitch(1)} so that shear is then given by
$$ {\hat s}=max({\hat s}_{min},{\hat s}_{cyl}) \eqno{\tt zshat} $$

The relevant coding for the calculations just described is:

\begin{verbatim}
c
      zscyl=max(abs(zshear),zepslon)
      zsmin=max(cswitch(1),zepslon)
      zshat=max(zsmin,zscyl)
c
\end{verbatim}

%**********************************************************************c

\section{Transport Models}

The computation of the anomalous transport coefficients is
now described.  Please note that all the heat flux is included in the
thermal diffusion and velocity coefficients.  There are no additional
``convective velocities''.
The mode abbreviations used here are
\begin{center}
\begin{tabular}{llll}
    &             &                                         &        \\
    & {\tt ig}    & $\eta_i$-mode and drift wave modes      &        \\
    & {\tt rb}    & resistive ballooning                    &        \\
    & {\tt kb}    & kinetic ballooning                      &        \\
    &             &                                         &
\end{tabular}
\end{center}

%**********************************************************************c

\subsection{$\eta_i$ Modes}
%%%%%

The $\eta_i$ and trapped electron mode model 
by Weiland et al\cite{nord90a} is implemented when
${\tt lswitch(1)}$ is set greater than 1.
When $ {\tt lswitch(1)} = 2 $, only the hydrogen equations are used
(with no trapped electrons or impurities) to compute only the 
$ \eta_i $ mode.
When $ {\tt lswitch(1)} = 4 $, trapped electrons are included,
but not impurities.
When $ {\tt lswitch(1)} = 6 $, a single species of impurity ions is
included as well as trapped electrons.
When $ {\tt lswitch(1)} = 7 $, the effect of collisions is included.
When $ {\tt lswitch(1)} = 8 $, parallel ion (hydrogenic) motion and 
the effect of collisions are included.
When $ {\tt lswitch(1)} = 9 $, finite beta effects and collisions are
included.
When $ {\tt lswitch(1)} = 10 $, parallel ion (hydrogenic) motion, 
finite beta effects, and the effect of collisions are included.
When $ {\tt lswitch(1)} = 11 $, parallel ion (hydrogenic and impurity) motion, 
finite beta effects, and the effect of collisions are included.
Finite Larmor radius corrections are included in all cases.
Values of {\tt lswitch(1)} greater than 11 are reserved for extensions
of this Weiland model.

The mode growth rate, frequency, and effective diffusivities are
computed in subroutine {\tt weiland14}.
Frequencies are normalized by $\omega_{De}$ and diffusivities are
normalized by $ \omega_{De} / k_y^2 $.
The order of the diffusivity equations is 
$ T_H $, $ n_H $, $ T_e $, $ n_Z $, $ T_Z $, \ldots
Note that the effective diffusivities can be negative.

The diffusivity matrix $ D = {\tt difthi(j1,j2)}$ 
is given above.


The impurity density gradient scale length is defined as 
$$g_{nz}=-R{{d\ }\over {dr}}\left(Zn_z\right)/(Zn_z)$$
The electron density gradient scale length is defined as
$$g_{ne}=(1-Zf_z-f_s)g_{nH}+Zf_zg_{nz}+f_sg_{ns}$$
where $ f \equiv n_Z / n_e $ and $ n_e = n_H + Z n_Z +n_s$.
For this purpose, all the impurity species are lumped together as 
one effective impurity species and all the hydrogen isotopes are lumped 
together as one effective hydrogen isotope.


\begin{verbatim}
c
        do j1=1,32
          iletai(j1) = 0
          cetain(j1) = 0.0
        enddo
c
        thiig(jz) = 0.0
        theig(jz) = 0.0
        thdig(jz) = 0.0
        thzig(jz) = 0.0
c
c..set the number of equations to use in the Weiland model
c
        if ( (lswitch(1) .lt. 2) .or. (lswitch(1) .gt. 11 )) then
          nerr = -10
          return
        elseif (lswitch(1) .eq. 3) then
          nerr = -10
          return
        else
          ieq = lswitch(1)
        endif
c
        cetain(11) = 1.0
c
c.. coefficient of k_parallel for parallel ion motion
c.. cswitch(19) for v_parallel in strong ballooning limit
c.. in 9 eqn model
c
        cetain(10) = cswitch(17)
        cetain(12) = cswitch(19)
        cetain(15) = cswitch(18)
        cetain(20) = cswitch(14)
c
        iletai(10) = 0
c
        idim   = km
c
c  Hydrogen species
c
        zthte  = zti / zte

        zbetae = 2. * zcmu0 * zckb * zne * zte / zbtor**2
c
c  Impurity species (use only impurity species 1 for now)
c  assume T_Z = T_H throughout the plasma here
c
        ztz    = zti
        znz    = densimp(jz)
        zmass  = amassimp(jz)
        zimpz  = avezimp(jz)
        zimpz  = max ( zimpz, cswitch(15) )
c
        ztzte  = zti / zte
        zfnzne = znz / zne
        zmzmh  = zmass / amasshyd(jz)
c
c  superthermal ions
c
c  zfnsne = ratio of superthermal ions to electrons
c  L_ns   = gradient length of superthermal ions
c
        zfnsne = max ( cswitch(16) * densfe(jz) / dense(jz) , 0.0 )
c
        zftrap = sqrt ( 2. * zrmin / ( zrmaj * ( 1. + zrmin / zrmaj )))
        if ( cswitch(20) .gt. zepslon ) zftrap = cswitch(20)
        if ( cswitch(20) .lt. -zepslon )
     &       zftrap = abs(cswitch(20))*zftrap
c
        if ( abs(cswitch(6)) .lt. zepslon ) then
          zkyrho = 0.316
        else
          zkyrho = cswitch(6)
        endif
c
c
c...Define a local copy of normalized ExB shearing rate : pis
c
        zomegde = 2.0 * zkyrho * zsound / zrmaj 
c
        zwexb = cswitch(21) * wexbs(jz) / zomegde 
c
        zbetah = 0.0
        zbetaz = 0.0
        zkparl = 1.0
        zperf  = 0.0
c
        cetain(30) = cswitch(23)
        iletai(6)  = 0
        if ( lswitch(2) .lt. 1 ) iletai(7) = 1
c
c  if lswitch(2) .lt. 1, compute only the effective diffusivities
c
        iletai(9) = lswitch(2)

           call etaw17diff ( 
     &   iletai,   cetain,   lprint,   ieq,      nprout,   zgne
     & , zgnh,     zgnz,     zgte,     zgth,     zgtz,     zthte
     & , ztzte,    zfnzne,   zimpz,    zmzmh,    zfnsne,   zbetae
     & , zbetah,   zbetaz,   zftrap,   znuhat,   zq,       zshat
     & , zelong,   zkyrho,   zkparl,   zwexb
     & , idim,     zomega,   zgamma,   zdfthi,   zvlthi,   zchieff
     & , imodes,   zperf,    nerr )
c
c  If nerr not equal to 0 an error has occured
c
	if (nerr .ne. 0) return
c
c
c  Growth rates for diagnostic output
c    Note that all frequencies are normalized by \omega_{De}
c      consequently, trapped electron modes rotate in the positive
c      direction (zomega > 0) while eta_i modes have zomega < 0.
c
        jm = 0
        do j1=1,imodes
          if ( zgamma(j1) .gt. zepslon ) then
            jm = jm + 1
            gamma(jm,jz) = zgamma(j1) / zomegde
            omega(jm,jz) = zomega(j1) / zomegde
          endif
        enddo
c
c  conversion factors for diffusion and convective velocity
c
        znormd = zelonf**cswitch(3) *
     &    2.0 * zsound * zrhos**2 / ( zrmaj * zkyrho )
        znormv = zelonf**cswitch(3) *
     &    2.0 * zsound * zrhos**2 / ( zrmaj**2 * zkyrho )
c
c  compute effective diffusivites for diagnostic purposes only
c
        thdig(jz) = fig(1) * znormd * zchieff(2)
        theig(jz) = fig(2) * znormd * zchieff(3)
        thiig(jz) = fig(3) * znormd * zchieff(1)
     &  + fig(3) * znormd * zchieff(5) * cswitch(22) * znz / zni
        thzig(jz) = fig(4) * znormd * zchieff(4)
c
c  start computing the fluxes
c
        vflux(1,jz) = vflux(1,jz) + thiig(jz) * zgth / zrmaj
        vflux(2,jz) = vflux(2,jz) + thdig(jz) * zgnh / zrmaj
        vflux(3,jz) = vflux(3,jz) + theig(jz) * zgte / zrmaj
        vflux(4,jz) = vflux(4,jz) + thzig(jz) * zgnz / zrmaj
c
c  compute diffusivity matrix
c
        do j1=1,matdim
          velthi(j1,jz) = 0.0
          vflux(j1,jz) = 0.0
          do j2=1,matdim
            difthi(j1,j2,jz) = 0.0
          enddo
        enddo
c
c..set diffthi and velthi
c
        if ( lswitch(2) .gt. 1 ) then
c
c  diagonal elements of matrix = effective diffusivities
c
          difthi(1,1,jz) = difthi(1,1,jz) + thiig(jz)
          difthi(2,2,jz) = difthi(2,2,jz) + thdig(jz)
          difthi(3,3,jz) = difthi(3,3,jz) + theig(jz)
          difthi(4,4,jz) = difthi(4,4,jz) + thzig(jz)
c
        else
c
c..full matrix form of model
c
          if ( ieq .eq. 2 ) then
            difthi(1,1,jz) = difthi(1,1,jz) +
     &        fig(3) * znormd * zdfthi(1,1)
            velthi(1,jz)   = velthi(1,jz) +
     &        fig(3) * znormv * zvlthi(1)
          elseif ( ieq .eq. 4 ) then
            do j2=1,3
              difthi(1,j2,jz) = difthi(1,j2,jz) +
     &          fig(3) * znormd * zdfthi(1,j2)
              difthi(2,j2,jz) = difthi(2,j2,jz) +
     &          fig(1) * znormd * zdfthi(2,j2)
              difthi(3,j2,jz) = difthi(3,j2,jz) + 
     &          fig(2) * znormd * zdfthi(3,j2)
              difthi(4,j2,jz) = difthi(4,j2,jz) + 
     &          fig(4) * znormd * zdfthi(2,j2)
            enddo
              velthi(1,jz)    = velthi(1,jz) +
     &          fig(3) * znormv * zvlthi(1)
              velthi(2,jz)    = velthi(2,jz) +
     &          fig(1) * znormv * zvlthi(2)
              velthi(3,jz)    = velthi(3,jz) +
     &          fig(2) * znormv * zvlthi(3)
              velthi(4,jz)    = velthi(4,jz) +
     &          fig(4) * znormv * zvlthi(2)
          else
            do j2=1,4
              difthi(1,j2,jz) = difthi(1,j2,jz) +
     &          fig(3) * znormd * zdfthi(1,j2)
              difthi(2,j2,jz) = difthi(2,j2,jz) +
     &          fig(1) * znormd * zdfthi(2,j2)
              difthi(3,j2,jz) = difthi(3,j2,jz) +
     &          fig(2) * znormd * zdfthi(3,j2)
              difthi(4,j2,jz) = difthi(4,j2,jz) +
     &          fig(4) * znormd * zdfthi(4,j2)
            enddo
              velthi(1,jz)    = velthi(1,jz) +
     &          fig(3) * znormv * zvlthi(1)
              velthi(2,jz)    = velthi(2,jz) +
     &          fig(1) * znormv * zvlthi(2)
              velthi(3,jz)    = velthi(3,jz) +
     &          fig(2) * znormv * zvlthi(3)
              velthi(4,jz)    = velthi(4,jz) +
     &          fig(4) * znormv * zvlthi(4)
          endif
c
        endif
c
c
c..transfer from diffusivity to convective velocity
c
        if ( lswitch(4) .gt. 0 ) then
c
          if ( thiig(jz) .lt. 0.0 ) then
            velthi(1,jz) = velthi(1,jz) - thiig(jz) * zgth / zrmaj
            thiig(jz) = 0.0
            do j2=1,4
              difthi(1,j2,jz) = 0.0
            enddo
          endif
c
          if ( thdig(jz) .lt. 0.0 ) then
            velthi(2,jz) = velthi(2,jz) - thdig(jz) * zgnh / zrmaj
            thdig(jz) = 0.0
            do j2=1,4
              difthi(2,j2,jz) = 0.0
            enddo
          endif
c
          if ( theig(jz) .lt. 0.0 ) then
            velthi(3,jz) = velthi(3,jz) - theig(jz) * zgte / zrmaj
            theig(jz) = 0.0
            do j2=1,4
              difthi(3,j2,jz) = 0.0
            enddo
          endif
c
          if ( thzig(jz) .lt. 0.0 ) then
            velthi(4,jz) = velthi(4,jz) - thzig(jz) * zgnz / zrmaj
            thzig(jz) = 0.0
            do j2=1,4
              difthi(4,j2,jz) = 0.0
            enddo
          endif
c
        else
c
c..shift from diffusion to convective velocity
c
          if ( abs(cswitch(10)) + abs(cswitch(11)) + abs(cswitch(12))
     &       + abs(cswitch(13)) .gt. zepslon ) then
c
            velthi(1,jz) = velthi(1,jz)
     &       + cswitch(10) * thiig(jz) * zgth / zrmaj
            velthi(2,jz) = velthi(2,jz)
     &       + cswitch(11) * thdig(jz) * zgnh / zrmaj
            velthi(3,jz) = velthi(3,jz)
     &       + cswitch(12) * theig(jz) * zgte / zrmaj
            velthi(4,jz) = velthi(4,jz)
     &       + cswitch(13) * thzig(jz) * zgnz / zrmaj
c
c..alter the effective diffusivities 
c  if they are used for more than diagnostic purposes
c
            thiig(jz) = ( 1.0 - cswitch(10) ) * thiig(jz)
            thdig(jz) = ( 1.0 - cswitch(11) ) * thdig(jz)
            theig(jz) = ( 1.0 - cswitch(12) ) * theig(jz)
            thzig(jz) = ( 1.0 - cswitch(13) ) * thzig(jz)
c
            do j2=1,4
              difthi(1,j2,jz) = ( 1.0 - cswitch(10) ) * difthi(1,j2,jz)
              difthi(2,j2,jz) = ( 1.0 - cswitch(11) ) * difthi(2,j2,jz)
              difthi(3,j2,jz) = ( 1.0 - cswitch(12) ) * difthi(3,j2,jz)
              difthi(4,j2,jz) = ( 1.0 - cswitch(13) ) * difthi(4,j2,jz)
            enddo
c
          endif
c
        endif
c
c..end of Weiland model
c
c---:----1----:----2----:----3----:----4----:----5----:----6----:----7-c
c
\end{verbatim}

\subsection{Drift Alfv\'en Model from Bruce Scott}

\begin{verbatim}
c
c..Bruce Scott's drift Alfven model from 24 Sept 1998
c
      iswitchda = klswitch
c
        do j1=1,iswitchda
          isdasw(j1) = 0
          zsdasw(j1) = 0.0
        enddo
c
        zfldath = 0.0
        zfldanh = 0.0
        zfldate = 0.0
        zfldanz = 0.0
        zfldatz = 0.0
c
        zdfdath = 0.0
        zdfdanh = 0.0
        zdfdate = 0.0
        zdfdanz = 0.0
        zdfdatz = 0.0
c
        idim   = matdim
c
        zchrgns  = 1.0
c
        isdasw(1) = lsuper
c
          call sda05dif ( isdasw, zsdasw, iswitchda, idim
     &   , lprint, nprout
     &   , zgne, zgnh, zgnz, zgte, zgth, zgtz, zthte, ztzte
     &   , zfnzne, zimpz, zmzmh, zfnsne, zchrgns
     &   , zbetae, znuhat, zq, zshat, zelong
     &   , zfldath, zfldanh, zfldate, zfldanz, zfldatz
     &   , zdfdath, zdfdanh, zdfdate, zdfdanz, zdfdatz
     &   , zdfthi, zvlthi
     &   , nerr )
c
c..Set total effective diffusivities
c
        znormd    = zsound * zrhos**2 / zrmaj
        znormv    = zsound * zrhos    / zrmaj
c
c  Note: temporarily we are using abs ( diffusio coeff )
c
        thdrb(jz) = frb(1) * abs(zdfdanh) * znormd * zne / zni
        therb(jz) = frb(2) * abs(zdfdate) * znormd
        thirb(jz) = frb(3) * abs(zdfdath) * znormd
     &    * zne * zte/(zni * zti)
        thzrb(jz) = frb(4) * abs(zdfdanh) * znormd * zne / zni
c
c  add to the fluxes
c
        vflux(1,jz) = vflux(1,jz) + thirb(jz) * zgth / zrmaj
        vflux(2,jz) = vflux(2,jz) + thdrb(jz) * zgnh / zrmaj
        vflux(3,jz) = vflux(3,jz) + therb(jz) * zgte / zrmaj
        vflux(4,jz) = vflux(4,jz) + thzrb(jz) * zgnz / zrmaj
c
        difthi(1,1,jz) = difthi(1,1,jz) + thirb(jz)
        difthi(2,2,jz) = difthi(2,2,jz) + thdrb(jz)
        difthi(3,3,jz) = difthi(3,3,jz) + therb(jz)
        difthi(4,4,jz) = difthi(4,4,jz) + thzrb(jz)
c

\end{verbatim}
%**********************************************************************c
%%%%%%

\subsection{Kinetic Ballooning}

Transport driven by the kinetic ballooning mode is computed using
a mode developed by Aaron Redd \cite{redd98b}
multiplied by coefficients $F_1^{KB}=$ {\tt fkb(1)}, $F_2^{KB}=$ {\tt fkb(2)},
$F_3^{KB}=$ {\tt fkb(3)}.

The relevant coding is:

\begin{verbatim}
c ..................................
c .  the kinetic ballooning model  .
c ..................................
c
c
      do j1=1,25
        ikbmodels(j1) = 0
        zkbmodels(j1) = 0.0
      enddo
c
      ikbmodels(17) = 2
c
c  zero out zbetac1 and zbetac2 since they are not used in
c    Aaron Redd's kinetic ballooning mode model
c
      zbetac1 = 0.0
      zbetac2 = 0.0
c
c  recompute zthte, zfnsne
c    as they were before calling the Weiland model above
c
      zthte = zti / zte
      zfnsne = max ( cswitch(16) * densfe(jz) / dense(jz) , 0.0 )
      zfnzne = znz / zne
c
      zchifact = zsound * zlari**2 / zrmaj
c
c  Use zshear rather than zshat to avoid minimum shear cthery(1)
c
c  define zepm to avoid radii too close to the magnetic axis
c
      zepm = max ( zep, 0.05 )
c
      call kbmodels(      ikbmodels, zkbmodels,
     &                    zq, zshear, zepm, zelong, zdelta, zgpr,
     &                    zbeta, zbetac1, zbetac2,
     &                    zthte, zfnsne, zfnzne,
     &                    zchifact,
     &                    zchiikb, zchiekb, zdifhkb, zdifzkb,
     &                    nerr)
c
c
        thikb(jz) = fkb(3) * zchiikb
        thekb(jz) = fkb(2) * zchiekb
        thdkb(jz) = fkb(1) * zdifhkb
        thzkb(jz) = fkb(4) * zdifzkb
c
c  add to the fluxes
c
        vflux(1,jz) = vflux(1,jz) + thikb(jz) * zgth / zrmaj
        vflux(2,jz) = vflux(2,jz) + thdkb(jz) * zgnh / zrmaj
        vflux(3,jz) = vflux(3,jz) + thekb(jz) * zgte / zrmaj
        vflux(4,jz) = vflux(4,jz) + thzkb(jz) * zgnz / zrmaj
c
        if ( thikb(jz) .gt. 0.0 ) then
          difthi(1,1,jz) = difthi(1,1,jz) + thikb(jz)
        else
          velthi(1,jz) = velthi(1,jz) - thikb(jz) * zgth / zrmaj
        endif
c
        if ( thdkb(jz) .gt. 0.0 ) then
          difthi(2,2,jz) = difthi(2,2,jz) + thdkb(jz)
        else
          velthi(2,jz) = velthi(2,jz) - thdkb(jz) * zgnh / zrmaj
        endif
c
        if ( thekb(jz) .gt. 0.0 ) then
          difthi(3,3,jz) = difthi(3,3,jz) + thekb(jz)
        else
          velthi(3,jz) = velthi(3,jz) - thekb(jz) * zgte / zrmaj
        endif
c
        if ( thzkb(jz) .gt. 0.0 ) then
          difthi(4,4,jz) = difthi(4,4,jz) + thzkb(jz)
        else
          velthi(4,jz) = velthi(4,jz) - thzkb(jz) * zgnz / zrmaj
        endif
c
c
 300  continue
c
c
c   end of the main do-loop over the radial index, "jz"----------
c
      return
      end
\end{verbatim}
 
%**********************************************************************c

\begin{thebibliography}{99}

\bibitem{bate98a}
Glenn Bateman, Arnold~H. Kritz, Jon~E. Kinsey, Aaron~J. Redd, and Jan Weiland,
``Predicting temperature and density profiles in tokamaks,''
{\em Physics of Plasmas,} {\bf 5} (1998) 1793--1799.

%\bibitem{Comments} C. E. Singer, ``Theoretical Particle and Energy
%Flux Formulas for Tokamaks,'' Comments on Plasma Physics and Controlled
%Fusion {\bf 11} (1988) 165.

\bibitem{nord90a} H. Nordman, J. Weiland, and A. Jarmen, 
``Simulation of toroidal drift mode turbulence driven by 
temperature gradients and electron trapping,'' 
Nucl. Fusion {\bf 30} (1990) 983--996.

%\bibitem{Singer} C.E.Singer, G.Bateman, and D.D.Stotler,
%``Boundary Conditions for OH, L, and H-mode Simulations,''
%Princeton University Plasma Physics Report PPPL-2527 (1988).


\bibitem{redd98b} A.~J.~Redd,
{\it Pressure-driven Transport in the Core of Tokamak Plasmas},
PhD Dissertation, Lehigh University Department of Physics (1998).

\end{thebibliography}

%**********************************************************************c
\end{document}             % End of document.
