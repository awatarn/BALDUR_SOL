%  To extract the fortran source code, obtain the utility "xtverb" from
%  Glenn Bateman (bateman@plasma.physics.lehigh.edu) and type:
%  xtverb < mixerba.tex > mixerba.f
%
\documentstyle{article}    % Specifies the document style.

\headheight 0pt \headsep 0pt  \topmargin 0pt  \oddsidemargin 0pt
\textheight 9.0in \textwidth 6.5in

\begin{document}
\begin{center}
{\LARGE Subroutine for Computing Particle and Energy Fluxes\\ \vskip8pt
Using the Mixed Transport Model}\vskip1.0cm
Version 1.0: January 18, 1999 \\ 
Implemented by M. Erba, G. Bateman, Arnold H. Kritz\\
 Lehigh University
\end{center}
For questions about this routine, please contact: \\
Matteo Erba, Lehigh:  {\tt erba@plasma.physics.lehigh.edu}\\
Arnold Kritz, Lehigh: {\tt kritz@plasma.physics.lehigh.edu}
Glenn Bateman, Lehigh: {\tt glenn@plasma.physics.lehigh.edu}


\begin{verbatim}
c@mixed.tex
c--------1---------2---------3---------4---------5---------6---------7-c
c
      subroutine mixerba (
     &   shear,       wexb,       rmaj,       cwexb
     & , chibohm,     zq,         zrlpe,      tekev,       tikev      
     & , gradte,      zra,        zdte,       btor      
     & , chii,        chie,       dh,         dimp,        ierr
     & , chbe,        chbi,       chgbe,      chgbi
     & , zcoef1,      zcoef2,     zcoef3,     zcoef4)
c
c Mixed Transport Model (by M.Erba, V.V.Parail, A.Taroni)
c
c Inputs:
c    chibohm:   Bohm diffusivity, defined as T_e/eB, in MKS units.
c    tekev:     Electron temperature in kev
c    gradte:    Electron Temperature Gradient in keV/m
c    btor:      Toroidal field in Tesla
c    rmaj:      Major Radius
c         *** all other inputs are dimensionless ***
c    zq:        The local value of q, the safety factor
c    zrlpe:      R/L_pe, where R is the major radius of the plasma,
c                    and L_pe is the local electron pressure scale length
c    zra:       normalized minor radius r/a
c    zdte:       Non-local dependence on edge electron temperature:
c               abs{[Te(0.8)-Te(1.0)]/Te(1.0)]}
c    zcoef1:    Coefficient for empirical hydrogen diffusivity
c    zcoef2:    Coefficient for empirical hydrogen diffusivity
c    zcoef3:    Coefficient for empirical impurity diffusivity
c    zcoef4:    Coefficient for empirical impurity diffusivity
c    cwexb:     Multiplier of wexb
c
c Outputs:
c    ierr:      Error code
c    chii:      The ion thermal diffusivity, in chibohm's units
c    chie:      The electron thermal diffusivity, in chibohm's units
c    chgbe:     The electron gyro-Bohm term, in chibohm's units
c    chgbi:     The ion gyro-Bohm term, in chibohm's units
c    chbe:      The electron Bohm term, in chibohm's units
c    chbi:      The ion Bohm term, in chibohm's units
c    dh:        The hydrogenic ion particle diffusivity, in [m^2/sec]
c    dimp:      The impurity ion particle diffusivity, in [m^2/sec]
c
      IMPLICIT NONE
c
c Declare variables
c NAMING CONVENTION: Dimensionless variables begin with a 'z'
c
      REAL
     &   shear,       wexb,       rmaj,       cwexb     
     & , chibohm,     zq,         zrlpe,      tekev,   tikev      
     & , gradte,      zra,        zdte,       btor      
     & , chii,        chie,       dh,         dimp
     & , chbe,        chbi,       chgbe,      chgbi
     & , zcoef1,      zcoef2,     zcoef3,     zcoef4
     & , func,        gamma
c
      INTEGER ierr
c
c check input for validity
c
      ierr = 0
      if ((zq .lt. 0.0) .or. (zq .gt. 100.0)) then 
         ierr=1
         return
      elseif (abs(zrlpe) .gt. 1000.0) then
         ierr=2
         return
      elseif ((tekev .lt. 0.01) .or. (tekev .gt. 100.0)) then
         ierr=3
         return
      elseif ((zdte .lt. 0.0) .or. (zdte .gt. 1000.0))  then
         ierr=5
         return
      elseif ((btor .lt. 0.0) .or. (btor .gt. 15.0))  then
         ierr=6
         return
      endif
\end{verbatim}

\section{The Mixed Bohm/gyro-Bohm model}

The Mixed Bohm/gyro-Bohm transport model derives from an
originally purely Bohm-like model for electron transport
developed for the JET Tokamak\cite{tar94}. This preliminary model has 
subsequently been extended to describe ion transport\cite{erb95},
and a gyro-Bohm term has been added in order to simulate data from
different machines\cite{erb98}.  

\subsection{Bohm term}

The mixed model is derived using the dimensional analysis approach,
whereby the diffusivity in a Tokamak plasma can be written as:\\
\[ \chi = \chi_0 F(x_1, x_2, x_3, ...)\]

where $\chi_0$ is some basic transport coefficient and F is a function
of the plasma dimensionless parameters. We choose for $\chi_0$ the Bohm diffusivity:\\
\[ \chi_0 = \frac{cT_e}{eB}\]

The expression of the dimensionless function F is chosen according to
the following criteria:\
\begin{itemize}
\item{The diffusivity must be bowl-shaped, increasing towards the plasma
boundary}
\item{The functional dependencies of F must be in agreement with 
scaling relationships of the global confinement time, reflecting
trends such as power degradation and linear dependence on plasma
current}
\item{The diffusivity must provide the right degree of resilience
of the temperature profile}
\end{itemize}

It easily shown that a very simple expression of F that satisfies
the above requirements is:\\
\[ F = aq^2/|L_{pe}^*|\]

where q is the safety factor and $L_{pe}^*=(dp_e/dr)^{-1}/a$, being a the
plasma minor radius. The resulting expression of the diffusivity
can be written as:\\
\[ \chi \propto |v_d| \Delta G\]

where $v_d$ is the plasma diamagnetic velocity, $\Delta=a$ and $G=q^2$,
so that it is clear that this model represents transport due to 
long-wavelength turbulence.\\
The evidence coming up from the simulation of non-stationary
JET experiments \cite{erb97}(such as ELMs, cold pulses, sawteeth, etc.)
suggested that the above Bohm term should depend non-locally
on the plasma edge conditions through the temperature
gradient averaged over a region near the edge:\\

\[ <L_{T_e}^*>_{\Delta V}^{-1} = \frac{T_e(x=0.8) - T_e(x=1)}{T_e(x=1)}\]

where x is the normalized toroidal flux coordinate. The final
expression of the Bohm-like model is:\\

\[ \chi_{e,i}^B = \alpha_{Be,i} \frac{cT_e}{eB} L_{pe}^* q^2\]

where $\alpha_B$ is a parameter to be determined empirically,
both for ions and electrons.\\ 

\subsection{gyro-Bohm term}

The Bohm-like expression so derived proved to be very successful
in simulating JET discharges, but failed badly in smaller Tokamaks
such as START\cite{roa96}.\\
For this reason a simple gyro-Bohm-like term, also based on 
dimensional analysis, was added:

\[ \chi_{e,i}^{gB} = \alpha_{gBe,i} \frac{cT_e}{eB} L_{Te}^* \rho^*\]

where $\rho^*$ is the normalized larmor radius:

\[ \rho^* =  \frac {M^{1/2}cT_e^{1/2}}{Z_ieB_t}\]

This expression is what can be expected from small scale
drift-wave turbulence. It is important to note that in large
Tokamaks such as JET and TFTR the gyro-Bohm term is negligible,
while in smaller machines, with larger values of $\rho^*$, the
gyro-Bohm term can play a role especially near the plasma centre.\\ 

\subsection{Final Model}
The resulting expressions of the diffusivities are:

\[ \chi_{e,i}=\chi_{Be,i}+\chi_{gBe,i}\]

where the Bohm and gyro-Bohm terms are defined above and the adopted values
of the empirical parameters are:\\

\[ \alpha_{Be} = 8\times10^{-5} ,\alpha_{Bi} = 2\times\alpha_{Be}\]
\[ \alpha_{gBe} = 3.5\times10^{-2} , \alpha_{gBi} = \alpha_{gBe}/2\]

The relevant coding is as follows:

\begin{verbatim}
c *
c * Definition of the mixed model
c *
c 
c Declare the correction coeficient
c
c
c Calculate function for EXB and magnetic shear stabilization
c
         gamma = 3.0959e5 * sqrt(tikev) / (zq*rmaj)
c	 func  = (1.0 - abs(500 * wexb / gamma))
	 func  = (0.1 + shear - cwexb * abs(wexb / gamma))
c
c Calculate Bohm and gyro-Bohm terms
c

         chbe   = 8.0e-5 * zq * zq * zrlpe * zdte * chibohm
         chgbe  = 0.15811 * sqrt(tekev)*gradte/(btor* btor) 
         chbi   = 2.0 * chbe
         chgbi  = chgbe / 2.0

	 if (func.lt.0) then
           chbe = 0.0
           chbi = 0.0
	 endif
c
c Now determine the actual electron and ion thermal and particle diffusivities.
c
         chie   = chbe + chgbe
         chii   = chbi + chgbi
         dh     = (zcoef1 + (zcoef2-zcoef1) * zra) 
     &             * chie * chii / (chie + chii)
c
c The impurity diffusivity is not included in the mixed
c model described in Ref. [1] but is defined here using a simple
c empirical model.
c
         dimp   = zcoef3 + zcoef4 * zra * zra
c
      return
      end
\end{verbatim}
%**********************************************************************c

\begin{thebibliography}{99}
\bibitem{tar94}
A. Taroni, M. Erba, E. Springmann and Tibone F.,
{\em Plasma Physics and Controlled Fusion,} {\bf 36} (1994) 1629.
\bibitem{erb95}
M. Erba, V. Parail, E. Springmann and A. Taroni,
{\em Plasma Physics and Controlled Fusion,} {\bf 37} (1995) 1249.
\bibitem{erb98}
M. Erba, et al.,
{\em Nuclear Fusion,} {\bf 38} (1998) 1013.
\bibitem{erb97}
M. Erba, et al.,
{\em Plasma Physics and Controlled Fusion,} {\bf 39} (1997) 261.
\bibitem{roa96}
C.M. Roach, 
{\em Plasma Physics and Controlled Fusion,} {\bf 38} (1996) 2187.
\end{thebibliography}
%**********************************************************************c
\end{document}             % End of document.
