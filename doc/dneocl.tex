%  20.26 11:00 20-mar-94 /11040/baldur/code/bald/dneocl.tex, Bateman, PPPL
%
%   This is a LaTeX input file documenting and providing the FORTRAN
% source code for the neoclassical transport in the BALDUR code.
%
%   To typeset this document type:
%
% LATEX DNEOCL
%
%   To extract and compile the fortran source code, type:
%
% LIB WBALDN3 ^ X XTVERB ^ END           ! to get utility program XTVERB
% XTVERB INPUT=DNEOCL OUTPUT=TNEOCL      ! to extract FORTRAN code
% COSMOS TNEOCL                          ! to compile and update YBALDLIB
%

\documentstyle {article}    % Specifies the document style.
\oddsidemargin 0pt \textwidth 6.5in

\title{Neoclassical Transport in BALDUR}
\author{Glenn Bateman and the authors of BALDUR \\
        Princeton Plasma Physics Laboratory}

\begin{document}           % End of preamble and beginning of text.

\maketitle                 % Produces the title.

\begin{verbatim}
c  BALDUR  file DNEOCL   Bateman, PPPL
c--------1---------2---------3---------4---------5---------6---------7-c
c
c  To obtain this file, type           (use appropriate date for yymmdd)
c cfs get /11040/bald94/byymmdd/wcode.tar
c tar xf wcode.tar  (this creates directory .../code and subdirectories)
c cd code/bald      (this moves to subdirectory .../code/bald)
c
c--------1---------2---------3---------4---------5---------6---------7-c
c
c       Contents of file DNEOCL:
c  TRNEO1 - simple neoclassical transport coefficients
c  NCFLUX - Hawryluk-Hirshman neoclassical particle transport
c
c--------1---------2---------3---------4---------5---------6---------7-c
c@trneo1  /11040/bald92/wbaldn1 DNEOCL
c  ap  15-feb-00 changed Hollerith character representation into '...' 
c                in call mesage(...) and call error(...)
c  rgb 07-feb-00 removed equivalence statement
c rgb 20-mar-94 move common /cmneo1/ to file com/cbaldr.m
c rgb 14-jul-92 computed xeneo2(jz) if lneocl(3)=1 after Stringer 1991
c   les  nov-90 d3he; add coppi-sharky anomalous inward pinch
c           vxcmg=0 if lcmg=0
c   les  nov-90 ; d3he ions different
c   impurities have v-ware = veware ???
c   les 90 - note aired (defined in auxval/default) for imp=1-4
c rgb 20.15 07-oct-91  corrected problem with Ware pinch at r=0
c rgb 20.14 06-oct-91  corrections to zalpha in Chang-Hinton \chi_i
c rgb 20.11 27-aug-91  correct Chang-Hinton \chi_i and improve document
c rgb 20.06 05-aug-91  1986 Chang-Hinton implemented for impure plasmas
c rgb 20.05 28-jul-91  sbrtn trneo1 to compute neoclassical transport
c rgb 18.37 11-jun-90 ahalfs(jz,1) to ahalfs(jz,jb) in comp of zdramp
c     make sure zft(jz) = 0.0 at the magnetic axis, zft(jz)=0. 
c     after do 1081
c  dps 16-dec-87 reinstate zshift specification by input for 
c     Chang-Hinton  chi-i; used only if cfutz(282) > 0.
c  dps 07-may-87 introduce switch on trapped fraction calculation
c                     in ware pinch.
c  dps 04-mar-87 add bootstrap current cf. Mike Hughes; here
c                     install improved trapped fraction calculation.
c  rgb 16-mar-86 ramp trapping factor in ware pinch near axis
c  rgb 28-sep-85 cfutz(282) no longer used to define Shafranov
c                     shift in Chang-Hinton neoclassical ki
c                     instead zshift = d rmids(jz,2) / d ahalfs(jz,2)
c
c--------1---------2---------3---------4---------5---------6---------7-c
c
c     Input variables:
c
c cfutz(9)   1.0  coeff of simplified neoclassical contribution to
c                   the diagonal hydrogenic paricle diffusivities
c cfutz(10)  1.0  coeff of simplified neoclassical contribution to
c                   the diagonal impurity   paricle diffusivities
c cfutz(11)  1.0  coeff of simplified neoclassical contribution to
c                   the electron thermal diffusivity
c cfutz(12)  1.0  coeff of simplified neoclassical contribution to
c                   the ion      thermal diffusivity
c cfutz(19)  1.0  coeff multiplying Ware pinch of hydrogen isotopes
c                   and electron energy
c cfutz(110) 0.0  activate Hawryluk-Hirshman ion-transport to replace
c                   the simplified off-diagonal neoclassical diffusivities
c cfutz(139) 0.0  multiplier for simplified off-diagonal neoclassical 
c                   diffusivities
c cfutz(281) 0.0  switch for neoclassical ion thermal diffusivity model
c                 = 0.0 for Hinton-Hazeltine (1976)
c                 = 1.0 for Chang-Hinton (1982)
c                 = 1.2 for Chang-Hinton (1986) (effect of impurities)
c                 = 2.0 for Bolton-Ware (1983)
c cfutz(282) 0.0  fraction of minor radius that magnetic axis is 
c                   shifted outward for computing \chi_i^{CH}
c                   ( = R_0' ).  Maximum value is 0.95.
c cfutz(365) 0.0  to include additional electron energy Ware pinch
c                   and omit convective energy flow controlled by CPVELC
c     (to restore the treatment of ware pinch on electron-energy flow
c      as used in older versions of the baldur code (prior to 19-may-83)
c      set "cfutz(365)=1.0".  default is "cfutz(365)=0.0".)
c cfutz(366) 0.0  multiplier for hydrogen-hydrogen simplified
c                   off-diagonal neoclassical transport diffusivities
c
c lneocl(3)  0  switch for simplified neoclassical contribution to
c                   the electron thermal diffusivity
c          = 0  for the original model in BALDUR
c          = 1  for multiple of neoclassical ion thermal diffusivity
c                 corresponding to theory by Stringer 1991 
c
c--------1---------2---------3---------4---------5---------6---------7-c
c
      subroutine trneo1
c
c
      include '../com/cbaldr.m'
      include '../com/commhd.m'
      include '../com/cbparm.m'
      include '../com/comtrp.m'
      include '../com/cmg.m'
c
      dimension zk(21)      , zft(55)           , zaspin(55)
     &  , zifreq(idxion,55) , zgyro2(idxion,55) , znstar(idxion,55)
c
cgb      equivalence   (nlzzzz(1),zgyro2(1,1))
c
\end{verbatim}

\section{Neoclassical Ion Thermal Diffusivity}

A selection of simple neoclassical thermal diffusivity models is 
controlled by the input variable {\tt cfutz(281)}
Effects in multispecies plasmas are approximated by assuming
that diffusive parts of the ion-heat flux are additive.

The following formulae are used below
$$ \nu_i = \sqrt{\bar{A}_H / A_i} Z_i^2  / \tau_i 
                                     \eqno{\tt zifreq(ip,jz)} $$
$$ \rho_{i\theta}^2 = ( c^2 / e^2 ) ( M_p / A_p ) A_i T_i
                   / ( Z_i^2 B_p^2 ) \eqno{\tt zgyro2(ip,jz)} $$
$$ \nu_{*i} = Z_i^2 \nu_H^*          \eqno{\tt znstar(ip,jz)} $$
corresponding to Eq(6.113) of Hinton and Hazeltine\cite{hint76a}
where
$$ \nu_H^* = \frac{C_{\nu H}B_T}{\tau_i |B_p|}
 \sqrt{\frac{R^3 \bar{A}_H}{r T_i}}  \eqno{\tt xnuhyd(ip,jz)} $$
$$ \tau_i = ( C_{\tau i} T_i^{3/2} \bar{A}_H )
    / ( Z_2 \lambda_i)               \eqno{\tt tions(ip,jz)}  $$
$$ Z_2 = \sum_{\mbox{all ions}} n_i \langle Z^2 \rangle       $$
are computed in sbrtn GETCHI.

\begin{verbatim}
c
        ipmax  = mhyd + max0(mimp,0)
c
        z0     = 2.0 * (fxc-fxes) + fxau
        zgyro0 = ( 2.0*fcc*fcc*fcau ) * (10.0**z0) / (fces*fces)
c
      do 10 jz = lcentr, mzones
c
        j0 = jz
        if ( xbouni(j0) .le. epslon ) j0 = lcentr + 1
c
c  zaspin(jz) = half-width / major radius to midpoint at zone bndries
c
          zaspin(jz) = ahalfs(j0,1) / rmids(j0,1)
c
        do 6010 ip=1,ipmax
          ii     = ip - mhyd
          zavez2 = 1.0
          if ( ii .gt. 0 ) zavez2 = c2mean(ii,1,j0)
          zmass0 = sqrt( ahmean(1,j0) / (aspec(ip)+epslon) )
          zifreq(ip,jz) = zmass0 * zavez2 / tions(1,j0)
          zgyro2(ip,jz) = zgyro0 * ( aspec(ip) * tis(1,j0) ) /
     &                  ( zavez2 * bpols(1,j0)**2 )
          znstar(ip,jz) = zavez2 * xnuhyd(1,j0)
 6010   continue
c
  10  continue
c
\end{verbatim}

\subsection{Hinton-Hazeltine Model}

The choice {\tt cfutz(281)} = 0.0 turns on the 
Hinton-Hazeltine\cite{hint76a} model

\begin{verbatim}
c
      if ( cfutz(281) .lt. 0.1 ) then
c
      do 12 jz = lcentr, mzones
c
        if ( abs(xbouni(jz)) .gt. epslon ) then
c
c  ion-gyroradius factor
c
          zgyroi = (fcc*sqrt(2.*ahmean(1,jz)*fcmp)/fces)*
     &      10.**(fxc+.5*fxnucl-fxes)
c
          z0 = 0.66 * sqrt(zaspin(jz)) * zgyroi**2 * tis(1,jz)
          z1 = 1.0 + 1.03 * sqrt( xnuhyd(1,jz) + 0.31 * xnuhyd(1,jz) )
          z2 = 1.0 + 0.74 * ( zaspin(jz) * sqrt(zaspin(jz)) )
     &         * xnuhyd(1,jz)
          z3 = 1.76645 * ( zaspin(jz) * sqrt(zaspin(jz)) )
     &           * ( zaspin(jz) * sqrt(zaspin(jz)) ) * xnuhyd(1,jz)
          xineo1(jz) = rhoins(1,jz) * ( z0 * (z2+z1*z3) ) /
     &      ( z1 * z2 * bpols(1,jz)**2 * tions(1,jz) + epslon )
c
        else
c
          xineo1(jz) = rhoins(1,jz) * ( 4.25806 * fcc**2 *
     &      sqrt( ahmean(1,jz) * fcmp * tis(1,jz) ) *
     &      tis(1,jz) * ahalfs(jz+1,1) ) /
     &      ( fces**2 * rmajs**2 * bzs * bpols(1,jz+1) ) *
     &      ( 10.0**( 2.0 * (fxc-fxes) + 0.5 * fxnucl ) )
c
        endif
c
  12  continue
c
\end{verbatim}

\subsection{Chang-Hinton $\chi_i^{CH}$ Models}

The choice {\tt cfutz(281)} = 1.0 turns on the
Chang-Hinton (1982) model\cite{chang82a} 
while the choice {\tt cfutz(281)} = 1.2 turns on the extension to the
Chang-Hinton model to include impurities (1986)\cite{chang86a}.
The difference between these two versions of the Chang-Hinton model
hinges entirely on the definition of $\alpha$
$$  \begin{array}{lcl} \alpha = \sum_{\rm impurities} n_I Z_I^2
 / \sum_{\rm hydrogen} n_i Z_i^2  & {\rm if} & {\tt cfutz(281)} = 1.2 \\
  \alpha = 0  & {\rm if} & {\tt cfutz(281)} = 1.0. \end{array} 
                                           \eqno{\tt zalpha} $$

Once $\alpha$ is established, then the ion thermal diffusivity\
is computed on zone boundaries
$$ \chi_i^{CH} = n_i \delta^{1/2} (\rho_{i\theta}^2 / \tau_{ii}) K_2
                                      \eqno{\tt xineo1(jz)}  $$
where $n_i$ is the hydrogenic density ({\tt zdens}),
$ \delta = r / R_0 = $ half-width / major radius to midpoint
at BALDUR zone boundaries, 
is the inverse aspect ratio ({\tt zaspin(jz)}),
$ \rho_{i\theta} $ is the poloidal ion gyroradius ({\tt zgyro2(ip,jz)}),
$ n_i $ is the hydrogenic ion density ({\tt zdens}),
and $K_2$ is defined by
$$ K_2 = K_2^{(0)} \left(
\frac{\hat{K}_2/K_2^{(0)}}{1+a_2\mu_{*i}^{1/2}+b_2\mu_{*i}}
   + \frac{(c_2^2/b_2)\mu_{*i}\delta^{3/2}}{1+c_2\mu_{*i}\delta^{3/2}}
     H_p F \right)                         \eqno{\tt zk2}     $$
where
$$ K_2^{(0)} = 0.66                          \eqno{\tt zk20}  $$
$$ a_2 = 1.03                                \eqno{\tt za2}   $$
$$ b_2 = 0.31                                \eqno{\tt zb2}   $$
$$ c_2 = 0.74                                \eqno{\tt zc2}   $$
$$ \hat{K}_2 = [ 0.66 (1+1.54\alpha) + (1.88\delta^{1/2}-1.54\delta)
  (1+3.75\alpha)] \langle B_0^2/B^2 \rangle \eqno{\tt zk2hat} $$
$$ H_p = 1+1.33\alpha(1+0.60\alpha)/(1+1.79\alpha) \eqno{\tt zhp} $$
The Pfirsch-Schl\"{u}ter factor is
$$ F = \frac{1}{2} \delta^{-1/2} \left( 
  \left\langle \frac{B_0^2}{B^2}\right\rangle
  - \left\langle \frac{B^2}{B_0^2} \right\rangle^{-1} \right)
                                              \eqno{\tt zf0}  $$
$$ \left\langle \frac{B_0^2}{B^2}\right\rangle =
   \frac{1 + (3/2)\delta(\delta+R_0') + (3/8)\delta^3R_0'
   }{1 + \delta R_0' / 2}                     \eqno{\tt zb1}  $$
$$ \left\langle \frac{B^2}{B_0^2}\right\rangle^{-1} =
   \frac{(1-\delta^2)^{1/2}(1+\delta R_0'/2)}{1 +
   (R_0'/\delta)[(1-\delta^2)^{1/2} - 1]}     \eqno{\tt zb2}  $$
$$ \mu_{*i} = \nu_{*ii} ( 1 + 1.54\alpha ) \eqno{\tt zmustr}  $$ 
$$ \begin{array}{lcl} R_0' = d R_0(r) / d r 
    & {\rm if} & |{\tt cfutz(282)}| \leq \epsilon \\
    R_0' = {\rm sign} [ \min (|{\tt cfutz(282)}, 0.95 ), 
    {\tt cfutz(282)} ] & {\rm if} & |{\tt cfutz(282)}| > \epsilon
     \end{array}                           \eqno{\tt zshift}  $$

Note that {\tt znstar} and {\tt zifreq} are divided by 
$ 1 + {\tt zalpha} $ below to make their definitions consistent
with the Chang-Hinton paper.

\begin{verbatim}
c
      elseif ( cfutz(281) .lt. 1.9 ) then
c
        zk20 = 0.66
        za20 = 1.03
        zb20 = 0.31
        zc20 = 0.74
c
      do 14 jz = lcentr, mzones
c
        zalpha = 0.0
        zdenh  = 0.0
        zdenzi = 0.0
      if ( cfutz(281) .gt. 1.1  .and.  mimp .gt. 0 ) then
        do 23 jh=1,mhyd
          zdenh = zdenh + rhohs(jh,1,jz)
  23    continue
        do 24 ji=1,mimp
          zdenzi = zdenzi + rhois(ji,1,jz) * c2mean(ji,1,jz)
  24    continue
          zalpha = zdenzi / zdenh
      endif
c
        zshift=0.0
      if (abs(cfutz(282)) .gt. epslon ) then
        zshift = sign ( min(abs(cfutz(282)),.95) , cfutz(282) )
      else
        zshift = (rmids(jz,2)-rmids(jz-1,2))
     &              / (ahalfs(jz,2)-ahalfs(jz-1,2))
      endif
c
        zb1 = ( 1.0 + 1.5 * ( zaspin(jz)**2 + zaspin(jz) * zshift )
     &           + 0.375 * zaspin(jz)**3 * zshift )
     &             / ( 1.0 + 0.5 * zaspin(jz) * zshift )
c
        zb2  = zaspin(jz) * sqrt(1.0-zaspin(jz)**2) 
     &    * (1.0 + 0.5*zaspin(jz)*zshift)
     &    / ( zaspin(jz) + zshift*( sqrt(1.0 - zaspin(jz)**2) - 1.0 ) )
c
        zf0 = ( zb1 - zb2 ) / ( 2.0 * sqrt(zaspin(jz)) )
c
        zhp = 1.0 + 1.33*zalpha*(1.0+0.60*zalpha)/(1.0+1.79*zalpha)
c
        zk2hat = zb1 * ( 0.66 * (1.0 + 1.54*zalpha)
     &    + (1.88*sqrt(zaspin(jz)) - 1.54*zaspin(jz))
     &      * (1.0 + 3.75*zalpha) )
c
          xineo1(jz) = 0.0
c
        do 6030 ip=1,ipmax
c
          zmustr = max( abs(znstar(ip,jz) * ( 1.0 + 1.54 * zalpha ))
     &      / ( 1.0 + zalpha) , epslon )
c
          zk2 = zk2hat / ( 1.0 + za20 * sqrt(zmustr) + zb20 * zmustr )
     &          + zk20 * ( zc20**2/zb20 ) * zmustr
     &            * zaspin(jz)*sqrt(zaspin(jz)) * zhp * zf0
     &            / ( 1.0 + zc20 * zmustr*zaspin(jz)*sqrt(zaspin(jz)) )
c
 
          if ( ip .gt. mhyd   ) then
             zdens = rhois(ip - mhyd,1,jz)
          else
            zdens = rhohs(ip,1,jz)
          endif
c
          xineo1(jz) = xineo1(jz)
     &      + zdens * sqrt(zaspin(jz)) * zgyro2(ip,jz) 
     &        * zk2 * zifreq(ip,jz) / ( 1.0 + zalpha )
 6030   continue
c
  14  continue
c
\end{verbatim}

The choice {\tt cfutz(281)} = 2.0 turns on the
Bolton-Ware model\cite{bolt83a}

\begin{verbatim}
c
      else
c
      do 16 jz = lcentr, mzones
c
        zanc = ( .66 + 2.441 * sqrt(zaspin(jz))
     &    -3.87 * zaspin(jz) + 2.19 * (zaspin(jz)*sqrt(zaspin(jz))) )
        zbnc = ( 0.9362 - 3.109 * zaspin(jz)
     &    + 4.087 * zaspin(jz)**2 ) 
     &      / sqrt( zaspin(jz) * sqrt(zaspin(jz)) )
        zcnc = ( 0.241 + 3.40 * zaspin(jz)
     &    - 2.54 * zaspin(jz)**2 ) / (zaspin(jz)*sqrt(zaspin(jz)))
        zdnc = ( 0.2664 - 0.352 * zaspin(jz)
     &    + 0.44 * zaspin(jz)**2 ) / (zaspin(jz)*sqrt(zaspin(jz)))
c
        zaps = ( 0.364 - 2.76 * zaspin(jz)
     &    + 2.21 * zaspin(jz)**2 ) * (zaspin(jz)*sqrt(zaspin(jz)))
        zbps = ( 0.553 + 2.41 * zaspin(jz)
     &    - 3.42 * zaspin(jz)**2 ) * (zaspin(jz)*sqrt(zaspin(jz)))
        zcps = ( 1.18 + 0.292 * zaspin(jz) + 1.07 * zaspin(jz)**2 )
        zdps = ( 0.0188 + 0.180 * zaspin(jz) - 0.127 * zaspin(jz)**2 )
c
        xineo1(jz) = 0.0
        za   = 1.3293404 * (zaspin(jz)*sqrt(zaspin(jz)))
c
        do 6020 ip=1,ipmax
          ii  = ip - mhyd
          z0  = za * znstar(ip,jz)
          z1  = sqrt(z0)
          z2  = z0 * z1
          zk1 = zanc / ( 1.0 + zbnc * z1 + zcnc * z0 + zdnc * z0 * z0 )
          zk2 = 1.57 * (zaspin(jz)*sqrt(zaspin(jz)))
     &      + (zaps + zbps*z1) / (1.0 + zcps*z2 + zdps*z0*z2 )
          zka = zk1 + zk2
c
          if ( ip .gt. mhyd   ) then
            zdens = rhois(ii,1,j0)
          else
            zdens = rhohs(ip,1,j0)
          endif
c
          xineo1(jz) = xineo1(jz)
     &      +  sqrt(zaspin(jz)) * zka * zdens
     &           * zgyro2(ip,jz) * zifreq(ip,jz)
 6020   continue
c
  16  continue
c
      endif
c
\end{verbatim}

\subsection{Neoclassical Electron Thermal Diffusivity}

The diagonal electron thermal diffusivity is
$$ \chi_e^{\rm neo} = \frac{C_E q^s \lambda_e n_e}{B_Z^2 T_e^{1/2}}
  \left\{ \frac{0.73 ( 1.6 + 2 Z_{\rm eff}}{[1
  + 0.1 ( 1.6 + 2 Z_{\rm eff} ) \nu_e^* ] \epsilon^{3/2}} + 1.13
  + 0.5 Z_{\rm eff} + \frac{0.55 Z_{\rm eff}}{0.59 + Z_{\rm eff}}
  \right\}      \eqno{\tt xeneo1(jz)} $$

\begin{verbatim}
c
      do 28 jz = lcentr, mzones
c
        if ( xnuhyd(1,jz) .lt. epsinv ) then
c
c..neither b-poliodal or r are near 0
c
      xeneo1(jz) = cdetes * max ( q(jz)**2, 1.0 )
     &  * cloges(1,jz) * rhoels(1,jz) /
     1  (bzs**2 * sqrt(tes(1,jz))) * (0.73*(1.6 + 2.0*xzeff(1,jz)) /
     2  (1.0 + 0.1*(1.6 + 2.0*xzeff(1,jz))
     3  * xnuel(1,jz) ) * sqrt( rmids(jz,1) / ahalfs(jz,1) )**3 +
     4  1.13 + 0.50*xzeff(1,jz) + 0.55*xzeff(1,jz)/(0.59 + xzeff(1,jz)))
c
        else
c
c..either b-poloidal or r are near 0
c
c  zbr is b/r
c  note--it is assumed that b-poloidal is 0, although
c  r need not be.  i.e., if r=0, b-poloidal must be 0
c
      zbr = bpols(1,jz+1) / (rmins * xbouni(jz+1))
c
c  zetae = r**(3/2) * zbr * xnuel(1,jz)
c
c  i.e., zetah and zetae are xnuhyd and xnuel
c  without the b-poloidal and r**(1/2) terms
c
      zetae = cnuel * rmajs * bzs / telecs(1,jz) *
     &          sqrt(rmajs / tes(1,jz))
c
      xeneo1(jz) = cdetes * q(jz)**2 * cloges(1,jz) * rhoels(1,jz) /
     1          (bzs**2 * sqrt(tes(1,jz))) *
     2  (zbr * 0.73*(1.6 + 2.0*xzeff(1,jz)) / (zetae * 0.1*(1.6 +
     3  2.0*xzeff(1,jz))) * sqrt(rmajs)**3 +
     4  1.13 + 0.50*xzeff(1,jz) + 0.55*xzeff(1,jz)/(0.59 + xzeff(1,jz)))
c
        endif
c
  28  continue
c
\end{verbatim}

If {\tt lneocl(3)} = 1, we include a contribution to the electron heat
flux that is a multiple of the neoclassical ion heat flux (using 
whichever model is chosen above) corresponding to a generalization
of the model developed by Stringer.\cite{stri91a}
$$ q_e^{\rm neocl} = q_{e1}^{\rm neocl} + q_i^{\rm neocl} 
\frac{(T_e/T_i)^3/(1 + T_e/T_i )}{3 + (T_e/T_i)^2/(1 + T_e/T_i )}. $$
Here we set up a second neoclassical electron thermal diffusivity
$$ q_e^{\rm neocl} = - {\tt xeneo1(jz)} \nabla T_e
 - {\tt xeneo2(jz)} \nabla T_i.  $$

\begin{verbatim}
c
      call resetr ( xeneo2, mzones, 0.0 )
c
      if ( lneocl(3) .eq. 1 ) then
        do 29 jz = lcentr, mzones
          ztau = tes(1,jz) / tis(1,jz)
          xeneo2(jz) = xineo1(jz) * ztau**3 / ( ( 1.0 + ztau )
     &      * ( 3.0 + ztau**2 / ( 1.0 + ztau ) ) )
  29    continue
      endif
c
\end{verbatim}

\section{Diagonal Hydrogen Diffusivity}

$$ D_{aa}^{\rm neo} = C_H n_e q^2 \lambda_e \Lambda_E^{-1}
  (T_e + T_i) / ( T_e^{3/2} B_Z^2 )       \eqno{\tt dneo1(jz)} $$

\begin{verbatim}
c
      do 30 jz = lcentr, mzones
c
        dneo1(jz) = cdnhs * rhoels(1,jz) *  q(jz)**2 * cloges(1,jz) *
     &          gspitz(1,jz) * (tes(1,jz) + tis(1,jz)) /
     &          (sqrt(tes(1,jz))**3 * bzs**2)
c
  30  continue
c
c
\end{verbatim}

\section{Cross-diffusion Terms}

Allow cross-diffusion terms to remain zero if
${\tt cfutz(139)} < {\tt epslon}$ or if
Hawryluk-Hirshman ion-transport terms are active 
(${\tt cfutz(110)} > {\tt epslon}$).

\begin{verbatim}
c
      if ( cfutz(139) .lt. epslon ) go to 1180
      if ( cfutz(110) .gt. epslon ) go to 1180
c
\end{verbatim}

\subsection{Hydrogen-Impurity Off-diagonal Terms}

First compute the density diffusion of hydrogen species jh due to
gradients of impurity species ji.
Note that the coefficient of diffusion due to impurity density gradients 
does not depend in any way on the type of impurity.
$$ D_{Ia}^{Ib} = {\tt cfutz(366)}
  ( C_h q^2 \lambda_i \mu_{ab}^{1/2} )
  / ( B_Z^2 T_i^{1/2} )       \eqno{\tt dnhis and dnihs} $$

\begin{verbatim}
c
        cdnh0=cfutz(139)*cdnhis
        cdni0=cfutz(139)*cdniis
c
      if ( mimp .gt. 0 ) then
c
        do 40 jz = lcentr, mzones
c
          do 1120 jh = 1, mhyd
c
             zdnhis = cdnh0 * q(jz)**2 * clogis(1,jz)
     &                 * sqrt(aspec(jh) / tis(1,jz)) / bzs**2
c
             do 1120 ji = 1, mimp
               dnhis(jh,ji,jz) = zdnhis
               dnihs(ji,jh,jz) = zdnhis
 1120        continue
c
  40    continue
c
      endif
c
\end{verbatim}

\subsection{Hydrogen-Hydrogen Off-diagonal Terms}

Hydrogen-hydrogen and impurity-impurity terms are:
$$ D_{Ia}^{Ib} = {\tt cfutz(366)}
  ( C_I q^2 \lambda_i \mu_{ab}^{1/2} )
  / ( B_Z^2 T_i^{1/2} )       \eqno{\tt dnhhs(ji,ji2,jz)} $$
where
$$ \mu_{ab} = A_a A_b / ( A_a A_b )  $$
is the reduced hydrogen ion mass in atomic mass units. 

\begin{verbatim}
c
      do 1160 jh = 1, mhyd
      do 1160 jh2 = 1, mhyd
c
      if ( jh2 .eq. jh ) then
c
        do 42 jz = lcentr, mzones
          dnhhs(jh,jh,jz)=0.0
  42    continue
c
      else
c
        do 44 jz = lcentr, mzones
          dnhhs(jh,jh2,jz) = cfutz(366) * cdni0
     &         * clogis(1,jz) * q(jz)*2
     &         / ( bzs**2 *  sqrt( tis(1,jz) / 
     &      (aspec(jh)*aspec(jh2))/(aspec(jh)+aspec(jh2)) )  )
  44    continue
c
      endif
c
 1160 continue
c
\end{verbatim}

\subsection{Impurity-Impurity Off-diagonal Terms}

$$ D_{Ia}^{Ib} = ( C_I q^2 \lambda_i \mu_{ab}^{1/2} )
  / ( B_Z^2 T_i^{1/2} )       \eqno{\tt dniis(ji,ji2,jz)} $$

\begin{verbatim}
c
c  ...For comparison with BALDP86m, needed to use ".le.0" instead of 
c  ...".lt.2" in the following line.
c
      if ( mimp .lt. 2 ) go to 1180
c
      do 1170 ji = 1, mimp
      do 1170 ji2 = 1, mimp
c
c  ...Also, for comparison with BALDP86m, remove this if-then-else structure,
c  ...leaving the non-trivial assignment statement (this was a bug).
c
        if ( ji2 .eq. ji ) then
c
          do 46 jz = lcentr, mzones
            dniis(ji,ji2,jz) = 0.0
  46      continue
c
        else
c
          do 48 jz = lcentr, mzones
            dniis(ji,ji2,jz) = cdni0 * clogis(1,jz) * q(jz)**2 /
     &               (sqrt(tis(1,jz)/aired(ji,ji2)) * bzs**2)
  48      continue
c
        endif
c
 1170 continue
c
 1180 continue
c
\end{verbatim}

\section{Computation of $L_{13}$ and $L_{23}$}

$$ L_{13} = (1+K_1) K_{11} [ 1-(K_1 K_{11})/(1+K_1)] \eqno{\tt rl13} $$
$$ L_{23} = (2.5 +K_2) K_{12} [ 1-(K_1 K_{12})/(2.5 + K_2)] 
      \eqno{\tt rl23} $$
$$ K_{1j} = f_T / [1. + (\beta_j \nu_e^*)^{1/2} + \zeta_j \nu_e^* ]
      \eqno{tt zk11 and zk12} $$
$$ K_1 = (0.53 + Z_{\rm eff})/[Z_{\rm eff}(1+1.32 Z_{\rm eff})]
      \eqno{\tt zk1} $$
$$ K_2 = 0.82 (3-Z_{\rm eff})/[Z_{\rm eff} (2.57 + Z_{\rm eff})]
      \eqno{\tt zk2} $$
$$ \beta_1 = 0.4 + 0.11 Z_{\rm eff}    \eqno{\tt z1beta} $$
$$ \zeta_1 = 0.55 + 0.255 Z_{\rm eff}  \eqno{\tt z1zeta} $$
$$ \beta_2 = 0.05 + 0.0345 Z_{\rm eff} \eqno{\tt z2beta} $$
$$ \zeta_2 = 0.25 + 0.143 Z_{\rm eff}  \eqno{\tt z2zeta} $$.

\begin{verbatim}
c
      do 52 jz = lcentr, mzones
c
c       the following relations are taken from an analytic approx-
c       imation for the ware flux worked out by s. p. hirshman and
c       r. j. hawryluk
c
c       the first quantities defined are used in assembling the
c       analytic approximations.  the characters after "z" cor-
c       respond to hirshman and hawryluk's notation
c
C.MHH Dec. 23 1986 - require these coefficients at both cell centres
c     (bootstrap current) and cell boundaries (pinch effect).
c
      do 1081 jb=2,1,-1
c
        zft(jz) = 0.0
c
c..14.03: should only skip calculation of rl13 and rl23 when jb=1
c..and jz=lcentr - allows zone center values at jz=lcentr to be found.
c
      if ( bpols(jb,jz) .le. epslon ) then
        rl13(jz,jb) = 0.0
        rl23(jz,jb) = 0.0
        rly(jz,jb)  = 3.0
      else
c
      zk1    = (0.53+xzeff(jb,jz)) /
     &  (xzeff(jb,jz)*(1.0+1.32*xzeff(jb,jz)))
      zcr1   = zk1 / (1.0+zk1)
c
      zk2    = (0.82*(3.0-xzeff(jb,jz)))/
     &  (xzeff(jb,jz)*(2.57+xzeff(jb,jz)))
      zcr2   = zk2 / (2.5+zk2)
c
      z1beta = 0.4  + 0.11  * xzeff(jb,jz)
      z1zeta = 0.55 + 0.255 * xzeff(jb,jz)
      z2beta = 0.05 + 0.0345* xzeff(jb,jz)
      z2zeta = 0.25 + 0.143 * xzeff(jb,jz)
c
c  zft(jz) is the trapped fraction
c
      zd = abs ( ahalfs(jz,jb) / rmids(jz,jb) )
c
c...Replace this assignment with the following for comparison with
c      BALDP86m.
c      zd = abs ( ahalfs(jz,2) / rmids(jz,2) )
c..........
c
c..smooth cutoff for zft(j), bateman, 16-feb-86
c  use ramp function to smooth ftrap near magnetic axis
c  zdramp = 1.0 for r/a > cemprc(20)
c  zdramp = r / (a * cemprc(20)) for r/a < cemprc(20)
c
      zdramp = min ( 1.0 ,
     & ahalfs(jz,jb)/(ahalfs(mzones,1)*max(cemprc(20),epslon) ) )
c
      zft(jz) = 1.0 -  (1.0 - zd)**2
     &  / (sqrt(abs(1.0 - zd**2))*(1.0 + 1.46*sqrt( abs(zd*zdramp))))
c
C.MHH Dec 23 1986 - select new trapped particle calculation
c
      iopt=1
      if ( cfutz(481) .ne. 0.0) zft(jz) = ft(jz,jb,iopt)
c
c
      zk11 = zft(jz)
     &  / ( sqrt(z1beta*xnuel(jb,jz))+z1zeta*xnuel(jb,jz)+1.0 )
      zk12 = zft(jz)
     &  / ( sqrt(z2beta*xnuel(jb,jz))+z2zeta*xnuel(jb,jz)+1.0 )     
c
c..Introduce factors to "speed up" transition out of banana regime
c..near the axis; i.e., increase rate with which zl13 and zl23 approach
c..0 as r -> 0, analogous to procedure used on neoclassical resistivity
c..in subroutine getchi.
c
            ztrans=1.
        if ((ctrnsp(19).gt.epslon).and.(ctrnsp(20).gt.epslon)) then
          if (jb.eq.1) 
     &      ztrans=1./(1.+(ctrnsp(19)/xbouni(jz))**ctrnsp(20))
          if (jb.eq.2)
     &      ztrans=1./(1.+(ctrnsp(19)/xzoni(jz))**ctrnsp(20))
        end if
c
c       principal factors in the analytic expressions
c
C.MHH Dec 23 1986 - save coefficients - use RHH expression for *y*
c
      rl13(jz,jb) = (1.0+zk1) * zk11 * (1.0-zcr1*zk11) * ztrans
      rl23(jz,jb) = (2.5+zk2) * zk12 * (1.0-zcr2*zk12) * ztrans
      rly(jz,jb)  = (1.33+3.0*xnuhyd(jb,jz))/(1.0+xnuhyd(jb,jz))
c
      endif
c
 1081    continue
  52  continue
c
\end{verbatim}

\section{Ware Pinch}

The Ware pinch describes the radial flow of hydrogen isotopes due to the
presence of a toroidal electric field.
$$ v_{\rm Ware}^H = {\tt cfutz(19)} L_{13} (Z_{\rm eff},
\nu_{e,\epsilon}) ( c E / B_\theta )    \eqno{\tt vnwars(jz)} $$

{\tt vnwars(jz)} represents a neoclassical ware-pinch effect on
the flow of hydrogenic ions; 
while, {\tt vewars(jz)} expresses the older way of approximating
ware-pinch effects on electron energy flow.
To restore the older treatment set ${\tt cfutz(365)} = 1.0$.

\begin{verbatim}
c
      do 61 jz=1,mzones
        vnwars(jz) = 0.0
  61  continue
c
      if ( cfutz(19) .gt. epslon ) then
c
        zcc = fcc * 10.0**fxc
c
        do 62 jz = 3, mzones
c
c..Rewrite Ware pinch in terms of loop voltage
c
        if (versno.gt.14.01) then
          zk15 = uisv * 0.5 * (vloopi(jz,2)+vloopi(jz-1,2))
     &             * zcc / ( 2.*fcpi*r0ref*bpols(1,jz) )
        else
          zk15 = zcc * eta(1,jz) * ajzs(1,jz) / bpols(1,jz)
        end if
c
        vnwars(jz) = cfutz(19) * rl13(jz,1) * zk15
c
        vewars(jz) = cfutz(19) * rl23(jz,1) * zk15 * cfutz(365)
c
  62  continue
c
      endif
c
c     add ware-pinch effects to the generalized pinch velocities
c      (ware-pinch effects on impurities are presently ignored)
c   les  nov-90 d3he; add coppi-sharky anomalous inward pinch
c           vxcmg=0 if lcmg=0
c
      do 64 jh=1,mhyd
        do 63 jz = 1, mzones
          vxemps(jh,jz) = vxemps(jh,jz) + vnwars(jz) - vxcmg(jz)
  63    continue
 64   continue
c
c   les  nov-90 ; d3he ions different
c   impurities have v-ware = veware ???
c
      if ( limpn .gt. mhyd ) then
        do 1103 ji=mhyd+1,limpn
          vxemps(ji,jz) = vxemps(ji,jz) - vxcmg(jz)
 1103   continue
      endif
c
c  les  protons have vnwars
c
      if ( cfutz(490) .gt. epslon )
     &   vxemps(lprotn,jz) = vnwars(jz) + vxemps(lprotn,jz)
c
      return
      end
c--------1---------2---------3---------4---------5---------6---------7-c
c@ncflux  /11040/bald91/wbaldn1 DNEOCL
c  rgb 02-jun-96 remove if(.not.inital) go to 1000
c    save zxtaa, zxtexp, zxlar, zlarex, zxomeg, zxlar2
c       les  nov-90  reorder ion species for d-3he fusion
c       fgps 1-feb-83 revised to handle more than 4 ionic species
c       aes 15-apr-82 fix form feeds in ncfprt formats
c       aes 12-mar-82 make times printed in ncfprt same as in mprint
c       aes 20-jan-82 added label5 to page headers
c       aes 19-nov-81 add 'data incflx' --> control logic for ncfprt
c       aes 19-nov-81 move error lines 9000 to before entry ncfprt
c               -- change label to 8599; clean up labeling order
c       aes 17-nov-81 edit printout --> entry ncfprt
c       aes 29-oct-81 array dimensions 52 -->55 in common/tempry,neoion/
c       aes 28-oct-81 put all data statements before executable code
c       fgps 20-jul-79 adapted subroutine ncflux (hawryluk 1-jun-
c                      79) for inclusion in baldur
c
c**********************************************************************c
c
        subroutine ncflux
c
c
cl      2.22    neoclassical particle transport redefined
c               according to hawryluk and hirshman
c
c
c**********************************************************************c
c
c
      include '../com/cbaldr.m'
      include '../com/commhd.m'
      include '../com/cbparm.m'
c
c       do not equivalence arrays; but, when there are 4 impurity
c       species, mxions = 6.
c
c------------------------------------------------------------------------------
c       june 1,1979
c
c       this routine calculates the neoclassical impurity fluxes
c       as well as the hydrogenic fluxes. electron-ion fluxes are
c       not calculated.
c       no mass ratio expansion is used in these calculations.
c       this calculation is based upon pppl-1473.       -appendix a
c
      dimension
     1zmass(idxion)        ,zcharg(idxion)       ,zdens(idxion)        ,
     2zab(idxion,idxion)   ,zc11(idxion,idxion)  ,zc12(idxion,idxion)  ,
     3zxc11(idxion,idxion) ,zxc12(idxion,idxion) ,zxb1b(idxion,idxion) ,
     4zxb2b(idxion,idxion) ,zxi11(idxion,idxion) ,zxi01(idxion,idxion) ,
     5zxi00(idxion,idxion) ,ztaa(idxion)         ,zlar2(idxion)        ,
     6zmu1(idxion)         ,zmu2(idxion)
c
c       june 8,1979
c       the following arrays are required for the pfirsch-schluter fluxes
c       the calculations are based on the work by
c       s.p.hirshman ,phys.fluids, pg 589(1977)
c
      dimension
     1zxn00(idxion,idxion) ,zxn01(idxion,idxion) ,zxn02(idxion,idxion) ,
     2zxn11(idxion,idxion) ,zxn12(idxion,idxion) ,zxn22(idxion,idxion) ,
     3zxm00(idxion,idxion) ,zxm01(idxion,idxion) ,zxm02(idxion,idxion) ,
     4zxm11(idxion,idxion) ,zxm12(idxion,idxion) ,zxm22(idxion,idxion) ,
     5zxhm22(idxion,idxion),zl11(idxion,idxion)  ,zxmmnn(idxion,idxion),
     6zl12(idxion,idxion)  ,zl22(idxion,idxion)  ,zm00(idxion)         ,
     7zm01(idxion)         ,zm11(idxion)         ,zm12(idxion)         ,
     8zalpha(idxion,idxion),zbeta(idxion,idxion) ,zdelta(idxion,idxion),
     9zhalph(idxion,idxion),zhbeta(idxion,idxion),zxa22(idxion,idxion) ,
     &zm20(idxion)         ,zm21(idxion)         ,zvstr(idxion)        ,
     1zw(idxion)           ,zm22(idxion)         ,zm02(idxion)
c
c       the following arrays are related to the strong
c       temperature equilibration approximation
c
      dimension
     1zc2(idxion,idxion)   ,zlh22(idxion)        ,zin2(idxion,idxion)  ,
     2zin22(idxion,idxion) ,zch2(idxion)         ,zs11(idxion,idxion)  ,
     3zs12(idxion,idxion)  ,ziii(idxion,idxion)  ,zzl22(idxion,idxion) ,
     4ipivot(idxion)
c
c
        logical         inital       , ladjst
c
c
c
        data    iclass /2/,     isub /22/
c
cahk
        save zxtaa, zxtexp, zxlar, zlarex, zxomeg, zxlar2
c
        if (.not.nlomt2(isub)) go to 10
        call mesage(' *** 2.22 subroutine ncflux bypassed ')
        return
   10   continue
c
c------------------------------------------------------------------------
c
c       cfutz(incflx) .ne. 0 allows subroutine convrt to call
c       subroutine ncflux.  arguments of cfutz-factors which ad-
c       just the levels of regimes are:  iclflx for classical,
c       ibpflx for banana-plateau, and ipsflx for pfirsch-
c       schluter.
c
        data    incflx,iclflx,ibpflx,ipsflx /110,111,112,113/
c
        data    inital,ladjst /.true.,.false./
c
c
cbate        if(.not.inital) go to 1000
        inital=.false.
        if(cfutz(iclflx).lt.epslon) cfutz(iclflx)=1.0
        if(cfutz(ibpflx).lt.epslon) cfutz(ibpflx)=1.0
        if(cfutz(ipsflx).lt.epslon) cfutz(ipsflx)=1.0
        if(cfutz(iclflx).ne.1.0) ladjst=.true.
        if(cfutz(ibpflx).ne.1.0) ladjst=.true.
        if(cfutz(ipsflx).ne.1.0) ladjst=.true.
c
c       initialize several constants used in the calculation.
c
        zxtaa=3.*sqrt(fcau)/(4.*sqrt(2.*fcpi)*fces**4)
        zxtexp=0.5*fxnucl-4.*fxes
        zxtaa=zxtaa*10.**zxtexp
        zxlar=fcc*sqrt(2.*fcau)/fces
        zlarex=fxc+0.5*fxnucl-fxes
        zxlar=zxlar*10.**zlarex
        zxomeg=sqrt(2./fcau)*10.**(-0.5*fxnucl)
        zxlar2=zxlar*zxlar
c
c       initialize several arrays which depend only upon
c       the atomic mass of the ions.
c
c       limpn=number of ion species
c       mhyd=number of hydrogen species
c       mimp=number of impurity species
c       aspec(ih)=atomic weight of hydrogen, ih=1,mhyd
c       aspec(ii)=atomic weight of impurities, ii=mhyd+1,limpn
c       cmean=mean z of the impurities--the difference between
c               the mean z and the mean z*z has been ignored
c
c       limit set equal to number of active ion species
c
        ihmax=limpn
        if(mimp.le.0) ihmax=mhyd
c
c       inverse aspect ratio at outer! edge of scrapeoff region
c       if such exists, otherwise at outer edge of plasma
c
        zasp0=rmins/rmajs
c
c       store at. wt. of ionic species into zmass(*)
c
c   les  nov-90  d3he fusion -- reorder ions by atomic weigth
c
      ipstrt=1
      if ( cfutz(490).gt.epslon ) then
        zmass(1)=aspec(lprotn)
        zmass(2)=aspec(ldeut)
        zmass(3)=aspec(ltrit)
        ipstrt=4
      endif
c
      do 200 ip=ipstrt,ihmax
        zmass(ip)=aspec(ip)
 200  continue
c
        do 300 ih=1,ihmax
        zxc11(ih,ih)=0.
        zxc12(ih,ih)=0.
c
        do 350 ihh=1,ihmax
c
        zx2=(zmass(ih)/zmass(ihh))
        zxa22(ih,ihh)=1./zx2
        zx4=zx2*zx2
        zx6=zx4*zx2
        zx8=zx6*zx2
        zx=sqrt(zx2)
        zxa1=1./(1.+zx2)
        zx12=sqrt(zxa1)
        zxa12=1./zx12
        zx32=zx12*zxa1
        zx52=zx32*zxa1
        zx72=zx52*zxa1
        zx92=zx72*zxa1
c
c       calculate the constant factor which is part of the classical
c       fluxes  (eq. a5a,b)
c       t(ih)=t(ihh) is assumed throughout.
c
        if(ih .eq. ihh) go to 220
        zxc11(ih,ihh)=zx12
        zxc12(ih,ihh)=-1.5*zx2*zx32
220     continue
c
c       to calculate the anisotropy driven fluxes, several arrays
c       must be calculated-eq a11.
c
c
c       see eq a11a and a11b.
c
        zxb1b(ih,ihh)=zxa12+zx2*log(zx/(1.+zxa12))
        zxb2b(ih,ihh)=zx12
c
c       see eq a12,a13,a14
c
        zxi00(ih,ihh)=(3.+5.*zx2)*zx32
        zxi01(ih,ihh)=(4.5+10.5*zx2)*zx52
        zxi11(ih,ihh)=(35.*zx6+38.5*zx4+46.25*zx2+12.75)*zx72
c
c       the constant coefficients required for the p-s
c       fluxes will be calculated:(sph equation a3)
c
c       the n00 etc. arrays are divided by zx compared with sph
c
        zxm00(ih,ihh)=-zx12
        zxn00(ihh,ih)=zx*zx12
        zxm01(ih,ihh)=-1.5*zx32
        zxn01(ihh,ih)=1.5*zx*zx32
        zxm02(ih,ihh)=-1.875*zx52
        zxn02(ihh,ih)=1.875*zx*zx52
        zxm11(ih,ihh)=-zx52*(7.5*zx4+4.*zx2+3.25)
        zxm12(ih,ihh)=-zx72*(15.75*zx4+6.*zx2+4.3125)
        zxm22(ih,ihh)=-zx92*
     .  (21.875*zx8+28.*zx6+57.375*zx4+17.*zx2+6.765)
        zxn11(ih,ihh)=6.75*zx52*zx2
        zxn12(ih,ihh)=14.0625*zx72*zx4
        zxn22(ih,ihh)=41.016*zx92*zx4
c
c       see equations a19 in ppl 1473
c
        zxhm22(ih,ihh)=-zxm22(ih,ihh)
        zxmmnn(ih,ihh)=3.*zx72*(2.5*zx4+2.*zx2+3.25)*zx2
350     continue
        zxi00(ih,ih)=zxi00(ih,ih)-.707107
        zxi01(ih,ih)=zxi01(ih,ih)-1.06066
        zxi11(ih,ih)=zxi11(ih,ih)-2.65165
c
c       see equations a19 in ppl 1473
c
        zxhm22(ih,ih)=zxhm22(ih,ih)-1.81
        zxmmnn(ih,ih)=zxmmnn(ih,ih)-.33145
c
300     continue
c
c
1000    continue
c
c-----------------start of main do-loop over radial index j---------------
c
        do 10000 j=lcentr,mzones
c
c
c       create a local array composed of the hydrogenic and impurity
c       species and their charge
c
c   les  nov-90  for d3he, reorder ions by atomic weight
c
      if ( cfutz(490).le.epslon ) go to 1009
        zdens(1)=rhois(lprotn-lhydn,1,j)
        zdens(2)=rhohs(ldeut,1,j)
        zdens(3)=rhohs(ltrit,1,j)
        zcharg(1)=1.
        zcharg(2)=1.
        zcharg(3)=1.
        do 1002 ii=4,ihmax
          zdens(ii)=rhois(ii-lhydn,1,j)
 1002  zcharg(ii)=cmean(ii-lhydn,1,j)
      go to 1220
 1009  continue
c
        do 1100 ih=1,mhyd
        zdens(ih)=rhohs(ih,1,j)
        zcharg(ih)=1.
1100    continue
c
c
        if(mimp.le.0) go to 1220
        do 1200 ih=1,mimp
        ii=ih+lhydn
        zdens(ii)=rhois(ih,1,j)
        zcharg(ii)=cmean(ih,1,j)
1200    continue
1220    continue
c
c       calculate the effective charge
c
        do 1300 ih=1,ihmax
        do 1300 ihh=1,ihmax
1300    zab(ih,ihh)=zcharg(ihh)*zcharg(ihh)*zdens(ihh)/
     1  (zcharg(ih)*zcharg(ih)*zdens(ih))
c
c       calculate the mu coefficients eq a8-a9
c
        zmsum=0.
        zeps=xbouni(j)*zasp0
        if(zeps .le. 1.e-10) zeps=1.e-10
        zeps32=zeps**1.5
        z1taa=zxtaa*tis(1,j)**1.5/clogis(1,j)
        z1lar2=zxlar2*tis(1,j)/(bzs*bzs)
        z1omeg=zxomeg*sqrt(tis(1,j))/(q(j)*rmajs+epslon)
c
c
        do 2000 ih=1,ihmax
c
c
        ztaa(ih)=z1taa*sqrt(zmass(ih))/(zdens(ih)*zcharg(ih)**4)
        zlar2(ih)=z1lar2*zmass(ih)/(zcharg(ih)*zcharg(ih))
        zomeg=z1omeg/sqrt(zmass(ih))
        zvstr(ih)=1./(ztaa(ih)*zomeg*zeps32)
c
c       see eq a11-a13
c
        zb1b=0.
        zb2b=0.
        zi11=0.
        zi01=0.
        zi00=0.
        do 2500 ihh=1,ihmax
        zb1b=zb1b + zab(ih,ihh)*zxb1b(ih,ihh)
        zb2b=zb2b + zab(ih,ihh)*zxb2b(ih,ihh)
        zi11=zab(ih,ihh)*zxi11(ih,ihh)+zi11
        zi01=zab(ih,ihh)*zxi01(ih,ihh)+zi01
        zi00=zab(ih,ihh)*zxi00(ih,ihh)+zi00
2500    continue
c
        zb1p=.607/zvstr(ih)
        zb2p=3.*zb1p
        ziot1=2.5*zi11/(zi11*zi00 -zi01*zi01)
        ziot2=2.5*(zi11+3.5*zi01)/(zi11*zi00-zi01*zi01)
        zden=1./(zvstr(ih)*zvstr(ih)*zeps32)
        zb1ps=.514*ziot1*zden
        zb2ps=.514*(ziot2+2.5*ziot1)*zden
        zb1=zb1b/((1.+zb1b/zb1p)*(1.+zb1p/zb1ps))
        zb2=zb2b/((1.+zb2b/zb2p)*(1.+zb2p/zb2ps))
c
c       normalize the original mu matrix coefficients by the proton mass.
c
        zmu1(ih)=zdens(ih)*zmass(ih)*zb1/ztaa(ih)
        zmu2(ih)=zmu1(ih)*(zb2-2.5*zb1)/zb1
        zmsum=zmsum+zmu1(ih)
c
2000    continue
c
c
c       calculate the arrays required for the classical
c       coefficients
c
        do 2600 ih=1,ihmax
        zc11(ih,ih)=0.
        zc12(ih,ih)=0.
        do 2650 ihh=1,ihmax
        if(ih .eq. ihh) go to 2650
        zc11(ih,ihh)=zab(ih,ihh)*zxc11(ih,ihh)
        zc12(ih,ihh)=zab(ih,ihh)*zxc12(ih,ihh)
        zc11(ih,ih)=zc11(ih,ih)-zc11(ih,ihh)
        zc12(ih,ih)=zc12(ih,ih)-zc12(ih,ihh)*zxa22(ih,ihh)
2650    continue
2600    continue

c
c       now both the classical (cl11 and cl12)
c       and the anisotropy drive coefficients (bp11,bp12) can be calculated
c
        zb1=.73*q(j)*q(j)/zeps32
        do 3000 ih=1, ihmax
        za1=0.5*zlar2(ih)*zdens(ih)*zcharg(ih)/ztaa(ih)
        do 3500 ihh=1,ihmax
c
c       classical coefficients
c
        za=za1/zcharg(ihh)
        cl11(ih,ihh,j)=za*zc11(ih,ihh)
        cl12(ih,ihh,j)=za*zc12(ih,ihh)
c
c       anisotropy drive fluxes
c
        za=zb1*zcharg(ihh)/zcharg(ih)
     1  *zlar2(ihh)/zmass(ihh)
        z1=zmu1(ih)/zmsum
        if(ih .eq. ihh) z1=z1-1.
c
        bp11(ih,ihh,j)=za*zmu1(ihh)*z1
        bp12(ih,ihh,j)=za*zmu2(ihh)*z1
c
3500    continue
3000    continue
c
c
c       we will now calculate the friction coefficients
c       in the pfirsch-schluter regime.
c

c       calculate the m arrays --sph equation 15.
c
        do 4000 ih=1,ihmax
        zm00(ih)=0.
        zm01(ih)=0.
        zm11(ih)=0.
        zm12(ih)=0.
        zm02(ih)=0.
        zm22(ih)=0.
c
        z1=zdens(ih)/ztaa(ih)
        do 4100 ihh=1,ihmax
        zm00(ih)=zm00(ih)+zab(ih,ihh)*zxm00(ih,ihh)
        zm11(ih)=zm11(ih)+zab(ih,ihh)*zxm11(ih,ihh)
        zm12(ih)=zm12(ih)+zab(ih,ihh)*zxm12(ih,ihh)
        zm22(ih)=zm22(ih)+zab(ih,ihh)*zxm22(ih,ihh)
        zm02(ih)=zm02(ih)+zab(ih,ihh)*zxm02(ih,ihh)
        zm01(ih)=zm01(ih)+zab(ih,ihh)*zxm01(ih,ihh)
4100    continue
        zm00(ih)=zm00(ih)*z1
        zm11(ih)=zm11(ih)*z1
        zm12(ih)=zm12(ih)*z1
        zm22(ih)=zm22(ih)*z1
        zm02(ih)=zm02(ih)*z1
        zm01(ih)=zm01(ih)*z1
4000    continue
c
        do 4150 ih=1,ihmax
        zm20(ih)=0.
        zm21(ih)=0.
        do 4175 ihh=1,ihmax
        z1=zdens(ihh)/ztaa(ihh)
        zm20(ih)=zm20(ih)+z1*zab(ihh,ih)*zxm02(ihh,ih)
        zm21(ih)=zm21(ih)+z1*zab(ihh,ih)*zxm12(ihh,ih)
4175    continue
4150    continue
c
c       calculate the alpha,beta,and delta arrays
c       see sph equation 14
c
        do 4200 ih=1,ihmax
        z1=zdens(ih)/ztaa(ih)
        zden=1./(zm22(ih)+z1*zxn22(ih,ih))
c
        do 4250 ihh=1,ihmax
        if(ih .eq. ihh) go to 4250
        z2=zdens(ihh)/ztaa(ihh)*zab(ihh,ih)*zxa22(ih,ihh)*zden
        zalpha(ih,ihh)=zxn02(ihh ,ih)*z2
        zbeta(ih,ihh)=zxn12(ihh,ih)*z2
        zdelta(ih,ihh)=-zxn22(ihh,ih)*z2
4250    continue
        zalpha(ih,ih)=zden*(zm20(ih)+z1*zxn02(ih,ih))
        zbeta(ih,ih)=zden*(zm21(ih)+z1*zxn12(ih,ih))
        zdelta(ih,ih)=0.
4200    continue
c
c       calculate the w arrays
c       see equation a21c in ppl 1473
c
        do 4500 ih=1,ihmax
        zmmnn=0.
        zmm22=0.
        do 4550 ihh=1,ihmax
        zmmnn=zmmnn+zab(ih,ihh)*zxmmnn(ih,ihh)
        zmm22=zmm22+zab(ih,ihh)*zxhm22(ih,ihh)
4550    continue
        zw(ih)=1./(1.+9.57/
     1  (zeps32*zeps32*zvstr(ih)*zvstr(ih)*zmm22*zmmnn))
4500    continue
c
c       calculate the alpha and beta hat arrays
c       see equation a21 in ppl 1473
c
        do 4700 ih=1,ihmax
        do 4750 ihh=1,ihmax
        zsuma=0.
        zsumb=0.
        do 4800 ihk=1,ihmax
        zsuma=zsuma+zw(ihk)*zdelta(ih,ihk)*zalpha(ihk,ihh)
        zsumb=zsumb+zw(ihk)*zdelta(ih,ihk)*zbeta(ihk,ihh)
4800    continue
        zhalph(ih,ihh)=zw(ih)*(zalpha(ih,ihh)+zsuma)
        zhbeta(ih,ihh)=zw(ih)*(zbeta(ih,ihh)+zsumb)
4750    continue
4700    continue
c
c       calculate the friction coefficients
c       see sph eq 20a-c
c
c       the friction coefficients have been normalized by the proton mass
c
        do 5000 ih=1,ihmax
        z1=zdens(ih)/ztaa(ih)
        do 5100 ihh=1,ihmax
        zsuma=0.
        zsumb=0.
        zsumc=0.
c
        do 5150 ihk=1,ihmax
        z2=z1*zab(ih,ihk)
        zsuma=zsuma+z2*zxn02(ih,ihk)*zhalph(ihk,ihh)
        zsumb=zsumb+z2*zxn02(ih,ihk)*zhbeta(ihk,ihh)
        zsumc=zsumc+z2*zxn12(ih,ihk)*zhbeta(ihk,ihh)
5150    continue
        z2=z1*zab(ih,ihh)
        z11=-zm02(ih)*zhalph(ih,ihh)+z2*zxn00(ih,ihh)-zsuma
        z12=zm02(ih)*zhbeta(ih,ihh)-z2*zxn01(ih,ihh)+zsumb
        z22=-zm12(ih)*zhbeta(ih,ihh)+z2*zxn11(ih,ihh)-zsumc
        if(ih .ne. ihh) go to 5160
        z11=z11+zm00(ih)
        z12=z12-zm01(ih)
        z22=z22+zm11(ih)
c
5160    zl11(ih,ihh)=zmass(ih)*z11
        zl12(ih,ihh)=zmass(ih)*z12
        zl22(ih,ihh)=zmass(ih)*z22
5100    continue
5000    continue
c
c       now the coefficients will be evaluated in the strong temperature
c       equilibration approximation
c       see sph section iii.b
c
c       invert the l22 matrix
c
c       create a dummy l22 array and an identity matrix
c
c
        do 6000 ih=1,ihmax
        do 6050 ihh=1,ihmax
        ziii(ih,ihh)=0.
6050    zzl22(ih,ihh)=zl22(ih,ihh)
        ziii(ih,ih)=1.
6000    continue
c
        call matrx1(zzl22,mxions,ihmax,ipivot,ierror)
c
        if(ierror .ne. 0) go to 9000
        call matrx2(zin22,ziii,zzl22,ipivot,mxions,ihmax,1,ihmax,1)
c
c       calculate the c2 and lhat22 matrix
c       see sph equation 35a and b
c
        zlhd=0.
        do 6200 ih=1,ihmax
        zsumb=0.
        do 6250 ihh=1,ihmax
        zsuma=0.
        do 6300 ihk =1,ihmax
        zsuma=zsuma+zl12(ihh,ihk)*zin22(ih,ihk)
6300    continue
        zc2(ih,ihh)=2.5*zdens(ih)/zdens(ihh)*zsuma
        zsumb=zsumb+zin22(ih,ihh)*zdens(ihh)
6250    continue
        zlh22(ih)=2.5*zsumb
        zlhd=zlh22(ih)*zdens(ih)+zlhd
6200    continue
c
c       evaluate zch2--see sph equation 40
c
        do 6400 ih=1,ihmax
        zch2(ih)=0.
        do 6350 ihh=1,ihmax
        zch2(ih)=zch2(ih)+zc2(ihh,ih)
6350    continue
6400    continue
c
c       calculate the effective l matrix (friction coefficients)
c       for strong temperature equilibration
c
        do 7000 ih=1,ihmax
        do 7500 ihh=1,ihmax
        zsum=0.
        do 7600 ihk=1,ihmax
        zsum=zsum+zl12(ih,ihk)*
     1  (zlh22(ihk)*zch2(ihh)/zlhd-zc2(ihk,ihh)/zdens(ihk))
7600    continue
        zs11(ih,ihh)=zl11(ih,ihh)+0.4*zdens(ihh)*zsum
        zs12(ih,ihh)=zdens(ih)*zdens(ihh)*zch2(ih)/zlhd
7500    continue
7000    continue
c
c       calculate the pfirsch-schluter coefficients (ps11 and ps12)
c       in the strong temperature approximation
c
c
c       if the weak temperature approximation is to be used
c       replace zs11 and zs12 respectively with zl11 and zl12 in
c       the following equations.
c
c
        do 8000 ih=1,ihmax
        za1=q(j)*q(j)*zlar2(ih)*zcharg(ih)/zmass(ih)
        do 8100 ihh=1,ihmax
        ps11(ih,ihh,j)=za1/zcharg(ihh)*zs11(ih,ihh)
        ps12(ih,ihh,j)=za1/zcharg(ihh)*zs12(ih,ihh)
8100    continue
8000    continue
c
        if(.not.ladjst) go to 8500
        do 8200 ihh=1,ihmax
        do 8200 ih=1,ihmax
        cl11(ih,ihh,j)=cfutz(iclflx)*cl11(ih,ihh,j)
        cl12(ih,ihh,j)=cfutz(iclflx)*cl12(ih,ihh,j)
        bp11(ih,ihh,j)=cfutz(ibpflx)*bp11(ih,ihh,j)
        bp12(ih,ihh,j)=cfutz(ibpflx)*bp12(ih,ihh,j)
        ps11(ih,ihh,j)=cfutz(ipsflx)*ps11(ih,ihh,j)
        ps12(ih,ihh,j)=cfutz(ipsflx)*ps12(ih,ihh,j)
 8200   continue
c
 8500   continue
c
10000   continue
c
c   les  nov-90  for d3he fusion, reorder ions back to d,t,p,3he,4he
c
      if ( cfutz(490).le.epslon ) then
        call reorder(ihmax,mzones,cl11)
        call reorder(ihmax,mzones,cl12)
        call reorder(ihmax,mzones,bp11)
        call reorder(ihmax,mzones,bp12)
        call reorder(ihmax,mzones,ps11)
        call reorder(ihmax,mzones,ps12)
      endif
c
c-----------------end of main do-loop over the radial index j-------------
c
c
        return
 9000   continue
        call error(1,iclass,isub,2, 
     >         ' *** error *** in solution for l22 ')
        return
c**********************************************************************
c
        entry ncfprt
c
c       edit print-out of neoclassical particle transport coefficients
c
        if(cfutz(incflx).le.epslon) return
c
        lpage = lpage + 1
c
        z0=uist*1.e+03
        zt=tai*z0
        zdt=dtoldi*z0
        write (nprint,9010) label1(1:48),label5(1:72),
     1          lpage,nstep,zt,zdt
        write(nprint,9020)
        do 8600 jz=lcentr,mzones
        write(nprint,9030) cl11(1,1,jz),cl12(1,1,jz),
     1  cl11(1,2,jz),cl12(1,2,jz),bp11(1,1,jz),bp12(1,1,jz),
     2  bp11(1,2,jz),bp12(1,2,jz),ps11(1,1,jz),ps12(1,1,jz),
     3  ps11(1,2,jz),ps12(1,2,jz)
 8600   continue
c
        return
 9010   format(1h1,2x,a48,10x,a72/
     1  '  -',i2,'-  *** time step ',i5,' ***',14x,'time =',
     2  0pf12.3,'  millisecs.',12x,'dt =',0pf12.6,'  millisecs.')
 9020 format(29x,' hawryluk-hirshman neoclassical particle transport'
     & //2x,'cl11(1,1)',1x,'cl12(1,1)',1x,'cl11(1,2)',1x,
     & 'cl12(1,2)',1x,'bp11(1,1)',1x,'bp12(1,1)',1x,'bp11(1,2)',1x,
     & 'bp12(1,2)',1x,'ps11(1,1)',1x,'ps12(1,1)',1x,'ps11(1,2)',1x,
     & 'ps12(1,2)')
 9030 format(1x,12(1x,1pe9.2))
c
        end
c--------1---------2---------3---------4---------5---------6---------7-c
\end{verbatim}

\begin{thebibliography}{99}

\bibitem{hint76a}  F.~L. Hinton and R.~D. Hazeltine,
``Theory of plasma transport in toroidal confinement systems,''
Reviews of Modern Physics, {\bf 48} (1976) 239--308.

\bibitem{chang82a} C.~S. Chang and F.~L. Hinton,
``Effect of finite aspect ratio on the neoclassical ion thermal 
conductivity in the banana regime,'' 
Phys. Fluids {\bf 25} (1982) 1493--1494.

\bibitem{chang86a} C.~S. Chang and F.~L. Hinton,
``Effect of impurity particles on the finite aspect-ratio neoclassical
ion thermal conductivity in a tokamak,''
Phys. Fluids {\bf 29} (1986) 3314--3316.

\bibitem{bolt83a} C. Bolton and A.~A. Ware,
Phys. Fluids {\bf 26} (1983) 459--467

\bibitem{stri91a} T.~E. Stringer,
``Inclusion of poloidal potential variation in neoclassical transport,''
Phys. Fluids B {\bf 3} (1991) 981--988.

\bibitem{ruth74a} P.~H. Rutherford,
``Impurity transport in the Pfirsch-Schl\"{u}ter regime,''
Phys. Fluids {\bf 17} (1974) 1782--1784.

\bibitem{ruth76a} P.~H. Rutherford, S.~P. Hirshman, R. Jenson,
D. Post and F.~P.~G. Seidl,
``Impurity transport in tokamaks,''
Princeton Plasma Physics Laboratory report PPPL-1297
(October, 1976).

\bibitem{hirs76a} S.~P. Hirshman,
``Transport properties of a toroidal plasma in a mixed 
collisionality regime,''
Phys. Fluids {\bf 19} (1976) 155--158.

\bibitem{hawr79a} R.~J. Hawryluk, S. Suckewer, and S.~P. Hirshman,
``Low-Z impurity transport in tokamaks,''
Nuclear Fusion {\bf 19} (1979) 607--632.

\bibitem{hirs81a} S.~P. Hirshman and D.~J. Sigmar,
``Neoclassical transport of impurities in tokamak plasmas,''
Nuclear Fusion {\bf 21} (1981) 1079--1201.

\bibitem{hirs77a} S.~P. Hirshman, R.~J. Hawryluk, and B. Birge,
``Neoclassical conductivity of a tokamak plasma,''
Nuclear Fusion {\bf 17} (1977) 611--613.

\bibitem{hirs78a} S.~P. Hirshman,
``Neoclassical current in a toroidally-confined multispecies plasma,''
Phys. Fluids {\bf 21} (1978) 1295--1301.


\end{thebibliography}

\end{document}
