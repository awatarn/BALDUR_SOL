\documentstyle {article}    % Specifies the document style.
\headheight 0pt \headsep 0pt  \topmargin 0pt  \oddsidemargin 0pt
\textheight 9.0in \textwidth 6.5in

\title{ {\tt ptheory}: a BALDUR Subroutine \\
 Interface between the BALDUR Transport Code \\
 and Subroutine Ntheory}     % title.
\author{
        Ping Zhu \\ UT Austin}
                           % Declares the author's name.
                           % Deleting the \date{} produces today's date.
\begin{document}           % End of preamble and beginning of text.
\maketitle                 % Produces the title.

This report documents a subroutine called {\tt vpolprof}, which 
interfaces with different routines computing plasma poloidal velocity 
using various neoclassical theory-based models.

Current included models:
\begin{enumerate}
\item Model implemented by P. Strand. ({\tt imod = 1})
\item Model implemented by M. Erba. ({\tt imod = 2})
\item NEO model implemented by P. Zhu. ({\tt imod = 3})
\end{enumerate}

The default is NEO model, which was most recently reported in Zhu~{\it etal}~\cite{zhu99}.

\begin{verbatim}
c@vpolprof ../code/baldur/bald/vpolprof.tex
c pzhu 06-Sep-99 included the NEO model
c   See rest of changes list at end of file
c--------1---------2---------3---------4---------5---------6---------7-c
\end{verbatim}

\begin{verbatim}
c
      subroutine vpolprof(imod, ns, ma, za, nx, den, temp, vpol, 
     &  rmin, rmaj, btor, bpol, qpsi, ierr)
c
      implicit none
c
      integer imod              !model index
                                !1: Strand formula
                                !2: Erba formula
                                !3: NEO model
      integer :: ierr       	!error flag
                                !0: no error upon return
      integer :: iprint = 0	!control parameter for output
c      
      integer ns                !number of particle species including electron
                                !1: electron 
                                !2: main ion 
                                !3: impurity ion
      real ma(ns)               !particle mass [u]
      real za(ns)               !particle charge [e]
      integer nx                !number of radial grid points
      real den(ns,nx)           !density [m^-3]
      real temp(ns,nx)          !temperature [eV]
      real vpol(ns,nx)          !poloidal velocity [m/s]
      real rmin(nx)             !minor radius of plasma [m]
      real rmaj(nx)             !major radius of plasma [m]
      real btor(nx)             !toroidal B field [T]
      real bpol(nx)             !poloidal B field [T]
      real qpsi(nx)             !safty factor 
c
      integer :: intrp  = 2     !use Akima spline
      integer :: kextrp = 1     !extrapolate linearly
      integer :: kderiv = 1     !calculate first derivatives
      integer :: kskip  = 1     !use only 1-d arrays as input
      integer kxlast
      integer is                !local species index
      integer jx                !local radial grid index
      real :: zcoefpol = 1.d0   !control parameter 
      real zepsr(nx)            !inverse aspect ratio
      real zdti(nx)             !temperature gradient
      real zfc, zk1
c
      real xarr(nx)
      real xbarr(nx)
c
      select case (imod)
c      
      case(1)
c-----------------------------------------------------
c     Neoclassical expression for vpol (by P.Strand)        
c-----------------------------------------------------
        do jx = 1, nx-1
          zepsr(jx) = rmin(jx)/rmaj(jx)
        end do
c
        do is = 1, ns
          call int1d(intrp,kextrp,kderiv,nx-1,kskip,rmin,temp(is,:),nx-1,rmin
     &      ,kxlast,zdti)
          do jx = 1, nx-1
            zfc = 1.0 - 1.46*sqrt(zepsr(jx)) + 
     &			0.46*zepsr(jx)*sqrt(zepsr(jx))
            zk1 = 0.8839 * zfc/ (0.3477 + 0.4058 * zfc)
            vpol(jx,is) = zcoefpol * zk1 * zdti(jx) * btor(jx+1) /
     &        (btor(jx+1)**2 + bpol(jx+1)**2)
          end do
        end do
c
      case (2)
c-----------------------------------------------------
c     Experimental expression for vpol (by M.Erba)
c-----------------------------------------------------
        vpol = 1.e3*temp
c
      case (3)
c------------------------------------------------------------------
c     Neoclassical routine for vpol based on NEO model (by P.Zhu)
c------------------------------------------------------------------
	xarr = rmin/rmin(nx)
c 
        call neo_pol(ns, ma, za, 
     &	  nx-2, xarr(3:nx), xarr(3:nx), 
     &	  den(1:ns,3:nx), temp(1:ns,3:nx), vpol(1:ns,3:nx), 
     &    rmin(nx), rmaj(3), btor(3), qpsi(3:nx), ierr)
        if(ierr.ne.0) then
          write(*,*)'vpolprof: neo_pol failed!'
          stop
        end if
	vpol(1:ns,2) = 0.e0
	vpol(1:ns,1) = -vpol(1:ns,3)
      case default
        write(*,*)'vpolprof: imod out of range!'
        ierr = 1
        stop
      end select
c
      if(iprint.eq.1) then
      	open(1, file='vp.d')
      	do jx = 2, Nx-1
      	write(1,*)xarr(jx), (vpol(is,jx),is=1,Ns)
      	end do
      	close(1)
      end if
c
      return
      end
\end{verbatim}

\end{document}
