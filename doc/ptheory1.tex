% 15:00 09-Oct-98 .../baldur/code/bald/ptheory.tex  Bateman, Lehigh
%
%  This is a LaTeX ASCII file.  To typeset document type: latex theory
%  To extract the fortran source code, obtain the utility "xtverb" from
%  Glenn Bateman (bateman@pppl.gov) and type:
%  xtverb < ptheory.tex > ptheory.f
%
\documentstyle {article}    % Specifies the document style.
\headheight 0pt \headsep 0pt  \topmargin 0pt  \oddsidemargin 0pt
\textheight 9.0in \textwidth 6.5in

\title{ {\tt ptheory}: a BALDUR Subroutine \\ 
 Interface between the BALDUR Transport Code \\
 and Subroutine Theory}     % title.
\author{
        Glenn Bateman \\ Princeton Plasma Physics Laboratory}
                           % Declares the author's name.
                           % Deleting the \date{} produces today's date.
\begin{document}           % End of preamble and beginning of text.
\maketitle                 % Produces the title.

This report documents a subroutine called {\tt theory}, which computes plasma 
transport coefficients using various microinstability theory-based models.
The default model used in subroutine {\tt theory} was developed by
C.~E. Singer as documented in
``Theoretical Particle and Energy Flux Formulas for Tokamaks,''
Comments on Plasma Physics and Controlled Fusion, {\bf 11}, 165 (1988)
(\cite{Comments}, hereafter referred to as the Comments paper).

\begin{verbatim}
c@ptheory  .../baldur/code/bald/ptheory.tex
c tho 14-may-99 Implemented Onjun's version  mixed Bohm/gyro-Bohm model
c sep-98 Matteo Erba added the ohe and mixed Bohm/gyro-Bohm models
c rgb 13-oct-98 added sbrtn mmm98b using sda04dif
c   rearranged output
c rgb 09-oct-98 added sbrtn mmm98
c rgb 30-sep-98 implemented limit on time rate of change of ITG
c rgb 29-sep-98 if lthery(24) = 1, set zdesnsi = zdensh + zdensimp
c   and zgrdni = zgrdnh + zgrdnz
c   and zgrdpr computed from zgrdti + zgrdni + zgrdte + zgrdne
c rgb 28-sep-98 zero out diffusivities at magnetic axis
c rgb 19-sep-98 added mmm95 and removed arrays that were not used
c pis 02-jul-98 added zvrotxb for toroidal rotation
c pis 02-jul-98 added interpolation for zvrotxb using wexbint
c pis 12-jun-98 added zwexbxb for flow shear
c rgb 15-dec-96 added diagnostic output sbrtn theory argument list
c rgb 21-jan-96 changed common /cnvect/ from (55,12) to (55,9)
c rgb 11-apr-95 common blocks -> argument list
c   See rest of changes list at end of file
c--------1---------2---------3---------4---------5---------6---------7-c
\end{verbatim}

Subroutine {\tt theory}'s calling argument,
{\tt nkthe}, is used to control the printout
(printout occurs if ${\tt nkthe} = 3$).

\begin{verbatim}
c--------1---------2---------3---------4---------5---------6---------7-c
c**********************************************************************c
      subroutine ptheory(knthe)
c
c
cl      2.21  theoretical particle and energy fluxes
c
c
c  lthery(21) < 1 to call sbrtn theory
c             = 1 for sbrtn mmm95 rather than sbrtn theory
c             = 2 for sbrtn mmm98
c             = 3 for sbrtn mmm98b
c             = 4 for sbrtn mmm98c
c             = 5 for sbrtn mmm98d
c             = 6 for sbrtn ohe model
c             = 7 for sbrtn mixed_merba
c             = 8 for sbrtn mixed_model (Mixed Bohm/gyro-Bohm model)
c
c  lthery(22) = 0 for the default Weiland model
c
c
          include '../com/cbaldr.m'
          include '../com/commhd.m'
          include '../com/cd3he.m'
c
      parameter ( kmatdim = 12 )
c
c  kmatdim = first dimension of matricies
c
      real zgrdne(55), zgrdni(55), zgrdnh(55), zgrdnz(55)
     & , zgrdte(55), zgrdti(55), zgrdpr(55), zgrdq(55)
c
c  Normalized gradients:
c
c  zgrdne(jz) = - R ( d n_e / d r ) / n_e
c  zgrdni(jz) = - R ( d n_i / d r ) / n_i
c     n_i = thermal ion density (sum over hydrogenic and impurity)
c  zgrdnh(jz) = - R ( d n_h / d r ) / n_h
c     n_h = thermal hydrogenic density (sum over hydrogenic species)
c  zgrdnz(jz) = - R ( d Z n_Z / d r ) / ( Z n_Z )
c     n_Z = thermal impurity density,  Z = average impurity charge
c           sumed over all impurities
c  zgrdte(jz) = - R ( d T_e / d r ) / T_e
c  zgrdti(jz) = - R ( d T_i / d r ) / T_i
c  zgrdpr(jz) = - R ( d p   / d r ) / p    for thermal pressure
c  zgrdq (jz) = R ( d q   / d r ) / q    related to magnetic shear
c
      real zsgrdne(55), zsgrdni(55), zsgrdnh(55), zsgrdnz(55)
     & , zsgrdte(55), zsgrdti(55), zsgrdpr(55), zsgrdq(55)
c
c  zsgrd* are smoothed and preprocessed normalized gradients
c
      integer iaxis, iedge, iseprtx, indim, iatdim, iprint
c
      real zrminor(55), zrmajor(55), zelong(55), ztriang(55)
     & , zindent(55), zaimass(55)
     & , zdense(55), zdensi(55), zdensh(55), zdensf(55), zdensfe(55)
     & , zxzeff(55), ztekev(55), ztikev(55), ztfkev(55)
     & , zq(55), zvloop(55), zbtor(55), zresist(55), zwexbxb(55)
     & , zvrotxb(55)
     & , zdensimp(55), zmassimp(55), zavezimp(55), zmasshyd(55)
c
      real  zrhois(55), zrsist(55), ztcrit(55)
     & , zslne(55), zslni(55), zslte(55), zslti(55), zsshr(55)
     & , zrhohs(55), zmlnh(55), zlnhs(55), zslnh(55)
     & , zrhozs(55), zlnzs(55), zmlnz(55), zslnz(55)
     & , zlpr(55), zslpr(55)
c
      real
     &   zthiig(55),   zthdig(55),    ztheig(55),    zthzig(55)
     & , zthirb(55),   zthdrb(55),    ztherb(55),    zthzrb(55)
     & , zthikb(55),   zthdkb(55),    zthekb(55),    zthzkb(55)
c
c
c  zthiig(jz) = ion thermal diffusivity from the Weiland model
c  zthdig(jz) = hydrogenic ion diffusivity from the Weiland model
c  ztheig(jz) = elelctron thermal diffusivity from the Weiland model
c  zthzig(jz) = impurity ion diffusivity from the Weiland model
c	    
c  zthirb(jz) = ion thermal diffusivity from resistive ballooning modes
c  zthdrb(jz) = hydrogenic ion diffusivity from resistive ballooning modes
c  ztherb(jz) = elelctron thermal diffusivity from resistive ballooning modes
c  zthzrb(jz) = impurity ion diffusivity from resistive ballooning modes
c	    
c  zthikb(jz) = ion thermal diffusivity from kinetic ballooning modes
c  zthdkb(jz) = hydrogenic ion diffusivity from kinetic ballooning modes
c  zthekb(jz) = elelctron thermal diffusivity from kinetic ballooning modes
c  zthzkb(jz) = impurity ion diffusivity from kinetic ballooning modes
c
c  Arrays for time averaging when cthery(60) > 0.0
c
      real zoetai(55), zoetae(55), zoetad(55), zoetaz(55)
     &  ,  zdleti(55), zdlete(55), zdletd(55), zdletz(55)
c
      dimension zrmajm(110), znemaj(110), ztemaj(110), ztimaj(110)
     &  , zefmaj(110)
c
c   Arrays for printing out profiles along the major radius
c zrmajm(jz) major radius in meters
c znemaj(jz) electron density in m^-3
c ztemaj(jz) electron temperatur in keV
c ztimaj(jz) ion temperature in keV
c zefmaj(jz) Z_{eff}
c
c
      dimension ztemp1(55), ztemp2(55), ztemp3(55), ztemp4(55)
c
      real  zdhthe(55), zvhthe(55), zdzthe(55), zvzthe(55)
     &  , zxethe(55), zxithe(55), zweithe(55)
     &  , zvithe(55), zvethe(55)
c
c  zdhthe(jz)   = hydrogenic diffusivity ( m^2/sec )
c  zvhthe(jz)   = hydrogenic convective velocity ( m/sec )
c  zdzthe(jz)   = impurity diffusivity ( m^2/sec )
c  zvzthe(jz)   = impurity convective velocity ( m/sec )
c  zxethe(jz)   = electron thermal diffusivity ( m^2/sec )
c  zxithe(jz)   = ion thermal diffusivity ( m^2/sec )
c  zweithe(jz)  = anomalous electron-ion equipartition
c  zvithe(jz)   = ion thermal convective velocity (m/sec)
c  zvethe(jz)   = electron thermal convective velocity (m/sec)
c
c..control variables for Multi-Mode model mmm95
c
      integer lsuper, lreset, lmmm95(8), ier
c
      real    cmmm95(25)
c
      real zgamma(kmatdim,55), zomega(kmatdim,55)
     &  , zvflux(kmatdim,55)
c
c..total diffusities for diagnostic output
c
      real zxetot(55), zxitot(55), zdhtot(55), zdztot(55)
c
c..effective convective velocities
c
        common /cnvect/ vftot(55,9), flxtot(55,9), srctot(55,9)
c
c  vftot(jz,ji)  = total effective convective velocities
c  flxtot(jz,ji) = total fluxes of particles and energies
c  srctot(jz,ji) = total sources of particles and energies
c    (these are computed in sbrtn convrt)
c
c
c..variables in common blocks /comth*/
c
c                 electron/ion thermal diffusivity from:
c  thdre/i(j)   drift waves (trapped electron modes)
c  thtie/i(j)   ion temperature gradient (eta_i) modes
c  thrme/i(j)   rippling modes
c  thrbe/i(j)   resistive ballooning modes
c  thrbgb,thrbb(j)   gyro-Bohm and Bohm parts of thrbe(j)
c  thnme/i(j)   neoclassical MHD modes
c  thkbe/i(j)   kinetic ballooning modes
c  thhfe/i(j)   eta_e mode
c  thrlwe/i(j)  Rebut-Lallia-Watkins model
c
c  thdte(j)  = D_{te}
c  thdi(j)   = D_i
c
c  difthi(j1,j2,jz) = full matrix of anomalous transport diffusivities
c  velthi(j1,jz)    = convective velocities
c
\end{verbatim}

The full matrix form of anomalous transport has the form
$$ \frac{\partial}{\partial t}
 \left( \begin{array}{c} n_H T_H  \\ n_H \\ n_e T_e \\ 
    n_Z \\ n_Z T_Z \\ \vdots
    \end{array} \right)
 = \nabla \cdot
\left( \begin{array}{llll} 
D_{1,1} n_H & D_{1,2} T_H & D_{1,3} n_H T_H / T_e \\
D_{2,1} n_H / T_H & D_{2,2} & D_{2,3} n_H / T_e \\
D_{3,1} n_e T_e / T_H & D_{3,2} n_e T_e / n_H & D_{3,3} n_e & \vdots \\
D_{4,1} n_Z / T_H & D_{4,2} n_Z / n_H & D_{4,3} n_Z / T_e \\
D_{5,1} n_Z T_Z / T_H & D_{5,2} n_Z T_Z / n_H & 
        D_{5,3} n_Z T_Z / T_e \\
 & \ldots & & \ddots
\end{array} \right)
 \nabla
 \left( \begin{array}{c}  T_H \\ n_H \\  T_e \\ 
   n_Z \\  T_Z \\ \vdots
    \end{array} \right)
$$
$$
 + \nabla \cdot
\left( \begin{array}{l} {\bf v}_1 n_H T_H \\ {\bf v}_2 n_H \\
   {\bf v}_3 n_e T_e \\
   {\bf v}_4 n_Z \\ {\bf v}_5 n_Z T_Z \\ \vdots \end{array} \right) +
 \left( \begin{array}{c} S_{T_H} \\ S_{n_H} \\ S_{T_e} \\
    S_{n_Z} \\ S_{T_Z} \\ \vdots
    \end{array} \right) $$
Note that all the diffusivities in this routine are normalized by
$ \omega_{De} / k_y^2 $, 
convective velocities are normalized by $ \omega_{De} / k_y $, 
and all the frequencies are normalized by $ \omega_{De} $.

\begin{verbatim}
c
c    Lengths:
c  thlni(j)  = L_{ni}
c  thlti(j)  = L_{T_i}
c  thlsh(j)  = L_s = R q / s\hat
c  thlpr(j)  = L_p
c  thlarp(j) = \rho_s
c  thrhos(j) = \rho_{\theta i}
c
c    Velocities:
c  thvthe(j) = v_{the}
c  thvthi(j) = v_{thi}
c  thsoun(j) = c_s
c  thalfv(j) = v_A
c
c     Dimensionless:
c  threti(j) = \eta_i
c  thdias(j) = resistive ballooning mode diamagnetic stabilization factor
c  thdinu(j) = \omega_e^\ast / \nu_{eff}
c  thfith(j) = f_{ith} as in eq (35)
c  thbpbc(j) = \beta^{\prime} / \beta_{c1}^{\prime}
c  thdia(j)  = \omega_e^\ast / k_\perp^2
c  thnust(j) = \nu_e^*
c  thlamb(j) = \Lambda multiplier for resistive ballooning modes
c
c  thbeta(j) = \beta
c  thetth(j) = \eta_i^{th}  threshold for \eta_i mode
c  thsrhp(j) = S = \tau_R / \tau_{hp} = r^2 \mu_0 v_A / ( \eta R_0 )
c  thvalh(jz)= Parameter determining validity of Hahm-Tang CTEM model
c
c-----------------------------------------------------------------------
\end{verbatim}

The coding continues with the
OLYMPUS  ({\it cf.} \cite{BALDUR}) number (2.21), 
and use of the OLYMPUS form for bypassing subroutines,
and comments on the common blocks and variables modified.
\begin{verbatim}
c
      data  istep /1/, itdiag /1/
c
      data               iclass /2/   , isub /21/
c
      save istep, itdiag
c
      if ( nlomt2(isub) ) then
        if (nstep .lt. 2)
     & call mesage(' *** 2.21 subroutine theory bypassed            ')
        return
      endif
c
c-----------------------------------------------------------------------
c
c
cl       common blocks and variables modified
c
c  eithes(jz), dxthes(ix,jz), vxthes(ix,jz), xethes(jz), xithes(jz),
c  weithe(jz), weiths(jz)  in /comthe/
c
c-----------------------------------------------------------------------
c
c..go to printout directly if knthe = 3
c
cbate      if ( knthe .eq. 3 ) go to 800
c
c..initialize arrays
c
      do 10 jz=1,mzones
        eithes(jz) = 0.
        xethes(jz) = 0.
        xithes(jz) = 0.
        weithe(jz) = 0.
        weiths(jz) = 0.
        thdre(jz)  = 0.
        thdri(jz)  = 0.
        thige(jz)  = 0.
        thigi(jz)  = 0.
        thtie(jz)  = 0.
        thtii(jz)  = 0.
        thrme(jz)  = 0.
        thrmi(jz)  = 0.
        thrbgb(jz) = 0.
        thrbb(jz)  = 0.
        thrbe(jz)  = 0.
        thnme(jz)  = 0.
        thnmi(jz)  = 0.
        thcee(jz)  = 0.
        thcei(jz)  = 0.
        thhme(jz)  = 0.
        thrbi(jz)  = 0.
        thkbe(jz)  = 0.
        thkbi(jz)  = 0.
        thhfe(jz)  = 0.
        thhfi(jz)  = 0.
        thrlwe(jz) = 0.
        thrlwi(jz) = 0.
        ztemp1(jz) = 0.
        ztemp2(jz) = 0.
  10  continue
c
      do 12 jz=1,mzones
        do 12 ix=1,6
          dxthes(ix,jz) = 0.
          vxthes(ix,jz) = 0.
  12  continue
c
c  converting the physical constants into local variables
c
      zcmu0=emu0
      zceps0=eps0
      zckb=cfev*10.**(fxk)
      zcme=fcme*10.**(fxme)*usim
      zcmp=fcmp*10.**(fxnucl)*usim
      zce=fce*10.**(fxe)*10.
      zcc=1./sqrt(zcmu0*zceps0)
c
      zlgeps=log(epslon)
c
c
\end{verbatim}

There is a problem with using the electron density scale length $L_{ne}$
in the computations below because the electron density is influenced by
fast beam ions which are subject to Monte-Carlo noise.
(Hence, $L_{ne}$ is observed to have random spatial and temporal fluctuations
due to this numerical artifact in the BALDUR and similar codes, 
especially near the magnetic axis).
It is preferable, therefore, to compute the density scale length using
the thermal ion density ({\tt rhoins(2,jz)} in the BALDUR code).
This is computed following the same methods used in BALDUR subroutine
XSCALE.
As an alternative, {\tt zrhohs(jz)} is computed from the sum of the
thermal hydrogen densities in standard units 
and {\tt zlnhs(jz)} is the gradient scale length of {\tt zrhohs(jz)}.

Note $ L_{n_Z} = {tt zlnzs} = Z n_Z / [ d ( Z n_Z ) / d r ] $.

\begin{verbatim}
c
c..compute thermal ion density scale length
c
c  zrhois(jz) = rhoins(2,jz)
c  zrhohs(jz) = hydrogen density
c  zrhozs(jz) = sum of Z_i n_i for impurities
c  rhisms(jz) = smoothed array of rhoins(2,jz)
c  slnis (jz) = 1. / ( d ln (rhoins(2,jz)) / d r )
c
c  temporarily put zrhohs in zmlnh and put zrhozs in zmlnz
c
      do jz=1,mzones
        zrhois(jz) = rhoins(2,jz)
        zmlnh(jz) = 0.0
        zmlnz(jz) = 0.0
        do js=1,mhyd
          zmlnh(jz) = zmlnh(jz) + rhohs(js,2,jz)
        enddo
        do ji=1,mimp
          zmlnz(jz) = zmlnz(jz) + cmean(ji,2,jz) * rhois(ji,2,jz)
        enddo
      enddo
c
c  Smooth profiles, filling arrays rhisms, zrhohs, and zrhozs
c
      call smooth (zrhois,rhisms,mzones,2,1,1,lcentr,dx2i,
     &             smrlow,smlwcy,lsmord)
c
      call smooth (zmlnh,zrhohs,mzones,2,1,1,lcentr,dx2i,
     &             smrlow,smlwcy,lsmord)
c
      call smooth (zmlnz,zrhozs,mzones,2,1,1,lcentr,dx2i,
     &             smrlow,smlwcy,lsmord)
c
      do jz=1,mzones
        zmlnh(jz) = 0.0
        zmlnz(jz) = 0.0
      enddo
c
      zsndni = -1.
      do jz=lcentr+1,mzones
        if ( abs ( rhisms(jz) - rhisms(lcentr) ) .gt.
     &       abs ( epslon * rhisms(lcentr) ) ) then
          zsndni = ( rhisms(jz) - rhisms(lcentr) )
     &             / abs ( rhisms(jz) - rhisms(lcentr) )
          go to 24
        endif
      enddo
  24  continue
c
      zsndnh = -1.
      do jz=lcentr+1,mzones
        if ( abs ( zrhohs(jz) - zrhohs(lcentr) ) .gt.
     &       abs ( epslon * zrhohs(lcentr) ) ) then
          zsndnh = ( zrhohs(jz) - zrhohs(lcentr) )
     &             / abs ( zrhohs(jz) - zrhohs(lcentr) )
          go to 25
        endif
      enddo
  25  continue
c
      zsndnz = -1.
      do jz=lcentr+1,mzones
        if ( abs ( zrhozs(jz) - zrhozs(lcentr) ) .gt.
     &       abs ( epslon * zrhozs(lcentr) ) ) then
          zsndnz = ( zrhozs(jz) - zrhozs(lcentr) )
     &             / abs ( zrhozs(jz) - zrhozs(lcentr) )
          go to 26
        endif
      enddo
  26  continue
c
      do jz=lcentr,mzones
        slnis(jz) = -1. / ( epslon * zsndni
     &    + ( rhisms(jz) - rhisms(jz-1) )
     &    / ( ( armins(jz,2) - armins(jz-1,2) )
     &        * 0.5 * ( rhisms(jz) + rhisms(jz-1) ) ) )
        zlnhs(jz) = -1. / ( epslon * zsndnh
     &    + ( zrhohs(jz) - zrhohs(jz-1) )
     &    / ( ( armins(jz,2) - armins(jz-1,2) )
     &        * 0.5 * ( zrhohs(jz) + zrhohs(jz-1) ) ) )
        zlnzs(jz) = -1. / ( epslon * zsndnz
     &    + ( zrhozs(jz) - zrhozs(jz-1) )
     &    / ( ( armins(jz,2) - armins(jz-1,2) )
     &        * 0.5 * ( zrhozs(jz) + zrhozs(jz-1) ) ) )
      enddo
c
\end{verbatim}

Set up arrays in MKS units (keV for temperatures)
for argument list of sbrtn theory.


\begin{verbatim}
c
        do 32 jz=1,mzones
          zslne(jz) = slnes(jz) * usil
          zslni(jz) = slnis(jz) * usil
          zslnh(jz) = zlnhs(jz) * usil
          zslnz(jz) = zlnzs(jz) * usil
          zslte(jz) = sltes(jz) * usil
          zslti(jz) = sltis(jz) * usil
          zslpr(jz) = (armins(jz,1)/max(epslon,slprs(jz)))*usil
          zsshr(jz) = shear(jz)
  32    continue
c
\end{verbatim}
\begin{verbatim}
c
c
      iaxis = lcentr
      iedge = mzones
      isep = mzones
      if (nadump(1) .gt. lcentr) isep = nadump(1)
c
      do jz=1,mzones
        zelong(jz)  = elong(jz,1)
        ztriang(jz) = triang(jz,1)
        zindent(jz) = dent(jz,1)
        zaimass(jz) = aimass(1,jz)
        zdense(jz)  = rhoels(1,jz)*usid
        zdensi(jz)  = rhoins(1,jz)*usid
        zdensh(jz)  = zrhohs(jz) * usid
        ztekev(jz)  = tes(1,jz)*useh
        ztikev(jz)  = tis(1,jz)*useh
        ztfkev(jz)  = 0.0
        zxzeff(jz)  = xzeff(1,jz)
       zlne = zslne(jz)
       zlni = zslni(jz)
       zlnh = zslnh(jz)
       zlnz = zslnz(jz)
       zlte = zslte(jz)
       zlti = zslti(jz)
ces   temporary numerical overflow protection, cf. statement 22 below
       zlpr(jz)=(armins(jz,1)/max(epslon,slprs(jz)))*usil
      zshear = zsshr(jz)
      zrsep  = max ( armins(isep,1), epslon ) * usil
      zrminor(jz) = max ( armins(jz,1), epslon ) * usil
      zrmajor(jz) = armajs(jz,1)*usil
      zvloop(jz)  = abs(vloopi(jz,2))
      zbtor(jz)   = bzs*usib
      zresist(jz) = eta(1,jz) * usir
c
      zdensf(jz) = ( rhobis(1,jz) + rh1fst(1,jz) + rh2fst(1,jz) )
     &  * usid
c
      zdensfe(jz) = ( rhobes(1,jz) + rh1fst(1,jz) + 2.0 * rh2fst(1,jz) )
     &  * usid
c
        znz    = 0.0
        zmass  = 0.0
        zimpz  = 0.0
c
        if ( mimp .gt. 0 ) then
          do jimp=1,mimp
            znz    = znz   + rhois(jimp,1,jz)
            zimpz  = zimpz + rhois(jimp,1,jz) * cmean(1,1,jz)
            zmass  = zmass + rhois(jimp,1,jz) * aspec(lhydn+1)
          enddo
            zimpz  = zimpz / znz
            zmass  = zmass / znz
            znz    = znz * usid
        endif
c
          zdensimp(jz) = znz
          zmassimp(jz) = max ( zmass, 1.0 )
          zavezimp(jz) = max ( zimpz, 1.0 )
c
        zmasshyd(jz) = ahmean(1,jz)
c
      enddo
c
c..set up gradient scale length arrays
c
      do jz=1,mzones
        zgrdne(jz) = zrmajor(jz)
     &    / sign(max(abs(zslne(jz)),epslon),zslne(jz))
        zgrdni(jz) = zrmajor(jz)
     &    / sign(max(abs(zslni(jz)),epslon),zslni(jz))
        zgrdnh(jz) = zrmajor(jz)
     &    / sign(max(abs(zslnh(jz)),epslon),zslnh(jz))
        zgrdnz(jz) = zrmajor(jz)
     &    / sign(max(abs(zslnz(jz)),epslon),zslnz(jz))
        zgrdte(jz) = zrmajor(jz)
     &    / sign(max(abs(zslte(jz)),epslon),zslte(jz))
        zgrdti(jz) = zrmajor(jz)
     &    / sign(max(abs(zslti(jz)),epslon),zslti(jz))
        zgrdpr(jz) = zrmajor(jz)
     &    / sign(max(abs(zslpr(jz)),epslon),zslpr(jz))
        zgrdq(jz)  = zsshr(jz) * zrmajor(jz) / zrminor(jz)
      enddo
c
      zgrdne(iaxis) = 0.0
      zgrdni(iaxis) = 0.0
      zgrdnh(iaxis) = 0.0
      zgrdnz(iaxis) = 0.0
      zgrdte(iaxis) = 0.0
      zgrdti(iaxis) = 0.0
      zgrdpr(iaxis) = 0.0
      zgrdq(iaxis)  = 0.0
c
c..if lthery(24) = 1, recompute zdensi, zgrdni, and zgrdpr
c
      if ( lthery(24) .eq. 1 ) then
c
        do jz=1,mzones
          zdensi(jz) = zdensh(jz) + zdensimp(jz)
        enddo
c
      endif
c
c     ztime  = tai * uist
c
\end{verbatim}

After setting up the variables in the argument list, call sbrtn theory.

\begin{verbatim}
c
 800  continue
c
      ztime  = tai * uist
      indim   = 6
      imatdim = 12
c
c Interpolate the ExB shearing rate on to the xb grid at time ztime
c
c pis 1-jun-98: Current guess of xb = xbouni, and nxb = mzones
c pis 6-jul-98: Implemented test for vrota = 0
c


      kxdim = 55 ! this has to be passed down from somewhere!!!

      call wexbint (kxdim, nxwexba, xwexba, ntwexba, twexba, wexba
     & , mzones, xbouni, ztime, zwexbxb)

      zvrotmax = 0.0
      DO J = 1, ntwexba
         DO JI = 1,nxwexba
            zvrotmax = max (zvrotmax, abs(vrota(jI,J)))
         END DO
      END DO

      if (zvrotmax .gt. epslon) then
         call wexbint (kxdim, nxwexba, xwexba, ntwexba, twexba, vrota
     &                 , mzones, xbouni, ztime, zvrotxb)
         call wexbprof (mzones,xbouni,flpols,zdensi,ztikev
     &     ,bpoli, zbtor, zvrotxb, zrminor, zrmajor, xbouni, zwexbxb
     &     ,knthe, nprint)
      end if 

      iprint = 0
      if ( knthe .gt. 2 ) iprint = - max ( 1, lthery(29) )
c
c..call the transport model
c
      if ( lthery(21) .lt. 1 ) then
c
        call theory( lthery, cthery
     & , iaxis, iedge, isep, imatdim
     & , zrminor, zrmajor, zelong, ztriang
     & , zdense, zdensi, zdensh, zdensimp, zdensf, zdensfe
     & , zxzeff, ztekev, ztikev, ztfkev, q, zvloop, zbtor, zresist
     & , zavezimp, zmassimp, zmasshyd, zaimass, zwexbxb 
     & , zgrdne, zgrdni, zgrdnh, zgrdnz, zgrdte, zgrdti, zgrdpr, zgrdq
     & , fdr, fig, fti, frm, fkb, frb, fhf, fec, fmh, fdrint
     & , zdhthe, zvhthe, zdzthe, zvzthe, zxethe, zxithe, zweithe
     & , difthi, velthi
     & , nstep, ztime, nprint, iprint)
c
c..Some pages of long printout
c
        if ( knthe .eq. 3  .and.  lthery(29) .gt. 2 ) then
c
          zt  = tai * uist * 1000.0
          zdt = dtoldi * uist * 1000.0
c
c..Total diffusivities if more printout is desired
c%%%%%%%%%%%%%%%%%%%%%%%%%%%%%%%%%%%%%%%%%%%%%%%%%
c
c
c..total theory-based diffusivities
c
      write(nprint,103) label1(1:48),label5(1:72),lpage,nstep,zt,zdt
c
       write(nprint,10007)
       write(nprint,10009)
       write(nprint,10008)
c
      do jz=iaxis,mzones
        write(nprint,101) jz, zrminor(jz), zxethe(jz)
     &     , zxithe(jz), zweithe(jz), zdhthe(jz), zdzthe(jz)
      enddo
c
c.. total power interchanged
c  rgb 15-apr-95 appended factor elong(jz) to volume expression
c
      zptot=0.0
c
      do jz=iaxis,mzones
        zdvol = 2.0 * fcpi * zrmajor(jz) * fcpi
     &     * (zrminor(jz)**2 - zrminor((jz-1))**2) * zelong(jz)
c
        zptot = zptot + weithe(jz)*zdvol
      enddo
c
      write(nprint,102) zptot
c
c..totals, including neoclassical and empirical effective diffusivities
c
      do jz=1,mzones
        zxetot(jz) = zxethe(jz)
     &    + (xeneo1(jz) + xeneo2(jz) + xeemps(jz))*usil**2
        zxitot(jz) = zxithe(jz)
     &    + (xineo1(jz)/rhoins(1,jz) + xiemps(jz))*usil**2
      enddo
c
c..electron thermal diffusivities
c
      lpage=lpage+1
      write(nprint,103) label1(1:48),label5(1:72),lpage,nstep,zt,zdt
c
      write(nprint,104)
      write(nprint,105)
c
      do jz=iaxis,mzones
        write(nprint,106) jz, zrminor(jz), zxethe(jz)
     &  , (xeneo1(jz)+xeneo2(jz))*usil**2, xeemps(jz)*usil**2
     &  , zxetot(jz)
      enddo
c
c
c..ion thermal diffusivities
c
      lpage=lpage+1
      write(nprint,103) label1(1:48),label5(1:72),lpage,nstep,zt,zdt
c
      write(nprint,107)
      write(nprint,108)
c
      do jz=iaxis,mzones
        write(nprint,106) jz, zrminor(jz), zxithe(jz)
     &    , xineo1(jz)*usil**2/rhoins(1,jz), xiemps(jz)*usil**2
     &    , zxitot(jz)
      enddo
c
        endif
c
c..end of printout after sbrtn theory, beginning of other models
c
      else
c
c..preprocess gradients for the Multi-Mode model
c
c  zsgrd* are the normalized gradients for preprocessing
c
        do jz=1,mzones
          zsgrdne(jz) = zgrdne(jz)
          zsgrdni(jz) = zgrdni(jz)
          zsgrdnh(jz) = zgrdnh(jz)
          zsgrdnz(jz) = zgrdnz(jz)
          zsgrdte(jz) = zgrdte(jz)
          zsgrdti(jz) = zgrdti(jz)
          zsgrdpr(jz) = zgrdpr(jz)
          zsgrdq(jz)  = zgrdq(jz)
c--Define a local copy of normalized ExB shearing rate
	  zwexbxb(jz) = cthery(129) * zwexbxb(jz)
        enddo
c
c  use smoothing when lthery(32) .ne. 0
c
        if ( lthery(32) .ne. 0 ) then
c
          i1     = iaxis + 1
          i2     = iedge - 1
          ismord = abs( lthery(32) )
          zlmin  = 1.e-4
          zmix   = 0.0
c
c  ismord = number of times smoothing is applied
c  zlmin  = minimum gradient scale length
c  zmix   = fraction of original array added to smoothed array
c
c  if smoothing is to be done, first divide by the minor radius
c
          zepsqrt = sqrt ( epslon )
c
          do jz=1,mzones
            zsgrdne(jz) = zsgrdne(jz)
     &        / sign ( max( abs( zrminor(jz) ), zepsqrt ), zrminor(jz) )
            zsgrdni(jz) = zsgrdni(jz)
     &        / sign ( max( abs( zrminor(jz) ), zepsqrt ), zrminor(jz) )
            zsgrdnh(jz) = zsgrdnh(jz)
     &        / sign ( max( abs( zrminor(jz) ), zepsqrt ), zrminor(jz) )
            zsgrdnz(jz) = zsgrdnz(jz)
     &        / sign ( max( abs( zrminor(jz) ), zepsqrt ), zrminor(jz) )
            zsgrdte(jz) = zsgrdte(jz)
     &        / sign ( max( abs( zrminor(jz) ), zepsqrt ), zrminor(jz) )
            zsgrdti(jz) = zsgrdti(jz)
     &        / sign ( max( abs( zrminor(jz) ), zepsqrt ), zrminor(jz) )
            zsgrdpr(jz) = zsgrdpr(jz)
     &        / sign ( max( abs( zrminor(jz) ), zepsqrt ), zrminor(jz) )
            zsgrdq(jz)  = zsgrdq(jz)
     &        / sign ( max( abs( zrminor(jz) ), zepsqrt ), zrminor(jz) )
          enddo
c
c  change the values at jz=maxis+1 in order to make sure zml** arrays
c    are monotonic near the magnetic axis
c    Note that this process used to be applied to zml*
c
          if ( lthery(31) .eq. 1 ) then
            zsgrdne(i1) = 2.0 * zsgrdne(i1+1) - zsgrdne(i1+2)
            zsgrdni(i1) = 2.0 * zsgrdni(i1+1) - zsgrdni(i1+2)
            zsgrdnh(i1) = 2.0 * zsgrdnh(i1+1) - zsgrdnh(i1+2)
            zsgrdnz(i1) = 2.0 * zsgrdnz(i1+1) - zsgrdnz(i1+2)
            zsgrdte(i1) = 2.0 * zsgrdte(i1+1) - zsgrdte(i1+2)
            zsgrdti(i1) = 2.0 * zsgrdti(i1+1) - zsgrdti(i1+2)
            zsgrdpr(i1) = 2.0 * zsgrdpr(i1+1) - zsgrdpr(i1+2)
            zsgrdq(i1)  = 2.0 * zsgrdq(i1+1)  - zsgrdq(i1+2)
          endif
c
c  smoothing
c
          call smooth2 ( zsgrdne, 1, ztemp1, ztemp2, 1, i1, i2
     &      , ismord, zmix )
          call smooth2 ( zsgrdni, 1, ztemp1, ztemp2, 1, i1, i2
     &      , ismord, zmix )
          call smooth2 ( zsgrdnh, 1, ztemp1, ztemp2, 1, i1, i2
     &      , ismord, zmix )
          call smooth2 ( zsgrdnz, 1, ztemp1, ztemp2, 1, i1, i2
     &      , ismord, zmix )
          call smooth2 ( zsgrdte, 1, ztemp1, ztemp2, 1, i1, i2
     &      , ismord, zmix )
          call smooth2 ( zsgrdti, 1, ztemp1, ztemp2, 1, i1, i2
     &      , ismord, zmix )
          call smooth2 ( zsgrdpr, 1, ztemp1, ztemp2, 1, i1, i2
     &      , ismord, zmix )
          call smooth2 ( zsgrdq,  1, ztemp1, ztemp2, 1, i1, i2
     &      , ismord, zmix )
c
c  undoing the preconditioning
c
          do jz=1,mzones
            zsgrdne(jz) = zsgrdne(jz) * zrminor(jz)
            zsgrdni(jz) = zsgrdni(jz) * zrminor(jz)
            zsgrdnh(jz) = zsgrdnh(jz) * zrminor(jz)
            zsgrdnz(jz) = zsgrdnz(jz) * zrminor(jz)
            zsgrdte(jz) = zsgrdte(jz) * zrminor(jz)
            zsgrdti(jz) = zsgrdti(jz) * zrminor(jz)
            zsgrdpr(jz) = zsgrdpr(jz) * zrminor(jz)
            zsgrdq(jz)  = zsgrdq(jz) * zrminor(jz)
          enddo
c
c  end of smoothing when lthery(32) .ne. 0
c
        endif
c
c..minimum gradient
c
         if ( cthery(50) .gt. 0.5 ) then
           do jz=1,mzones
             zsgrdne(jz) = sign ( max ( abs ( zsgrdne(jz) ),
     &         1.0 / cthery(50) ), zsgrdne(jz) )
           enddo
         endif
c
         if ( cthery(51) .gt. 0.5 ) then
           do jz=1,mzones
             zsgrdni(jz) = sign ( max ( abs ( zsgrdni(jz) ),
     &         1.0 / cthery(51) ), zsgrdni(jz) )
             zsgrdnh(jz) = sign ( max ( abs ( zsgrdnh(jz) ),
     &         1.0 / cthery(51) ), zsgrdnh(jz) )
             zsgrdnz(jz) = sign ( max ( abs ( zsgrdnz(jz) ),
     &         1.0 / cthery(51) ), zsgrdnz(jz) )
           enddo
         endif
c
         if ( cthery(52) .gt. 0.5 ) then
           do jz=1,mzones
             zsgrdte(jz) = sign ( max ( abs ( zsgrdte(jz) ),
     &         1.0 / cthery(52) ), zsgrdte(jz) )
           enddo
         endif
c
         if ( cthery(53) .gt. 0.5 ) then
           do jz=1,mzones
             zsgrdti(jz) = sign ( max ( abs ( zsgrdti(jz) ),
     &         1.0 / cthery(53) ), zsgrdti(jz) )
           enddo
         endif
c
         if ( cthery(54) .gt. 0.5 ) then
           do jz=1,mzones
             zsgrdpr(jz) = sign ( max ( abs ( zsgrdpr(jz) ),
     &         1.0 / cthery(54) ), zsgrdpr(jz) )
           enddo
         endif
c
c%%%%%%%%%
c--------1---------2---------3---------4---------5---------6---------7-c
c
c..call sbrtn mmm95
c
        if ( lthery(21) .eq. 1 ) then
c
c..set switches
c
          i1 = 3
          ipoints = iedge + 1 - i1
c
          lsuper = lthery(22)
          lreset = 0
c
          call mmm95 (
     &   zrminor(i1),  zrmajor(i1),   zelong(i1)
     & , zdense(i1),   zdensh(i1),    zdensimp(i1),  zdensfe(i1)
     & , zxzeff(i1),   ztekev(i1),    ztikev(i1),    q(i1)
     & , zbtor(i1),    zavezimp(i1),  zmassimp(i1),  zmasshyd(i1)
     & , zaimass(i1),  zwexbxb(i1)
     & , zsgrdne(i1),  zsgrdni(i1),   zsgrdnh(i1),   zsgrdnz(i1)
     & , zsgrdte(i1),  zsgrdti(i1),   zsgrdq(i1)
     & , zthiig(i1),   zthdig(i1),    ztheig(i1),    zthzig(i1)
     & , zthirb(i1),   zthdrb(i1),    ztherb(i1),    zthzrb(i1)
     & , zthikb(i1),   zthdkb(i1),    zthekb(i1),    zthzkb(i1)
     & , zgamma(1,i1), zomega(1,i1)
     & , difthi(1,1,i1),  velthi(1,i1), zvflux(1,i1)
     & , imatdim,  ipoints,   nprint,    iprint,  ierr
     & , lsuper,   lreset,    lmmm95,    cmmm95
     & , fig,      frb,       fkb)
c
c..call sbrtn mmm98
c
        else if ( lthery(21) .eq. 2 ) then
c
c..set switches
c
          i1 = 3
          ipoints = iedge + 1 - i1
c
          lsuper = lthery(22)
          lreset = 0
c
          call mmm98 (
     &   zrminor(i1),  zrmajor(i1),   zelong(i1),    ztriang(i1)
     & , zdense(i1),   zdensh(i1),    zdensimp(i1),  zdensfe(i1)
     & , zxzeff(i1),   ztekev(i1),    ztikev(i1),    q(i1)
     & , zbtor(i1),    zavezimp(i1),  zmassimp(i1),  zmasshyd(i1)
     & , zaimass(i1),  zwexbxb(i1)
     & , zsgrdne(i1),  zsgrdni(i1),   zsgrdnh(i1),   zsgrdnz(i1)
     & , zsgrdte(i1),  zsgrdti(i1),   zsgrdq(i1)
     & , zthiig(i1),   zthdig(i1),    ztheig(i1),    zthzig(i1)
     & , zthirb(i1),   zthdrb(i1),    ztherb(i1),    zthzrb(i1)
     & , zthikb(i1),   zthdkb(i1),    zthekb(i1),    zthzkb(i1)
     & , zgamma(1,i1), zomega(1,i1)
     & , difthi(1,1,i1),  velthi(1,i1), zvflux(1,i1)
     & , imatdim,  ipoints,   nprint,    iprint,  ierr
     & , lsuper,   lreset,    lmmm95,    cmmm95
     & , fig,      frb,       fkb)
c
c..call sbrtn mmm98b
c
        else if ( lthery(21) .eq. 3 ) then
c
c..set switches
c
          i1 = 3
          ipoints = iedge + 1 - i1
c
          lsuper = lthery(22)
          lreset = 0
c
          call mmm98b (
     &   zrminor(i1),  zrmajor(i1),   zelong(i1),    ztriang(i1)
     & , zdense(i1),   zdensh(i1),    zdensimp(i1),  zdensfe(i1)
     & , zxzeff(i1),   ztekev(i1),    ztikev(i1),    q(i1)
     & , zbtor(i1),    zavezimp(i1),  zmassimp(i1),  zmasshyd(i1)
     & , zaimass(i1),  zwexbxb(i1)
     & , zsgrdne(i1),  zsgrdni(i1),   zsgrdnh(i1),   zsgrdnz(i1)
     & , zsgrdte(i1),  zsgrdti(i1),   zsgrdq(i1)
     & , zthiig(i1),   zthdig(i1),    ztheig(i1),    zthzig(i1)
     & , zthirb(i1),   zthdrb(i1),    ztherb(i1),    zthzrb(i1)
     & , zthikb(i1),   zthdkb(i1),    zthekb(i1),    zthzkb(i1)
     & , zgamma(1,i1), zomega(1,i1)
     & , difthi(1,1,i1),  velthi(1,i1), zvflux(1,i1)
     & , imatdim,  ipoints,   nprint,    iprint,  ierr
     & , lsuper,   lreset,    lmmm95,    cmmm95
     & , fig,      frb,       fkb)
c
c..call sbrtn mmm98c
c
        else if ( lthery(21) .eq. 4 ) then
c
c..set switches
c
          i1 = 3
          ipoints = iedge + 1 - i1
c
          lsuper = lthery(22)
          lreset = 0
c
          call mmm98c (
     &   zrminor(i1),  zrmajor(i1),   zelong(i1),    ztriang(i1)
     & , zdense(i1),   zdensh(i1),    zdensimp(i1),  zdensfe(i1)
     & , zxzeff(i1),   ztekev(i1),    ztikev(i1),    q(i1)
     & , zbtor(i1),    zavezimp(i1),  zmassimp(i1),  zmasshyd(i1)
     & , zaimass(i1),  zwexbxb(i1)
     & , zsgrdne(i1),  zsgrdni(i1),   zsgrdnh(i1),   zsgrdnz(i1)
     & , zsgrdte(i1),  zsgrdti(i1),   zsgrdq(i1)
     & , zthiig(i1),   zthdig(i1),    ztheig(i1),    zthzig(i1)
     & , zthirb(i1),   zthdrb(i1),    ztherb(i1),    zthzrb(i1)
     & , zthikb(i1),   zthdkb(i1),    zthekb(i1),    zthzkb(i1)
     & , zgamma(1,i1), zomega(1,i1)
     & , difthi(1,1,i1),  velthi(1,i1), zvflux(1,i1)
     & , imatdim,  ipoints,   nprint,    iprint,  ierr
     & , lsuper,   lreset,    lmmm95,    cmmm95
     & , fig,      frb,       fkb)
c
c..call sbrtn mmm98d
c
        else if ( lthery(21) .eq. 5 ) then
c
c..set switches
c
          i1 = 3
          ipoints = iedge + 1 - i1
c
          lsuper = lthery(22)
          lreset = 0
c
          call mmm98d (
     &   zrminor(i1),  zrmajor(i1),   zelong(i1),    ztriang(i1)
     & , zdense(i1),   zdensh(i1),    zdensimp(i1),  zdensfe(i1)
     & , zxzeff(i1),   ztekev(i1),    ztikev(i1),    q(i1)
     & , zbtor(i1),    zavezimp(i1),  zmassimp(i1),  zmasshyd(i1)
     & , zaimass(i1),  zwexbxb(i1)
     & , zsgrdne(i1),  zsgrdni(i1),   zsgrdnh(i1),   zsgrdnz(i1)
     & , zsgrdte(i1),  zsgrdti(i1),   zsgrdq(i1)
     & , zthiig(i1),   zthdig(i1),    ztheig(i1),    zthzig(i1)
     & , zthirb(i1),   zthdrb(i1),    ztherb(i1),    zthzrb(i1)
     & , zthikb(i1),   zthdkb(i1),    zthekb(i1),    zthzkb(i1)
     & , zgamma(1,i1), zomega(1,i1)
     & , difthi(1,1,i1),  velthi(1,i1), zvflux(1,i1)
     & , imatdim,  ipoints,   nprint,    iprint,  ierr
     & , lsuper,   lreset,    lmmm95,    cmmm95
     & , fig,      frb,       fkb)
c
c
c..call sbrtn ohe
c
        else if ( lthery(21) .eq. 6 ) then
c
c..set switches
c
          i1 = 3
          ipoints = iedge + 1 - i1
c
          lsuper = lthery(22)
          lreset = 0
c
      call ohe_model (
     &   zrminor(i1),  zrmajor(i1),   zelong(i1)    
     & , ztekev(i1),    ztikev(i1),    q(i1)
     & , zbtor(i1), zmasshyd(i1)
     & , zsgrdnz(i1),  zsgrdte(i1),   zsgrdq(i1),   zsgrdnh(i1)    
     & , zwexbxb(i1)
     & , zsgrdne(i1),  zsgrdni(i1)
     & , zdense(i1),   zdensh(i1),    zdensimp(i1)
     & , zsgrdti(i1)
     & , zthiig(i1),   zthdig(i1),    ztheig(i1),    zthzig(i1)
     & , zthirb(i1),   zthdrb(i1),    ztherb(i1),    zthzrb(i1)
     & , zthikb(i1),   zthdkb(i1),    zthekb(i1),    zthzkb(i1)
     & , difthi(1,1,i1),  velthi(1,i1), zvflux(1,i1)
     & , imatdim,  ipoints,   nprint,    iprint,  ierr
     & , lsuper,   lreset,    lmmm95,    cmmm95 
     & , fig,      frb,       fkb)
c
c
c..call Matteo Erba's version of the Mixed Bohm/gyro-Bohm model
c
        else if ( lthery(21) .eq. 7 ) then
c
c..set switches
c
          i1 = 3
          ipoints = iedge + 1 - i1
c
          lsuper = lthery(22)
          lreset = 0
c
      call mixed_merba (
     &   zrminor(i1),  zrmajor(i1),   zelong(i1)    
     & , ztekev(i1),    ztikev(i1),    q(i1)
     & , zbtor(i1), zmasshyd(i1)
     & , zsgrdnz(i1),  zsgrdte(i1),   zsgrdq(i1),   zsgrdnh(i1)    
     & , zwexbxb(i1)
     & , zsgrdne(i1),  zsgrdni(i1)
     & , zdense(i1),   zdensh(i1),    zdensimp(i1)
     & , zsgrdti(i1)
     & , zthiig(i1),   zthdig(i1),    ztheig(i1),    zthzig(i1)
     & , zthirb(i1),   zthdrb(i1),    ztherb(i1),    zthzrb(i1)
     & , zthikb(i1),   zthdkb(i1),    zthekb(i1),    zthzkb(i1)
     & , difthi(1,1,i1),  velthi(1,i1), zvflux(1,i1)
     & , imatdim,  ipoints,   nprint,    iprint,  ierr
     & , lsuper,   lreset,    lmmm95,    cmmm95 
     & , fig,      frb,       fkb)
c
c
c..call Onjun's version of the Mixed Bohm/gyro-Bohm model
c
        else if ( lthery(21) .eq. 8 ) then
c
c..set switches
c
          i1 = 3
          ipoints = iedge + 1 - i1
c
          lsuper = lthery(22)
          lreset = 0
c
      call mixed_model (
     &   zrminor(i1),  zrmajor(i1),   zelong(i1)    
     & , ztekev(i1),    ztikev(i1),    q(i1)
     & , zbtor(i1), zmasshyd(i1)
     & , zsgrdnz(i1),  zsgrdte(i1),   zsgrdq(i1),   zsgrdnh(i1)    
     & , zwexbxb(i1)
     & , zsgrdne(i1),  zsgrdni(i1)
     & , zdense(i1),   zdensh(i1),    zdensimp(i1)
     & , zsgrdti(i1)
     & , zthiig(i1),   zthdig(i1),    ztheig(i1),    zthzig(i1)
     & , zthirb(i1),   zthdrb(i1),    ztherb(i1),    zthzrb(i1)
     & , zthikb(i1),   zthdkb(i1),    zthekb(i1),    zthzkb(i1)
     & , difthi(1,1,i1),  velthi(1,i1), zvflux(1,i1)
     & , imatdim,  ipoints,   nprint,    iprint,  ierr
     & , lsuper,   lreset,    lmmm95,    cmmm95 
     & , fig,      frb,       fkb)
c
        endif
c
c..zero out diffusivities up to magnetic axis
c
          do jz=1,2
c
            zthiig(jz) = 0.0
            zthdig(jz) = 0.0
            ztheig(jz) = 0.0
            zthzig(jz) = 0.0
            zthirb(jz) = 0.0
            zthdrb(jz) = 0.0
            ztherb(jz) = 0.0
            zthzrb(jz) = 0.0
            zthikb(jz) = 0.0
            zthdkb(jz) = 0.0
            zthekb(jz) = 0.0
            zthzkb(jz) = 0.0
c
            do j1=1,4
              velthi(j1,jz) = 0.0
              do j2=1,4
                difthi(j1,j2,jz) = 0.0
              enddo
            enddo
c
          enddo
c
c..set local diffusivities as needed
c
        if ( lreset .eq. 0 ) then
c
          do jz=iaxis,iedge
            zxithe(jz) = difthi(1,1,jz)
            zvithe(jz) = velthi(1,jz)
            zdhthe(jz) = difthi(2,2,jz)
            zvhthe(jz) = velthi(2,jz)
            zxethe(jz) = difthi(3,3,jz)
            zvethe(jz) = velthi(3,jz)
            zdzthe(jz) = difthi(4,4,jz)
            zvzthe(jz) = velthi(4,jz)
            weithe(jz) = 0.0
c
            velthi(1,jz) = 0.0
            velthi(3,jz) = 0.0
            difthi(1,1,jz) = 0.0
            difthi(2,2,jz) = 0.0
            difthi(3,3,jz) = 0.0
            difthi(4,4,jz) = 0.0
          enddo
\end{verbatim}

In order to limit the time rate of change of the theory-based 
diffusivities, the old values are kept in local arrays.
The difference between the old and the new values is computed
\[ \Delta \chi_j = \chi^{N}_j - \chi^{N-1}_j \]
and stored in local arrays.
This difference is then spatially averaged
\[  \bar{ \Delta \chi_j }
    = ( \Delta \chi_{j-1} + 2 \Delta \chi_j +  \Delta \chi_{j+1} ) / 4 \]
Finally, the adjusted diffusivity is computed
\[ \chi^N_j = \chi^{N-1}_j
 + \Delta \chi_j /
 ( 1. + c_{60} |  \Delta \chi_j - \bar{ \Delta \chi_j} | ). \]
Here $ c_{60} = {\tt cthery(60)} $ with default value 0.0.
A recommended value is $ {\tt cthery(60)} = 10.0 $
if the diffusivities show the pattern of a numerical instability.
This adjustment suppresses large local changes in the diffusivities.
Here, this algorithm is applied only to the ITG ($\eta_i$) mode.

\begin{verbatim}
      if ( cthery(60) .gt. epslon ) then
c
c..First, subtract the ITG diffusivites from the total diffusivities
c
        do jz=1,iedge
          zxithe(jz) = zxithe(jz) - zthiig(jz)
          zdhthe(jz) = zdhthe(jz) - zthdig(jz) 
          zxethe(jz) = zxethe(jz) - ztheig(jz)
          zdzthe(jz) = zdzthe(jz) - zthzig(jz)
        enddo
c
c..Store the ITG diffusivities in zoeta* before first step
c
        if ( nstep .lt. 1 ) then
          do jz=1,iedge
            zoetai(jz) = zthiig(jz)
            zoetae(jz) = ztheig(jz)
            zoetad(jz) = zthdig(jz)
            zoetaz(jz) = zthzig(jz)
          enddo
        endif
c
c..Compute difference between diffusivities now and before
c
        do jz=1,iedge
          zdleti(jz) = zthiig(jz) - zoetai(jz)
          zdlete(jz) = ztheig(jz) - zoetae(jz)
          zdletd(jz) = zthdig(jz) - zoetad(jz)
          zdletz(jz) = zthzig(jz) - zoetaz(jz)
        enddo
c
c..Spatially average diffusivities
c
        do jz=iaxis+1,iedge-1
          ztemp1(jz)=0.25*(zdleti(jz-1)+2.0*zdleti(jz)+zdleti(jz+1))
          ztemp2(jz)=0.25*(zdlete(jz-1)+2.0*zdlete(jz)+zdlete(jz+1))
          ztemp3(jz)=0.25*(zdletd(jz-1)+2.0*zdletd(jz)+zdletd(jz+1))
          ztemp4(jz)=0.25*(zdletz(jz-1)+2.0*zdletz(jz)+zdletz(jz+1))
        enddo
c
c..compute adjusted diffusivities
c
        do jz=iaxis+1,iedge-1
          zthiig(jz) = zoetai(jz) + zdleti(jz)
     &     / ( 1. + cthery(60) * abs ( zdleti(jz) - ztemp1(jz) ) )
          ztheig(jz) = zoetae(jz) + zdlete(jz)
     &     / ( 1. + cthery(60) * abs ( zdlete(jz) - ztemp2(jz) ) )
          zthdig(jz) = zoetad(jz) + zdletd(jz)
     &     / ( 1. + cthery(60) * abs ( zdletd(jz) - ztemp3(jz) ) )
          zthzig(jz) = zoetaz(jz) + zdletz(jz)
     &     / ( 1. + cthery(60) * abs ( zdletz(jz) - ztemp4(jz) ) )
        enddo
c
c..store the new diffusivites if this step has not been repeated
c
        if (  nstep .gt. istep ) then
          istep = nstep
          do jz=1,iedge
            zoetai(jz) = zthiig(jz)
            zoetae(jz) = ztheig(jz)
            zoetad(jz) = zthdig(jz)
            zoetaz(jz) = zthzig(jz)
          enddo
        endif
c
c..add the ITG diffusivities back into the total diffusivities
c
        do jz=1,iedge
          zxithe(jz) = zxithe(jz) + zthiig(jz)
          zdhthe(jz) = zdhthe(jz) + zthdig(jz) 
          zxethe(jz) = zxethe(jz) + ztheig(jz)
          zdzthe(jz) = zdzthe(jz) + zthzig(jz)
        enddo
c
      endif
c
        endif
c
c..additional output from the Multi-Mode model
c
c..Some pages of long printout
c
        if ( knthe .eq. 3  .and.  lthery(29) .gt. 2 ) then
c
          zt  = tai * uist * 1000.0
          zdt = dtoldi * uist * 1000.0
c
c..Total diffusivities if more printout is desired
c%%%%%%%%%%%%%%%%%%%%%%%%%%%%%%%%%%%%%%%%%%%%%%%%%
c
c
c..total theory-based diffusivities
c
        write(nprint,103) label1(1:48),label5(1:72),lpage,nstep,zt,zdt
c
         write(nprint,10007)
         write(nprint,10009)
         write(nprint,10008)
c
        do jz=iaxis,mzones
          write(nprint,101) jz, zrminor(jz), zxethe(jz)
     &       , zxithe(jz), zweithe(jz), zdhthe(jz), zdzthe(jz)
        enddo
c
c..totals, including neoclassical and empirical effective diffusivities
c
        do jz=1,mzones
          zxetot(jz) = zxethe(jz)
     &      + (xeneo1(jz) + xeneo2(jz) + xeemps(jz))*usil**2
          zxitot(jz) = zxithe(jz)
     &      + (xineo1(jz)/rhoins(1,jz) + xiemps(jz))*usil**2
        enddo
c
c..ion thermal diffusivities
c
        lpage=lpage+1
        write(nprint,103) label1(1:48),label5(1:72),lpage,nstep,zt,zdt
c
        write(nprint,107)
        write(nprint,122)
c
        do jz=iaxis,mzones
          write(nprint,120)  zrminor(jz)
     &      , zthiig(jz), zthirb(jz), zthikb(jz), zxithe(jz)
     &      , xineo1(jz)*usil**2/rhoins(1,jz), xiemps(jz)*usil**2
     &      , zxitot(jz)
        enddo
c
c..electron thermal diffusivities
c
        lpage=lpage+1
        write(nprint,103) label1(1:48),label5(1:72),lpage,nstep,zt,zdt
c
        write(nprint,104)
        write(nprint,124)
c
        do jz=iaxis,mzones
          write(nprint,120)  zrminor(jz)
     &      , ztheig(jz), ztherb(jz), zthekb(jz), zxethe(jz)
     &      , (xeneo1(jz)+xeneo2(jz))*usil**2, xeemps(jz)*usil**2
     &      , zxetot(jz)
        enddo
c
c
c..hydrogenic particle diffusivities
c
        lpage=lpage+1
        write(nprint,103) label1(1:48),label5(1:72),lpage,nstep,zt,zdt
c
        write(nprint,125)
        write(nprint,126)
c
        do jz=iaxis,mzones
          write(nprint,120)  zrminor(jz)
     &      , zthdig(jz), zthdrb(jz), zthdkb(jz), zdhthe(jz)
        enddo
c
c..impurity particle diffusivities
c
        lpage=lpage+1
        write(nprint,103) label1(1:48),label5(1:72),lpage,nstep,zt,zdt
c
        write(nprint,127)
        write(nprint,128)
c
        do jz=iaxis,mzones
          write(nprint,120)  zrminor(jz)
     &     , zthzig(jz), zthzrb(jz), zthzkb(jz), zdzthe(jz)
        enddo
c
        lpage=lpage+1
        write(nprint,103) label1(1:48),label5(1:72),lpage,nstep,zt,zdt
c
        write (nprint,*)
        write (nprint,*) 'Normalized gradients:'
c
        write (nprint,138)
 138    format ('zrminor',t15,'grdne',t27,'grdni'
     &    ,t39,'grdnh',t51,'grdnz',t63,'grdte',t75,'grdti'
     &    ,t87,'grdpr',t99,'grdq')
        do j=1,iedge
          write (nprint,152) zrminor(j), zgrdne(j), zgrdni(j)
     &      , zgrdnh(j), zgrdnz(j), zgrdte(j), zgrdti(j)
     &      , zgrdpr(j), zgrdq(j)
        enddo
c
        lpage=lpage+1
        write(nprint,103) label1(1:48),label5(1:72),lpage,nstep,zt,zdt
c
        write (nprint,*)
        write (nprint,*) 'Smoothed normalized gradients:'
c
        write (nprint,139)
 139    format ('zrminor',t15,'sgrdne',t27,'sgrdni'
     &    ,t39,'sgrdnh',t51,'sgrdnz',t63,'sgrdte',t75,'sgrdti'
     &    ,t87,'sgrdpr',t99,'sgrdq')
        do j=1,iedge
          write (nprint,152) zrminor(j), zsgrdne(j), zsgrdni(j)
     &      , zsgrdnh(j), zsgrdnz(j), zsgrdte(j), zsgrdti(j)
     &      , zsgrdpr(j), zsgrdq(j)
        enddo
c
        endif
c
      endif
c
c..diagnostic output as needed
c  when nstep = lthery(26)
c  or when tai*uiet .ge. tplot(j) .gt. epslon  and  lthery(26) .gt. 0
c
      idiag = 0
      if ( nstep .eq. lthery(26) ) idiag = 1
      if ( nstep .lt. -lthery(26) ) idiag = 1
      if ( itdiag .lt. 20  .and.  lthery(26) .gt. 0 ) then
        do j=itdiag,20
          if ( tplot(j) .gt. epslon
     &         .and. tai*uiet .gt. tplot(j) ) then
            idiag = 1
            itdiag = j + 1
          endif
        enddo
      endif
c
      if ( idiag .gt. 0 ) then
c
        write (nprint,150)
 150    format(/'Diagnostic output from sbrtn ptheory')
c
        write (nprint,151) nstep, ztime
 151    format ('# nstep = ',i5,' time = ',0pf14.6)
 152    format (1p10e12.4)
c
        write (nprint,154) (lthery(j),j=1,50)
 154    format ('# lthery'/,(10i5))
c
        write (nprint,155) (cthery(j),j=1,150)
 155    format ('# cthery(j)'/,(1p10e12.4))
c
        write (nprint,156) iaxis, iedge, isep, imatdim
     &    , nprint, iprint
 156    format ('# maxis = ',i5,'  medge = ',i5,'  mseprtx = ',i5
     &    ,'  matdim = ',i5,'  nprint = ',i5,'  lprint = ',i5)
c
        write (nprint,157)
 157    format ('#  xbouni',t15,'rminor',t27,'rmajor',t39,'elong'
     &    ,t51,'triang')
        do j=1,iedge
          write (nprint,152) xbouni(j), zrminor(j), zrmajor(j)
     &      , zelong(j), ztriang(j)
        enddo
c
        write (nprint,158)
 158    format ('#  xbouni',t15,'dense',t27,'densi'
     &    ,t39,'densh',t51,'densimp',t63,'densf',t75,'densfe')
        do j=1,iedge
          write (nprint,152) xbouni(j), zdense(j), zdensi(j)
     &      , zdensh(j), zdensimp(j), zdensf(j), zdensfe(j)
        enddo
c
        write (nprint,159)
 159    format ('#  xbouni',t15,'xzeff',t27,'tekev'
     &    ,t39,'tikev',t51,'tfkev',t63,'q',t75,'vloop'
     &    ,t87,'btor',t99,'resist')
        do j=1,iedge
          write (nprint,152) xbouni(j), zxzeff(j), ztekev(j)
     &      , ztikev(j), ztfkev(j), q(j), zvloop(j)
     &      , zbtor(j), zresist(j)
        enddo
c
        write (nprint,160)
 160    format ('#  xbouni',t15,'avezimp',t27,'amassimp'
     &    ,t39,'amasshyd',t51,'aimass')
        do j=1,iedge
          write (nprint,152) xbouni(j), zavezimp(j), zmassimp(j)
     &      , zmasshyd(j), zaimass(j)
        enddo
c
        write (nprint,161)
 161    format ('#  xbouni',t15,'grdne',t27,'grdni'
     &    ,t39,'grdnh',t51,'grdnz',t63,'grdte',t75,'grdti'
     &    ,t87,'grdpr',t99,'grdq')
        do j=1,iedge
          write (nprint,152) xbouni(j), zgrdne(j), zgrdni(j)
     &      , zgrdnh(j), zgrdnz(j), zgrdte(j), zgrdti(j)
     &      , zgrdpr(j), zgrdq(j)
        enddo
c
        write (nprint,181)
 181    format ('#  xbouni',t15,'sgrdne',t27,'sgrdni'
     &    ,t39,'sgrdnh',t51,'sgrdnz',t63,'sgrdte',t75,'sgrdti'
     &    ,t87,'sgrdpr',t99,'sgrdq')
        do j=1,iedge
          write (nprint,152) xbouni(j), zsgrdne(j), zsgrdni(j)
     &      , zsgrdnh(j), zsgrdnz(j), zsgrdte(j), zsgrdti(j)
     &      , zsgrdpr(j), zsgrdq(j)
        enddo
c
        write (nprint,162)
 162    format ('#  fdr',t15,'fig',t27,'fti'
     &    ,t39,'frm',t51,'fkb',t63,'frb',t75,'fhf'
     &    ,t87,'fec',t99,'fmh')
        do j=1,5
          write (nprint,152) fdr(j), fig(j), fti(j)
     &      , frm(j), fkb(j), frb(j), fhf(j)
     &      , fec(j), fmh(j)
        enddo
c
        write (nprint,*) '# fdrint = ',fdrint
c
        write (nprint,163)
 163    format ('#  xbouni',t15,'dhtot',t27,'vhtot'
     &    ,t39,'dztot',t51,'vztot',t63,'xetot',t75,'xitot'
     &    ,t87,'wiethe')
        do j=1,iedge
          write (nprint,152) xbouni(j), zdhthe(j), zvhthe(j)
     &      , zdzthe(j), zvzthe(j), zxethe(j), zxithe(j)
     &      , zweithe(j)
        enddo
c
      write (nprint,*)
      write (nprint,*) ' Diffusion matrix:'
      write (nprint,171)

      do jr=iaxis,mzones
        write (nprint,110) zrminor(jr)
     &    , velthi(1,jr),   difthi(1,1,jr), difthi(1,2,jr)
     &    , difthi(1,3,jr), difthi(1,4,jr), difthi(1,5,jr)
      enddo
c
      write (nprint,*)
      write (nprint,172)

      do jr=iaxis,mzones
        write (nprint,110) zrminor(jr)
     &    , velthi(2,jr),   difthi(2,1,jr), difthi(2,2,jr)
     &    , difthi(2,3,jr), difthi(2,4,jr), difthi(2,5,jr)
      enddo
c
      write (nprint,*)
      write (nprint,173)

      do jr=iaxis,mzones
        write (nprint,110) zrminor(jr)
     &    , velthi(3,jr),   difthi(3,1,jr), difthi(3,2,jr)
     &    , difthi(3,3,jr), difthi(3,4,jr), difthi(3,5,jr)
      enddo
c
      write (nprint,*)
      write (nprint,174)

      do jr=iaxis,mzones
        write (nprint,110) zrminor(jr)
     &    , velthi(4,jr),   difthi(4,1,jr), difthi(4,2,jr)
     &    , difthi(4,3,jr), difthi(4,4,jr), difthi(4,5,jr)
      enddo
c
      endif
c
\end{verbatim}

Convert to cgs units, as needed.

\begin{verbatim}
c
      do jz=iaxis,iedge
c
        xethes(jz) = zxethe(jz)
        xithes(jz) = zxithe(jz)
c
        do ji=lhyd1,lhydn
          dxthes(ji,jz) = zdhthe(jz)
cbate          vxthes(ji,jz) = zvhthe(jz)
        enddo
c
        do ji=limp1,limpn
          dxthes(ji,jz) = zdzthe(jz)
cbate          vxthes(ji,jz) = zvzthe(jz)
        enddo
c
c  put 'weithe' in standard units to be used in sub. convrt (dsolver)
c
        weiths(jz) = weithe(jz) * uesh * uisd
c
c  rgb 15-apr-95 appended factor elong(jz) to volume expression
c
        zvolum=2*fcpi*zrmajor(jz)*fcpi*(zrminor(jz)**2-
     #       zrminor((jz-1))**2)*zelong(jz)
        eithes(jz) = weithe(jz)*zvolum
c
      enddo
c
\end{verbatim}

\subsection{Printout}

\begin{verbatim}
c-----------------------------------------------------------------------
c
c  print theory's output
c
      if ( knthe .ne. 3 ) go to 990
c
cbate      entry prethprnt
c
 900  continue
c
      zt  = tai * uist * 1000.0
      zdt = dtoldi * uist * 1000.0
c
c..Profiles as a function of major radius
c%%%%%%%%%%%%%%%%%%%%%%%%%%%%%%%%%%%%%%%%
c
      icntr = lcentr
      iedge = ledge + 1
c
      ir = 0
      do 802 jz=iedge,icntr,-1
        ir = ir + 1
        zrmajm(ir) = ( rmids(jz,2) - ahalfs(jz,2) ) * usil
        znemaj(ir) = rhoels(2,jz) * usid
        ztemaj(ir) = tes(2,jz) * useh
        ztimaj(ir) = tis(2,jz) * useh
        zefmaj(ir) = xzeff(2,jz)
 802  continue
c
      do 804 jz=icntr,iedge
        ir = ir + 1
        zrmajm(ir) = ( rmids(jz,2) + ahalfs(jz,2) ) * usil
        znemaj(ir) = rhoels(2,jz) * usid
        ztemaj(ir) = tes(2,jz) * useh
        ztimaj(ir) = tis(2,jz) * useh
        zefmaj(ir) = xzeff(2,jz)
 804  continue
c
      irmax = ir
c 
      write(nprint,103) label1(1:48),label5(1:72),lpage,nstep,zt,zdt
c
      write (nprint,130)
c
      do 806 jz=1,irmax
        write (nprint,132) zrmajm(jz),znemaj(jz),ztemaj(jz),ztimaj(jz)
     &    ,zefmaj(jz)
 806  continue

c
c..Densities as a function of minor radius
c%%%%%%%%%%%%%%%%%%%%%%%%%%%%%%%%%%%%%%%%%
c
      if ( lthery(29) .gt. 2 ) then
c
      lpage=lpage+1
      write(nprint,103) label1(1:48),label5(1:72),lpage,nstep,zt,zdt
c
      write ( nprint, 133)
c
       do jz=iaxis,mzones
        write (nprint,106) jz, zrminor(jz)
     &    , zdense(jz), zdensi(jz)
     &    , zdensh(jz), zdensimp(jz)
     &    , zdensf(jz), zdensfe(jz), zxzeff(jz), zaimass(jz)
      enddo
c
      endif
c
c..print diffusivity matrix
c%%%%%%%%%%%%%%
c
        lpage=lpage+1
        write(nprint,103) label1(1:48),label5(1:72),lpage,nstep,zt,zdt
c
      write (nprint,*)
      write (nprint,*) ' Convective velocities (not used):'
      write (nprint,170)
 170  format (/t2,'rminor'
     &  ,t10,'zvithe',t21,'zvhthe',t33,'zvethe'
     &  ,t45,'zvzthe')

      do jr=iaxis,mzones
        write (nprint,110) zrminor(jr)
     &    , zvithe(jz), zvhthe(jz), zvethe(jz), zvzthe(jz)
      enddo
c
        lpage=lpage+1
        write(nprint,103) label1(1:48),label5(1:72),lpage,nstep,zt,zdt
c
      write (nprint,*)
      write (nprint,*) ' Diffusion matrix:'
      write (nprint,171)
 171  format (/t2,'rminor'
     &  ,t10,'vel(1,jr)',t21,'dif(1,1,jr)',t33,'dif(1,2,jr)'
     &  ,t45,'dif(1,3,jr)',t57,'dif(1,4,jr)',t69,'dif(1,5,jr)',
     &   /t4,'[m]',t12,'[m/s]',t23,'[m^2/s]',t35,'[m^2/s]',
     &   t47,'[m^2/s]',t59,'[m^2/s]',t71,'[m^2/s]')

      do jr=iaxis,mzones
        write (nprint,110) zrminor(jr)
     &    , velthi(1,jr),   difthi(1,1,jr), difthi(1,2,jr)
     &    , difthi(1,3,jr), difthi(1,4,jr), difthi(1,5,jr)
      enddo
c
      write (nprint,*)
      write (nprint,172)
 172  format (/t2,'rminor'
     &  ,t10,'vel(2,jr)',t21,'dif(2,1,jr)',t33,'dif(2,2,jr)'
     &  ,t45,'dif(2,3,jr)',t57,'dif(2,4,jr)',t69,'dif(2,5,jr)',
     &   /t4,'[m]',t12,'[m/s]',t23,'[m^2/s]',t35,'[m^2/s]',
     &   t47,'[m^2/s]',t59,'[m^2/s]',t71,'[m^2/s]')

      do jr=iaxis,mzones
        write (nprint,110) zrminor(jr)
     &    , velthi(2,jr),   difthi(2,1,jr), difthi(2,2,jr)
     &    , difthi(2,3,jr), difthi(2,4,jr), difthi(2,5,jr)
      enddo
c
      write (nprint,*)
      write (nprint,173)
 173  format (/t2,'rminor'
     &  ,t10,'vel(3,jr)',t21,'dif(3,1,jr)',t33,'dif(3,2,jr)'
     &  ,t45,'dif(3,3,jr)',t57,'dif(3,4,jr)',t69,'dif(3,5,jr)',
     &   /t4,'[m]',t12,'[m/s]',t23,'[m^2/s]',t35,'[m^2/s]',
     &   t47,'[m^2/s]',t59,'[m^2/s]',t71,'[m^2/s]')

      do jr=iaxis,mzones
        write (nprint,110) zrminor(jr)
     &    , velthi(3,jr),   difthi(3,1,jr), difthi(3,2,jr)
     &    , difthi(3,3,jr), difthi(3,4,jr), difthi(3,5,jr)
      enddo
c
      write (nprint,*)
      write (nprint,174)
 174  format (/t2,'rminor'
     &  ,t10,'vel(4,jr)',t21,'dif(4,1,jr)',t33,'dif(4,2,jr)'
     &  ,t45,'dif(4,3,jr)',t57,'dif(4,4,jr)',t69,'dif(4,5,jr)',
     &   /t4,'[m]',t12,'[m/s]',t23,'[m^2/s]',t35,'[m^2/s]',
     &   t47,'[m^2/s]',t59,'[m^2/s]',t71,'[m^2/s]')

      do jr=iaxis,mzones
        write (nprint,110) zrminor(jr)
     &    , velthi(4,jr),   difthi(4,1,jr), difthi(4,2,jr)
     &    , difthi(4,3,jr), difthi(4,4,jr), difthi(4,5,jr)
      enddo
c
c..print effective convective velocities
c%%%%%%%%%%%%%%%%%%%%%%%%%%%%%%%%%%%%%%%
c
        lpage=lpage+1
        write(nprint,103) label1(1:48),label5(1:72),lpage,nstep,zt,zdt
c
      write ( nprint, 140)
c
      do jz=iaxis,mzones
c
        write (nprint,106) jz, zrminor(jz)
     &    , vftot(jz,lhyd1), vftot(jz,limp1)
     &    , vftot(jz,lion), vftot(jz,lelec)
      enddo
c
c..print fluxes
c%%%%%%%%%%%%%%
c
        lpage=lpage+1
        write(nprint,103) label1(1:48),label5(1:72),lpage,nstep,zt,zdt
c
      write ( nprint, 141)
c
      do jz=iaxis,mzones
c
        write (nprint,106) jz, zrminor(jz)
     &    , flxtot(jz,lhyd1), flxtot(jz,limp1)
     &    , flxtot(jz,lion), flxtot(jz,lelec)
      enddo
c
c..print sources
c%%%%%%%%%%%%%%%
c
        lpage=lpage+1
        write(nprint,103) label1(1:48),label5(1:72),lpage,nstep,zt,zdt
c
      write ( nprint, 142)
c
      do jz=iaxis,mzones
        write (nprint,106) jz, zrminor(jz)
     &    , srctot(jz,lhyd1), srctot(jz,limp1)
     &    , srctot(jz,lion), srctot(jz,lelec)
      enddo
c
c%%%%%%%%%%%%%%%%%%%%%%%%%%%%%%%%%%%%%%%%%%%%%%%%%%%%%%%%%%%%%%%%%%
c
10007 format(/,10x,'transport coefficients from theory',/)
10008 format(4x,'zone',6x,'radius',9x,'chi-elc',8x,'chi-ion',
     #       8x,'intrchg',10x,'zdifh',10x,'zdifz')
10009 format(16x,'m',13x,'m*m/s',10x,'m*m/s',12x,'w',
     #       12x,'m*m/s')
c
 101  format(5x,i2,6x,0pf6.3,8x,5(1pe11.4,4x))
 102  format(/,15x,'total interchange power = ',2x,e11.4,3x,'watts')
 103  format(/2x,a48,10x,a72/
     &  2x,'-',i2,'-',2x,'*** time step',i5,' ***',14x,
     &          'time =',f12.3,2x,'millisecs.',12x,'dt =',f12.6,2x,
     &          'millisecs.')
 104  format(/,10x,'electron thermal diffusion coefficients',/,
     #       10x,43('-'))
 105  format(12x,'m',9x,'m2/s',7(8x,'m2/s'),/
     & 5x,'jz',3x,'radius',5x,'xethe',8x,'neocl',6x
     &         ,'empirc',6x,'xetot')
 106  format(5x,i2,3x,0pf6.3,4x,9(1pe10.3,2x))
 107  format(/,10x,'ion thermal diffusion coefficients',/,
     #       10x,37('-'))
 108  format(12x,'m',9x,'m2/s',7(8x,'m2/s'),/
     &  5x,'jz',3x,'radius',5x,'xithe'
     &         ,8x,'neocl',7x,'empirc',6x,'xitot')
 109  format(12x,'m',9x,'m2/s',7(8x,'m2/s'),/
     &  5x,'jz',3x,'radius',6x,'dhdr',8x,'dhig',8x,'dhti',8x
     &         ,'dhrm',8x,'dhrb',8x,'dhkb',8x,'dhnm'
     &         ,8x,'dhhf',8x,'dhtot')
c
 110  format (0p1f6.3,1p6e12.3)
c
 120  format(0pf6.3,4x,9(1pe10.3,2x))
 122  format(2x,'m',9x,'m2/s',7(8x,'m2/s'),/
     &  'radius',t12,'thiig',t24,'thirb',t36,'thikb'
     &  ,t48,'xithe',t60,'neocl',t72,'empirc',t84,'xitot')
 124  format(2x,'m',9x,'m2/s',7(8x,'m2/s'),/
     &  'radius',t12,'theig',t24,'therb',t36,'thekb'
     &  ,t48,'xethe',t60,'neocl',t72,'empirc',t84,'xetot')
 125  format(/,10x,'hydrogenic particle diffusion coefficients',/,
     #       10x,37('-'))
 126  format(2x,'m',9x,'m2/s',7(8x,'m2/s'),/
     &  'radius',t12,'thdig',t24,'thdrb',t36,'thdkb'
     &  ,t48,'dhthe',t60,'neocl',t72,'empirc',t84,'dhtot')
 127  format(/,10x,'impurity particle diffusion coefficients',/,
     #       10x,37('-'))
 128  format(2x,'m',9x,'m2/s',7(8x,'m2/s'),/
     &  'radius',t12,'tzdig',t24,'tzdrb',t36,'thzkb'
     &  ,t48,'dzthe',t60,'neocl',t72,'empirc',t84,'dztot')
c
 130  format (
     & /,10x,'Profiles as a function of major radius'
     & /,t4,'rmajor(m)',t17,'ne(m^-3)',t30,'Te(keV)',t43,'Ti(keV)'
     &  ,t56,'Zeff')
 132  format (5(2x,1pe11.4))
c
 133  format (
     & /,10x,'Densities as a function of minor radius',/
     &  ,5x,'jz',3x,'radius',8x,'ne',10x,'ni',10x,'nh',10x,'nz'
     &  ,10x,'ns',10x,'nse',8x,'zeff',9x,'mi')
c
 135  format (5x,'jz',3x,'radius',2x,'diffusivity',4x,'velthi(*)'
     &  ,'  difthi(*,1) difthi(*,2) difthi(*,3) difthi(*,4)'
     &  ,t95,'perform')
c
 140  format (/5x,'jz',3x,'radius',t22,'vftot_H'
     &  ,t34,'vftot_I',t46,'vftot_Ti',t58,'vftot_Te')
c
 141  format (/5x,'jz',3x,'radius',t22,'flxtot_H'
     &  ,t34,'flxtot_I',t46,'flxtot_Ti',t58,'flxtot_Te')
c
 142  format (/5x,'jz',3x,'radius',t22,'srctot_H',t34,'srctot_I'
     &  ,t46,'srctot_Ti',t58,'srctot_Te')
c
 990  return
      end
\end{verbatim}

\end{document}             % End of document.
