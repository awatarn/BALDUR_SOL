% 21:00 30-Mar-98  Glenn Bateman and Jon Kinsey, Lehigh University
%
%  This is a LaTeX ASCII file.  To typeset document type: latex theory
%  To extract the fortran source code, obtain the utility "xtverb" from
%  Glenn Bateman (bateman@pppl.gov) and type:
%  xtverb < theory.tex > theory.f
%
\documentstyle{article}    % Specifies the document style.
\headheight 0pt \headsep 0pt  \topmargin 0pt  \oddsidemargin 0pt
\textheight 9.0in \textwidth 6.5in

\title{ {\tt theory}: a BALDUR Subroutine \\ % The preamble begins here.
 for Computing Theory-based Tokamak \\
 Particle and Energy Fluxes}     % title.
\author{E. S. Ghanem, C. E. Singer, \\ University of Illinois \\ \\
        Jon Kinsey, Glenn Bateman,  \\ Lehigh University
        }
                           % Declares the author's name.
                           % Deleting the \date{} produces today's date.
\begin{document}           % End of preamble and beginning of text.
\maketitle                 % Produces the title.

This report documents a subroutine called {\tt theory}, which computes plasma 
transport coefficients using various microinstability theory-based models.
The default model used in subroutine {\tt theory} was developed by
C.~E. Singer as documented in
``Theoretical Particle and Energy Flux Formulas for Tokamaks,''
Comments on Plasma Physics and Controlled Fusion, {\bf 11}, 165 (1988)
(\cite{Comments}, hereafter referred to as the Comments paper).

\begin{verbatim}
c@theory.tex
c--------1---------2---------3---------4---------5---------6---------7-c
c
      subroutine theory( lthery, cthery
     & , maxis, medge, mseprtx, matdim
     & , rminor, rmajor, elong, triang
     & , dense, densi, densh, densimp, densf, densfe
     & , xzeff, tekev, tikev, tfkev, q, vloop, btor, resist
     & , avezimp, amassimp, amasshyd, aimass, wexbs
     & , grdne, grdni, grdnh, grdnz, grdte, grdti, grdpr, grdq
     & , fdr, fig, fti, frm, fkb, frb, fhf, fec, fmh, fdrint
     & , dhtot, vhtot, dztot, vztot, xetot, xitot, weithe
     & , difthi, velthi
     & , nstep, time, nprint, lprint)
c
c  lthery(jc), j=1,50    integer control variables (see below)
c  cthery(jc), jc=1,150  general control variables (see below)
c  maxis   = zone boundary at magnetic axis
c  medge   = zone boundary at edge of plasma
c  mseprtx = zone boundary at separatrix
c
c  matdim  = first and second dimension of transport matricies
c              difthi(j1,j2,jz) and velthi(j1,jz)
c
cbate     & , slne, slni, slnh, slnz, slte, slti, slpr, sshr
c
c    All the following 1-D arrays are assumed to be on zone boundaries
c    including the densities (in m^-3) and temperatures (in keV).
c
c  rminor(jz) = minor radius (half-width) of zone boundary (m)
c  rmajor(jz) = major radius to geometric center of zone bndry (m)
c  elong(jz)  = local elongation of zone boundary
c  triang(jz) = triangularity of zone boundary
c
c  aimass(jz) = mean atomic mass of thermal ions (AMU)
c  dense(jz)  = electron density (m^-3)
c  densi(jz)  = thermal ion density (m^-3)
c  densh(jz)  = thermal hydrogen ion density (m^-3)
c  densimp(jz) = sum over impurity ion densities (m^-3)
c  densf(jz)  = fast (non-thermal) ion density (m^-3)
c  densfe(jz) = electron density from fast (non-thermal) ions (m^-3)
c  xzeff(jz)  = Z_eff
c  tekev(jz)  = T_e (keV)  (electron temperature)
c  tikev(jz)  = T_i (keV)  (temperature of thermal ions)
c  tfkev(jz)  = T_f (keV)  (effective temperature of fast ions)
c  q(jz)      = magnetic q-value
c  vloop(jz)  = loop voltage  (volts)
c  btor(jz)   = ( R B_tor ) / rmajor(jz)  (tesla)
c  resist(jz) = plasma resistivity ( Ohm m )
c
c  avezimp(jz) = average density weighted charge of impurities
c  amassimp(jz) = average density weighted atomic mass of impurities
c  amasshyd(jz) = average density weighted atomic mass of hydrogen ions
c
c  wexbs(jz)   = ExB shearing rate in [rad/s]
c
c    All of the following normalized gradients are at zone boundaries.
c    r = half-width, R = major radius to center of flux surface
c    carry out any smoothing before calling sbrtn theory
c
c  grdne(jz) = R ( d n_e / d r ) / n_e
c  grdni(jz) = R ( d n_i / d r ) / n_i
c     n_i = thermal ion density (sum over hydrogenic and impurity)
c  grdnh(jz) = R ( d n_h / d r ) / n_h
c     n_h = thermal hydrogenic density (sum over hydrogenic species)
c  grdnz(jz) = R ( d Z n_Z / d r ) / ( Z n_Z )
c     n_Z = thermal impurity density,  Z = average impurity charge
c           sumed over all impurities
c  grdte(jz) = R ( d T_e / d r ) / T_e
c  grdti(jz) = R ( d T_i / d r ) / T_i
c  grdpr(jz) = R ( d p   / d r ) / p    for thermal pressure
c  grdq (jz) = R ( d q   / d r ) / q    related to magnetic shear
c
c  fdr,..., fmh  coefficients for transport contributions (see below)
c
c  dhtot(jz)   = hydrogenic diffusivity ( m^2/sec )  (output)
c  vhtot(jz)   = hydrogenic convective velocity ( m/sec )  (output)
c  dztot(jz)   = impurity diffusivity ( m^2/sec )  (output)
c  vztot(jz)   = impurity convective velocity ( m/sec )  (output)
c  xetot(jz)   = electron thermal diffusivity ( m^2/sec ) (output)
c  xitot(jz)   = ion thermal diffusivity ( m^2/sec )  (output)
c  weithe(jz)  = anomalous electron-ion equipartition (output)
c
c  nstep  = current time-step
c  time   = current time (sec)
c
c  nprint = output unit number for long printout
c  lprint controls the amount of printout ( 0 -> no printout)
c           higher values yield more diagnostic printout
c  mprint = 0 no printout from etaw14.f
c
c   The input control variables are documented on the following lines:
c
c  Input control variables:
c  ------------------------
c
c     contributions to fluxes and interchanges: (all defaults = 0.0)
c
c  fdr(1)   particle transport from drift modes (trapped electron)
c  fdr(2)   electron thermal transport from drift modes (trapped electron)
c  fdr(3)   ion      thermal transport from drift modes (trapped electron)
c
c  fig(1)   particle transport from ITG (eta_i) mode
c  fig(2)   electron thermal transport from ITG (eta_i) mode
c  fig(3)   ion      thermal transport from ITG (eta_i) mode
c
c  fti(1)   particle transport from trapped ion modes
c  fti(2)   electron thermal transport from trapped ion modes
c  fti(3)   ion      thermal transport from trapped ion modes
c
c  frm(1)   particle transport from rippling mode
c  frm(2)   electron thermal transport from rippling mode
c  frm(3)   ion      thermal transport from rippling mode
c
c  frb(1)   particle transport from resistive ballooning mode
c  frb(2)   electron thermal transport from resistive ballooning mode
c  frb(3)   ion      thermal transport from resistive ballooning mode
c
c  fkb(1)   particle transport from kinetic ballooning mode
c  fkb(2)   electron thermal transport from kinetic ballooning mode
c  fkb(3)   ion      thermal transport from kinetic ballooning mode
c
c  fhf(1)   particle transport from the eta_e mode
c  fhf(2)   electron thermal transport from the eta_e mode
c  fhf(3)   ion      thermal transport from the eta_e mode
c
c  fmh(1)   particle transport from the neoclassical mhd mode
c  fmh(2)   electron thermal transport from the neoclassical mhd mode
c  fmh(3)   ion thermal transport from the neoclassical mhd mode
c
c  fec(1)   particle transport from the circulating electron mode
c  fec(2)   electron thermal transport from the circulating electron mode
c  fec(3)   ion thermal transport from the circulating electron mode
c
c  fdrint   electron-ion energy interchange coefficient
c
c
c     lthery(j), j=1,50   integer control variables: (all defaults = 0)
c
c  lthery(1)  = 0 for the original shear scale length model
c
c  lthery(2)  = 1 to extrapolate density * diffusivity to edge grid point
c
c  lthery(3)  = 0 to set zln = zlnj = zlne (electron density scale length)
c             = 1   set zln = zlne and zlnj = zlni
c             = 2   set zln = zlnj = zlni (thermal ion density scale length)
c             = 3   use zlnh in sbrtn etaw14
c             = 4   use zlnh and zlnz in sbrtn etaw14
c
c  lthery(4)  = 0 to use Spitzer resistivity throughout (default)
c             = 1 to use neoclassical resistivity computed elsewhere
c
c  lthery(5)  < 0 for no model
c             = 0 for the original dissipative trapped electron mode
c             = 1 for the Hahm-Tang trapped electron mode theories
c                   implemented by Bateman (1990)
c  lthery(6)  < 0 for no model
c             = 0 for the original collisionless trapped electron mode
c             = 1 for min [ c_{20}, c_{21} 0.1 / \nu_e^* ] transition
c                     suggested by Greg Rewoldt
c                   Implementations by M. Redi and J. Cummings (1990):
c             = 2 for the Hahm-Tang CTEM/DTEM model (IAEA 1990).
c             = 3 for original CTEM with no transition
c             = 4 for Hahm-Tang CTEM only
c             = 5 for Kadomtsev-Pogutse DTEM model
c             = 6 for K-P model with Rewoldt transition
c             = 7 for Hahm-Tang CTEM with Rewoldt transition
c             = 8 for Hahm-Tang CTEM with K-P DTEM
c  lthery(7)  < 0 for no model
c             = 0 for the original eta_i mode
c             = 1 for Lee-Diamond eta_i mode transport
c             = 2 for Hamaguchi-Horton eta_i mode
c             = 4 for Ottaviani-Horton-Erba ``Santa Barbara'' eta_i mode
c             = 6 for Kim-Horton eta_i mode
c             = 10 for Weiland-Nordman NF 30 (1990) 983 eta_i model
c             = 11 for both eta_i and TEM models from NF 30 (1990) 983 
c                  Note: this option overrides previous TEM models
c             = 12 Full matrix form of Weiland-Nordman model
c                  Note: effective diffusivities are computed only for 
c                    diagnostic purposes
c             = 21 Weiland model Hydrogen \eta_i mode only
c             = 22 Weiland model with Hydrogen and trapped electrons
c             = 23 Weiland model Hydrogen, trapped electrons,
c                    and one impurity species
c             = 24 Weiland model Hydrogen, trapped electrons,
c                  one impurity species, and collisions
c             = 25 Weiland model Hydrogen, trapped electrons,
c                  one impurity species, collisions, and parallel
c                  ion (hydrogenic) motion
c             = 26 Weiland model Hydrogen, trapped electrons,
c                  one impurity species, collisions, and finite beta
c             = 27 Weiland model Hydrogen, trapped electrons,
c                  one impurity species, collisions, parallel
c                  ion (hydrogenic) motion, and finite beta
c             = 28 Weiland model Hydrogen, trapped electrons,
c                  one impurity species, collisions, parallel
c                  ion (hydrogenic, impurity) motion, and finite beta
c
c  lthery(8)  < 0 for no model
c             = 0 for the original threshold for the eta_i mode
c             = 1 for Mattor-Diamond, Hahm-Tang threshold
c             = 2 for Dominguez-Rosenbluth threshold
c             = 20 normalize diffusivity matrix with velthi = 0.0
c             > 20 use effective diffusivities and set difthi=velthi=0.0
c
c  lthery(9)  < 0 for no model
c             = 0 for the original form of f_{ith}
c             = 1 for (\eta_i-\eta_i^{th})/\eta_i^{th}
c             = 2 for linear ramp form of f_{ith}
c
c  lthery(10) = 0 for the original rippling mode
c  lthery(11) = 0 for the original \chi_e^{RM} rippling mode
c             = 1 for estimate of \chi_e^{RM} from Hahm, Diamond, et al
c                   PF 30 (1987) 1452 [Eq (53]
c
c  lthery(12) = 1 use (1+\kappa^2)/2 instead of \kappa scaling
c                 otherwise use \kappa scaling
c                 raised to exponents (cthery(12) - ctheory(16))
c
c  lthery(13) = 0 for the original resistive ballooning mode
c             = 1 Carreras-Diamond PF B 1 (1989) 1011-1017
c                   model for resistive ballooning modes
c             = 2 hybrid of new and old resistive ballooning mode models
c             = 3 Guzdar-Drake drift/resistive ballooning model
c             = 4 Bruce Scott's Drift Alfven model
c
c  lthery(14) = 0 single iteration approximation for lambda
c                   in the Carreras-Diamond resistive ballooning mode
c             = n for n iterations for lambda in RB mode
c
c  lthery(15) = 1 for neoclassical MHD driven transport
c             = 2 for neoclassical MHD with Callen corrections
c
c  lthery(16) = 1 for the circulating electron/high-m tearing modes
c                 cthery(82) multiplies contribution from high-m tearing
c
c  lthery(17) = 0 for the original kinetic ballooning mode
c             = 1 for 1995 kinetic ballooning mode
c
c  lthery(19) > 1 to smooth fast ion relative density lthery(19) times
c
c  lthery(20) = 0 for the original eta_e mode
c
c  lthery(21) < 1 to call sbrtn theory
c             = 1 for sbrtn mmm95 rather than sbrtn theory
c             = 2 for sbrtn mmm98
c             = 3 for sbrtn mmm98b
c             = 4 for sbrtn mmm98c
c             = 5 for sbrtn mmm98d
c             = 6 for sbrtn ohe model
c             = 7 for sbrtn mixed_merba
c             = 8 for sbrtn mixed_model (Mixed Bohm/gyro-Bohm model)
c             = 9 for sbrtn mmm99
c
c  lthery(22) = 0 for the default Weiland model in sbrtn theory
c             = -1 weiland14 called from sbrtn theory
c             = -2 etaw14diff called from sbrtn theory
c             = -3 etaw14a called from sbrtn theory
c             = -4 etaw17diff called from sbrtn theory
c             = -5 etaw17a called from sbrtn theory
c
c  lthery(23) = 0 use major radius to geometric center of flux surfaces
c             = 1 use major radius to outboard edge of flux surfaces
c
c  lthery(24) = 1  to set desnsi = densh + densimp
c                    and grdni = grdnh + grdnz
c                    and grdpr = grdti + grdni + grdte + grdne
c
c  lthery(25) = 1 for the Rebut-Lallia-Watkins model
c
c  lthery(26) =   timestep for diagnostic output for etaw* model
c
c  lthery(27) > 0 to replace negative diffusivity with velocity
c
c  lthery(28) > 0 to smooth diffusivities lthery(28) times
c
c  lthery(29) = 0 to minimize diagnostic printout
c             > 0 larger values produce more diagnostic printout
c
c  lthery(30) = 0 take absolute value of gradient scale lengths
c             = 1 to retain sign of gradient scale lengths
c
c  lthery(31) = 1 to make r * ( gradient scale lengths ) monotonic
c                   near the magnetic axis by changing value at maxis+1
c
c  lthery(32) = 0 for the original gradient scale lengths
c             .gt. 0 for smoothing over lthery(32) orders (like lsmord)
c             .lt. 0 to smooth 1 / gradient scale lengths
c
c  lthery(33) = 0 print unsmoothed gradient scale lengths
c             .gt. 0 to print smoothed values
c
c    The following switches control mmm99, called when lthery(21) = 9
c
c  lthery(34) = 0 use sbrtn weiland14 from mmm95 for ITG mode in mmm99 
c             = 1 use etaw17 from mmm98d for ITG in mmm99 
c
c  lthery(35) = 0 use Guzdar-Drake resistive balloong mode
c                   from mmm95 in mmm99
c             = 1 use drift Alfven mode (Scott model) from MMM98d
c                   in mmm99
c
c  lthery(36) = 0 use old kinetic ballooning mode from MMM95 in mmm99
c             = 1 use new kinetic ballooning mode (by Redd) from MMM98d
c                   in mmm99
c
c  lthery(37) = 0 use Hahm-Burrel ExB shearing rate to substract
c                   from weiland model growth rates in mmm99
c             = 1 use Hamaguchi-Horton ExB shearing parameter
c                   to multiply all long-wavelength transport coeffs
c                   in mmm99
c
c  lthery(38) = 0 do not use any ETG model in mmm99
c             = 1 use Horton's ETG model in mmm99
cap
c  lthery(39) >=0 use zxithe, zvithe ... rather than difthi/velthi
c             < 0 use difthi/velthi rather than zxithe, zvithe ... 
c
c             |...| 5   don't produce transport matrix, only eff.diffusiv. 
c                       (default)
c                   3   only diagonal elements of diffusion matrix 
c                   1   full diffusion matrix
c                  x2   rescale transport matrix with velthi=0 
c                  x1   don't rescale transport matrix with velthi=0 
c                       (default)
c
c  lthery(40)  = 0  (default) do not include E&M effects in Weiland model
c                1  include E&M effects in Weiland model
c
c  lthery(41)  = 0  (default) do not multiply zchieff(1) by znh/zni
c                1  multiply zchieff(1) by znh/zni in mmm99.tex
c
c*************
c
c     cthery(j), j=1,150  general control variables:
c
c  variable  default  meaning
c  --------  -------  -------
c
c  cthery(1)  0.0  divertor shear
c  cthery(2)  1.0  radius of artificial separatrix relative to BALDUR
c                    separatrix
c  cthery(3)  0.5  minimum shear
c  cthery(4)  0.0  logarithmic gradient of Zeff
c  cthery(5)  6.0  fully charged state of impurity
c  cthery(6)  1.0  coefficient of f_{ith} used to cut off \eta_i mode
c  cthery(7)  6.0  f_{ith} = 1./(1. + exp[-cthery(7)*(\eta_i-\eta_i^{th})])
c                  or when lthery(9)=2, cthery(7) controls the width
c                  of the linear ramp starting from \eta_i^{th}
c  cthery(8)  6.0  coeff cutoff of kinetic ballooning mode in CPP Eq. (58)
c  cthery(9)  6.0  coeff in cutoff of \eta_e mode in CPP Eq. (63)
c  cthery(10) 0.0  
c  cthery(11) 1.0  minimum value of \eta_i^{th} CPP (34) when lthery(8)=0
c  cthery(12) 0.0  exponent of local elongation multiplying drift waves
c  cthery(13) 0.0  exponent of local elongation multiplying rippling mode
c  cthery(14) 0.0  exponent of local elongation multiplying resistive
c                    balllooning modes
c  cthery(15) 0.0  exponent of local elongation multiplying
c                    kinetic balllooning modes
c  cthery(16) 0.0  exponent of local elongation multiplying eta_e mode
c  cthery(17) 1.0  coef of convective flux subtracted from thermal flux
c  cthery(19) -1.5 coefficient of f_ith added to 5/2 in drift wave
c                    electron thermal diffusivity
c  cthery(20) 1.0  transition from collisional to collisionless
c                    trapped electron mode in CPP Eq. 37 in \hat{D}_{te}
c  cthery(21) 1.0  coef of \omega^*_e/\nu_{eff} in \hat{D}_{te}
c                  or 0.1 / \nu_{e*} in \hat{D}_{te} when lthery(6)=1
c  cthery(22) 0.0  exp[-cthery(22)*(Ti/Te-1)**2] multiplying TEM
c  cthery(23) 0.95 Numerical correction in D^{DR} in Rewoldt TEM model
c
c  cthery(24) 0.0  multiplies $D_i^{RLW}$ Rebut-Lallia-Watkins model
c  cthery(25) 0.0  multiplies $\chi_e^{RLW} Rebut-Lallia-Watkins model
c  cthery(26) 0.0  multiplies $\chi_i^{RLW} Rebut-Lallia-Watkins model
c  cthery(27) 0.0  exponent of local elongation multiplying RLW model
c
c  cthery(29) 0.0  multiplies threshold \eta_i in OHE model
c
c  cthery(30) 1.0  baseline \eta_i^{th} when lthery(8)=1
c  cthery(31) 1.9  coeff of (1+T_i/T_e) L_n/L_s when lthery(8)=1
c                    \eta_i threshhold by Mattor-Diamond, Hahm-Tang
c  cthery(33) 0.0  collisional cutoff in Lee-Diamond \chi_e^{IG} \eta_i-mode
c
c  cthery(34) 0.0  exp[-cthery(34)*(Ti/Te-1)**2] multiplying eta_i mode
c                   (suggested by R. Dominguez, 24 April 1990)
c  cthery(35) 0.0  exp[ - min ( cthery(35) * Ln , cthery(36) * L_Ti ) / Ls ]
c  cthery(36) 0.0    in Hamaguchi-Horton theory \eta_i mode theory
c  cthery(37) 0.0  eta_i mode diffusivities multiplied by q(jz)**cthery(37)
c  cthery(38) 0.0  k_y \rho_s (= 0.316 if abs(cthery(38)) < zepslon)
c  cthery(39) 0.0  gammain for sbrtn e3bsub
c
c  cthery(41) 5.0  toroidal mode number for resistive ballooning modes
c  cthery(42) 0.0  coeff of f_\star in resistive ballooning mode CPP (52-53)
c  cthery(43) 0.0  exponential used in diamagnetic res ball mode stabilization
c                    zfdias = ( 1. + cthery(42) * zfstar )**(-cthery(43))
c                    should be 1./4. (old theory) to 1./6. (new theory)
c  cthery(44) 0.0  \chi_e = \chi_e^{RB} + cthery(44) * D^{RB} + ...
c  cthery(45) 0.0  correction applied to resistive ballooning mode 
c                    diffusivity to more closely match analytic solution
c                    (use cthery(45) = 1.0 for normal correction)
c  cthery(47) 8.0  exponent of q in denominator of lambda in RB model'
c  cthery(48) 0.69135 used to approximate lambda with single iteration
c  cthery(49) 1.0  exponent of shear in D^{RB} in RB model
c
c     The following effects are turned on only when cthery(*) .gt. zepslon
c  cthery(50) 0.0  upper bound on L_{ne} / R_{major}
c  cthery(51) 0.0  upper bound on L_{ni} / R_{major}
c  cthery(52) 0.0  upper bound on L_{Te} / R_{major}
c  cthery(53) 0.0  upper bound on L_{Ti} / R_{major}
c  cthery(54) 0.0  upper bound on L_{p } / R_{major}
c
c  cthery(56) 0.0  upper bound on eta_e
c  cthery(57) 0.0  upper bound on eta_i
c
c  cthery(60) 0.0  limits local rate of change of ITG mode diffusities
c  cthery(61) 0.0  limits local rate of change of TEM mode diffusities
c
c  cthery(68) 0.0  convective coefficient for electron thermal transport
c  cthery(69) 0.0  convective coefficient for ion thermal transport
c     Note:  many theories have convection built into the total thermal
c     diffusivities.  cthery(68) and cthery(69) are the coefficients of
c     any additional convective thermal transport (ie, 3/2 or 5/2).
c
c  cthery(70) 1.0  Multiplier in neoclassical mhd - zgammh
c  cthery(71) 0.0  Elongation exponent in neoclassical mhd
c  cthery(72) 0.0  coef of chi_e^{NM} added to D^{NM}
c  cthery(73) 0.0  coef of D_p^{NM} added to \chi_e^{NM}
c  cthery(74) 0.0  coef of chi_e^{NM} added to \chi_i^{NM}
c  cthery(75) 1.0  coeff of \omega_*^2/\gamma^2 in diamagnetic stabilization
c  cthery(76) 0.0  exponential used in diamagnetic stabilization
c                  zfdia3 = (1.+cthery(75)*\omega_*^2/\gamma^2)^cthery(76)
c  cthery(77) 1.0  toroidal mode number = max [ 1.0, cthery(77) ]
c
c  cthery(78) 1.0  coeff of beta_prime_1 in kinetic ballooning mode
c  cthery(79) 1.0  coeff of beta_prime_2 in kinetic ballooning mode
c  cthery(80) 0.0  multiplier in D^{CE}
c  cthery(81) 0.0  elongation exponent in circulating electron mode
c  cthery(82) 0.0  multiplier in high-m tearing
c  cthery(83) 0.0  elongation exponent in high-m tearing
c
c  cthery(85) 2.0  diamagnetic correction to Guzdar-Drake model
c                  = 1.0 analytic expression for alpha
c                  = 2.0 to prescribe alpha using c_86
c  cthery(86) 0.15 alpha in diamagnetic stabilization in GD model
c
c  cthery(87)  0.0 ( normalized gradient pressure )**cthery(87)
c                  in the drift Alfven mode from Bruce Scott
c  cthery(88)  0.0 diagnostic printout time
c
c  cthery(111) 0.0 transfer from thigi(jz) to velthi(1,jz)
c  cthery(112) 0.0 transfer from zddig(jz) to velthi(2,jz)
c  cthery(113) 0.0 transfer from thige(jz) to velthi(3,jz)
c  cthery(114) 0.0 transfer from zdzig(jz) to velthi(4,jz)
c
c  cthery(119) 1.0 include effect of finite beta in etaw14
c  cthery(120) 0.0 min value of impurity charge state zimpz
c  cthery(121) 0.0 include superthermal ions
c  cthery(122) 0.0 -> 1.0 for effect of elongation in etaw16
c  cthery(123) 1.0 effect of parallel ion motion in etaw14
c  cthery(124) 0.0 -> 1.0 for effect of collisions in etaw14
c  cthery(125) 0.0 -> 1.0 for v_parallel in strong ballooning limit
c  cthery(126) 0.0 trapping fraction used in etaw14 (when > 0.0)
c                  multiplies electron trapping fraction when < 0.0
c  cthery(127) 0.0 fraction of computed diffusivities to be mixed with
c                    smoothed diffusivities
c  cthery(128) 0.0 tolerance used in eigenvalue finder in sbrtn etaw14
c  cthery(129) 0.0 multiplier for flow shear rate wexbs 
c  cthery(130) 0.0 -> 1.0 adds impurity heat flow to total ionic heat 
c                  flow for the weiland model
c***********************************************************************
c
cbate          include '../com/cbaldr.m'
cbate          include '../com/setimp.m'
cbate          include '../com/commhd.m'
cbate          include '../com/cd3he.m'
c
      integer kr, kn, kd, kc
c
      parameter ( kr = 55, kn = 6, kd = 9, kc = 150 )
c
c  kr = max number of radial grid points
c  kn = max number of charged particle species
c  kd = max number of transport matrix elements (transport channels)
c  kc = max number of elements in conrol arrays cthery and lthery
c
      real  cthery(*)
     & , rminor(*), rmajor(*), elong(*), triang(*)
     & , aimass(*), dense(*), densi(*), densh(*), densimp(*)
     & , densf(*), densfe(*)
     & , xzeff(*), tekev(*), tikev(*), tfkev(*)
     & , q(*), vloop(*), btor(*), resist(*)
     & , avezimp(*), amassimp(*), amasshyd(*), wexbs(*)
     & , grdne(*), grdni(*), grdnh(*), grdnz(*)
     & , grdte(*), grdti(*), grdpr(*), grdq(*)
cbate     & , slne(*), slni(*), slnh(*), slnz(*)
cbate     & , slte(*), slti(*), slpr(*), sshr(*)
c
      real  
     &   fdr(*), fig(*), fti(*), frm(*), fkb(*), frb(*)
     & , fhf(*), fec(*), fmh(*)
     & , dhtot(*), vhtot(*), dztot(*), vztot(*)
     & , xetot(*), xitot(*)
     & , weithe(*)
     & , difthi(matdim,matdim,*), velthi(matdim,*)
c
      real time
c
      integer lthery(*), maxis, medge, mseprtx, matdim
     &  , nstep, nprint, lprint, mprint
c
c..physical and computer constants
c
      real zpi, zcc, zcmu0, zceps0, zckb, zcme, zcmp, zce
     &  , zepslon, zepsinv, zlgeps, zepsqrt
c
      save zpi, zcc, zcmu0, zceps0, zckb, zcme, zcmp, zce
     &  , zepslon, zepsinv, zlgeps, zepsqrt
c
c  zpi    = pi
c  zcc    = speed of light ( m / sec )
c  zcmu0  = vacuum magnetic permeability   ( henrys / m )
c  zceps0 = vacuum electrical permittivity ( farads / m )
c  zckb   = Joule / keV
c  zcme   = electron mass ( kg )
c  zcmp   = proton mass ( kg )
c  zce    = electron charge ( Coulomb )
c  zepslon = machine epsilon ( smallest number st 1.0+zepslon > 1.0 )
c  zepsinv = 1.0 / zepslon
c  zlgeps  = ln ( zepslon )
c  zepsqrt = sqrt ( zepslon )
c
      real
     &   thdre(55) , thrme(55) , thrbe(55) , thkbe(55) , thhfe(55)
     & , thdri(55) , thrmi(55) , thrbi(55) , thkbi(55) , thhfi(55)
     & , thige(55) , thigi(55) , thtie(55) , thtii(55)
     & , threti(55), thdinu(55), thfith(55), thdte(55) , thdi(55)
     & , thbeta(55), thlni(55) , thlti(55) , thdia(55) , thnust(55)
     & , thlsh(55),  thlpr(55),  thlarp(55), thrhos(55), thrstr(55)
     & , thvthe(55), thvthi(55), thsoun(55), thalfv(55), thtau(55)
     & , thbpbc(55), thetth(55), thsrhp(55), thdias(55), thlamb(55)
     & , thrlwe(55), thrlwi(55), thnme(55),  thnmi(55), thhme(55)
     & , thcee(55), thcei(55)
     & , thrbgb(55), thrbb(55),  thvalh(55)
c
c
c..variables in common blocks /comth*/
c
c                 electron/ion thermal diffusivity from:
c  thdre/i(j)   drift waves (trapped electron modes)
c  thige/i(j)   ion temperature gradient (eta_i) modes
c  thtie/i(j)   trapped ion modes
c  thrme/i(j)   rippling modes
c  thrbe/i(j)   resistive ballooning modes
c  thrbgb,thrbb(j)   gyro-Bohm and Bohm contributions to res. ball.
c  thkbe/i(j)   kinetic ballooning modes
c  thhfe/i(j)   eta_e mode
c  thrlwe/i(j)  Rebut-Lallia-Watkins model
c
c  thdte(j)  = D_{te}
c  thdi(j)   = D_i
c
c    Lengths:
c  thlni(j)  = L_{ni}
c  thlti(j)  = L_{T_i}
c  thlsh(j)  = L_s = R q / s\hat
c  thlpr(j)  = L_p
c  thlarp(j) = \rho_s
c  thrhos(j) = \rho_{\theta i}
c
c    Velocities:
c  thvthe(j) = v_{the}
c  thvthi(j) = v_{thi}
c  thsoun(j) = c_s
c  thalfv(j) = v_A
c
c     Dimensionless:
c  thnust(j) = \nu_e^*
c  thrstr(j) = \rho_* = \rho_s/a
c  thbeta(j) = \beta
c  threti(j) = \eta_i
c  thsrhp(j) = S = \tau_R / \tau_{hp} = r^2 \mu_0 v_A / ( \eta R_0 )
c
c  thdinu(j) = \omega_e^\ast / \nu_{eff}
c  thetth(j) = \eta_i^{th}  threshold for \eta_i mode
c  thfith(j) = f_{ith} as in eq (35)
c  thbeta(j) = thbpbc(j) = \beta^{\prime} / \beta_{c1}^{\prime}
c  thdia(j)  = \omega_e^\ast / k_\perp^2
c  thdias(j) = resistive ballooning mode diamagnetic stabilization factor
c  thlamb(j) = \Lambda in Carreras-Diamond resistive ballooning mode
c              model PF B1 (1989) 1011-1017.
c  thvalh(j) = array of Hahm model criterion values
c
c
      dimension zfstarrb(55), zfdiarb(55), zgdtot(55), zgdlp(55)
     & , zalphz(55), zdelez(55), zlambz(55), zflamz(55)
     & , znuiz(55), zcmiz(55), zgammz(55), zfgamz(55), zfdiaz(55)
     & , zxnmz(55), zexbnz(55), zwstrnm(55), zprfmx(55), zfnsnea(55)
c
      dimension  zrhois(55), zrsist(55), ztcrit(55)
     & , zmlne(55), zmlni(55), zmlte(55), zmlti(55), zmshr(55)
     & , zslne(55), zslni(55), zslte(55), zslti(55), zsshr(55)
     & , zmlnh(55), zlnhs(55), zslnh(55), zslpr(55)
     & , zrhozs(55), zlnzs(55), zmlnz(55), zslnz(55)
c
      save zslne, zslni, zslnh, zslnz, zslte, zslti, zslpr, zsshr
c
      dimension zh1tem(55), zk1tem(55), zhdtem(55), zhctem(55)
     & , zddtem(55), zdztem(55), zd1tem(55), zddig(55), zdzig(55)
     & , zchie(55), zchii(55), zdifh(55), zdifz(55)
     & , zdshat(55), zdnuhat(55), zdbetae(55)
     & , zdbetah(55), zdbetaz(55), zkpar(55)
c
c zddig(jz) = hydrogen diffusivity from eta_i mode
c zdzig(jz) = impurity diffusivity from eta_i mode
c zchie(jz) = total effective electron thermal diffusivity for printout
c zchii(jz) = total effective ion      thermal diffusivity for printout
c zdifh(jz) = total effective hydrogen diffusivity for printout
c zdifz(jz) = total effective impurity diffusivity for printout
c zprfmx(jz) = maximum performance index
c zfnsnea(jz) = densfe(jz) / dense(jz)
c
      dimension  zgmitg(55), zomitg(55), zgm2nd(55), zom2nd(55)
     &  , zgmtem(55), zomtem(55), zomegde(55), zomegse(55)
     &  , zkinvsq(55)
c
c zgmitg(jz) = growth rate of ITG mode (sec^{-1})
c zomitg(jz) = frequency of ITG mode (sec^{-1})
c zgm2nd(jz) = growth rate of mode after ITG and TEM (sec^{-1})
c zom2nd(jz) = frequency of mode after ITG and TEM (sec^{-1})
c zgmtem(jz) = growth rate of trapped electron mode (sec^{-1})
c zomtem(jz) = frequency of trapped electron mode (sec^{-1})
c zomegde(jz) = omega_{De} (sec^{-1})
c zomegse(jz) = omega_{*e} (sec^{-1})
c zkinvsq(jz) = 1.0 / k_perp^2 (m^{-2})
c
      dimension  zbprima(55), zbc1a(55), zbc2a(55), zdka(55)
c
c  arrays associated with the kinetic ballooning mode
c
c  arrays z**tem(jz) are contributions to the Hahm-Tang trapped electron
c  mode calculation (see below)
c
      dimension zoetai(55), zdleti(55)
     &        , zoetae(55), zdlete(55)
     &        , zoetad(55), zdletd(55)
     &        , zoetaz(55), zdletz(55)
c
c  zoetai(jz) = thigi(jz) at the last time step
c  zdleti(jz) = thigi(jz) - zoetai(jz)
c  zdaeti(jz) = spatial average of zdleti(jz) ...
c
      dimension zotmai(55), zdltmi(55)
     &        , zotmae(55), zdltme(55)
     &        , zotmad(55), zdltmd(55)
     &        , zotmaz(55), zdltmz(55)
c
c  zotmai(jz) = xtmhes(jz) at the last time step
c  zdltmi (jz) = xtmhes(jz) - zotmai(jz)
c  zdatmi(jz) = spatial average of zdltmi(jz) ...
c
      dimension ztemp1(55), ztemp2(55), ztemp3(55), ztemp4(55)
c
      dimension zdfthi(12,12), zvlthi(12)
     &  , zchieff(12), zomega(12), zgamma(12), zperf(12), znerr(12)
     &  , zflux(12)
c
c  zdfthi(j1,j2) = full matrix of anomalous transport diffusivities
c  zvlthi(j1)    = convective velocities
c  zchieff(j1)   = effective diffusivities corresponding to zdifthi
c  zperf(j1)     = performance index
c  znerr(j1)     = error index
c  zflux(j1)     = normalized flux from Weiland model
c
c..control variables for eta_i mode models
c
      dimension iletai(32), zcetai(32)
c
c..hydrogen particle diffusivities
c
      dimension zdti(55), zdrm(55), zdrb(55), zdkb(55)
     &  , zdnm(55), zdhf(55)
c
      real zlastime, znormd, znormv
c
      save zlastime
c
c  zlastime     time at previous call to sbrtn theory
c  znormd       factor to convert normalized diffusivities
c  znormv       factor to convert normalized convective velocities
c
c..variables in common blocks /comth*/
c
c                 electron/ion thermal diffusivity from:
c  thdre/i(j)   drift waves (trapped electron modes)
c  thtie/i(j)   ion temperature gradient (eta_i) modes
c  thrme/i(j)   rippling modes
c  thrbe/i(j)   resistive ballooning modes
c  thrbgb,thrbb(j)   gyro-Bohm and Bohm parts of thrbe(j)
c  thnme/i(j)   neoclassical MHD modes
c  thkbe/i(j)   kinetic ballooning modes
c  thhfe/i(j)   eta_e mode
c  thrlwe/i(j)  Rebut-Lallia-Watkins model
c
c  thdte(j)  = D_{te}
c  thdi(j)   = D_i
c
c  difthi(j1,j2,jz) = full matrix of anomalous transport diffusivities
c  velthi(j1,jz)    = convective velocities
c
      integer isdasw(8), iswitch, idim, iprint
c
      real zsdasw(8)
     & , zfldath, zfldanh, zfldate, zfldanz, zfldatz
     & , zdfdath, zdfdanh, zdfdate, zdfdanz, zdfdatz
c
c Local copy of ExB shearing rate
c
      real zwexb
c

\end{verbatim}

The full matrix form of anomalous transport has the form
$$ \frac{\partial}{\partial t}
 \left( \begin{array}{c} n_H T_H  \\ n_H \\ n_e T_e \\ 
    n_Z \\ n_Z T_Z \\ \vdots
    \end{array} \right)
 = \nabla \cdot
\left( \begin{array}{llll} 
D_{1,1} n_H & D_{1,2} T_H & D_{1,3} n_H T_H / T_e \\
D_{2,1} n_H / T_H & D_{2,2} & D_{2,3} n_H / T_e \\
D_{3,1} n_e T_e / T_H & D_{3,2} n_e T_e / n_H & D_{3,3} n_e & \vdots \\
D_{4,1} n_Z / T_H & D_{4,2} n_Z / n_H & D_{4,3} n_Z / T_e \\
D_{5,1} n_Z T_Z / T_H & D_{5,2} n_Z T_Z / n_H & 
        D_{5,3} n_Z T_Z / T_e \\
 & \ldots & & \ddots
\end{array} \right)
 \nabla
 \left( \begin{array}{c}  T_H \\ n_H \\  T_e \\ 
   n_Z \\  T_Z \\ \vdots
    \end{array} \right)
$$
$$
 + \nabla \cdot
\left( \begin{array}{l} {\bf v}_1 n_H T_H \\ {\bf v}_2 n_H \\
   {\bf v}_3 n_e T_e \\
   {\bf v}_4 n_Z \\ {\bf v}_5 n_Z T_Z \\ \vdots \end{array} \right) +
 \left( \begin{array}{c} S_{T_H} \\ S_{n_H} \\ S_{T_e} \\
    S_{n_Z} \\ S_{T_Z} \\ \vdots
    \end{array} \right) $$
Note that all the diffusivities in this routine are normalized by
$ \omega_{De} / k_y^2 $, 
convective velocities are normalized by $ \omega_{De} / k_y $, 
and all the frequencies are normalized by $ \omega_{De} $.

\begin{verbatim}
c
c    Lengths:
c  thlni(j)  = L_{ni}
c  thlti(j)  = L_{T_i}
c  thlsh(j)  = L_s = R q / s\hat
c  thlpr(j)  = L_p
c  thlarp(j) = \rho_s
c  thrhos(j) = \rho_{\theta i}
c
c    Velocities:
c  thvthe(j) = v_{the}
c  thvthi(j) = v_{thi}
c  thsoun(j) = c_s
c  thalfv(j) = v_A
c
c     Dimensionless:
c  threti(j) = \eta_i
c  thdias(j) = resistive ballooning mode diamagnetic stabilization factor
c  thdinu(j) = \omega_e^\ast / \nu_{eff}
c  thfith(j) = f_{ith} as in eq (35)
c  thbpbc(j) = \beta^{\prime} / \beta_{c1}^{\prime}
c  thdia(j)  = \omega_e^\ast / k_\perp^2
c  thnust(j) = \nu_e^*
c  thrstr(j) = \rho_* = \rho_s/a
c  thlamb(j) = \Lambda multiplier for resistive ballooning modes
c
c  thbeta(j) = \beta
c  thetth(j) = \eta_i^{th}  threshold for \eta_i mode
c  thsrhp(j) = S = \tau_R / \tau_{hp} = r^2 \mu_0 v_A / ( \eta R_0 )
c  thvalh(jz)= Parameter determining validity of Hahm-Tang CTEM model
c
c-----------------------------------------------------------------------
\end{verbatim}

The coding continues with the
OLYMPUS  ({\it cf.} \cite{BALDUR}) number (2.21), 
and use of the OLYMPUS form for bypassing subroutines,
and comments on the common blocks and variables modified.
\begin{verbatim}
c
      logical initial
      data    initial /.true./
      save    initial
c
      integer istep, iclass, isub
      data    istep /1/,   iclass /2/   , isub /21/
      save    istep, iclass, isub
c
c-----------------------------------------------------------------------
c
c..physical and computer constants
c
      if ( initial ) then
        zpi     = atan2 ( 0.0, -1.0 )
        zcc     = 2.997925e+8
        zcmu0   = 4.0e-7 * zpi
        zceps0  = 1.0 / ( zcc**2 * zcmu0 )
        zckb    = 1.60210e-16
        zcme    = 9.1091e-31
        zcmp    = 1.67252e-27
        zce     = 1.60210e-19
        zepslon = 1.0e-34
        zepsinv = 1.0 / zepslon
        zlgeps  = log ( zepslon )
        zepsqrt = sqrt ( zepslon )
        zlastime = time
        initial = .false.
      endif
c
c
c..skip directly to printout if lprint < 0
c
      if ( lprint .lt. 0 ) go to 900
c
c..initialize arrays
c
      do 10 jz=1,medge
        xetot(jz) = 0.
        xitot(jz) = 0.
        weithe(jz) = 0.
        thdre(jz)  = 0.
        thdri(jz)  = 0.
        thige(jz)  = 0.
        thigi(jz)  = 0.
        thtie(jz)  = 0.
        thtii(jz)  = 0.
        thrme(jz)  = 0.
        thrmi(jz)  = 0.
        thrbgb(jz) = 0.
        thrbb(jz)  = 0.
        thrbe(jz)  = 0.
        thnme(jz)  = 0.
        thnmi(jz)  = 0.
        thcee(jz)  = 0.
        thcei(jz)  = 0.
        thhme(jz)  = 0.
        thrbi(jz)  = 0.
        thkbe(jz)  = 0.
        thkbi(jz)  = 0.
        thhfe(jz)  = 0.
        thhfi(jz)  = 0.
        thrlwe(jz) = 0.
        thrlwi(jz) = 0.
        zddtem(jz) = 0.
        zdztem(jz) = 0.
        zddig(jz)  = 0.
        zdti(jz)   = 0.
        zdrm(jz)   = 0.
        zdrb(jz)   = 0.
        zdkb(jz)   = 0.
        zdnm(jz)   = 0.
        zdhf(jz)   = 0.
        zfnsnea(jz) = 0.
        ztemp1(jz) = 0.
        ztemp2(jz) = 0.
        ztemp3(jz) = 0.
        ztemp4(jz) = 0.
  10  continue
c
      do jz=1,medge
        dhtot(jz) = 0.
        vhtot(jz) = 0.
        dztot(jz) = 0.
        vztot(jz) = 0.
      enddo
c
c
c..set up gradient scale length arrays
c
c      do jz=1,medge
c        zslne(jz) = slne(jz)
c        zslni(jz) = slni(jz)
c        zslnh(jz) = slnh(jz)
c        zslnz(jz) = slnz(jz)
c        zslte(jz) = slte(jz)
c        zslti(jz) = slti(jz)
c        zslpr(jz) = slpr(jz)
c        zsshr(jz) = sshr(jz)
c        zslne(jz) = slne(jz)
c      enddo
c
      do jz=1,medge
        zslne(jz) = rmajor(jz)
     &    / sign(max(abs(grdne(jz)),zepslon),grdne(jz)+zepsqrt)
        zslni(jz) = rmajor(jz)
     &    / sign(max(abs(grdni(jz)),zepslon),grdni(jz)+zepsqrt)
        zslnh(jz) = rmajor(jz)
     &    / sign(max(abs(grdnh(jz)),zepslon),grdnh(jz)+zepsqrt)
        zslnz(jz) = rmajor(jz)
     &    / sign(max(abs(grdnz(jz)),zepslon),grdnz(jz)+zepsqrt)
        zslte(jz) = rmajor(jz)
     &    / sign(max(abs(grdte(jz)),zepslon),grdte(jz)+zepsqrt)
        zslti(jz) = rmajor(jz)
     &    / sign(max(abs(grdti(jz)),zepslon),grdti(jz)+zepsqrt)
        zslpr(jz) = rmajor(jz)
     &    / sign(max(abs(grdpr(jz)),zepslon),grdpr(jz)+zepsqrt)
        zsshr(jz) = grdq(jz) * rminor(jz) / rmajor(jz)
      enddo
c
c  zsshr(jz) = ( r / q ) ( d q / d r )
c
c..smooth the relative superthermal ion density
c
      do jz=maxis, medge
        zfnsnea(jz) = cthery(121) * densfe(jz) / dense(jz)
      enddo
c
      if ( lthery(19) .gt. 0  .and. cthery(121) .gt. zepslon ) then
c
        ismooth = lthery(19)
        zsmooth = 0.0
        ilower  = maxis + 1
        iupper  = medge
c
        call smooth2 ( zfnsnea, 1, ztemp1, ztemp2, 1
     &    , ilower, iupper, ismooth, zsmooth )
c
      endif
c
\end{verbatim}

There are continuing numerical problems that can be traced back to noisy
gradient scale lengths.  To help combat these problems, apply smoothing
directly to the gradient scale lengths whenever ${\tt lthery(32)} > 0 $.
The numerical value of {\tt lthery(32)} will determine the order of 
smoothing.
Note that the gradient scale lengths vary like $1/r$ near the magnetic axis.
Hence, before smoothing, the gradient scale lengths will be pre-conditioned 
by multiplying them by the minor radius.
The output of this section will be the gradient scale lengths
{\tt zslne(jz)}, {\tt zslni(jz)}, {\tt zslte(jz)}, {\tt zslti(jz)},
all in meters.

Whenever ${\tt lthery(32)} < 0 $, the smoothing is applied to the 
reciprocal of the gradient scale lengths.

\begin{verbatim}
c
      if ( lthery(32) .ne. 0 ) then
c
        i1     = maxis + 1
        izones = medge + 1 - i1
        ismord = abs( lthery(32) )
        zlmin  = 1.e-4
c
c  ismord = number of times smoothing is applied
c  zlmin  = minimum gradient scale length
c
        do 33 jz=i1,medge
          zmlne(jz) = zslne(jz) * rminor(jz)
          zmlni(jz) = zslni(jz) * rminor(jz)
          zmlnh(jz) = zslnh(jz) * rminor(jz)
          zmlnz(jz) = zslnz(jz) * rminor(jz)
          zmlte(jz) = zslte(jz) * rminor(jz)
          zmlti(jz) = zslti(jz) * rminor(jz)
          zmshr(jz) = zsshr(jz) / rminor(jz)
  33    continue
c
c  change the values at jz=maxis+1 in order to make sure zml** arrays
c    are monotonic near the magnetic axis
c
        if ( lthery(31) .eq. 1 ) then
          zmlne(i1) = 2. * zmlne(i1+1) - zmlne(i1+2)
          zmlni(i1) = 2. * zmlni(i1+1) - zmlni(i1+2)
          zmlnh(i1) = 2. * zmlnh(i1+1) - zmlnh(i1+2)
          zmlnz(i1) = 2. * zmlnz(i1+1) - zmlnz(i1+2)
          zmlte(i1) = 2. * zmlte(i1+1) - zmlte(i1+2)
          zmlti(i1) = 2. * zmlti(i1+1) - zmlti(i1+2)
          zmshr(i1) = 2. * zmshr(i1+1) - zmshr(i1+2)
        endif
c
c  reciprocal of the gradient scale lengths
c
        if ( lthery(32) .lt. 0 ) then
          do jz=1,medge
            zmlne(jz) = 1.0 
     &        / sign ( max ( abs( zmlne(jz) ), zlmin ), zmlne(jz) )
            zmlni(jz) = 1.0 
     &        / sign ( max ( abs( zmlni(jz) ), zlmin ), zmlni(jz) )
            zmlnh(jz) = 1.0 
     &        / sign ( max ( abs( zmlnh(jz) ), zlmin ), zmlnh(jz) )
            zmlnz(jz) = 1.0 
     &        / sign ( max ( abs( zmlnz(jz) ), zlmin ), zmlnz(jz) )
            zmlte(jz) = 1.0 
     &        / sign ( max ( abs( zmlte(jz) ), zlmin ), zmlte(jz) )
            zmlti(jz) = 1.0 
     &        / sign ( max ( abs( zmlti(jz) ), zlmin ), zmlti(jz) )
            zmshr(jz) = 1.0 
     &        / sign ( max ( abs( zmshr(jz) ), zlmin ), zmshr(jz) )
          enddo
        endif
c
c  apply smoothing
c
        do 37 js=1,ismord
c
           do 34 jz=1,medge
             zslne(jz) = zmlne(jz)
             zslni(jz) = zmlni(jz)
             zslnh(jz) = zmlnh(jz)
             zslnz(jz) = zmlnz(jz)
             zslte(jz) = zmlte(jz)
             zslti(jz) = zmlti(jz)
             zsshr(jz) = zmshr(jz)
  34       continue
c
           znorm = 0.25
           do 35 jz=i1+1,medge-1
             zmlne(jz) = znorm*(zslne(jz-1)+2.0*zslne(jz)+zslne(jz+1))
             zmlni(jz) = znorm*(zslni(jz-1)+2.0*zslni(jz)+zslni(jz+1))
             zmlnh(jz) = znorm*(zslnh(jz-1)+2.0*zslnh(jz)+zslnh(jz+1))
             zmlnz(jz) = znorm*(zslnz(jz-1)+2.0*zslnz(jz)+zslnz(jz+1))
             zmlte(jz) = znorm*(zslte(jz-1)+2.0*zslte(jz)+zslte(jz+1))
             zmlti(jz) = znorm*(zslti(jz-1)+2.0*zslti(jz)+zslti(jz+1))
             zmshr(jz) = znorm*(zsshr(jz-1)+2.0*zsshr(jz)+zsshr(jz+1))
  35       continue
c
  37     continue
c
c  go from reciprocals back to the gradient scale lengths
c
        if ( lthery(32) .lt. 0 ) then
          do jz=1,medge
            zmlne(jz) = 1.0 
     &        / sign ( max ( abs( zmlne(jz) ), zlmin ), zmlne(jz) )
            zmlni(jz) = 1.0 
     &        / sign ( max ( abs( zmlni(jz) ), zlmin ), zmlni(jz) )
            zmlnh(jz) = 1.0 
     &        / sign ( max ( abs( zmlnh(jz) ), zlmin ), zmlnh(jz) )
            zmlnz(jz) = 1.0 
     &        / sign ( max ( abs( zmlnz(jz) ), zlmin ), zmlnz(jz) )
            zmlte(jz) = 1.0 
     &        / sign ( max ( abs( zmlte(jz) ), zlmin ), zmlte(jz) )
            zmlti(jz) = 1.0 
     &        / sign ( max ( abs( zmlti(jz) ), zlmin ), zmlti(jz) )
            zmshr(jz) = 1.0 
     &        / sign ( max ( abs( zmshr(jz) ), zlmin ), zmshr(jz) )
          enddo
        endif
c
c  undo preconditioning with minor radius
c
        do 38 jz=i1,medge
          zslne(jz) = zmlne(jz) / rminor(jz)
          zslni(jz) = zmlni(jz) / rminor(jz)
          zslnh(jz) = zmlnh(jz) / rminor(jz)
          zslnz(jz) = zmlnz(jz) / rminor(jz)
          zslte(jz) = zmlte(jz) / rminor(jz)
          zslti(jz) = zmlti(jz) / rminor(jz)
          zsshr(jz) = zmshr(jz) * rminor(jz)
  38    continue
c
      endif
\end{verbatim}

We then enter a loop over the spatial zones,
and set the following BALDUR {\tt common} variables
to variables local to subroutine {\tt theory}: the
mean atomic mass number of the thermal ions, $A_{i}$ ({\tt zai}),
the electron
and ion density and temperature, $n_{e}$ ({\tt zne}),
$n_{i}=\sum_{a}n_{a}$ ({\tt zni}),
$T_{e}$ ({\tt zte}), and $T_{i}$ ({\tt zti}),
the safety factor, $q$ ({\tt zq}),
the effective charge, $Z_{eff}$ ({\tt zeff}),
the scale lengths $L_{ne}$, $L_{ni}$, $L_{Te}$, $L_{Ti}$, $L_{p}$
({\tt zlne}, {\tt zlni}, {\tt zlte}, {\tt zlti}, {\tt zlpr}, which
are modified below after the poloidal gyroradius is calculated),
a quantity, $\theta_{shear}$ ({\tt zslbps},
discussed below and related to the shear),
the midplane halfwidth of the separatrix
(which BALDUR sets to the midplane halfwidth
of the outermost computational zone
if no scrapeoff zones are active), $r_{sep}$ ({\tt zrsep}),
the midplane halfwidth of a flux surface, $r$ ({\tt zrmin}),
the major radius, $R_{o}$ ({\tt zrmaj}), and
the toroidal field at a reference major radius, $B_{0}$ ({\tt zb}).
Additional variables local to subroutine {\tt theory} computed
from BALDUR {\tt common} variables are the inverse aspect
ratio for a flux surface, $\epsilon=r/R_{0}$ ({\tt zep})
and the local toroidal
electric field which is calculated using the variable VLOOPI
defined in BALDUR as $VLOOPI=2 \pi R_{O}*E_{O}.$
Therefore, the only variables not defined in subroutine
{\tt theory} that are needed to complete the rest of the
calculation are $\pi$ ({\tt zpi}), the small overflow
protection variable {\tt zepslon}, and the convection
constants $C_{pv}^{e}$ ({\tt cthery(68)}) 
and $C_{pv}^{i}$ ({\tt ctheory(69)}),
which are specified by BALDUR's preexisting {\tt namelist}
input (and defaulted to 1.5 if the inputs are zero.)
The points of defining so many local variables
are to compact the notation and to make it easier for other
transport code programmers to understand and possibly use
the main body of this subroutine without using BALDUR's
notation or array structure for various {\tt common} variables.
The relevant coding for the calculations just described is:

\begin{verbatim}
c
c.. start the main do-loop over the radial index "jz"..........
c
      jzmin = maxis + 1
c
c..diagnostic printout header
c
cbate      write (nprint,*) '#zz1   rminor    grdti     zgth'
c
      do 300 jz=jzmin,medge
c
c  transfer common to local variables to compact the notation
c
      zelong = max (zepslon,elong(jz))
      if ( lthery(12) .eq. 1 ) then
        zelonf = ( 1. + zelong**2 ) / 2.
      else
        zelonf = zelong
      endif
c
      zai = aimass(jz)
      zne = dense(jz)
      zni = densi(jz)
      znh = densh(jz)
      zte = tekev(jz)
      zti = tikev(jz)
      ztf = tfkev(jz)
      zq  = q(jz)
      zeff = xzeff(jz)
      zlne = zslne(jz)
      zlni = zslni(jz)
      zlnh = zslnh(jz)
      zlnz = zslnz(jz)
      zlte = zslte(jz)
      zlti = zslti(jz)
ces   temporary numerical overflow protection, cf. statement 22 below
      zlpr = zslpr(jz)
      zshear = zsshr(jz)
      zrsep = max( rminor(mseprtx), zepslon )
      zrmin = max( rminor(jz), zepslon )
      zrmaj = rmajor(jz)
      zvloop = abs(vloop(jz))
      zb    = btor(jz)
c
c  compute inverse aspect ratio
c
      zep = max( zrmin/zrmaj, zepslon )
c
      znfi = densf(jz)
      znfe = densfe(jz)
c
c
\end{verbatim}

To complete the rest of the calculation we then compute various
quantities needed for the transport flux formulas (as in Table 1 of
the Comments paper, from which
$\omega_{ce}$ was inadvertantly omitted).
To begin with, we compute only quantities
which do not involve scale heights.
In the order in which they are computed, algebraic notation for
these quantities is:
$$ p=n_e T_e + n_i T_i \ --- \ {\rm (thermal)}\eqno{\tt zprth} $$  
$$ \omega_{ci}=eB_{o}/(m_{p}A_{i}) \eqno{\tt zgyrfi} $$
$$ \beta=(2\mu_{o}k_{b}/B_{o}^{2})(n_{e}T_{e}+n_{i}T_{i})
 \eqno{\tt zbeta} $$
$$ \omega_{pe}=[n_{e}e^{2}/(m_{e}\epsilon_{o})]^{1/2} \eqno{\tt zfpe} $$
$$ v_{e}=(2k_{b}T_{e}/m_{e})^{1/2} \eqno{\tt zvthe} $$
$$ v_{i}=(2k_{b}T_{i}/m_{p}A_{i})^{1/2} \eqno{\tt zvthi} $$
$$ c_{s}=[k_{b}T_{e}/(m_{p}A_{i})]^{1/2} \eqno{\tt zsound} $$
$$ v_{A}=B_{o}/(\mu_{o}n_{e}m_{p}A_{i})^{1/2} \eqno{\tt zvalfv} $$
$$ \beta_{\theta}=\beta (q/\epsilon )^2 \eqno{\tt zbetap} $$
$$ \ln (\lambda)=37.8 - \ln (n_{e}^{1/2}T_{e}^{-1}) \eqno{\tt zlog} $$
$$ \nu_{ei}=4(2\pi)^{1/2}n_{e}(\ln \lambda)e^{4}Z_{eff}
               /[3(4\pi \epsilon_{o})^{2}m_{e}^{1/2}(k_{b}T_{e})^{3/2}]
 \eqno{\tt znuei} $$
$$ \nu_{ii}=4\pi ^{1/2}n_{e}(\ln \lambda)e^{4}
               /[3(4\pi \epsilon_{o})^{2}(m_{p}A_{i})^{1/2}
                 (k_{b}T_{i})^{3/2}] \eqno{\tt znuii} $$
$$ \eta=\nu_{ei}/(2\epsilon_{o}\omega_{pe}^{2}) \eqno{\tt zresis} $$
If $ {\tt lthery(4)} = 1 $, {\tt zresis} is set equal to the neoclasssical
resistivity {\tt eta(1,jz)} computed in BALDUR (in subroutine GETCHI).
$$ \nu_{eff}=\nu_{ei}/\epsilon \eqno{\tt znueff} $$
$$ \nu_{e}^{*}=\nu_{ei}qR_{o}/(\epsilon^{3/2}v_{e}) \eqno{\tt thnust} $$
$$ \nu_{i}^{*}=\nu_{ii}qR_{o}/(\epsilon^{3/2}v_{i}) \eqno{\tt znusti} $$
$$ \hat{\nu}=\nu_{eff}/\omega_{De} \eqno{\tt znuhat} $$
$$ \rho_{i}=v_{i}/\omega_{ci} \eqno{\tt zlarpo} $$
$$ \rho_{\theta i}=\rho_{i}q/\epsilon \eqno{\tt zlari} $$
$$ \rho_{s}=c_{s}/\omega_{ci} \eqno{\tt zrhos} $$
$$ \rho_{*}=\rho_{s}/a \eqno{\tt thrstr} $$
$$ k_{\perp}=0.3/\rho_{s} \eqno{\tt zwn} $$

The corresponding coding is:

\begin{verbatim}
c  start calculating table (1) of the comments paper
c
      zprth=zne*zte+zni*zti
      zgyrfi=zce*zb/(zcmp*zai)
      zbeta=(2.*zcmu0*zckb/zb**2)*(zne*zte+zni*zti)
      zfpe=zce*sqrt(zne/(zcme*zceps0))
      zvthe=sqrt(2.*zckb*zte/zcme)
      zvthi=sqrt(2.*zckb*zti/(zcmp*zai))
      zsound=sqrt(zckb*zte/(zcmp*zai))
      zvalfv=zb/sqrt(zcmu0*zne*zcmp*zai)
      zbetap=zbeta*(zq/zep)**2
      zlog=37.8-log(sqrt(zne)/zte)
       zcf=(4.*sqrt(zpi)/3.)
       zcf=zcf*(zce/(4.*zpi*zceps0))**2
      zcf=zcf*(zce/zckb)*zce/sqrt(zcme*zckb)
      znuei=zcf*sqrt(2.)*zne*zlog*zeff/(zte*sqrt(zte))
      znuii=zcf*zne*zlog/(sqrt(zcmp*zai/zcme)*zti**1.5)
      znustar=znuei*zq*zrmaj / ( zvthe*zep**1.5 )
      zresis=znuei/(2.*zceps0*zfpe**2)
c
      if ( lthery(4) .eq. 1 ) zresis = resist(jz)
c
      znueff=znuei/zep
      thnust(jz)=znuei*zq*zrmaj/(zvthe*zep**1.5)
      znusti = znuii * zq * zrmaj / ( zvthi * zep**1.5 )
      zlari=zvthi/zgyrfi
      zlarpo=max(zlari*zq/zep,zepslon)
      zrhos=zsound/zgyrfi
      thrstr(jz)=zrhos/ rminor(medge)
      zwn=0.3/zrhos
      znude=2*zwn*zrhos*zsound/zrmaj
      znuhat=znueff/znude
c
c
\end{verbatim}

Next we use the poloidal gyroradius to avoid singularities
associated with the scale lengths:

$$ L_{ne}=max(|-n_{e}/(\partial n_{e}/\partial r)|,\rho_{\theta i})
 \eqno{\tt zlne} $$
$$ L_{ni}=max(|-n_{i}/(\partial n_{i}/\partial r)|,\rho_{\theta i})
 \eqno{\tt zlni} $$
$$ L_{T_{e}}=max(|-T_{e}/(\partial T_{e}/\partial r|,\rho_{\theta i})
 \eqno{\tt zlte} $$
$$ L_{T_{i}}=max(|-T_{i}/(\partial T_{i}/\partial r|,\rho_{\theta i})
 \eqno{\tt zlti} $$
$$ L_{p}=max(|-\beta /(\partial \beta/\partial r)|,\rho_{\theta i})
 \eqno{\tt zlpr} $$
Here we are constructing the $L_p$ from the smoothed density and temperature
scale lengths.

If {\tt lthery(30)} = 0 (default), use only the absolute values
of the gradient scale lengths.

\begin{verbatim}

c  the following is to avoid singularities associated with
c  the scale lengths
c
      zsglne = 1.0
      zsglni = 1.0
      zsglnh = 1.0
      zsglnz = 1.0
      zsglte = 1.0
      zsglti = 1.0
      zsglpr = 1.0
c
      if ( lthery(30) .eq. 1 ) then
        zsglne = sign ( 1.0, zlne )
        zsglni = sign ( 1.0, zlni )
        zsglnh = sign ( 1.0, zlnh )
        zsglnz = sign ( 1.0, zlnz )
        zsglte = sign ( 1.0, zlte )
        zsglti = sign ( 1.0, zlti )
        zsglpr = sign ( 1.0, zlpr )
      endif
c
      zlne = max(abs(zlne),zlarpo) * zsglne
      zlni = max(abs(zlni),zlarpo) * zsglni
      zlnh = max(abs(zlnh),zlarpo) * zsglnh
      zlnz = max(abs(zlnz),zlarpo) * zsglnz
      zlte = max(abs(zlte),zlarpo) * zsglte
      zlti = max(abs(zlti),zlarpo) * zsglti
c      zlpr = max(abs(zlpr),zlarpo) * zsglpr
c
c  Compute the pressure scale length using smoothed and bound
c  density and temperature
c
      zdprth = zne*zte*(1./zlne+1./zlte) + zni*zti*(1./zlni+1./zlti)
      zsgdpr = sign ( 1.0, zdprth)
      zdpdr = max(abs(zdprth),zlarpo) * zsgdpr
      zlpr = zprth / zdpdr
c
c
\end{verbatim}

When any of the coefficients $c_{5*}$ are greater than {\tt zepslon},
upper bounds are placed on the scale lengths relative to the major radius
$$ L_{ne} = min ( |L_{ne}|, c_{50} R) \eqno{\tt zlne} $$
$$ L_{ni} = min ( |L_{ni}|, c_{51} R) \eqno{\tt zlni} $$
$$ L_{nh} = min ( |L_{nh}|, c_{51} R) \eqno{\tt zlnh} $$
$$ L_{nz} = min ( |L_{nz}|, c_{51} R) \eqno{\tt zlnz} $$
$$ L_{Te} = min ( |L_{Te}|, c_{52} R) \eqno{\tt zlte} $$
$$ L_{Ti} = min ( |L_{Ti}|, c_{53} R) \eqno{\tt zlti} $$
$$ L_{p } = min ( |L_{p }|, c_{54} R) \eqno{\tt zlp } $$
\begin{verbatim}
      if ( cthery(50) .gt. zepslon )
     &  zlne = min ( abs(zlne), cthery(50) * zrmaj ) * zsglne
      if ( cthery(51) .gt. zepslon )
     &  zlni = min ( abs(zlni), cthery(51) * zrmaj ) * zsglni
      if ( cthery(51) .gt. zepslon )
     &  zlnh = min ( abs(zlnh), cthery(51) * zrmaj ) * zsglnh
      if ( cthery(51) .gt. zepslon )
     &  zlnz = min ( abs(zlnz), cthery(51) * zrmaj ) * zsglnz
      if ( cthery(52) .gt. zepslon )
     &  zlte = min ( abs(zlte), cthery(52) * zrmaj ) * zsglte
      if ( cthery(53) .gt. zepslon )
     &  zlti = min ( abs(zlti), cthery(53) * zrmaj ) * zsglti
      if ( cthery(54) .gt. zepslon )
     &  zlpr = min ( abs(zlpr), cthery(54) * zrmaj ) * zsglpr
\end{verbatim}

In order to ensure backward compatibility, define $ {\tt zln} = {\tt zlne} $
if $ {\tt lthery(3)} \leq 1 $ and $ {\tt zln} = {\tt zlni} $ otherwise.
Also, define $ {\tt zlnj} = {\tt zlne} $ if $ {\tt lthery(3)} \leq 0 $
and $ {\tt zlnj} = {\tt zlni} $ otherwise.  The variable {\tt zlnj} is used
in the definition of $\eta_i$ {\tt zetai} and in formulae for the threshold
$\eta_{ith}$ {\tt zetith} below.
\begin{verbatim}
        zln  = zlni
        zlnj = zlni
      if ( lthery(3) .le. 1 ) zln  = zlne
      if ( lthery(3) .le. 0 ) zlnj = zlne
\end{verbatim}

Our formulas for the shear begin with
$$ {\hat s}_{cyl}=|(r/q)(\partial q/\partial r)| \eqno{\tt zscyl} $$
computed earlier in this subroutine.
We then provide an option
to accomodate users who want to simulate the effect of high
shear on transport
near a separatrix but find it inconvenient to specify
the boundary shape and use enough moments in the equilibrium
calculation to obtain a large shear there.  This is done by
defining
$$  k'=|1-\frac{r}{c_{2}r_{sep}}|^{1/2} \eqno{\tt zkprim} $$
and
$$ {\hat s}_{div}={\hat s}_{cyl}
   +c_{1}\left( \frac{1}{(k')^{2}|\ln (4/k')|} -\frac{1}{\ln 4} \right)
 \eqno{\tt zsdiv} $$
The function of $k'$
inside the parantheses here approximates the expression
$\frac{{\bf E}(\rho )}{{\bf K}(\rho )(1 - \rho^{2})} - 1$
({\it eg.} given
in the Comments paper
({\it cf.} \cite{HD} to within 2\% for $\rho \equiv r/(c_{2}r_{sep})=.96$)
for the expected situation where $k'\le 1$
Here ${\bf E}$ and ${\bf K}$ are complete elliptic integrals.
(When the user wants to
set the computational boundary inside the separatrix
location for a divertor simulation, then $c_{2}$ should
be set to the ratio of the outermost computational zone
location to the physical midplane halfwidth at the separatrix.)
The shear is then limited to be no larger than $r/\rho_{\theta_{i}}$
$$ {\hat s}_{lim}=min({\hat s}_{div},r/\rho_{\theta i}) \eqno{\tt zslim} $$

unless $r/\rho_{\theta_{i}}$ is less than a minimum prescribed shear
$$ {\hat s}_{min}=max(c_{3},0) \eqno{\tt zsmin} $$
since we use this to set a minimum on the shear of the form
$$ {\hat s}=max({\hat s}_{min},{\hat s}_{lim}) \eqno{\tt zshat} $$
Following the literature conventions used in the Comments paper,
the shear length is defined somewhat differently
than the other scale lengths, as
$$ L_{s}=R_{o}q/{\hat s} \eqno{\tt zlsh} $$
The default values, $c_{1}=0.0$, $c_{2}=1.0$, and $c_{3}=0.5$,
are set up so they can be modified by {\tt namelist} input.

The $Z_{eff}$ gradient is not presently calculated in BALDUR
and might not be very accurately modelled
in most simulations even if it were.
The effects of impurities are, in any case,
undergoing theoretical reexamination.
We, therefore, include the (normalized)
$Z_{eff}$ gradient $c_{4}=(\partial Z_{eff}/\partial r)/Z_{eff}$
for the time being merely as an input constant
(with default =0) when computing
the electrical conductivity scale height:
$$ L_{\sigma}=[1.5L_{T_{e}}^{-1}+c_{4}]^{-1} \eqno{\tt zlsig} $$

Some other generally useful quantities dependent on scale heights are:
$$ \eta_{e}=L_{ne}/L_{T_{e}} \eqno{\tt zetae} $$
$$ \mbox{if} \; \; c_{56} > \epsilon \; \; \mbox{then} \; \;
   \eta_e = \min ( \eta_e , c_{56} R ) \eqno{\tt zetae} $$
$$ \eta_{i}=L_{n}/L_{T_{i}} \eqno{\tt zetai} $$
$$ \mbox{if} \; \; c_{57} > \epsilon \; \; \mbox{then} \; \;
   \eta_i = \min ( \eta_i , c_{57} R ) \eqno{\tt zetae} $$
$$ \omega_{e}^{*}=k_{\perp}\rho_{s}c_{s}/L_{ne} \eqno{\tt zdiafr} $$

Note that
BALDUR already calculates $\ln (\lambda)$ and $\nu_{e}^{*}$
(and stores at least $\nu_{e}^{*}$ in {\tt common} as {\tt xnuel}).
Each of these existing calculations should only be used if and only if
they are identical to the formulas in Table 1 of the Comments paper,
in order to avoid unnecessary minor difficulties in communicating to
non-BALDUR users exactly what formulas were used.  (The idea
is that other people should, at least in principle,
be able to reproduce the transport flux formulas we use
in work based on the Comments paper
by referring only to the Comments paper, insofar as possible.)

The relevant coding for the calculations just described is:

\begin{verbatim}
c
      zscyl=max(abs(zshear),zepslon)
      zkprim=max(sqrt(abs(1.-zrmin/(cthery(2)*zrsep))),zepslon)
c       parentheses in the next line are OK in versions 15.06...
      zsdiv=zscyl
     & + cthery(1)*(1./(zkprim**2*abs(log(4./zkprim))) -1./log(4.))
      zslim=min(zsdiv,zrmin/zlarpo)
      zsmin=max(cthery(3),zepslon)
      zshat=max(zsmin,zsdiv)
      zlsh=zrmaj*zq/zshat
      zlsig=1./(1.5/zlte + cthery(4))
      zetae  = zlne/zlte
      if ( cthery(56) .gt. zepslon )
     &  zetae = min ( zetae, cthery(56) * zrmaj )
      zetai  = zlnj / zlti
      if ( cthery(57) .gt. zepslon )
     &  zetai = min ( zetai, cthery(57) * zrmaj )
      zdiafr = zwn * zrhos * zsound / zlne
c
\end{verbatim}
The magnetic Reynold's number is defined by
$$ S = \tau_R / \tau_{hp}  \eqno{\tt zsrhp} $$
$$ \tau_R = r^2 \mu_0 / \eta \equiv \mbox{local restive time},
      \eqno{\tt ztaur} $$
For the moment, we use the local Spitzer resistivity {\tt zresis},
although I think the neoclassical resistivity (which may be 3 times larger)
should be used in the future.
$$ \tau_{hp} = R_0 / v_A \equiv \mbox{local poloidal Alfven time}.
      \eqno{\tt ztauhp} $$
\begin{verbatim}
c
        ztaur  = zrmin**2 * zcmu0 / zresis
        ztauhp = zrmaj / zvalfv
        zsrhp  = ztaur / ztauhp
c
c  this is the end of calculating parameters in table (1) of
c  the comment paper.
\end{verbatim}

%**********************************************************************c

\section{Transport Models}

The computation of the anomalous transport coefficients is
now described.  We do this for the energy fluxes by computing
thermal diffusivities, despite the fact that the Comments
paper directly gives energy fluxes.  This is done
to maintain parallelism with previous methods of
computing anomalous energy fluxes in BALDUR.  It should be
noted that, for convenience in subtracting out what
are traditionally called convective energy fluxes, we
assume quasineutrality in a pure hydrogen plasma.  The
user can (and should) set the input parameters
${\tt cthery(68)}={\tt cthery(69)}=0.0$
when precise modelling of
other types of plasma is desired ({\it cf.} Section 3.7, below).

%**********************************************************************c

\subsection{Trapped Electron Modes}

Here, ``drift wave fluxes'' include dissipative trapped electron modes
and collisionless trapped electron modes.
(Note that $\eta_i$ modes have been moved to a separate section below.)

To begin with, allow all trapped electron mode contributions to be 
multiplied by
$$ \exp^{-c_{22}(T_i/T_e - 1)^2}.  \eqno{\tt zdtite} $$
as suggested by R. Dominguez.
By default, $c_{22} = 0.0$, so this factor has no effect.
Dominguez suggests the value $c_{22} = 1.0$.

\begin{verbatim}
c
c .......................................
c . trapped electron mode calculations  .
c .......................................
c
      zdtite = 1.0
      if ( abs(cthery(22)) .gt. zepslon )
     &  zdtite = exp( -cthery(22) * ((zti/zte)-1.)**2 )
c
\end{verbatim}

If ${\tt lthery(5)} = 1$, we use the collisionless trapped electron mode
transport theory by Hahm and Tang\cite{hahm90a}
together with an extension of the theory to include dissipative
trapped electron modes by Hahm and Tang.\cite{hahm90b}
First define some convenient factors used in these expressions:
$$ F_{\beta} = 1.0 \eqno{\tt zfbeta} $$
($F_{\beta}$ is a finite beta correction to be added later.)
$$ G = 1.2 \eqno{\tt zgtem} $$
(Note that $G$ is actually a function of magnetic shear and an average 
over the particle pitch angle.  
It will need to be generalized for noncircular geometry
and high $\beta_{pol}$.)
The following factor is bounded by unity as a condition for the validity
of the weak turbulence theory:
$$ H_1 = \min \left[ 4 \frac{2 \pi r}{R} \eta_e^2 
\left( \frac{R}{G L_n} \right)^3
        \left( \frac{R}{G L_n} - \frac{3}{2} \right)^2
        \exp \left( - \frac{2R}{G L_n} \right) , 1.0 \right]  
        \eqno{\tt zh1tem} $$
$$ K_1 = max [ 1.0, 1. / \sqrt{ G L_n / R} ]  \eqno{\tt zk1tem} $$
(Here $ K_1 \equiv K_M / K_L $ in the Hahm-Tang paper.)
$$ H_2 =  K_1 - \ln(K_1) - 1.0   \eqno{\tt zh2tem} $$
$$ H^{DTEM} = 3 (r/R)^3 (c_s/L_{T_e} \nu_{ei})^2 (1 + (T_i/T_e)(1+\eta_i))
                 \eqno{\tt zhdtem} $$
$$ H^{CTEM} = (2/3) H_1 H_2  \eqno{\tt zhctem} $$
$$ D_1 = \frac{8}{5 \pi} \frac{\rho_s^2 c_s}{L_n} \frac{q^2}{\hat{s}^2}
         \frac{T_e}{T_i} \frac{R^2}{L_n^2} 
         \frac{\sqrt{1+(T_i/T_e)(1+\eta_i)}}{1+5\eta_i/4}. \eqno{\tt zd1tem}
$$
Then the particle diffusivity produced by either the dissipative or
collisionless trapped electron mode is
$$ D^{TEM} = F_a^{DR} F_{\beta} \kappa^{c_{12}} {\tt zdtite}
               D_1 \min [ H^{DTEM} , c_{20} H^{CTEM} ].  \eqno{\tt zddtem} $$
The effective electron thermal diffusivity is
$$ \chi_e^{TEM} = F_e^{DR} F_{\beta} \kappa^{c_{12}} {\tt zdtite}
     ( D_1 / \eta_e ) \min [ 5 H^{DTEM} , c_{20} \frac{R}{G L_n} H^{CTEM} ].
     \eqno{\tt thdre} $$
The effective ion thermal diffusivity is
$$ \chi_i^{TEM} = F_i^{DR} F_{\beta} \kappa^{c_{12}} {\tt zdtite}
      2.75 \frac{1+1.93 \eta_i}{1+1.25 \eta_i} \frac{D_1}{\eta_i}
      \min [ H^{DTEM} , c_{20} H^{CTEM} ].  \eqno{\tt thdri} $$
The anomalous electron to ion energy exchange is
$$ \Delta^{DR} = F_{\Delta}^{DR} \frac{n_e T_e}{L_n^2} {\tt zdtite}
       D_1 \min [ H^{DTEM} , c_{20} H^{CTEM} ].  \eqno{\tt weithe} $$
Here $c_{20} = {\tt cthery(20)}$ is an adjustable coefficient
defaulted to 1.0.

\begin{verbatim}
c
      if ( lthery(5) .eq. 1 ) then
        zfbeta = 1.0
        zgtem  = 1.2
        zrgln  = zrmaj / ( zgtem * abs(zlne) )
        zh1tem(jz) = min ( 1.0,
     &    (8.*zpi*zrmin/zrmaj) * zetae**2 
     &     * zrgln**3 * ( zrgln - 1.5 )**2 * exp( - 2. * zrgln) )
        zk1tem(jz) = max ( 1.0, 1. / sqrt ( max(zepslon, 1./zrgln) ) )
        zh2tem = zk1tem(jz) - log( zk1tem(jz) ) - 1.0
        zhdtem(jz) = 3. * (zrmin/zrmaj)**3 * (zsound/(zlte*znuei))**2
     &    * ( 1.0 + (zti/zte)*(1.0+zetai) )
        zhctem(jz) = 2.0 * zh1tem(jz) * zh2tem / 3.0
        zd1tem(jz) = 8.0 * zrhos**2 * zsound * zq**2 * zte * zrmaj**2
     &    * sqrt ( 1.0 + (zti/zte)*(1.0+zetai) )
     &    / ( 5.0 * zpi * abs(zlne)**3 * zshat**2 * zti 
     &      * ( 1.0 + 5.0*zetai/4.0 ) )
c
        zddtem(jz) = fdr(1) * zfbeta * zelonf**cthery(12) * zdtite
     &    * zd1tem(jz) * min ( zhdtem(jz) , cthery(20) * zhctem(jz) )
        thdre(jz) = fdr(2) * zfbeta * zelonf**cthery(12) * zdtite
     &    * ( zd1tem(jz) / zetae )
     &    * min ( 5.0 * zhdtem(jz) , cthery(20) * zrgln * zhctem(jz) )
        thdri(jz) = fdr(3) * zfbeta * zelonf**cthery(12) *zdtite
     &    * 2.75 * ( (1.0+1.93*zetai)/(1.0+1.25*zetai) )
     &    * ( zd1tem(jz) / zetai )
     &    * min ( zhdtem(jz) , cthery(20) * zhctem(jz) )
        weithe(jz) = fdrint * zfbeta * zelonf**cthery(12) * zdtite
     &    * ( zne * zte / abs(zlne)**2 ) * zd1tem(jz)
     &    * min ( zhdtem(jz) , cthery(20) * zhctem(jz) )
c
      endif
\end{verbatim}

If $ {\tt lthery(6)} = 2 $, the Hahm-Tang toroidal collisionless trapped
electron drift wave model is calculated (IAEA, Washington, 1990),
as implemented by M. Redi and J. Cummings.

The following are the Hahm formulae for the anomalous fluxes:
\[
\Gamma_e = - \frac{C_{e}\epsilon}{G^{3}}
\left(\frac{R}{L_{n}}\right)^5
\left(\frac{R}{GL_{n}}-\frac{3}{2}\right)^{2}
exp\left(-\frac{2R}{GL_{n}}\right)
\frac{\eta_{e}\frac{T_{e}}{T_{i}}
\frac{q^{2}}{\hat{s}^{2}}((\frac{R}{L_{n}})^{\frac{1}{2}} -
ln(\frac{R}{L_{n}})^{\frac{1}{2}} - 1)}
{(1 + \frac{5}{4}\eta_{i})(1 + \frac{T_{i}}{T_{e}}(1 +
\eta_{i}))^{\frac{1}{2}} }
\frac{cT_{e}}{eB_{0}}
\frac{\rho_{s}}{L_{Te}}
\left(\frac{\partial n_{e}}{\partial r}\right)
\]

\[
Q_{e} = \left(\frac{R}{GL_{n}}\right)T_{e}\Gamma_{e}
\]
\[
Q_{i} = \left(\frac{85 \eta_{i} + 44}{20 \eta_{i} + 16}\right) T_{i}\Gamma_{e}
\]

$G = 1.2$ typically, but it is a function of $q(r)$ and $\hat{s}.$
$C_{e}$ is almost a constant $(C_{e} \sim 10).$
It is a very weak function of various parameters.

The anomalous electron$\rightarrow$ion energy exchange is
\[
\Delta^{DR} = \frac{T_{e}}{L_{n}}\Gamma_{e}
\]
 

The validity regime of the Hahm weak turbulence theory is
\[
2
\left(\frac{2\pi
r}{R}\right)^\frac{1}{2}\eta_{e}\left(\frac{R}{L_{n}G}\right)^\frac{3}{2}\left(\frac{R}{L_{n}G}
- \frac{3}{2}\right)exp\left(-\frac{R}{L_{n}G}\right) < 1 ,\] 
where $G\approx 1.2$ and r is the local minor radius. This condition
depends primarily on profile broadness.

When $\nu_{e}^{*}$ exceeds 0.1, these formulae are replaced by the
Hahm-Tang toroidal dissipative trapped electron drift wave model.
The new anomalous fluxes as predicted by the Hahm-Tang DTEM model are:

\[
\Gamma_{e}=-\frac{24}{5\pi}\epsilon^{3}\left(\frac{R}{L_{n}}\right)^{2}
           \left(\frac{c_{s}}{L_{Te}\nu_{ei}}\right)^{2}
           \frac{\frac{T_{e}}{T_{i}}\frac{q^{2}}{\hat s^{2}}}
           {(1+\frac{5}{4}\eta_{i})(1+\frac{T_{i}}{T_{e}}
           (1+\eta_{i}))^{\frac{3}{2}}}\frac{cT_{e}}{eB_{0}}
           \frac{\rho_{s}}{L_{n}}\left(\frac{\partial{n_{e}}}
           {\partial{r}}\right)
\]

\[
Q_{e}=5T_{e}\Gamma_{e}
\]

The diffusivity computed from the DTEM model is multiplied by the 
constant $c_{45}$ in an effort to make it match the predicted diffusivity
of the CTEM model at the $ \nu_{e}^{*}=0.1 $ threshold.  The formulae
for $Q_{i}$ and $\Delta^{DR}$ remain unchanged from the CTEM model.
 
If $ {\tt lthery(6)} = 3 $, the original CTEM formula is used, without any
transition to the dissipative regime.  If $ {\tt lthery(6)} = 4 $,
the Hahm CTEM model is used with no dissipative transition. 

If $ {\tt lthery(6)} = 5 $, the Kadomtsev-Pogutse DTEM model is used,
with no transition to CTEM.  The Kadomtsev-Pogutse DTEM formula is 
\[
{\hat D}_{te}=\epsilon^{3/2}\eta_{e}\frac{\omega_{e}^{*}}{k_{\perp}^{2}}
\frac{\omega_{e}^{*}}{\nu_{ei}}
\]
If $ {\tt lthery(6)} = 6 $, the Kadomtsev-Pogutse DTEM model is 
switched on with the Rewoldt transition.

If $ {\tt lthery(6)} = 7 $, the Hahm-Tang CTEM model is used with the
Rewoldt transition to the dissipative regime.

If $ {\tt lthery(6)} = 8 $, the Hahm-Tang CTEM model is used, with the 
Kadomtsev-Pogutse DTEM model for the dissipative regime.

The relevant coding is:

\begin{verbatim}
c
c  start calculating the diffusivities using the transport
c  formulae given in the comments paper.
c
c ......................................
c . the drift wave model calculations  .
c ......................................
c
      if ( lthery(6) .lt. 0 ) then
        zddtem(jz) = 0.0
        thdre(jz) = 0.0
        thdri(jz) = 0.0
      else
c
c  For Hahm CTEM model, calculate validity condition parameter.
c  If zvhahm < 1, plasma is in collisionless regime.
c
      if ( lthery(6) .eq. 2 .or. lthery(6) .eq. 4 .or. lthery(6) .eq. 7
     &    .or. (lthery(6) .eq. 8 .and. thnust(jz) .le. 0.1) ) then
        zvhahm = 2.0 * sqrt(2.0 * zpi * zep) * abs(zetae) *
     &       (zrmaj / abs(zln*1.2))**1.5 * (zrmaj/ abs(zln*1.2) - 1.5)
     &          * exp(-1.0 * zrmaj / abs( zln * 1.2 ))
      else
        zvhahm = 0.0
      endif
c
c  drift wave model options controlled by lthery(6).
c
c  The standard MMM drift wave model with Rewoldt transition.
c
      if ( lthery(6) .eq. 1 ) then
        zdte=(sqrt(zep)*zdiafr/zwn**2)
     &    * min ( cthery(20),
     &              cthery(21) * 0.1 / max(thnust(jz),zepslon) )
c
c  New formulae for Dte from Hahm DTEM model.
c
      elseif ( lthery(6) .eq. 2 ) then
        if ( thnust(jz) .gt. 0.1) then
          zdte = (24.0/5.0/zpi) * abs(zln*zdiafr/zwn) * (zq/zshat)**2
     &           * zep**3 * (zsound/zlte/znuei)**2 * (zte/zti) 
     &           / (1.0+1.25*abs(zetai))
     &           / (1.0+(zti/zte)*(1.0+abs(zetai)))**1.5 *
     &           (zrmaj/zln)**2 * abs(zrhos/zln) * cthery(45)
c
c  Hahm CTEM model.
c
        else
          zdte = (2.0**7/15.0) * (sqrt(abs(zrmaj/zln)) -
     &           log(sqrt(abs(zrmaj/zln))) -
     &           1.0) * (zrmaj/zln)**2 *
     &           abs(zvhahm**2/8.0/zpi/zetae) *
     &           (zte/zti) * ( (zq/zshat)**2
     &           / (1.0 + 1.25*abs(zetai)) /
     &           sqrt(1.0 + (zti/zte)*(1.0 + abs(zetai))) ) *
     &           abs(zln*zdiafr/zwn) * (zrhos/zlte)
        endif
c
c  The standard MMM drift wave model with no dissipative transition.
c
      elseif (lthery(6) .eq. 3 ) then
        zdte=(sqrt(zep)*abs(zdiafr)/zwn**2)
c
c  Hahm CTEM model only.
c
      elseif (lthery(6) .eq. 4 ) then
        zdte = (2.0**7/15.0) * (sqrt(abs(zrmaj/zln)) -
     &         log(sqrt(abs(zrmaj/zln))) -
     &         1.0) * (zrmaj/zln)**2 *
     &         abs(zvhahm**2/(8.0*zpi*zetae)) *
     &         (zte/zti) * ( (zq/zshat)**2 / (1.0 + 1.25*abs(zetai)) /
     &         sqrt(1.0 + (zti/zte)*(1.0 + abs(zetai))) ) *
     &         abs(zln*zdiafr/zwn) * abs(zrhos/zlte)
c
c  Kadomtsev-Pogutse DTEM model with no collisionless transition.
c
      elseif (lthery(6) .eq. 5 ) then
        zdte=zep**1.5 * zdiafr**2 * abs(zetae) / ( zwn**2 * znuei )
c
c  Kadomtsev-Pogutse DTEM model with "inverse" Rewoldt transition
c  to the collisionless regime.
c
      elseif (lthery(6) .eq. 6 ) then
        zdte=(zep**1.5 * zdiafr**2 * abs(zetae) / zwn**2 / znuei)
     &       * thnust(jz)/0.1
     &       * min ( cthery(20),
     &       cthery(21) * 0.1 / max(thnust(jz),zepslon) )
c
c  Hahm-Tang CTEM model with Rewoldt transition.
c
      elseif (lthery(6) .eq. 7) then
          zdte = (2.0**7/15.0) * (sqrt(abs(zrmaj/zln)) -
     &           log(sqrt(abs(zrmaj/zln))) -
     &           1.0) * (zrmaj/zln)**2 *
     &           abs(zvhahm**2/8.0/zpi/zetae) *
     &           (zte/zti) * ( (zq/zshat)**2 / (1.0 + 1.25*abs(zetai))
     &           / sqrt(1.0 + (zti/zte)*(1.0 + abs(zetai))) ) *
     &           abs(zln*zdiafr/zwn) * abs(zrhos/zlte)
     &           * min ( cthery(20),
     &           cthery(21) * 0.1 / max(thnust(jz),zepslon) )
c
c  Hahm-Tang CTEM model and Kadomtsev-Pogutse DTEM model.
c
      elseif (lthery(6) .eq. 8) then
        if (thnust(jz) .gt. 0.1) then
          zdte=zep**1.5 * zdiafr**2 * abs(zetae) / zwn**2 / znuei
     &         * cthery(46)
        else
          zdte = (2.0**7/15.0) * (sqrt(abs(zrmaj/zln)) -
     &           log(sqrt(abs(zrmaj/zln))) -
     &           1.0) * (zrmaj/zln)**2 *
     &           abs(zvhahm**2/8.0/zpi/zetae) *
     &           (zte/zti) * ( (zq/zshat)**2 / (1.0 + 1.25*abs(zetai))
     &           / sqrt(1.0 + (zti/zte)*(1.0 + abs(zetai))) ) *
     &           abs(zln*zdiafr/zwn) * abs(zrhos/zlte)
        endif
c
c  lthery(6)=0, so use original formula from Comments paper.
c
      else
        zdte = abs(sqrt(zep)*zdiafr/zwn**2)
     &    * min ( cthery(20), cthery(21) * abs(zdiafr) / znueff )
      endif
c
c  Hahm model: don't include beta ratio or elongation in drift wave D.
c
      zelfdr = zelong**cthery(12)
      zbprim = abs(zbeta/zlpr)
      zbc1   = abs(zshat/(1.7*zq**2*zrmaj))
      zbpbc1 = zbprim/zbc1
      zratio = (1.0+zbpbc1)/(1.0+zbpbc1**3)
      if ( lthery(6) .eq. 2 .or. lthery(6) .eq. 4 .or. lthery(6) .eq. 7
     &    .or. (lthery(6) .eq. 8 .and. thnust(jz) .le. 0.1) ) then
        zdd = zdte
        zddtem(jz) = zdd * fdr(1) * zdtite
      else
        zdd = zratio * zdte
        zddtem(jz) = zdd * fdr(1) * zelfdr * zdtite
      endif
c
c  if lthery(6) = 2,4,7 or 8 use Hahm formulae for Xe, Xi.
c
      if ( lthery(6) .eq. 2 .or. lthery(6) .eq. 4 .or. lthery(6) .eq. 7
     &    .or. (lthery(6) .eq. 8 .and. thnust(jz) .le. 0.1) ) then
        if ( lthery(6) .eq. 2 .and. thnust(jz) .gt. 0.1) then
          thdre(jz) = fdr(2) * 5.0 * abs(zdte/zetae) * zdtite
        else
          thdre(jz) = fdr(2) * abs(zrmaj/ (1.2*zln))
     &                * abs(zdte/zetae) * zdtite
        endif
        thdri(jz) = fdr(3) * (85.0*abs(zetai) + 44.0)
     &              /(20.0*abs(zetai) + 16.0) *
     &              abs(zdte*zne*zlti/zni/zlne) * zdtite
      else
        thdre(jz) = 2.5 * fdr(2) * zratio * zdte * zelfdr * zdtite
        thdri(jz) = 2.5 * fdr(3) * zratio * zdte * zelfdr * zdtite
      endif
c
c..end of trapped electron mode section
c
      endif
c
\end{verbatim}

The drift wave contributions cited in the Comments paper\cite{Comments}
have been moved to the end of the section on $\eta_i$ modes because
the original drift wave and $\eta_i$ models were intertwined.

%**********************************************************************c

\subsection{$\eta_i$ Modes}

First consider the threshold for $\eta_i$ modes given by $\eta_{i}^{th}$:
For the default model, we use a form given by Romanelli:\cite{Romanelli}
$$ \eta_{i}^{th} = max [ c_{11}, c_{11}+2.5( \frac{L_{n}}{R_{o}}-.2)]
 \eqno{\tt zetith} $$
(This makes quantitative
a similar suggestion which Dominguez and Waltz noted to be
important at the 1988 Sherwood theory meeting \cite{Sherwood}.
It constitutes
the most significant difference from the Comments paper to be incorporated
in the default model.)

If ${\tt lthery(8)} = 1$, the $\eta_i$ threshold by 
Mattor-Diamond\cite{matt89a}
and by Hahm-Tang\cite{hahm89a} [Eq(13a)] is used,
with coefficients controlled by ${\tt cthery(30)} = 1.0$ and
${\tt cthery(31)} = 1.9$:
$$ \eta_i^{th} = c_{30} + c_{31} | (1+T_i/T_e) ( L_n / L_s) |.
      \eqno{\tt zetith} $$

If ${\tt lthery(8)} = 2$, forms of the $\eta_i$ threshold suggested by
Dominguez-Rosenbluth\cite{domn89a} are used
$$ \eta_i^{th} = max [ c_{30}, c_{31} L_n / R, c_{32} L_n / ( R q ) ].
      \eqno{\tt zetith} $$
Recommended values are ${\tt cthery(30)} = 1.0$ (the default value)
and ${\tt cthery(31)} = 5.0$ 
(diferent from the default value ${\tt cthery(31)} = 1.9$)
or ${\tt cthery(32)} = 20.0$.

The function that turns the $\eta_i$ mode on or off $f_{ith}$
is the same as the form used by Dominguez and Waltz:
$$  f_{ith}=c_{6}\{1+\exp[-c_{7}(\eta_{i}-\eta_{i}^{th})]\}^{-1}
 \eqno{\tt zfith} $$
If ${\tt lthery(9)} = 1$, the argument of the exponent is normalized
by $\eta_{i}^{th}$ in order to avoid excessively large swings in 
$f_{ith}$ in regions of flat density profile where $\eta_{i}^{th}$
is very large:
$$ f_{ith}=c_{6}\{1+\exp[-c_{7}(\eta_{i}-\eta_{i}^{th})/\eta_{i}^{th}
    ]\}^{-1}   \eqno{\tt zfith}$$
If ${\tt lthery(9)} = 2$, a linear ramp form is used for $f_{ith}$:
\[ f_{ith} = \left\{ \begin{array}{ll}
0. & \mbox{ if $\eta_i < \eta_i^{th}$} \\
c_6 ( \eta_i / \eta_i^{th} - 1 ) / c_7
     & \mbox{if $\eta_i^{th} \leq \eta_i \leq (1+c_7) \eta_i^{th}$ } \\
c_6 & \eta_i > \mbox{if $ (1+c_7) \eta_i^{th}$ }  \end{array}  \right. \]
If the input value of $c_7$ is less than {\tt zepslon}, then 1.0 is used.

\begin{verbatim}
c
c..threshold eta_i
c
      if ( lthery(8) .lt. 0 ) then
        zetith = 0.0
c
      elseif ( lthery(8) .eq. 2 ) then
        zetith = max ( cthery(30), cthery(31) * zlnj / zrmaj,
     &    cthery(32) * zlnj / ( zrmaj * zq ) )
c
      elseif ( lthery(8) .eq. 1 ) then
        zetith = cthery(30) + cthery(31) * abs( ( 1.0 + zti / zte )
     &           * ( zlnj / max ( zlsh, zepslon ) ) )
c
      else
        zetith = max ( cthery(11)
     &    , cthery(11) + 2.5 * ( zlnj / zrmaj - 0.2 ) )
      endif
c
c..onset function
c
      if ( lthery(9) .lt. 0 ) then
        zfith = 0.0
      elseif ( lthery(9) .eq. 1 ) then
        zexdr = -cthery(7) * (zetai-zetith) / zetith
        zovfdr = min( max(zexdr,zlgeps), -zlgeps )
        zfith = cthery(6) / ( 1.0 + exp( zovfdr ) )
      else if ( lthery(9) .eq. 2 ) then
        z7 = cthery(7)
        if ( z7 .lt. zepslon ) z7 = 1.0
          zfith = 0.0
        if ( zetai .gt. (1. + z7)*zetith ) then
          zfith = cthery(6)
        else if ( zetai .gt. zetith ) then
          zfith = cthery(6) * (zetai-zetith) / ( z7 * zetith )
        endif
      else
        zexdr = -cthery(7) * (zetai-zetith)
        zovfdr = min( max(zexdr,zlgeps), -zlgeps )
        zfith = cthery(6) / ( 1.0 + exp( zovfdr ) )
      endif
\end{verbatim}

On 24 April 1990, it was suggested by R. Dominguez that the transport from
$eta_i$ modes sould be reduced by the factor 
$$ \exp^{-c_{34}(T_i/T_e - 1)^2}.  \eqno{\tt zftite} $$
By default, $c_{34} = 0.0$, so this factor has no effect.
Dominguez suggests the value $c_{34} = 1.0$.

\begin{verbatim}
      zftite = 1.0
      if ( abs(cthery(34)) .gt. zepslon )
     &  zftite = exp( -cthery(34) * ((zti/zte)-1.)**2 )
\end{verbatim}

%**********************************************************************c

\subsubsection{The Ottoviani-Horton-Erba ``Santa Barbara'' $\eta_i$ model}

The 1996 Ottoviani-Horton-Erba ITG/TEM mode transport model \cite{hortoncomm}
is selected by setting {\tt lthery(7) = 4}.
This model is derived from the assumption that the ion thermal diffusivity
will be related to the radial correlation length $\lambda_c$ and
correlation time $\tau_c$ for ITG-driven turbulence:
\[ \chi_i^{\rm ITG} \propto \frac{\lambda_c^2}{\tau_c} \]

The correlation length $\lambda_c$ is estimated from the large-scale
poloidal cutoff ($k_{\theta c}$) of the turbulent spectrum:
\[ \lambda_c \simeq \frac{1}{k_{\theta c}} \simeq
   \frac{q R \rho_s}{L_{T_i}} \]
where $\rho_s$ is a quantity that has units of length and is related
to the electron and ion Larmor radii ($\rho_e$ and $\rho_i$, respectively):
\[ \rho_s = \frac{ \sqrt{m_i T_e}}{eB} =
   \sqrt{ \frac{m_i}{m_e} } \rho_e =
   \sqrt{ \frac{T_e}{T_i} } \rho_i \]

In order to give the correct scaling with plasma current, the
correlation time $\tau_c$ is not simply the growth rate of the
fastest-growing ITG-driven instability.
Instead, $\tau_c$ is given by:
\[ \tau_c \simeq \frac{ \sqrt{ R L_{T_i} } }{v_i} \]
where $v_i$ is the ion thermal velocity.

Putting these estimates together, we find that the ion thermal diffusivity
is (in MKS units):
\[ \chi_i^{\rm ITG} = C_i \left( \frac{T_e}{eB} \right) q^2
   \left( \frac{\rho_i}{L_{T_i}} \right)
   max \left[ 0, \frac{R}{L_{T_i}} \right] \]
where $C_i$ is a constant to be calibrated (see below).

Notice that the OHE model does not include a critical gradient or any other
mechanism for ``shutting off'' the ITG-driven transport.
The derivation of this model assumes that the plasma is not near
marginal stability, so that neglecting the critical gradient is a
good approximation.
If the plasma is near marginal stability, however, the ITG-driven flux
will decrease linearly with the difference
\[ \frac{R}{L_{T_i}} - \frac{R}{L_{T_i}^{\rm crit}} \]
where $L_{T_i}^{\rm crit}$ denotes the critical ion temperature scale
length.
So, the ITG-driven flux will have the form:
$$ \chi_i^{\rm ITG} = C_i \left( \frac{T_e}{eB} \right) q^2
  \left( \frac{\rho_i}{L_{T_i}} \right)
  \left( \frac{R}{L_{T_i}} - 
                  c_{29} \frac{R}{L_{T_i^{\rm crit}}} \right)
  \eqno{\tt thigi} $$
where selecting {\tt cthery(29) = 0} would make the critical gradient term
vanish.
Ottoviani, {\it et al} do not provide a method for calculating the critical
$T_i$ gradient.
However, there are many models included in the BALDUR code for finding
threshold $\eta_i$'s, and any of these could be used.

With regards to the electron thermal diffusivity $\chi_e^{\rm ITG}$,
Ottoviani, {\it et al} \cite{hortoncomm} simplify the situation considerably
by assuming that
the electron heat energy will be conducted only by the trapped electrons.
Then, simplify further by assuming that $\chi_e^{\rm ITG}$ will be equal
to $\chi_i^{\rm ITG}$ multiplied by the trapped particle fraction
($\sqrt{\epsilon}$, where $\epsilon$ is the inverse aspect ratio r/R):
$$ \chi_e^{\rm ITG} = C_e \left( \frac{T_e}{eB} \right) q^2
  \sqrt{\epsilon}
  \left( \frac{\rho_i}{L_{T_i}} \right)
  max \left[ 0, \frac{R}{L_{T_i}} - \frac{R}{L_{T_i}^{\rm crit}} \right]
  \eqno{\tt thige} $$
where $C_e$ is a constant.
These expressions were calibrated against the medium power L-mode JET
discharge \#19649.
The optimum fit between the JETTO runs and the experimental data were
acheived when:
\[ C_i = C_e = 0.014 \]

The OHE model, in its given form, does not include particle transport,
as it assumes an adiabatic electron response.
However, in the BALDUR code, we are following four transport channels:
ion and electron thermal, and hydrogenic and impurity particles.
Therefore, we extend this model by obtaining a particle diffusivity
$D^{\rm ITG}$ that is equal to the ion thermal diffusivity $\chi_i^{\rm ITG}$:
$$ D^{\rm ITG} = \chi_i^{\rm ITG} = C_i \left( \frac{T_e}{eB} \right) q^2
  \left( \frac{\rho_i}{L_{T_i}} \right)
  max \left[ 0, \frac{R}{L_{T_i}} - \frac{R}{L_{T_i}^{\rm crit}} \right]
  \eqno{\tt zddig,zdzig} $$

The relevant coding is:
\begin{verbatim}

c     *
c     * The Ottoviani-Horton-Erba ITG/TEM model
c     *
      if ( lthery(7) .eq. 4) then
c
c        First, include the critical gradient effects
c
         if ( cthery(29) .gt. zepslon) then
            zdig = max(0.0,
     &         zrmaj/zslti(jz) - cthery(29)*zrmaj*zetith/zslne(jz))
         else
            zdig = max(0.0,zrmaj/zslti(jz))
            zdig = zdig**(1.5)
          end if
c
c        Calculate chi_i according to Ottoviani, et al
c
         zdig = (0.014)*zq*zq*zlari*zdig/zslti(jz)
         zdig = (zckb*zte*zdig)/(zce*zb)
c
c        Now find the actual diffusivities
c
         zddig(jz) = fig(1)*zdig*zftite*zelonf**cthery(14)
         zdzig(jz) = fig(1)*zdig*zftite*zelonf**cthery(14)
         thigi(jz) = fig(2)*zdig*zftite*zelonf**cthery(14)
         thige(jz) = fig(3)*sqrt(zep)*zdig*zftite*zelonf**cthery(14)
c
      end if

\end{verbatim}

Note that the coding includes two additional modifiers to the diffusivities:
{\tt zftite} and {\tt zelonf**cthery(14)}.
The first factor reduces
the transport to take account of $T_i/T_e$, but will be equal to unity
when {\tt cthery(34) = 0.0}.
The second factor modifies the diffusivity based upon the elongation.
Normally, {\tt cthery(14) = -4}, giving a $\kappa^{-4}$ dependence.

Good results have been obtained with the OHE model thermal transport
when the particle transport is given by BALDUR's empirical model
(described elsewhere).

%**********************************************************************c

The $\eta_i$-mode model by Kim and Horton\cite{kim92a} is implemented
when ${\tt lthery(7)} = 6$.
Note that the diffusivity from this model is normalized by 
$ \rho_s^2 c_s q^2 / L_n $ and it is computed only when $ L_n > 0 $.
The frequencies are normalized by $ c_s / L_n $.

\begin{verbatim}
c
c..no eta_i mode model if lthery(7) < 0
c
      if ( lthery(7) .lt. 0 ) then
c
        zddig(jz) = 0.0
        zdzig(jz) = 0.0
        thige(jz) = 0.0
        thigi(jz) = 0.0
c
c..Kim-Horton-Coppi eta_i mode model if lthery(7) =6
c
      elseif ( lthery(7) .eq. 6 ) then
c
        do jc=1,32
          iletai(jc) = 0
          zcetai(jc) = 0.0
        enddo
c
        iletai(3)  = 1
        iprint  = 0
        ztauie  = zti / zte
        zepsn  = zln / zrmaj
        zbetai = 2. * zcmu0 * zckb * zni * zti / zb**2
        zftrap = sqrt ( 2. * zrmin / ( zrmaj * ( 1. + zrmin / zrmaj )))
c
        zqprr  = 0.0
        if ( abs(cthery(38)) .lt. zepslon ) then
          zkyrho = 0.5
        else
          zkyrho = cthery(38)
        endif
        zkparl = zln / ( zq * zrmaj )
c
        zomegain  = 0.2
        zgammain  = 0.04
        if ( cthery(39) .gt. zepslon ) zgammain = cthery(39)
c
        zwnprin = 0.5
        zwnpdel = 0.1
        zerrabs = 0.01
        imaxfun = 20
c
        zxmr    = 10.0
        insig   = 7
        iitmax  = 20
        zdel    = 0.01
        zftest  = 1.e-7
c
        write (nprint,*)
     &    'sbrtn e3bsub not available at present'
c
cbate        call e3bsub ( iletai, zcetai, iprint
cbate     &   , zetai, zetae, ztauie, zepsn, zbetai
cbate     &   , zftrap, zqprr, zkyrho, zkparl, zomegain, zgammain
cbate     &   , zwnprin, zwnpdel, zerrabs, imaxfun
cbate     &   , zxmr, insig, iitmax, zdel, zftest
cbate     &   , zomegab, zgammab, zdifetai )
c
cahk        zgmeti(jz) = zgammab
c
        if ( zln .lt. zepslon ) then
c
          zddig(jz) = 0
          zdzig(jz) = 0
          thige(jz) = 0
          thigi(jz) = 0
c
        else
c
          znorm = zelonf**cthery(12) *
     &      zrhos**2 * zsound * zq**2 / zln
c
cahk          zddig(jz) = fig(1) * znorm * zdifetai
cahk          zdzig(jz) = fig(1) * znorm * zdifetai
cahk          thige(jz) = fig(2) * znorm * zdifetai
cahk          thigi(jz) = fig(3) * znorm * zdifetai
c
        endif
c
\end{verbatim}

%**********************************************************************c

The $\eta_i$ and trapped electron mode model 
by Weiland et al\cite{nord90a} is implemented when
${\tt lthery(7)}$ is set between 21 and 28.
When $ {\tt lthery(7)} = 21 $, only the hydrogen equations are used
(with no trapped electrons or impurities) to compute only the 
$ \eta_i $ mode.
When $ {\tt lthery(7)} = 22 $, trapped electrons are included,
but not impurities.
When $ {\tt lthery(7)} = 23 $, a single species of impurity ions is
included as well as trapped electrons.
When $ {\tt lthery(7)} = 24 $, the effect of collisions is included.
When $ {\tt lthery(7)} = 25 $, parallel ion (hydrogenic) motion and 
the effect of collisions are included.
When $ {\tt lthery(7)} = 26 $, finite beta effects and collisions are
included.
When $ {\tt lthery(7)} = 27 $, parallel ion (hydrogenic) motion, 
finite beta effects, and the effect of collisions are included.
When $ {\tt lthery(7)} = 28 $, parallel ion (hydrogenic and impurity) motion, 
finite beta effects, and the effect of collisions are included.
Finite Larmor radius corrections are included in all cases.
Values of {\tt lthery(7)} between 21 and 35 are reserved for extensions
of this Weiland model.

The mode growth rate, frequency, and effective diffusivities are
computed in subroutine {\tt etaw14}.
Frequencies are normalized by $\omega_{De}$ and diffusivities are
normalized by $ \omega_{De} / k_y^2 $.
The order of the diffusivity equations is 
$ T_H $, $ n_H $, $ T_e $, $ n_Z $, $ T_Z $, \ldots
Note that the effective diffusivities can be negative.

The diffusivity matrix $ D = {\tt difthi(j1,j2)}$
is given in the following form:
$$ \frac{\partial}{\partial t}
 \left( \begin{array}{c} n_H T_H  \\ n_H \\ n_e T_e \\ 
    n_Z \\ n_Z T_Z \\ \vdots
    \end{array} \right)
 = \nabla \cdot
\left( \begin{array}{llll} 
D_{1,1} n_H & D_{1,2} T_H & D_{1,3} n_H T_H / T_e \\
D_{2,1} n_H / T_H & D_{2,2} & D_{2,3} n_H / T_e \\
D_{3,1} n_e T_e / T_H & D_{3,2} n_e T_e / n_H & D_{3,3} n_e & \vdots \\
D_{4,1} n_Z / T_H & D_{4,2} n_Z / n_H & D_{4,3} n_Z / T_e \\
D_{5,1} n_Z T_Z / T_H & D_{5,2} n_Z T_Z / n_H & 
        D_{5,3} n_Z T_Z / T_e \\
 & \ldots & & \ddots
\end{array} \right)
 \nabla
 \left( \begin{array}{c}  T_H \\ n_H \\  T_e \\ 
   n_Z \\  T_Z \\ \vdots
    \end{array} \right)
$$
$$
 + \nabla \cdot
\left( \begin{array}{l} {\bf v}_1 n_H T_H \\ {\bf v}_2 n_H \\
   {\bf v}_3 n_e T_e \\
   {\bf v}_4 n_Z \\ {\bf v}_5 n_Z T_Z \\ \vdots \end{array} \right) +
 \left( \begin{array}{c} S_{T_H} \\ S_{n_H} \\ S_{T_e} \\
    S_{n_Z} \\ S_{T_Z} \\ \vdots
    \end{array} \right) $$
Note that all the diffusivities in this routine are normalized by
$ \omega_{De} / k_y^2 = 2 \rho_s c_s / ( R k_y ) $, 
convective velocities are normalized by 
$ \omega_{De} / R k_y^2 = 2 \rho_s c_s / ( R^2 k_y ) $,
and all the frequencies are normalized by $ \omega_{De} $.

For the moment, consider only the first impurity species.

Define the impurity density gradient scale length to be
\[ L_{nZ} \equiv Z n_Z / \frac{d Z n_Z}{d x}. \]
Then, it follows from charge neutrality that
\[ \frac{1}{L_{ne}} = \frac{ 1 - f Z }{L_{nH}} + \frac{ f Z }{L_{nZ}} \]
where $ f \equiv n_Z / n_e $ and $ n_e = n_H + Z n_Z $.
For this purpose, all the impurity species are lumped together as 
one effective impurity species and all the hydrogen isotopes are lumped 
together as one effective hydrogen isotope.

\begin{verbatim}
      elseif ( lthery(7) .ge. 21 .and. lthery(7) .le. 35 ) then
c
        do jc=1,32
          iletai(jc) = 0
          zcetai(jc) = 0.0
        enddo
c
        zcetai(11) = 1.0
c
c.. coefficient of k_parallel for parallel ion motion
c.. cthery(125) for v_parallel in strong ballooning limit
c.. in 9 eqn model
c
        zcetai(10) = cthery(123)
        zcetai(12) = cthery(125)
        zcetai(15) = cthery(124)
        zcetai(20) = cthery(119)
        zcetai(25) = cthery(122)
c
c  Use complex NAG routine f02gje - this is obsolete as of etaw17 
c
        iletai(10) = 0
c
        iprint = lthery(29) - 10
c
c       For up to 6 equations
c
        if ( lthery(7) .le. 23 ) then
          ieq = (lthery(7) - 20) * 2
        endif
c
        if ( lthery(7) .eq. 24 ) ieq = 7
        if ( lthery(7) .eq. 25 ) ieq = 8
        if ( lthery(7) .eq. 26 ) ieq = 9
        if ( lthery(7) .eq. 27 ) ieq = 10
        if ( lthery(7) .eq. 28 ) ieq = 11
c
        idim   = matdim
        idim2  = idim**2
c
c  Hydrogen species
c
        zthte  = zti / zte
        zepsnh = zlni / zrmaj
        if ( lthery(3) .ge. 3 ) zepsnh = zlnh / zrmaj
        zepsth = zlti / zrmaj
        zbetah = 2. * zcmu0 * zckb * zni * zti / zb**2
        zbetae = 2. * zcmu0 * zckb * zne * zte /   zb**2
c
        zepste = zlte / zrmaj
c
c  Impurity species (use only impurity species 1 for now)
c  assume T_Z = T_H throughout the plasma here
c
        ztz    = zti
        znz    = densimp(jz)
        zmass  = amassimp(jz)
        zimpz  = avezimp(jz)
        zimpz  = max ( zimpz, cthery(120) )
c
        ztzte  = zti / zte
        zepstz = zepsth
        zfnzne = znz / zne
        zmzmh  = zmass / amasshyd(jz)
        zbetaz = 2. * zcmu0 * zckb * znz * ztz / zb**2
c
c  compute zepsnz from zepsnh and zepsne or from zlnz directly
c
        if ( lthery(3) .ge. 4 ) then
          zepsnz = zlnz / zrmaj
        else
          zepsne = zlne / zrmaj
          zepsnz = zfnzne * zimpz * zepsnh
     &      / ( zfnzne * zimpz - 1.0 + zepsnh   / zepsne )
        endif
c
c  superthermal ions
c
c  zfnsne = ratio of superthermal ions to electrons
c  L_ns   = gradient length of superthermal ions
c  zepsne = L_ns / R
c
        zfnsne = max ( zfnsnea(jz), 0.0 )
c
        zgrdns = zrmaj * ( 1.0 / zlne
     &    - ( 1.0 - zimpz * zfnzne - zfnsne ) / zlnh
     &    - zimpz * zfnzne / zlnz ) / max ( zfnsne, 1.e-6 )
c
        zftrap = sqrt ( 2. * zrmin / ( zrmaj * ( 1. + zrmin / zrmaj )))
        if ( cthery(126) .gt. zepslon ) zftrap = cthery(126)
        if ( cthery(126) .lt. -zepslon )
     &       zftrap = abs(cthery(126))*zftrap
c
        if ( abs(cthery(38)) .lt. zepslon ) then
          zkyrho = 0.316
        else
          zkyrho = cthery(38)
        endif
        zkparl = zln / ( zq * zrmaj )
        zcetai(32) = cthery(128)
c
c...Define a local copy of normalized ExB shearing rate : pis
c
        zomegde(jz) = 2.0 * zkyrho * zsound / zrmaj 
        wexbs(jz) = cthery(129)*wexbs(jz)
        zwexb = wexbs(jz) / zomegde(jz) 
c
c
c  normalized gradients
c
        zgne = zrmaj / zlne
        zgnh = zrmaj / zlnh
        zgnz = zrmaj / zlnz
        zgns = zgrdns
        zgte = zrmaj / zlte
        zgth = zrmaj / zlti
        zgtz = zrmaj / zlti
c
c..diagnostic printout
c
cbate        write (nprint,199) zrmin, grdti(jz), zgth
 199    format ('#zz1',1p8e12.4)
c
c  Use NAG14 rather than IMSL routine - this is obsolete as of etaw17 
c
        iletai(6)  = 0
        if ( lthery(8) .eq. 20 ) iletai(7) = 1
c
cbate        if ( nstep .eq. lthery(26) ) then
cbate          mprint = nprint
cbate        else
cbate          mprint = 99
cbate        endif
c
        if ( iprint .gt. 0 ) write (nprint,191) jz, nstep
 191    format (/' jz = ',i4,'  nstep = ',i5,' call etaw17a')
c
        if (lthery(22) .eq. -1 ) then

          iletai(9) = 2  ! to compute only the effective diffusivities

          call weiland14 ( 
     &     iletai,   zcetai,   iprint,   ieq,      nprint,   zgne
     &   , zgnh,     zgnz,     zgte,     zgth,     zgtz,     zthte
     &   , ztzte,    zfnzne,   zimpz,    zmzmh,    zfnsne,   zbetae
     &   , zftrap,   znuhat,   zq,       zshat,    zkyrho,   zwexb
     &   , idim,     zomega,   zgamma,   zdfthi,   zvlthi,   zchieff
     &   , zflux,    imodes,   inerr )

c
        else if (lthery(22) .eq. -2 ) then

           call etaw14diff ( iletai, zcetai, iprint, ieq, nprint
     &    , zgne, zgnh, zgnz, zgte, zgth, zgtz, zthte, ztzte
     &    , zfnzne, zimpz, zmzmh, zfnsne, zbetae, zbetah, zbetaz
     &    , zftrap, znuhat, zq, zshat, zkyrho, zkparl, zwexb
     &    , idim, zomega, zgamma, zdfthi, zvlthi
     &    , zchieff, imodes, zperf, inerr )
c
        else if (lthery(22) .eq. -3 ) then

           call etaw14a ( iletai, zcetai, iprint, ieq, nprint
     &    , zgne, zgnh, zgnz, zgte, zgth, zgtz, zthte, ztzte
     &    , zfnzne, zimpz, zmzmh, zfnsne, zbetae, zbetah, zbetaz
     &    , zftrap, znuhat, zq, zshat, zkyrho, zkparl, zwexb
     &    , idim, zomega, zgamma, zdfthi, zvlthi
     &    , zchieff, imodes, zperf, inerr )
c
        else if (lthery(22) .eq. -4 ) then

           call etaw17diff ( iletai, zcetai, iprint, ieq, nprint
     &    , zgne, zgnh, zgnz, zgte, zgth, zgtz, zthte, ztzte
     &    , zfnzne, zimpz, zmzmh, zfnsne, zbetae, zbetah, zbetaz
     &    , zftrap, znuhat, zq, zshat, zelong, zkyrho, zkparl, zwexb
     &    , idim, zomega, zgamma, zdfthi, zvlthi
     &    , zchieff, imodes, zperf, inerr )

        else

           call etaw17a ( iletai, zcetai, iprint, ieq, nprint
     &    , zgne, zgnh, zgnz, zgte, zgth, zgtz, zthte, ztzte
     &    , zfnzne, zimpz, zmzmh, zfnsne, zbetae, zbetah, zbetaz
     &    , zftrap, znuhat, zq, zshat, zelong, zkyrho, zkparl, zwexb
     &    , idim, zomega, zgamma, zdfthi, zvlthi
     &    , zchieff, imodes, zperf, inerr )

        endif                   
c
c  Find maximum performance index
c
c      DO jm = 1, 5
c          write(54, '(5e12.4,A3,e12.4)') 
c     &    (zdfthi(jm,j1),j1=1,5) ,' | ', zvlthi(jm)
c      END DO
c      write(54,*)
c      write(54,'(5e12.4)') (zchieff(jm),jm=1,5)
c      write(54,*)
      zprfmx(jz) = 0.0
      do jm=1,ieq
        if ( zperf(jm) .gt. abs ( zprfmx(jz) ) )
     &    zprfmx(jz) = zperf(jm)
      enddo
c
c  Growth rates for diagnostic output
c    Note that all frequencies are normalized by \omega_{De}
c      consequently, trapped electron modes rotate in the positive
c      direction (zomega > 0) while eta_i modes have zomega < 0.
c
        zomegse(jz) = zomegde(jz) * 0.5 * zgne
        zkinvsq(jz) = zrhos**2 / zkyrho**2
c
        zgmitg(jz) = 0.0
        zomitg(jz) = 0.0
        zgmtem(jz) = 0.0
        zomtem(jz) = 0.0
        zgm2nd(jz) = 0.0
        zom2nd(jz) = 0.0
c
        do jm=1,ieq
          if ( zomega(jm) .gt. 0.0 ) then
            if ( zgamma(jm) .gt. zgmtem(jz) ) then
              zgmtem(jz) = zgamma(jm)
              zomtem(jz) = zomega(jm)
            else if ( zgamma(jm) .gt. zgm2nd(jz) ) then
              zgm2nd(jz) = zgamma(jm)
              zom2nd(jz) = zomega(jm)
            endif
          else
            if ( zgamma(jm) .gt. zgmitg(jz) ) then
              zgmitg(jz) = zgamma(jm)
              zomitg(jz) = zomega(jm)
            else if ( zgamma(jm) .gt. zgm2nd(jz) ) then
              zgm2nd(jz) = zgamma(jm)
              zom2nd(jz) = zomega(jm)
            endif
          endif
        enddo
c
c..convert growth rates and frequencies to (sec)^{-1}
c
        zgmitg(jz) = zgmitg(jz) * zomegde(jz)
        zomitg(jz) = zomitg(jz) * zomegde(jz)
        zgmtem(jz) = zgmtem(jz) * zomegde(jz)
        zomtem(jz) = zomtem(jz) * zomegde(jz)
        zgm2nd(jz) = zgm2nd(jz) * zomegde(jz)
        zom2nd(jz) = zom2nd(jz) * zomegde(jz)
c
c  compute diffusivity matrix
c
        znormd = zelonf**cthery(12) *
     &    2.0 * zsound * zrhos**2 / ( zrmaj * zkyrho )
        znormv = zelonf**cthery(12) *
     &    2.0 * zsound * zrhos**2 / ( zrmaj**2 * zkyrho )
c
c        call resetr ( difthi(1,1,jz), idim2, 0.0 )
c        call resetr ( velthi(1,jz), idim, 0.0 )
c
        do j1=1,matdim
          velthi(j1,jz) = 0.0
          do j2=1,matdim
            difthi(j1,j2,jz) = 0.0
          enddo
        enddo
c
c..full matrix form of model
c
        if ( lthery(8) .lt. 21 ) then
c
          if ( lthery(7) .eq. 21 ) then
            difthi(1,1,jz) = fig(3) * znormd * zdfthi(1,1)
            velthi(1,jz)   = fig(3) * znormv * zvlthi(1)
          elseif ( lthery(7) .eq. 22 ) then
            do j2=1,3
              difthi(1,j2,jz) = fig(3) * znormd * zdfthi(1,j2)
              difthi(2,j2,jz) = fig(1) * znormd * zdfthi(2,j2)
              difthi(3,j2,jz) = fig(2) * znormd * zdfthi(3,j2)
              difthi(4,j2,jz) = fig(1) * znormd * zdfthi(2,j2)
            enddo
              velthi(1,jz)    = fig(3) * znormv * zvlthi(1)
              velthi(2,jz)    = fig(1) * znormv * zvlthi(2)
              velthi(3,jz)    = fig(2) * znormv * zvlthi(3)
              velthi(4,jz)    = fig(1) * znormv * zvlthi(2)
          else
            do j2=1,4
              difthi(1,j2,jz) = fig(3) * znormd * zdfthi(1,j2)
              difthi(2,j2,jz) = fig(1) * znormd * zdfthi(2,j2)
              difthi(3,j2,jz) = fig(2) * znormd * zdfthi(3,j2)
              difthi(4,j2,jz) = fig(1) * znormd * zdfthi(4,j2)
            enddo
              velthi(1,jz)    = fig(3) * znormv * zvlthi(1)
              velthi(2,jz)    = fig(1) * znormv * zvlthi(2)
              velthi(3,jz)    = fig(2) * znormv * zvlthi(3)
              velthi(4,jz)    = fig(1) * znormv * zvlthi(4)
          endif
c
        endif
c
c  compute effective diffusivites for diagnostic purposes only
c
        zddig(jz) = fig(1) * znormd * zchieff(2)
        zdzig(jz) = fig(1) * znormd * zchieff(4)
        thige(jz) = fig(2) * znormd * zchieff(3)
        thigi(jz) = fig(3) * znormd * zchieff(1)
     &  + fig(3)*znormd*zchieff(5) * densimp(jz)/densi(jz)*cthery(130)
c
c
c..transfer from diffusivity to convective velocity
c
        if ( lthery(27) .gt. 0 ) then
c
          if ( thigi(jz) .lt. 0.0 ) then
            velthi(1,jz) = velthi(1,jz) - thigi(jz) / zlti
            thigi(jz) = 0.0
            do j2=1,4
              difthi(1,j2,jz) = 0.0
            enddo
          endif
c
          if ( zddig(jz) .lt. 0.0 ) then
            velthi(2,jz) = velthi(2,jz) - zddig(jz) / zlnh
            zddig(jz) = 0.0
            do j2=1,4
              difthi(2,j2,jz) = 0.0
            enddo
          endif
c
          if ( thige(jz) .lt. 0.0 ) then
            velthi(3,jz) = velthi(3,jz) - thige(jz) / zlte
            thige(jz) = 0.0
            do j2=1,4
              difthi(3,j2,jz) = 0.0
            enddo
          endif
c
          if ( zdzig(jz) .lt. 0.0 ) then
            velthi(4,jz) = velthi(4,jz) - zdzig(jz) / zlnz
            zdzig(jz) = 0.0
            do j2=1,4
              difthi(4,j2,jz) = 0.0
            enddo
          endif
c
        else
c
c  Note that the gradient scale lengths
c  zlti, zlnh, zlte, and zlnz must all be non-zero
c
        velthi(1,jz) = velthi(1,jz)
     &     + cthery(111) * thigi(jz) / zlti
        velthi(2,jz) = velthi(2,jz)
     &     + cthery(112) * zddig(jz) / zlnh
        velthi(3,jz) = velthi(3,jz)
     &     + cthery(113) * thige(jz) / zlte
        velthi(4,jz) = velthi(4,jz)
     &     + cthery(114) * zdzig(jz) / zlnz
c
c..alter the effective diffusivities 
c  if they are used for more than diagnostic purposes
c
        if ( lthery(8) .gt. 20 ) then
          thigi(jz) = ( 1.0 - cthery(111) ) * thigi(jz)
          zddig(jz) = ( 1.0 - cthery(112) ) * zddig(jz)
          thige(jz) = ( 1.0 - cthery(113) ) * thige(jz)
          zdzig(jz) = ( 1.0 - cthery(114) ) * zdzig(jz)
        endif
c
        do j1=1,4
          ii = 110 + j1
          do j2=1,4
            difthi(j1,j2,jz) = ( 1.0 - cthery(ii) ) * difthi(j1,j2,jz)
          enddo
        enddo
c
        endif
c
c---:----1----:----2----:----3----:----4----:----5----:----6----:----7-c
c
      if ( ( lprint .gt. 0 .or. nstep .eq. lthery(26) )
     &  .or. ( zlastime .lt. cthery(88) .and. cthery(88) .le. time ) )
     &  then
c
      if ( jz .eq. jzmin ) then
c
        write (nprint,150) nstep, znormd, znormv, zkyrho, zmzmh
     &    , zcetai(10), zcetai(12), zcetai(15), zcetai(20)
     &    , zcetai(32), ieq
 150    format (/' Diagnostic output from sbrtn theory at nstep',i6
     &    ,/' znormd =',1pe11.3,/' znormv =',1pe11.3
     &    ,/' zkyrho =',1pe11.3
     &    ,/' zmzmh  =',1pe11.3
     &    ,/' zcetai(10) =',1pe11.3,2x,'parallel ion motion'
     &    ,/' zcetai(12) =',1pe11.3,2x,'par. ion motion (9 eqns)'
     &    ,/' zcetai(15) =',1pe11.3,2x,'collisions'
     &    ,/' zcetai(20) =',1pe11.3,2x,'finite beta'
     &    ,/' zcetai(32) =',1pe11.3
     &    ,/' ieq =',i5)
c
        write (nprint,154)
 154    format (
     &    /t4,'radius',t15,'zgne',t26,'zgnh',t37,'zgnz'
     &    ,t48,'zgte',t59,'zgth',t70,'zthte',t81,'zfnzne'
     &    ,t92,'zimpz',t103,'zfnsne',t114,'zftrap',t125,'#e')
c
        write (nprint,155)
 155    format (t4,'radius',t15,'zchieff'
     &    ,t59,'thigi',t70,'zddig',t81,'thige',t92,'zdzig'
     &    ,t125,'#c')
c
        write (nprint,156)
 156    format (
     &    t4,'radius',t15,'zq',t26,'zshat',t37,'znuhat'
     &    ,t48,'zbetae',t59,'zbetah',t70,'zbetaz',t81,'zkparl'
     &    ,t92,'zelong',t125,'#b')
c
      endif
c
        write (nprint,165) zrmin, zgne, zgnh, zgnz, zgte, zgth
     &    , zthte, zfnzne, zimpz, zfnsne, zftrap
 165    format (1p11e11.3,4x,'#e')
c
        write (nprint,166) zrmin, (zchieff(jm),jm=1,4)
     &    , thigi(jz), zddig(jz), thige(jz), zdzig(jz)
 166    format (1p9e11.3,4x,'#c')
c
        write (nprint,167) zrmin, zq, zshat
     &    , znuhat, zbetae, zbetah, zbetaz, zelong, zkparl
 167    format (1p9e11.3,4x,'#b')
c
      endif
c
\end{verbatim}

%**********************************************************************c

xoAn early form  of the Hamaguchi-Horton theory\cite{hama89a}
of $\eta_i$ mode transport
is implemented when ${\tt lthery(7)} = 2$.  Here
$$ D_{\eta_i} = \frac{\rho_s^2 c_s}{L_n} (\eta_i - \eta_i^{th})
        \exp [ - \min ( c_{35} L_n , c_{36} L_{Ti} ) / L_s ]
   f_{ith} \exp^{-c_{34}(T_i/T_e - 1)^2} \kappa^{c_{12}}
   q^{c_{37}}                          \eqno{\tt zdetai} $$
$$ D^{IG} = fig(1) D_{\eta_i}  \eqno{\tt zddig(jz)} $$
$$ \chi_e^{IG} = F_e^{IG} D_{\eta_i}   \eqno{\tt thige} $$
$$ \chi_i^{IG} = F_i^{IG} D_{\eta_i}.  \eqno{\tt thigi} $$
The recommended values are $ c_{35} = 5.0 $ and $ c_{36} = 4.0 $.
(Note: at present, they are defaulted to 0.0).

\begin{verbatim}
      elseif ( lthery(7) .eq. 2 ) then
        zdetai = zfith * zftite * zelonf**cthery(12) * zq**cthery(37)
     &    * abs( zrhos**2 * zsound / zln ) * ( zetai - zetith )
     &    * exp ( - min (cthery(35) * abs(zln), cthery(36) * abs(zlti))
     &    / abs(zlsh) )
        zddig(jz) = fig(1) * zdetai
        zdzig(jz) = fig(1) * zdetai
        thige(jz) = fig(2) * zdetai
        thigi(jz) = fig(3) * zdetai
\end{verbatim}

%**********************************************************************c

The Lee and Diamond 1986 theory\cite{lee86a} of transport due to
ion temperature gradient driven turbulence ($\eta_i$-mode) is used 
when ${\tt lthery(7)} = 1$.
When $\eta_i > \eta_i^{th}$, the effective ion thermal diffusivity is
given by equation (92) in the Lee-Diamond paper
$$ \chi_i^{IG} = 0.4 F_i^{IG} f_{ith}
   \left[ \frac{\pi}{2} \ln ( 1+\eta_i) \right]^2
   \left[ \frac{1+\eta_i}{T_e/T_i} \right]^2 \frac{\rho_s^2 c_s}{L_s}
   \kappa^{c_{12}}.
   \eqno{\tt thigi} $$
The effective electron thermal diffusivity results from the 
dissipative trapped electron response to $\eta_i$-mode turbulence
given by equation (96) in the Lee-Diamond paper
$$ \chi_e^{IG} = 3.394 F_e^{IG} f_{ith} \epsilon^{1.5}
   \left[ \frac{\pi}{2} \ln ( 1+\eta_i) \right]^4
   \left[ \frac{1+\eta_i}{T_e/T_i} \right]^3
   \frac{1}{[\nu_{ei},c_{33} \epsilon v_e / q R ]_{max}}
   \frac{c_s^2 \rho_s^2}{L_s^2} \kappa^{c_{12}}.   \eqno{\tt thige} $$
Note that there is a cutoff when the effective electron-ion
collisionality becomes larger than the electron transit frequency,
with adjustable coefficient $c_{33}$.
\begin{verbatim}
c
c..Lee-Diamond theory
c
      elseif ( lthery(7) .eq. 1 ) then
c
        zddig(jz) = 0.
        zdzig(jz) = 0.
c
        if ( zetai .gt. 0. ) then
c
          thigi(jz) = 0.4 * fig(3) * zfith * zftite *
     &      zq**cthery(37) *
     &      ( (zpi/2.) * log(1.+abs(zetai)) )**2
     &      * zelonf**cthery(12) *
     &      ( ( 1. + abs(zetai) ) / ( zte / zti ) )**2 *
     &      zrhos**2 * zsound / abs(zlsh)
c
          thige(jz) = 3.3942 * fig(2) * zfith * zep**1.5 * zftite *
     &      zq**cthery(37) *
     &      ( (zpi/2.) * log(1.+abs(zetai)) )**4
     &      * zelonf**cthery(12) *
     &      ( ( 1. + abs(zetai) ) / ( zte / zti ) )**3 *
     &      zrhos**2 * zsound**2 / ( zlsh**2 *
     &        max ( znuei, cthery(33) * zep*zvthe/(zq*zrmaj) ) )
c
        endif
c
      endif
c
\end{verbatim}

The default models ({\tt lthery(5) = 0} for trapped electron modes
and {\tt lthery(7) = 0} for $\eta_i$ modes)
follow the Comments paper\cite{Comments}.

From the Comments paper\cite{Comments} (denoting its equations
as (C13) \ldots to distinguish them form
equation numbers in the present document), we then
compute (C13), (C12), (C11),
$\beta '/\beta_{c1}'$,
$(1+\beta '/\beta_{c1}')/[1+(\beta '/\beta_{c1}')^{3}]$, and
(C7).  We then use (C8)
to compute $Q_{e}^{DR}L_{Te}/(n_{e}T_{e})$
and (C10) to compute $Q_{i}^{DR}L_{Ti}/(n_{i}T_{i})$.  Algebraic
notation for these formulas is
$$  {\hat D}_{i}=\frac{\omega_{e}^{*}}{k_{\perp}^{2}}
  \left(\frac{2T_{i}}{T_{e}}
     \frac{L_{ni}}{L_{T_{i}}}\frac{L_{ni}}{R_{o}}\right)^{1/2}
     \kappa^{c_{12}} \eqno{\tt zdi} $$
$$  {\hat D}_{te}=\epsilon^{1/2}\frac{\omega_{e}^{*}}{k_{\perp}^{2}}
  \left[ c_{20}, c_{21}\frac{\omega_{e}^{*}}{\nu_{eff}}\right]_{min}
 \eqno{\tt zdte} $$
If $ {\tt lthery(6)} = 1 $, then use the form
for collisionless trapped electron modes suggested by Greg Rewoldt
\[
  {\hat D}_{te}=\epsilon^{1/2}\frac{\omega_{e}^{*}}{k_{\perp}^{2}}
  \left[ c_{20}, c_{21}\frac{0.1}{\nu_e^*}\right]_{min}
\]
Defining
$$ \beta ' = \beta /L_{p} \eqno{\tt zbprim} $$
$$ \beta_{c1}'=\hat{s}/(1.7 q^{2}R_{o}) \eqno{\tt zbc1} $$
$$ \beta_{1}=\beta '/\beta_{c1}' \eqno{\tt zbpbc1} $$
gives
$$  D^{DR} =
    \frac{1 + \beta_{1}}{1 + \beta_{1}^{3}}
    \left(1 - \frac{f_{ith}}{c_{23}+\nu_e^*}\right)
    \hat{D}_{te} \eqno{\tt zdd} $$

Quantities need to compute thermal diffusivities are approximated as
$$ D_{a}^{DR}=D^{DR}F_{a}^{DR} \kappa^{c_{12}} {\tt zdtite}
   \eqno{\tt zddtem} $$
$$  Q_{e}^{DR}\frac{L_{T_{e}}}{n_{e}T_{e}} = \frac{5}{2}
  \frac{1 + \beta_1}{1 + \beta_1^{3}}\hat{D}_{te}
  F_{e}^{DR} \kappa^{c_{12}} {\tt zdtite} \eqno{\tt thdre} $$
$$  Q_{i}^{DR}\frac{L_{T_{i}}}{n_{i}T_{i}}
  = \frac{5}{2} \hat{D}_{te}
  \frac{1 + \beta_{1}}{1 + \beta _{1}^{3}}
 F_{i}^{DR} \kappa^{c_{12}} {\tt zdtite} \eqno{\tt thdri} $$
The anomalous electron$\rightarrow$ion energy exchange
is also computed in subroutine {\tt theory} for transfer
to BALDUR subroutine {\tt trcoef}.
$$  \Delta^{DR} = (.89 - .54\eta_{i} -.6\beta '/\beta_{c1}')D^{DR}
  \frac{n_{e}T_{e}}{L_{ni}^{2}}F_{\Delta}^{DR} {\tt zdtite}
   \eqno{\tt weithe} $$

In this default model, the ``drift wave'' coefficients $F_{e}^{DR}$
and $F_{i}^{DR}$ are used to control
even the $\eta_i$ contributions to
the electron and ion thermal flux effective diffusivities
when {\tt lthery(7) = 0}:
$$  Q_{e}^{DR}\frac{L_{T_{e}}}{n_{e}T_{e}} =  c_{19} f_{ith}
  \frac{1 + \beta '/\beta_{c1}'}{1 + (\beta '/\beta_{c1}')^{3}}\hat{D}_{te}
  F_{e}^{DR}  \kappa^{c_{12}} {\tt zdtite} \eqno{\tt thige} $$
(The default for $c_{19} = - 3/2$.)
$$  Q_{i}^{DR}\frac{L_{T_{i}}}{n_{i}T_{i}}
  = \frac{5}{2}(f_{ith}\hat{D}_{i})
  \frac{1 + \beta_{1}}{1 + \beta _{1}^{3}}
   F_{i}^{DR} \kappa^{c_{12}} {\tt zdtite} \eqno{\tt thigi} $$

The relevant coding is:

\begin{verbatim}
c
c..original default theory
c
      if ( lthery(6) .lt. 2 .and.
     &  ( lthery(5) .eq. 0 .or. lthery(7) .eq. 0 ) ) then
c
        zdi=(zdiafr/zwn**2)*zln*sqrt(2*zti/(zte*zlti*zrmaj))
c
        if ( lthery(6) .eq. 1 ) then
          zdte=(sqrt(zep)*abs(zdiafr)/zwn**2)
     &      * min ( cthery(20),
     &                cthery(21) * 0.1 / max(thnust(jz),zepslon) )
        else
          zdte=(sqrt(zep)*abs(zdiafr)/zwn**2)
     &      * min ( cthery(20), cthery(21) * abs(zdiafr) / znueff )
        endif
c
        zbprim = abs(zbeta/zlpr)
        zbc1   = abs(zshat)/(1.7*zq**2*zrmaj)
        zbpbc1 = zbprim/zbc1
        zratio = (1.+zbpbc1)/(1.+zbpbc1**3)
        zdd    = zratio*(1.-zfith/(cthery(23)+thnust(jz)))*zdte
          zelfdr=zelonf**cthery(12)
        zddtem(jz) = zdd * fdr(1) * zelfdr * zdtite
c
c  trapped electron mode contributions
c
        if ( lthery(5) .eq. 0 ) then
          thdre(jz) = 2.5 * fdr(2) * zratio * zdte * zelfdr * zdtite
          thdri(jz) = 2.5 * fdr(3) * zratio * zdte * zelfdr * zdtite
        endif
c
c  eta_i mode contributions
c
        if ( lthery(7) .eq. 0 ) then
          thige(jz) = fdr(2) * cthery(19) * zfith * zratio
     &      * zdte * zelfdr * zftite
          thigi(jz) = 2.5 * fdr(3) * zfith * zratio
     &       * zdi * zelfdr * zftite
        endif
c
c..original expression for anomalous energy interchange
c
          weithe(jz) = fdrint * (0.89-0.54*abs(zetai)-0.6*zbpbc1)
     &      * zdd * zne * zte * zelfdr / zln**2
c
      endif
c
c  T.S. Hahm's formula for energy exchange.
c
      if ( lthery(6) .eq. 2 .or. lthery(6) .eq. 4 .or. lthery(6) .eq. 7
     &    .or. (lthery(6) .eq. 8 .and. thnust(jz) .le. 0.1) ) then
c
          weithe(jz) = abs((zte/zln)*abs(zdte*zne/zlne))*fdrint
c
      endif
c
\end{verbatim}

%**********************************************************************c

\subsection{Trapped Ion Modes}

\begin{verbatim}
c
      zdti(jz) = 0.
c
\end{verbatim}

%**********************************************************************c

\subsection{Rippling}

For rippling modes, we compute
$$  D_{\nabla \eta}=\left( \frac{E_{o}L_{s}}{B_{o}L_{\sigma}}\right) ^{4/3}
   \left( \frac{r^{2}L_{s}^{2}Z_{imp}^{2}\nu_{ii}}
   {25v_{i}^{2}}\right) ^{1/3} \eqno{\tt zdgret} $$
Before subtracting convective energy losses,
the diffusivities are approximated as
$$ D^{RM}_{a}=D_{\nabla \eta}F^{RM}_{a} \eqno{\tt zdrm(jz)} $$
$$  Q_{e}^{RM}\frac{L_{T_{e}}}{n_{e}T_{e}} = D_{\nabla \eta}F_{e}^{RM}
 \eqno{\tt thrme} $$
$$  Q_{i}^{RM}\frac{L_{T_{i}}}{n_{i}T_{i}} = D_{\nabla \eta}F_{i}^{RM}
 \eqno{\tt thrmi} $$
(Note that appropriate choices of input will zero the diffusivities
and make the numerical calculation of the resultant fluxes
purely convective.  However, there is no known numerical necessity
for doing this, and it will lead to the above-mentioned small
difficulties for plasmas with electrons due to impurities or
fast ions.)

If ${\tt lthery(11)} = 1$, the form for $\chi_e$ suggested by
Hahm et al\cite{hahm87a} [Eq(53)] is used
$$ \chi_e = (m_e/M_i)^{1/6} (L_T / L_n) D_{\nabla \eta}.
     \eqno{\tt thrme(jz)} $$

The relevant coding is:

\begin{verbatim}
c ......................
c . the rippling model .
c ......................
c
c
c  note that a typsetting error in the comments paper
c  for the resistivity scale height is corrected here
c
      zdgret=((zrmin*zlsh*cthery(5))**2*znuii/(25*zvthi**2))**(1./3.)
c
      zeloop=zvloop/(2*zpi*zrmaj)
c
      zdgret=zdgret*abs(zeloop*zlsh/(zb*zlsig))**(4./3.)
        zelfrm=zelonf**cthery(13)
      zdrm(jz) = zdgret*frm(1)*zelfrm
c
      if ( lthery(11) .eq. 1 ) then
        thrme(jz) = zdgret*frm(2)*zelfrm
     &    * abs ( (zcme/(zcmp*zai))**(1./6.) * abs(zlte/zln) )
      else
        thrme(jz)=zdgret*frm(2)*zelfrm
      endif
c
      thrmi(jz)=zdgret*frm(3)*zelfrm
c
\end{verbatim}

%**********************************************************************c

\subsection{Resistive Ballooning}

\subsubsection{Analytic correction to resistive ballooning modes}

Apply a correction to the resistive ballooning mode diffusivities in an
effort to make the finite difference solution more closely match the 
local analytic solution.
Consider the diffusion equation in a plane slab region
region with uniform flux and locally no sources.
Integrate once to get
\[ \chi \partial T / \partial x = - F_0. \]
Assuming the density profile is locally flatter than the temperature
profile, the thermal diffusivity may be approximated by
\[ \chi = \chi_0 |\partial T / \partial x | a \sqrt{T_0} / T^{3/2}. \]
Then, locally, the analytic solution between $x=a$ and $x=b$ is
\[ 16 \chi_0 a T_0 [ (T_a/T_0)^{1/4} - (T_b/T_0)^{1/4} ]^2 
    = F_0 (b-a)^2. \]
In finite difference form, the solution is
\[ \chi_0 a \sqrt{T_0} \left( \frac{2}{T_a + T_b} \right)^{3/2}
   \left( \frac{T_a - T_b}{b-a} \right)^2 = F_0. \]
In order to make $T_a$, $T_b$, $a$, $b$, and $F_0$ the same
we need to multiply the finite difference diffusivity by
$$ {\tt zrbfac} = \frac{8 (T_a + T_b)^{3/2}}{\sqrt{2}} 
  \left[ \frac{ T_a^{1/4} - T_b^{1/4} }{T_a - T_b} \right]^2.
  \eqno{\tt zrbfac} $$
This factor is controlled by {\tt cthery(45)} by setting
$$ {\tt zrbfac} = 1.0 + {\tt cthery(45)} * ( {\tt zrbfac} - 1.0 ). $$
By default $({\tt cthery(45)} = 0.0)$, there is no correction.
For the full normal correction, set ${\tt cthery(45)} = 1.0$.
\begin{verbatim}
c
      zte2  = 0.5 * ( tekev(jz) + tekev(jz+1) )
      zte2m = 0.5 * ( tekev(jz-1) + tekev(jz) )
c
      if ( abs( cthery(45) * (zte2m - zte2) ) .gt. 
     &         abs( zepslon * (zte2m + zte2) ) ) then
        zrbfac = 1.0 + cthery(45) * ( -1.0 +
     &    ( (zte2m**0.25 - zte2**0.25)
     &    / (zte2m - zte2) )**2
     &    * (zte2m + zte2)**1.5 * 8.0 / sqrt(2.0) )
      else
        zrbfac = 1.0
      endif
c
\end{verbatim}

For the original resistive ballooning mode model\cite{Comments} 
$({\tt lthery(13)} = 0)$,  
we compute (C20), (C21), (C19), and
then use (C18) to set up the thermal diffusivities.  In algebraic notation,
the required equations are:
$$  \chi_{e,res.ball}=\frac{3v_{e}\eta}{2\mu_{o}(2q)^{1/2}v_{A}}
         \left(\frac{\beta_{\theta}\epsilon^{2}L_{s}}{L_{p}}\right)^{3/2}
         \eqno{\tt zoldrb} $$
$$  \Lambda_{S} = \frac{4}{3\pi}\ln (\beta^{-1/2}R_{o}v_{A}\mu_{o}/\eta )
 \eqno{\tt thlamb} $$
$$  f_{*}=\left( \frac{\mu_{o}\omega_{ci}\rho_{i}^{3}}
  {\eta \beta q^{2}L_{ni}} \right) ^{2} \eqno{\tt zfstar} $$
$$  \chi^{RB} =
  (1 + c_{42} f_{*})^{-c_{43}}
  \Lambda_{S}^{2}\, \chi_{e,res.ball} \kappa^{c_{14}} \eqno{\tt zxrb} $$
where $c_{42} = 1.$ and $c_{43} = 1/4$.

To accomodate users who might want it, we
include an option for an associated particle
and ion energy diffusion coefficients
proportional to that for electron energy; but we note
that the present theory predicts much smaller values for the related fluxes.
Thus, we compute
$$ D_{a}^{RB}=\chi^{RB}F^{RB}_{a} \eqno{\tt zdrb(jz)} $$
$$ Q_{e}^{RB}\frac{L_{Te}}{n_{e}T_{e}}=\chi^{RB}F_{e}^{RB}
 \eqno{\tt thrbe} $$
$$ Q_{i}^{RB}\frac{L_{Ti}}{n_{i}T_{i}}=\chi^{RB}F_{i}^{RB}
 \eqno{\tt thrbi} $$
where $F^{RB}_{e}$ is identical to $F^{RB}$ of
the comments paper.

The relevant coding is:

\begin{verbatim}
c ..................................
c . the resistive ballooning model .
c ..................................
c
      zfstar=((zcmu0*zgyrfi*zlari**3)/(zresis*zbeta*zq**2*zln))**2
      zfdias = ( 1. + abs ( cthery(42) * zfstar ) )**( - cthery(43) )
c
      if ( lthery(13) .eq. 0 .or. lthery(13) .eq. 2 ) then
c
       zoldrb= (3.*zvthe*zresis)/(2.*zcmu0*sqrt(2*zq)*zvalfv)
      zoldrb = zoldrb*abs(zbetap*zep**2*zlsh/zlpr)**1.5
      zls    = 4.*log(zrmaj*zvalfv*zcmu0/(zresis*sqrt(zbeta)))/(3*zpi)
      thlamb(jz) = zls
      zxrb = zls**2 * zoldrb * abs(zfdias)
        zelfrb = zelonf**cthery(14)
      if ( lthery(13) .ne. 2 )
     &   zdrb(jz) = zxrb * frb(1) * zelfrb * zrbfac
      thrbe(jz) = zxrb * frb(2) * zelfrb * zrbfac
      thrbi(jz) = zxrb * frb(3) * zelfrb * zrbfac
c
      endif
\end{verbatim}

The 1989 resistive ballooning mode model by Carreras and Diamond \cite{carr89a} 
is selected by setting
$({\tt lthery(13)} \ne 0)$,
$$ D^{RB} = F_a^{RB}
  \frac{\beta R_0^2 q^2}{\sqrt{2} L_p R_c \hat{S}^{c_{49}} }
  \frac{r^2}{\tau_R} \Lambda^2 f_{dia} \kappa^{c_{14}}. \eqno{\tt zdrb} $$
The electron thermal diffusivity is a sum of the magnetic flutter and
$\bf E \times B$ contributions, respectively.
$$ \chi_e^{RB} = \left[ F_e^{RB}
  \frac{1}{2^{13/6} \langle n \rangle ^{2/3} S^{2/3} \hat{S} }
  \left( \beta \frac{R_0^2}{L_p R_c} q^2 \right)^{4/3}
  \frac{v_e r^2}{R_0}  \Lambda ^{4/3} \\
   + c_{44} \frac{\beta R_0^2 q^2}{\sqrt{2} L_p R_c \hat{S}^{c_{49}} }
            \frac{r^2}{\tau_R} \Lambda^2 \right] f_{dia}. \eqno{\tt thrbe} $$
For the moment, the ion thermal diffusivity will be taken to be
an adjustable fraction of the particle diffusivity
$$ \chi_i^{RB} = F_i^{RB}
  \frac{\beta R_0^2 q^2}{\sqrt{2} L_p R_c \hat{S}^{c_{49}} }
  \frac{r^2}{\tau_R} \Lambda^2 f_{dia} \kappa^{c_{14}}. \eqno{\tt thrbi} $$
Here, $c_{49}=1.0$, 
and the diamagnetic stabilization term is approximated by
$$ f_{dia} = ( 1 + c_{42} f_\ast )^{-c_{43}}.  \eqno{\tt zfdias} $$
$$      f_{*}=\left(    \frac{\mu_{o}\omega_{ci}\rho_{i}^{3}}
        {\eta   \beta   q^{2}L_{ni}}    \right) ^{2}    $$
where $c_{42}=1.0$ and $c_{43}=1/6$. Also,
\begin{displaymath}
\langle n \rangle = 
\mbox{rms value of the toroidal mode number}.
\end{displaymath}
Based upon Mirnov measurements on ISX-B, this should be in the range 
$ \langle n \rangle \simeq 5 \rightarrow 10$. However, g-mode calculations
near low-q surfaces indicate that $ \langle n \rangle = 2 $ is more appropriate
\cite{rosscom}.
Until an appropriate theoretical formula is developed for $ \langle n
\rangle$, it will be adjustable by input data and bounded above 1 by the expression
\begin{displaymath}
{\tt znmode} = \max ( {\tt cthery(21)}, 2 ).
\end{displaymath}
The variable $S=\tau_R/\tau_{hp}$ is defined in the preamble.
$$ R_c = {\tt zrcurv} \equiv \mbox{radius of curvature}. \eqno{\tt zrcurv} $$
For now, $R_c = R_0$, the radius of curvature is taken to be the major radius. \\
If ${\tt lthery(14)} \ne 0$, then $\Lambda$ is solved making $n$ iterations
on the equation
$$
\Lambda = \frac{2}{3 \pi} \ln \left[ \frac{256 S^2 L_p}{\beta R_0
                 \Lambda^3 q^{c_{47}} }
                 \left( \frac{\hat{S}}{\langle n \rangle} \right)^4 \right].
                  \eqno{\tt zlambd} $$
When ${\tt lthery(14)} = 0$, then $\Lambda$ is approximated by a single
iteration formula given by
 $$
 \Lambda = \frac{2}{3 \pi} \ln \left[ \frac{S^2 L_p}{\beta R_0 q^{c_{47}} }
                   \left( \frac{\hat{S}}{\langle n \rangle} \right)^4+ c_{48} \right] $$
Here, $c_{47}=8.0$ and $c_{48}= \ln (256/ \Lambda^{3}) \approx 0.7$ for TFTR.
\small
\begin{verbatim}
c
c..Carreras-Diamond theory (PF B 1 (1989) 1011-1017
c
      if ( lthery(13) .eq. 1 .or. lthery(13) .eq. 2 ) then
c
        znmode = max ( cthery(41), 2.0 )
        zrcurv = zrmaj
c
        zlambd = max ( thlamb(jz), 1.0 )

c     Single iteration approximation for lambda if lthery(14) = 0

        if ( lthery(14) .eq. 0 ) then
          zlambd = (2./(3.*zpi)) *
     &      log (abs( (zshat/znmode)**4 * abs(zsrhp**2*zlpr)
     &        / abs(zbeta*zrmaj*zq**cthery(47))
     &        + cthery(48) ) )
        else
          do 81 jt=1,lthery(14)
            zlamb1 = zlambd
            zlambd = (2./(3.*zpi)) *
     &        log ( (zshat/znmode)**4 * abs(256. * zsrhp**2 * zlpr)
     &          / abs(zbeta*zrmaj*zq**cthery(47)*zlambd**3) )
          if ( abs(zlambd-zlamb1) .lt. 1.e-5 ) go to 82
 81     continue
         endif
c
 82     continue
          thlamb(jz) = zlambd
          zfstarrb(jz) = zfstar
          zfdiarb(jz) = zfdias
c
          zxrb   = ( zbeta * zrmaj**2 * zq**2 * zrmin**2 * zlambd**2 )
     &           * zelonf**cthery(14)
     &           / ( sqrt(2.) * zlpr * zrcurv * zshat ** cthery(49)
     &           * ztaur )
c
         zdrb(jz) = frb(1) * zxrb * zfdias * zrbfac
         thrbi(jz) = frb(3) * zxrb * zfdias * zrbfac
c
         thrbb(jz) = frb(2) * zvthe * zrmin**2 * zrbfac
     &  * zfdias * zelonf**cthery(14) *
     &  ( zbeta * zrmaj**2 * zq**2 * zlambd / (zlpr * zrcurv) )**(4./3.)
     &  / ( 2.**(13./6.) * ( znmode * zsrhp )**(2./3.) * zshat * zrmaj )
         thrbgb(jz) = cthery(44) * zxrb * zfdias * zrbfac
         thrbe(jz) = thrbb(jz) + thrbgb(jz)
c
      endif

\end{verbatim}


For the hybrid resistive ballooning model combining the energy
transport from the old model and the energy transport from the
1989 Carreras-Diamond model, lthery(13) = 2. This yields: \\
$$ D^{RB} = F_a^{RB}
 \frac{\beta R_0^2 q^2}{\sqrt{2} L_p R_c \hat{S} }
 \frac{r^2}{\tau_R} \Lambda^2 f_{dia} \kappa^{c_{14}}. \eqno{\tt zdrb} $$
$$ Q_{e}^{RB}\frac{L_{Te}}{n_{e}T_{e}}=\chi_{old}^{RB}F_{e}^{RB}
 \eqno{\tt thrbe} $$
$$ Q_{i}^{RB}\frac{L_{Ti}}{n_{i}T_{i}}=\chi_{old}^{RB}F_{i}^{RB}
 \eqno{\tt thrbi} $$

%**********************************************************************c

\subsubsection{Guzdar-Drake Drift-Resistive Ballooning Model}

The 1993 $\bf E \times B$ drift-resistive ballooning mode model by
Guzdar and Drake \cite{drake93} is selected by setting
${\tt lthery(13)}= 3$,
$$
  D^{DB} = F_a^{DB}
  \left( 2\pi q_{a}^2 \right) \rho_e^2 \nu_{ei} \frac{R_o}{L_p}
$$
Here, $L_p$ has been substituted for $L_n$ given in their paper following
a comment made by Drake at the 1995 TTF Workshop\cite{drakecom2}.
Including diamagnetic and elongation effects, the particle diffusivity is
$$
  D^{DB} = \left( 2\pi q_{a}^2 \right) \rho_e^2 \nu_{ei} 
  \frac{R_o}{L_p} \left( \frac{1}{1 + \alpha^2} \right) \kappa^{c_{14}}  \eqno{\tt zgddb}
$$
where $\rho_e=v_e/\omega_{ce}$ and
$\alpha$ is the ratio of the diamagnetic frequency to the 
characteristic growth time,
$$
  \alpha = \frac{\rho_s c_s t_o}{L_p L_o} \eqno{\tt zgdalf}
$$
It can be calculated analytically by setting ${\tt cthery(85)}=1$ or
it can be numerically prescribed by setting ${\tt cthery(85)}=2$ and
${\tt cthery(86)}$ to the desired value.
Results from their 3-dimensional fluid simulations show that this 
parameter ranges from $\alpha = 0.3-0.6$ for L-mode discharges 
to $\alpha = 0.7-2.0$ for H-mode discharges~\cite{guzcomm}.
Here, $L_o$ is the characteristic scale length,
$$
  L_o = 2\pi q_a \left( \frac{\nu_{ei}R \rho_s}{2 \Omega_e} \right)^{1/2}
        \left( \frac{R}{L_p} \right)^{1/4} \eqno{\tt zgdln}
$$
and $t_o$ is growth rate,
$$
  t_o = \left( \frac{R L_p}{2} \right)^{1/2} \frac{1}{c_s} \eqno{\tt zgdtime}
$$
This is the ideal growth rate, but applied to shorter wavelength turbulence.

The electron  and ion thermal diffusivities are taken equal taken to be
an adjustable fraction of the particle diffusivity.
$$
 \chi_e^{DB} = \left( 2\pi q_{a}^2 \right) \rho_e^2 \nu_{ei} 
  \frac{R_o}{L_p} \left( \frac{1}{1 + \alpha^2} \right) \kappa^{c_{14}}
  \eqno{\tt thrbe}
$$
$$
 \chi_i^{DB} = \left( 2\pi q_{a}^2 \right) \rho_e^2 \nu_{ei} 
  \frac{R_o}{L_p} \left( \frac{1}{1 + \alpha^2} \right) \kappa^{c_{14}}
  \eqno{\tt thrbi}
$$

\begin{verbatim}
c
c..Guzdar-Drake theory (Phys Fluids B 5 (1993) 3712
c..L_p used instead of L_n
c
      if ( lthery(13) .eq. 3 ) then
c
c..   Compute pressure scale length w/o fast ions
c
        zgdtot1 = zne * zte * (1 / zlte + 1 / zlne)
     &  + zlti + znh * (1 / zlnh + 1 / zlti)
        zgdtot(jz) = zgdtot1 + zlti * znz
     &  * (1 / zlnz + 1 / zlti)
        if ( zgdtot(jz) .gt. zepslon ) then
        zgdlp(jz) = (zne * zte + znh * zti
     &   + znz * zti) / zgdtot(jz)
        endif
c
        zgyrfe = zce * zb / zcme  ! electron plasma frequency
        zlare = zvthe / zgyrfe    ! electron Larmor radius
c
        zgdtime = sqrt ( zrmaj*zlpr/2. ) / zsound  ! ideal growth rate
        zgdrlp = (zrmaj/zlpr)**.25                 ! (R/L_p)^.25
        zgdlna = 2. * zpi * zq * zgdrlp
        zgdln = zgdlna*sqrt((znuei*zrmaj*zrhos)/(2.*zgyrfe)) ! char length
        zgdalf = (zrhos*zsound*zgdtime) / ( zlpr*zgdln ) ! alpha
c
c..   Diamagnetic stabilization
c
        if ( nint(cthery(85)) .eq. 1 ) then
          zgddia = 1 / ( 1 + zgdalf**2 )
        elseif ( nint(cthery(85)) .eq. 2 ) then
          zgddia = cthery(86)
        else
          zgddia = 1.0
        endif
c
c..   Diffusivities
c
        zgdp = 2. * zpi * ((zq * zlare)**2.) * znuei
     &  * ( zrmaj / zlpr ) * 100. * zgddia
        zdrb(jz) = frb(1) * zgdp * zelonf**cthery(14) * zdtite
c        thrbgb(jz) = frb(2) * zgdp * zelonf**cthery(14) * zdtite
        thrbe(jz) = frb(2) * zgdp * zelonf**cthery(14) * zdtite
        thrbi(jz) = frb(3) * zgdp * zelonf**cthery(14) * zdtite
        thrbgb(jz) = thrbe(jz)
      endif
\end{verbatim}

Bruce Scott's 1998 Drift Alfven transport model is selected by setting
${\tt lthery(13)}= 4$.
See documentation in file {\tt sda01flx.tex}.

\begin{verbatim}
      if ( lthery(13) .eq. 4 .or. lthery(13) .eq. 5 ) then
c
        iswitch = 8
c
        do jc=1,iswitch
          isdasw(jc) = 0
          zsdasw(jc) = 0.0
        enddo
c
        iprint = lthery(29) - 10
c
        idim   = matdim
c
c  normalized gradients
c
        zgne = zrmaj / zlne
        zgnh = zrmaj / zlnh
        zgnz = zrmaj / zlnz
        zgns = zgrdns
        zgte = zrmaj / zlte
        zgth = zrmaj / zlti
        zgtz = zrmaj / zlti
c
c  ratios of temperatures
c
        zthte  = zti / zte
        ztzte  = zti / zte
        zbetae = 2. * zcmu0 * zckb * zne * zte /   zb**2
c
c  Impurity species (use only impurity species 1 for now)
c  assume T_Z = T_H throughout the plasma here
c
        ztz    = zti
        znz    = densimp(jz)
        zmass  = amassimp(jz)
        zimpz  = avezimp(jz)
        zimpz  = max ( zimpz, cthery(120) )
c
        zfnzne = znz / zne
        zmzmh  = zmass / amasshyd(jz)
c
c  superthermal ions
c  zfnsne = ratio of superthermal ions to electrons
c  zchrgns = charge of fast ions (assumed = 1.0 here)
c
        zfnsne = max ( zfnsnea(jz), 0.0 )
        zchrgns = 1.0
c
        zsdasw(1) = cthery(87)
c
        if ( lthery(13) .eq. 4) then
c
          call sda01dif ( isdasw, zsdasw, iswitch, idim
     &   , iprint, nprint
     &   , zgne, zgnh, zgnz, zgte, zgth, zgtz, zthte, ztzte
     &   , zfnzne, zimpz, zmzmh, zfnsne, zchrgns
     &   , zbetae, znuhat, zq, zshat, zelong
     &   , zfldath, zfldanh, zfldate, zfldanz, zfldatz
     &   , zdfdath, zdfdanh, zdfdate, zdfdanz, zdfdatz
     &   , zdfthi, zvlthi
     &   , inerr )
c
        else
c
          call sda04dif ( isdasw, zsdasw, iswitch, idim
     &   , iprint, nprint
     &   , zgne, zgnh, zgnz, zgte, zgth, zgtz, zthte, ztzte
     &   , zfnzne, zimpz, zmzmh, zfnsne, zchrgns
     &   , zbetae, znuhat, zq, zshat, zelong
     &   , zfldath, zfldanh, zfldate, zfldanz, zfldatz
     &   , zdfdath, zdfdanh, zdfdate, zdfdanz, zdfdatz
     &   , zdfthi, zvlthi
     &   , inerr )
c
        endif
c
c..Set total effective diffusivities
c
        znormd    = zsound * zrhos**2 / zrmaj
        znormv    = zsound * zrhos    / zrmaj
c
        zdrb(jz)  = frb(1) * zdfdanh * znormd * zne / zni

        thrbe(jz) = frb(2) * zdfdate * znormd
        thrbi(jz) = frb(3) * zdfdath * znormd
     &    * zne * zte / ( zni * zti )
c
        thrbgb(jz) = thrbe(jz)
c
        if ( zdrb(jz)  .lt. 0.0 ) zdrb(jz)  = 0.0
        if ( thrbe(jz) .lt. 0.0 ) thrbe(jz) = 0.0
        if ( thrbi(jz) .lt. 0.0 ) thrbi(jz) = 0.0
c
c..diagnostic printout
c
      if ( lprint .gt. 0 .or. nstep .eq. lthery(26)
     &  .or. ( zlastime .lt. cthery(88) .and. cthery(88) .le. time ) )
     &  then
c
      if ( jz .eq. jzmin ) then
c
        write (nprint,180) time, nstep, znormd, znormv
 180    format (/' Diagnostic output from sbrtn theory '
     &    ,' for the drift Alfven wave after sda01dif'
     &    ,/' time   =',1pe13.5,/' nstep  =',i5
     &    ,/' znormd =',1pe13.5,/' znormv =',1pe13.5)
c
        write (nprint,181) ( isdasw(jc),jc=1,iswitch )
 181    format (/' isdasw(jc) =',/(5i5))
c
        write (nprint,182) ( zsdasw(jc),jc=1,iswitch )
 182    format (/' zsdasw(jc) =',/(1p5e13.5))
c
        write (nprint,183)
 183    format (
     &    /t4,'radius',t17,'zgne',t30,'zgnh',t43,'zgnz'
     &    ,t56,'zgte',t69,'zgth',t82,'zthte',t95,'zfnzne'
     &    ,t108,'zimpz',t125,'#da1')
c
        write (nprint,184)
 184    format (
     &    /t4,'radius',t17,'zfnsne',t30,'zchrgns',t43,'zbetae'
     &    ,t56,'znuhat',t69,'zq',t82,'zshat',t95,'zelong'
     &    ,t108,' ',t125,'#da2')
c
        write (nprint,185)
 185    format (
     &    /t4,'radius',t17,'chi_i',t30,'d_h',t43,'chi_e'
     &    ,t56,'zdfdath',t69,'zdfdanh',t82,'zdfdate',t95,'zdfdanz'
     &    ,t108,'zdfdatz',t125,'#da3')
c
        write (nprint,186)
 186    format (
     &    /t4,'radius',t17,'zfldath',t30,'zfldanh',t43,'zfldate'
     &    ,t56,'zfldanz',t69,'zfldatz',t82,'normalized fluxes'
     &    ,t125,'#da4')
c
      endif
c
        write (nprint,187) zrmin, zgne, zgnh, zgnz, zgte, zgth
     &    , zthte, zfnzne, zimpz
 187    format (1p9e13.5,t125,'#da1')
c
        write (nprint,188) zrmin, zfnsne, zchrgns
     &    , zbetae, znuhat, zq, zshat, zelong
 188    format (1p8e13.5,t125,'#da2')
c
        write (nprint,189) zrmin
     &    , thrbi(jz), zdrb(jz), thrbe(jz)
     &    , zdfdath, zdfdanh, zdfdate, zdfdanz, zdfdatz
 189    format (1p9e13.5,4x,'#da3')
c
        write (nprint,190) zrmin
     &    , zfldath, zfldanh, zfldate, zfldanz, zfldatz
 190    format (1p6e13.5,4x,'#da4')
c
      endif
c
      endif
\end{verbatim}
%**********************************************************************c

\subsection{Neoclassical MHD Model}

The effective diffusivity driven by neoclassical MHD consists of an
$E \times B$ part, taken from page 300
of Kwon, Diamond and Biglari\cite{Kwon}
$$ D_p =  \frac{\eta}{2 \mu_0} \frac{q \beta R}{r}
          \frac{L_s}{L_p} \delta_e \Lambda_N^2  \eqno{\tt zexbnm} $$
and an effective electron diffusivity due to stochastic magnetic fields
as given by Eq. (32) of Kwon, Diamond and Biglari\cite{Kwon}
with corrections by Callen\cite{Callen}
$$
 \chi_{e0}=.046 (4 \pi)^{2/3} \frac{v_{e} L_s}{(n S)^{2/3}}
 \left( \frac{q \beta R}{L_{p}} \right) ^{4/3}
 \delta_{e}^{5/3}
 \Lambda_{N}^{7/3} f_{\Lambda} f_{\gamma}  \eqno{\tt zxnm}
$$
where
$$
 \delta_{e}^{-1}=1+\frac{\alpha_{e}(1+1.07
 \nu_{*e}^{1/2}+1.02\nu_{*e})
 (1+1.07\epsilon^{3/2}\nu_{*e})}{2.31\epsilon^{1/2}}  \eqno{\tt zdelm1}
$$
$$ \delta_e = 1. / \delta_e^{-1}   \eqno{\tt zdele} $$
$$
 \alpha_{e}=(1+1.198Z_{eff}+0.222Z_{eff}^{2})/
 (1+2.966Z_{eff}+0.753Z_{eff}^{2})   \eqno{\tt zalphe}
$$
$$
 \Lambda_{N} = \sqrt{3 \delta_{e}^{-1} - 1
 + 2 \sqrt{\delta_e^{-1} (2\delta_{e}^{-1}-1)}}. \eqno{\tt zlambn}
$$
Note, $\delta_e$ is always $ \leq 1.0 $ 
and $ \Lambda_{N} $ is always $ \geq 2.0 $.

In the original presentation of the theory\cite{Kwon},
which is accessed by setting ${\tt lthery(15)} = 1$,
$$ f_{\Lambda} = 1  \eqno{\tt zflamb} $$
$$ f_{\gamma}  = 1. \eqno{\tt zfgam}  $$
In more recent presentations\cite{Callen,IAEA1986},
accessed by setting ${\tt lthery(15)} \geq 2$,
$$
 f_{\Lambda}=(1-\Lambda_{N}^{-1})/
 \sqrt{1-\sqrt{2}/(\Lambda_{N}+\Lambda_{N}^{3})}   \eqno{\tt zflamb}
$$
and
$$
 f_{\gamma}=\left( \frac{\mu_{i}}{\mu_{i}+\gamma_0}
 + (1+2q^2) \frac{\epsilon^{2}}{q^{2}}\right) ^{1/2}   \eqno{\tt zfgam}
$$
where
$$
 \mu_{i}=0.66\frac{\epsilon^{1/2}\nu_{i}}
 {(1+1.03\nu_{*i}^{1/2}+0.31\nu_{*i})
 (1+0.66\nu_{*i}\epsilon^{3/2})}   \eqno{\tt zcmi}
$$
$$
 \nu_{i}=(n_{i}/n_{e}) \sqrt{A_{i}/2} [1+\sqrt{2}(Z_{eff}-1)]\nu_{ii}.
   \eqno{\tt znui}
$$
Here, $\nu_{ii}$ is the ion-ion collision frequency among only the hydrogenic
species,  $\nu_i$ is the ion-ion collision frequency including impurities,
and $ \nu_{*i} = \nu_i R q / ( \epsilon^{3/2} v_{Ti} ) $ is the ion-ion
collision frequency divided by the ion bounce frequency.

The growth rate of the mode is given by
$$
 \gamma_0 = c_{70}\left( \frac{\delta_{e}}{4S}\right) ^{1/3}
 \left( \frac{n q \beta R}{L_{p}} \right) ^{2/3}
 \frac{v_{A}}{R}   \eqno{\tt zgammh}
$$

Each contribution is multiplied by a correction due to diamagnetic effects 
derived by Diamond et al \cite{diam85a}
$$  f_{dia} = ( 1 + c_{75} \omega_*^2/\gamma^2 )^{-c_{76}}  $$ 
where 
$$  \omega_* = - n q \rho_s c_s / ( r L_{n_e} )  \eqno{\tt zwstrp} $$
and
$$ \gamma ( \gamma^2 + \omega_*^2 ) = \gamma_0^3  $$
for $ T_e = T_i $.  
Here, an approximate form for the diamagnetic stabilization is used:
$$ f_{dia} = ( 1 + c_{75} \omega_*^6 / \gamma_0^6 )^{-c_{76}}
              \eqno{\tt zfdia3} $$
which matches $f_{dia}$ for $\omega_* \gg \gamma_0 $
and reduces to $f_{dia} \simeq 1$ for $ \omega_* \ll \gamma_0$.

Note, in all of the above, the average toroidal mode number is taken to be
$$ n = max[1,{\tt cthery(77)}]  \eqno{\tt zntor} $$, 
where $ n = 1 $ gives the largest thermal diffusivity and the smallest
diamagnetic stabilization.

Each contribution is multiplied by an enhancement for elongated plasmas
$$  {\tt zelonf}^{\tt cthery(71)}  \eqno{\tt zelfnm}.  $$
Finally, {\tt zrbfac} is the correction to the finite difference
expressions which is defined at the beginning of the resistive
ballooning section.

The coding is:
\begin{verbatim}
c ...........................
c  Neoclassical mhd theory  .
c............................
c
      if ( lthery(15) .eq. 1 ) then
       zntor  = max( 1.0, cthery(77) )
       zalphe = ( 1.0 + 1.198*zeff + 0.222*zeff**2 )
     &        / ( 1.0 + 2.966*zeff + 0.753*zeff**2 )
       zdelm1 = 1.0 + zalphe *
     &  ( 1.0 + 1.07*sqrt(abs(thnust(jz))) + 1.02*thnust(jz) )
     &  / ( 2.31*sqrt(abs(zep)) )
       zdele  = 1. / zdelm1
       zlambn = sqrt( abs( 3.0*zdelm1 - 1.0
     &  + 2.0*sqrt(abs( zdelm1 * (2.*zdelm1 - 1.) ) ) ) )
       zflamb = ( 1.0 - 1.0/zlambn )
     &  / sqrt(abs(( 1.0 - sqrt(2.0)/( zlambn * ( 1.0 + zlambn**2 )))))
       znui = ( zni/zne ) * sqrt(zai/2.0)
     &  * ( 1.0 + sqrt(2.0) * (zeff-1.) ) * znuii
       zcmi = 0.66 * sqrt(abs(zep)) * znui
     &  / ( (1.0 + 1.03 * sqrt(abs(znusti)) + 0.31 * znusti)
     &    * (1.0 + 0.66 * znusti * abs(zep)**(1.5)) )
       zgammh = cthery(70) 
     &  *( abs(zdele/(4.0*zsrhp))
     &    * (zntor*zq*zbeta*zrmaj/zlpr)**2)**(1./3.)
     &  * zvalfv / zrmaj
       zfgam = sqrt(abs( zcmi/(zcmi+zgammh) + (zep/zq)**2 ))
       zxnm  = 0.2486 * zvthe * abs(zlsh)
     &  * abs( zq * zbeta * zrmaj / zlpr )**(4./3.)
     &  * abs(zdele)**(5./3.) * abs(zlambn)**(7./3.)
     &  / ( abs( zntor * zsrhp )**(2./3.) )
c
       zexbnm = 0.5 * zresis * zq * zrmaj * zbeta * abs(zlsh)
     &    * zdele* zlambn**2 / abs( zcmu0 * zrmin * zlpr )
c
      elseif ( lthery(15) .gt. 1 ) then
       zntor  = max( 1.0, cthery(77) )
       zalphe = ( 1.0 + 1.198*zeff + 0.222*zeff**2 )
     &        / ( 1.0 + 2.966*zeff + 0.753*zeff**2 )
       zdelm1 = 1.0 + zalphe *
     &  ( 1.0 + 1.07*sqrt(abs(thnust(jz))) + 1.02*thnust(jz) )
     &  * ( 1.0 + 1.07*abs(zep)**(1.5)*thnust(jz) )
     &  / ( 2.31 * sqrt(abs(zep)) )
       zdele  = 1. / zdelm1
       zlambn = sqrt( abs( 3.0*zdelm1 - 1.0
     &  + 2.0*sqrt(abs( zdelm1 * (2.*zdelm1 - 1.) ) ) ) )
       zflamb = ( 1.0 - 1.0/zlambn )
     &  / sqrt(abs(( 1.0 - sqrt(2.0)/( zlambn * ( 1.0 + zlambn**2 )))))
       znui = ( zni/zne ) * sqrt(zai/2.0)
     &  * ( 1.0 + sqrt(2.0) * (zeff-1.) ) * znuii
       zcmi = 0.66 * sqrt(abs(zep)) * znui
     &  / ( (1.0 + 1.03 * sqrt(abs(znusti)) + 0.31 * znusti)
     &    * (1.0 + 0.66 * znusti * abs(zep)**(1.5)) )
       zgammh = cthery(70) 
     &  *( abs(zdele/(4.0*zsrhp))
     &    * (zntor*zq*zbeta*zrmaj/zlpr)**2)**(1./3.)
     &  * zvalfv / zrmaj
       zfgam = sqrt(abs( zcmi/(zcmi+zgammh)
     &  + (1.0 + 2.0 * zq**2 ) * (zep/zq)**2 ))
       zxnm  = 0.2486 * zvthe * abs(zlsh)
     &  * abs( zq * zbeta * zrmaj / zlpr )**(4./3.)
     &  * abs(zdele)**(5./3.) * abs(zlambn)**(7./3.)
     &  * zflamb * zfgam / ( abs( zntor * zsrhp )**(2./3.) )
c
       zexbnm = 0.5 * zresis * zq * zrmaj * zbeta * abs(zlsh)
     &    * zdele* zlambn**2 / abs( zcmu0 * zrmin * zlpr )
      endif
c
      if ( lthery(15) .gt. 0 ) then
c
       zelfnm = zelonf**cthery(71)
c
       zwstrp = - zntor * zq * zrhos * zsound / ( zrmin * abs(zlne) )
       zfdia3 = 1.0
       if ( min( cthery(75), cthery(76), zgammh ) .gt. zepslon )
     &   zfdia3 = ( 1. + cthery(75)*(zwstrp/zgammh)**6 )**(-cthery(76))
c
       zalphz(jz) = zalphe
       zdelez(jz) = zdele
       zlambz(jz) = zlambn
c
       zflamz(jz) = zflamb
       znuiz(jz)  = znui
       zcmiz(jz)  = zcmi
       zgammz(jz) = zgammh
       zfgamz(jz) = zfgam
       zxnmz(jz)  = zxnm
       zexbnz(jz) = zexbnm
       zfdiaz(jz) = zfdia3
       zwstrnm(jz) = zwstrp
c
       zdnm(jz)  = ( fmh(1) * zexbnm + cthery(72) * zxnm )
     &             * zelfnm * zfdia3 * zrbfac
       thnme(jz) = ( fmh(2) * zxnm + cthery(73) * zexbnm )
     &             * zelfnm * zfdia3 * zrbfac
       thnmi(jz) = ( fmh(3) * zexbnm + cthery(74) * zxnm )
     &             * zelfnm * zfdia3 * zrbfac
      endif
c
\end{verbatim}

%**********************************************************************c

\subsection{Cirulating Electron and High-$m$ Tearing Modes}

Fluxes due to circulating electrons
and/or the high-$m$ tearing mode can sensibly
be added to those defined above.  To do this, let
$$
 \Gamma_{a}^{CE}=F^{CE}_{a}D^{CE}\kappa^{c_{83}}
 \frac{\partial n_{a}}{\partial r}
\eqno{\tt zxce} $$
$$
 Q_{e}^{CE}=F^{CE}_{e}D^{CE}\kappa^{c_{83}}
 n_{e}\frac{\partial T_{e}}{\partial r}
\eqno{\tt thcee} $$
$$
 Q_{i}^{CE}=F^{CE}_{i}D^{CE}\kappa^{c_{83}}
 n_{i}\frac{\partial T_{i}}{\partial r}
\eqno{\tt thcei} $$
\begin{equation}
 D^{CE}=(\nu_{ei}\omega_{e}^{*}/\omega_{te}^{2})
 (1+a_{19}f_{ith}\eta_{e}\epsilon_{n})Df_{\tau}
\end{equation}
where $\omega_{te}={\hat s}v_{e}/(qR)$.
For electron energy
fluxes due to high-$m$ tearing modes, let
$$
 Q_{e}^{HM}=c_{82}(\nu_{ei}\omega_{e}^{*}/\omega_{te}^{2})
 \eta_{e}^{2}Df_{\tau}
 \kappa^{c_{83}}\frac{\partial n_{a}}{\partial r}
\eqno{\tt thhme} $$

The coding is:

\begin{verbatim}
c ...................................................
c . the circulating electron & high-m tearing model .
c ...................................................
c
      if ( lthery(16) .eq. 1 ) then
       zepn=zlne/zrmaj
       zfte=(zshat*zvthe)/(zq*zrmaj)
       zdperp=zdiafr/zwn**2
       zdfte=znuei*zdiafr/zfte**2
       zcest=zfith*zetae*zepn
       zhmst=(zetae**2)*zdperp*zdtite
       zelfce=zelonf**cthery(81)
       zxce=zdfte*(1.+cthery(80)*zcest)*zdperp*zdtite
       zdce      = fec(1) * zxce * zelfce
       thcee(jz) = fec(2) * zxce * zelfce
       thcei(jz) = fec(3) * zxce * zelfce
       thhme(jz) = cthery(82)*zdfte*zhmst*zelonf**cthery(83)
      endif
c
\end{verbatim}

%**********************************************************************c

\subsection{Kinetic Ballooning}

For transport due to the kinetic ballooning mode, we compute $D^{KB}$ and
the thermal diffusivities.  The ideal and
second stability thresholds are given, respectively, as
$$  \beta_{c1}' = c_{78} \hat{s}/(1.7 q^{2}R_{o}) \eqno{\tt zbc1} $$ \\[-5mm]
$$  \beta_{c2}' = c_{79} 4 \hat{s}/(q^{2}R_{o}) \eqno{\tt zbc2} $$ \\[-2mm]
Here, $cthery(78,79)=1$ by default, but are included for flexibility.
For the original Singer-Tang-Rewoldt~\cite{Comments,Tang86,RewTang} 
kinetic ballooning mode model, chosen by setting lthery(17) = 0, we have
$$  f_{\beta th} = \left[ 1 +
  \exp \left[ -\frac{L_{p}}{c_{8}\rho_{\theta i}}(\beta ' -
  \beta_{c1}') \right] \right]^{-1} \eqno{\tt zfbth} $$
and
$$  D^{KB} = \omega_{e}^{*}\rho_{i}^{2}f_{\beta th}
  \left(1 + \frac{\beta '}{\beta_{c1}'}\right)
  \left[1 - \frac{\beta '}{\beta_{c2}'} , 0\right]_{\max} \eqno{\tt zdk} $$ \\
The diffusivities are then given as:
$$ D_{a}^{KB}=D^{KB}F^{KB}_{a} F_{\kappa} \eqno{\tt zdkb(jz)} $$
$$ Q_{e}^{KB}\frac{L_{Te}}{n_{e}T_{e}}=2.5D^{KB}F^{KB}_{e} F_{\kappa} \eqno{\tt thkbe} $$
$$ Q_{i}^{KB}\frac{L_{Ti}}{n_{i}T_{i}}=2.5D^{KB}F^{KB}_{i} F_{\kappa} \eqno{\tt thkbi} $$\\

\noindent
For the 1995 revised version~\cite{gbcomm} (lthery(17) = 1),
$$  f_{\beta th} = \exp \left[c_{8}\left( \frac{\beta '}{\beta_{c1}'} -1 \right) \right]
\eqno{\tt zfbth2} $$
along with
$$  D^{KB} = \frac{c_s \rho_s^2}{L_p} f_{\beta th} \eqno{\tt zdk} $$\\
The diffusivities are then given as:
$$ D_{a}^{KB}=D^{KB}F^{KB}_{a} F_{\kappa} \eqno{\tt zdkb(jz)} $$
$$ Q_{e}^{KB}\frac{L_{Te}}{n_{e}T_{e}}=D^{KB}F^{KB}_{e} F_{\kappa} \eqno{\tt thkbe} $$
$$ Q_{i}^{KB}\frac{L_{Ti}}{n_{i}T_{i}}=D^{KB}F^{KB}_{i} F_{\kappa} \eqno{\tt thkbi} $$
Note that the new version does not include the (5/2) factor in the thermal
diffusivities.

\noindent
The relevant coding is:

\begin{verbatim}
c ..................................
c .  the kinetic ballooning model  .
c ..................................
c
c       zbprim, zbc1, and zbpbc1 computed above under drift model
c
      if ( lthery(17) .gt. -1  .and.  abs(cthery(8)) .gt. zepslon
     &   .and.  zlpr .gt. 0.0  ) then
c
      zbprim = abs(zbeta/zlpr)
      zbcoef1 = 1.0
      zbcoef2 = 1.0
      if ( abs(cthery(78)) .gt. zepslon ) zbcoef1 = cthery(78)
      if ( abs(cthery(79)) .gt. zepslon ) zbcoef2 = cthery(79)
      zbc1   = zbcoef1 * abs(zshat)/(1.7*zq**2*zrmaj)
      zbpbc1 = zbprim/zbc1
      zbc2   = zbcoef2 * 4.* abs(zshat)/(zq**2*zrmaj)
      zelfkb = zelonf**cthery(15)
c
      if ( lthery(17) .eq. 1) then
        zfbthn = exp( min(abs(zlgeps),
     &     max(-abs(zlgeps),cthery(8)*(zbprim/zbc1 - 1.))) )
c
        zdk = abs( zsound * zrhos**2 * zfbthn / zlpr )
        zdkb(jz)  = zdk*fkb(1)*zelfkb
        thkbe(jz) = zdk*fkb(2)*zelfkb
        thkbi(jz) = zdk*fkb(3)*zelfkb
      else
        zexkb  = - zlpr*(zbprim-zbc1)/(cthery(8)*zlarpo)
        zovfkb = max(zexkb,-abs(zlgeps))
        zovfkb = min(zovfkb,abs(zlgeps))
        zfbth  = 1.0/(1.0+exp(zovfkb))
        zdk1   = max(1.-zbprim/zbc2,0.0)*(1.+zbpbc1)
        zdk = abs(zdk1*zdiafr*zlari**2.*zfbth)
        zdkb(jz)  = zdk*fkb(1)*zelfkb
        thkbe(jz) = 2.5*zdk*fkb(2)*zelfkb
        thkbi(jz) = 2.5*zdk*fkb(3)*zelfkb
      endif
c
c..arrays for diagnostic printout
c
        zbprima(jz) = zbprim
        zbc1a(jz)   = zbc1
        thbpbc(jz)  = zbpbc1
        zbc2a(jz)   = zbc2
        zdka(jz)    = zdk
c
      endif
c
\end{verbatim}

%**********************************************************************c

\subsection{$\eta_{e}$ Mode}

For the $\eta_{e}$ mode, we compute
$$ f_{eth}=\{ 1+\exp [-c_{9}(\eta_{e}-1)]\} ^{-1} \eqno{\tt zfeth} $$
$$ \chi_{e,Md}=0.1\left( \frac{c}{\omega_{pe}}\right) ^{2}
 \frac{v_{e}}{qR_{0}}(1+\eta_{e})\eta_{e}{\hat s} \eqno{\tt zxemd} $$
with the modified limit \cite{Guzdar}
$$ f_{lim}=\eta_{e}(1+\eta_{e})/[1+.015(T_{e}/T_{i})\eta_{e}(1+\eta_{e})]
 \eqno{\tt zflim} $$
$$ \chi^{HF}=f_{lim}f_{eth}\frac{1}{1+\nu_{eff}/\omega_{e}^{*}}\chi_{e,Md}
 \eqno{\tt zxhf} $$
As above, we give the user the option of adding similar
scalings for particle and ion energy diffusion coefficients
in the form
$$ D_{a}^{HF}=\chi^{HF}F^{HF}_{a} \eqno{\tt zdhf(jz)} $$
in addition to (C27) (with the new symbol $F^{HF}_{e}$ equivalent
to $F^{HF}$ in the Comments paper),
$$ Q^{HF}_{e}L_{Te}/(n_{e}T_{e})=\chi^{HF}F^{HF}_{e} \eqno{\tt thhfe} $$
As above, we also allow an ion energy tranport option of the form
$$ Q^{HF}_{i}L_{Ti}/(n_{i}T_{i})=\chi^{HF}F^{HF}_{i} \eqno{\tt thhfi} $$
The relevant coding is:

\begin{verbatim}
c ..........................
c .  the eta-e mode model  .
c ..........................
c
      if ( lthery(20) .lt. 0 ) then
        zdhf(jz)  = 0.0
        thhfe(jz) = 0.0
        thhfi(jz) = 0.0
      else
        zexhf   = -cthery(9)*(zetae-1.0)
         zovfhf = max(zexhf,zlgeps)
        zovfhf  = min(zovfhf,-zlgeps)
        zfeth   = 1./(1.+exp( min(15.0, max(-15.0, -6.*(zetae-1.))) ))
        z1      = zetae*(1.+abs(zetae))
        zxemd   = 0.1*(zcc/zfpe)**2*(zvthe/(zq*zrmaj))*z1*zshat
        zflim   = z1/(1.+0.015*(zte/zti)*z1)
        zxhf    = zflim*zfeth*zxemd/(1.+abs(znueff/zdiafr))
          zelfhf= zelonf**cthery(16)
        zdhf(jz)  = zxhf*fhf(1)*zelfhf
        thhfe(jz) = zxhf*fhf(2)*zelfhf
        thhfi(jz) = zxhf*fhf(3)*zelfhf
      endif
\end{verbatim}

%**********************************************************************c

\subsection{Rebut-Lallia-Watkins Model}

The Rebut-Lallia-Watkins transport model\cite{RLW88a,rebu91a} 
is turned on with the default coefficients by setting
${\tt lthery(25)} = 1$ together with
${\tt cthery(24)} = 1.0$ for particle diffusivity, 
${\tt cthery(25)} = 1.0$ for electron thermal diffusivity, and
${\tt cthery(26)} = 1.0$ for ion thermal diffusivity.
Note that convective transport should also be turned on when using this
model by setting ${\tt cthery(68)} = 1.5$ and ${\tt cthery(69)} = 1.5$.

For
\[ |\nabla T_{ec}| / |\nabla T_e| < 1.0 \;\;
   {\rm and} \;\; \nabla q > 0.0 \]
the expressions for the thermal and particle diffusivities are:
$$ \chi_e^{RLW} = C_{25} \chi_e^{an} ( 1.-\nabla T_{ec} / \nabla T_e ) 
    \eqno{\tt thrlwe(jz)} $$
$$ \chi_i^{RLW} = C_{26} \chi_i^{an} ( 1.-\nabla T_{ec} / \nabla T_e )
    \eqno{\tt thrlwi(jz)} $$
and
$$ D^{RLW} = C_{24} 0.7 \chi_i^{RLW}. \eqno{\tt zdrlw} $$
The anomalous electron thermal diffusivity is given by
$$ \begin{array}{rcl}
  \chi_e^{an} &  = &  0.5 c^2 \sqrt{\mu_0 m_i} 
  \left| \frac{\nabla T_e}{T_e} + \frac{2\nabla n_e}{n_e} \right|
  \left( \frac{T_e}{T_i} \right)^{1/2} 
  \left( \frac{q^2}{\nabla q B \sqrt{R}} \right) \\
  ( 1 - \sqrt{r/R} ) \sqrt{ 1 + Z_{eff} }
               & = & 0.5  
  \left[ \frac{1}{L_{T_e}} + \frac{2}{L_{n_e}} \right]
  \left( \frac{T_e}{T_i} \right)^{1/2}
  \frac{q}{\hat{s}} \frac{c^2}{v_A} \frac{r}{\sqrt{Rn_e}}
  ( 1 - \sqrt{r/R} ) \sqrt{ 1 + Z_{eff} }.
  \end{array} \eqno{\tt zchian} $$
The critical electron temperature gradient is given by
$$ \begin{array}{rcl}  \frac{\nabla T_{ec}}{\nabla T_e} & = &
  \frac{0.06}{q \nabla T_e} 
  \left[ \frac{\eta J B^3 e^2}{n T_e^{1/2} \mu_o m_e^{1/2}}
     \right]^{1/2} \\
  & = & \frac{0.12 L_{T_e}}{q \rho_e}
  \left[ \sqrt{2} \frac{\eta J}{B v_e} 
    \frac{1+\frac{n_i T_i}{n_e T_e}}{\beta}
  \right]^{1/2}
  \end{array} \eqno{\tt ztcrit} $$
where
$$ \rho_e = v_e / \omega_{ce} \eqno{\tt zrhoe} $$
Then the anomalous ion thermal diffusivity is given by
$$ \chi_i^{an} =  \chi_e^{an} (T_e / T_i )^{1/2} 
  2.0 Z_i / \sqrt{1 + Z_{eff}} $$
Each contribution is multiplied by an enhancement for elongated plasmas
$$  {\tt zelonf}^{\tt cthery(27)}  \eqno{\tt zelrlw}  $$
The relevant coding is:

\begin{verbatim}
c
      thrlwe(jz) = 0.0
      thrlwi(jz) = 0.0
      zdrlw      = 0.0
c
      if ( lthery(25) .eq. 1 ) then
       zrhoe  = zvthe * zcme / ( zce * zb )
c rgb 15-apr-95 define zefld from zvloop rather than ajzs(1,jz)
cbate       zefld  = zresis * ajzs(1,jz)*usij
       zefld = 0.0
       if ( zvloop .gt. zepslon) zefld  = zvloop / ( 2. * zpi * zrmaj)
       ztcrit(jz) = ( 0.12 * zlte / ( zrhoe * zq ) )
     &  * sqrt( sqrt(2.0) * zefld
     &   * ( 1.0 + ( zni * zti ) / ( zne * zte ) )
     &    / ( zb * zvthe * zbeta ) )
c
       if ( abs( ztcrit(jz) ) .lt. 1.0
     &     .and. q(jz+1) .gt. q(jz-1) ) then
         zchian = 0.5 * ( abs(1.0/zlte) + abs(2.0/zlne) )
     &    * sqrt( zte / ( zti * zrmaj * zne ) )
     &    * zq * zcc**2 / abs( zshat * zvalfv )
     &    * ( 1.0 - sqrt(abs(zrmin/zrmaj)) ) * sqrt( 1.0 + zeff )
         zelrlw = zelonf**cthery(27)
         thrlwe(jz) = cthery(25) * zchian * (1.0 - ztcrit(jz)) * zelrlw
         thrlwi(jz) = cthery(26) * zchian * sqrt( zte / zti )
     &    * ( 1.0 - ztcrit(jz) ) * zelrlw * 2.0 / sqrt( 1.0 + zeff )
         zdrlw = cthery(24) * 0.7 * zchian * sqrt( zte / zti )
     &    * ( 1.0 - ztcrit(jz) ) * zelrlw * 2.0 / sqrt( 1.0 + zeff )
       endif
      endif
c
\end{verbatim}

%**********************************************************************c

\subsection{Sawteeth}

As there is no immediately forseeable need for the
post-sawtooth transport enhancement from the Comments paper,
writing coding for evaluation of these fluxes
has been deferred until such time as they are required.
When this is included, it will be necessary to
have the time $t_{saw}$ of the last sawtooth crash.
It may be possible to compute this from the sawtooth-controlling
parameters already in {\tt common} and the present values of time and
timestep (and $q$ profile, to make sure a sawtooth crash actually
occured when allowed by the input data).  Alternatively,
it may be necessary to alter BALDUR elsewhere
and transfer $t_{saw}$ by {\tt common} or subroutine argument.
Then compute
(C32), (C31), and $Q_{e}^{RR}L_{Te}/(n_{e}T_{e})$.

%**********************************************************************c

\subsection{Neoclassical Fluxes}

Implementation of the simplified
neoclassical fluxes in the Comments paper
will also be deferred until after preliminary applications of
the anomalous flux formulas and evidence surfaces of a need
for the simplified neoclassical formulas.
To give some idea of how much additional coding these would require,
however, the needed formulas are outlined here.
For the neoclassical fluxes, an appropriate order of computation
would be something like the following. (The sign conventions here
should be rechecked, and the formulas should be rechecked
against the original references.  Here,
we follow the
convention used elsewhere
in BALDUR of defining the particle
fluxes as ``pinches''.  This should be reconsidered before
final implementation.)  Compute $v_{a}^{W}=-\Gamma_{a}^{W}$ from
(C1), $v_{I}^{P}=\Gamma_{I}^{P}/n_{I}$ from (C3),
$\sum_{a}v_{a}^{W}n_{a}L_{Te}/n_{e}$, and
$\sum_{I}^{P}(1-Z_{I})L_{Ti}n_{I}/n_{i}$ [{\it cf.} (C4)].
The following quantities in (C41) and the definitions immediately
thereafter would be needed: $\alpha_{nc}$, $\mu_{i}$, $\nu_{I}^{*}$,
$K_{2ps}$, $K_{2nc}$. Also, $K_{2}$, $\chi_{nc}$,
and $\chi_{nc-new}$ are needed, their interpretation for more
than one impurity needs to be clarified. It should also be checked whether
$K_{2}$ and $\chi_{nc}$ are properly listed.  For self-consistency,
ion heating due to damping of poloidal rotation should also be
included in a complete neoclassical treatment ({\it cf.} the
Hirshman-Sigmar Nuclear Fusion review \cite{Hirshman} and/or the
Hirshman-Jardin J. Comp. Physics \cite{JCP} paper for details).

%**********************************************************************c

\subsection{Totals}

The quantities computed above which have units
of diffusivity are combined in the sums
$$ D_{a}=D_{a}^{TEM}+D_{a}^{ITG}+D_{a}^{RM}+D_{a}^{RB}+D_{a}^{KB}+
 D_{a}^{NMHD}+D_{a}^{CE}+D_{a}^{HF}
 \eqno{\tt dhtot} $$
$$ \chi_{e}
 =(Q_{e}^{TEM} + Q_e^{ITG} + Q_e^{RM}
 +Q_{e}^{RB}+Q_{e}^{KB}+Q_{e}^{NMHD}+Q_{e}^{CE}+Q_{e}^{HF})
 \frac{L_{Te}}{n_{e}T_{e}} \eqno{\tt xetot} $$
$$ \chi_{i}
 =(Q_{i}^{TEM} + Q_i^{ITG} + Q_i^{RM}
 +Q_{i}^{RB}+Q_{i}^{KB}+Q_{i}^{NMHD}+Q_{i}^{CE}+Q_{i}^{HF})
 \frac{L_{Ti}}{n_{i}T_{i}} \eqno{\tt xitot} $$
and  the thermal diffusivities are
corrected for inclusion of convective energy transport using
$$  \chi_{e}=Q_{e}\frac{L_{T_{e}}}{n_{e}T_{e}}
               - c_{17} C_{pv}^{e}D_{a}/\eta_{e} \eqno{\tt xetot} $$
$$  \chi_{i}=Q_{i}\frac{L_{T_{i}}}{n_{i}T_{i}}
               - c_{17} C_{pv}^{i}D_{a}/\eta_{i} \eqno{\tt xitot} $$

and then returned to BALDUR subroutine {\tt trcoef} as is
presently done from {\tt empirc}. Those thermal diffusivities and the
electron-
ion energy interchange are printed out when {\tt theory} is called from
subroutine {\tt mprint}. The relevant coding is:

\begin{verbatim}
c   combined totals
c
        zdsum = zdti(jz) + zdrm(jz) + zdrb(jz) + zdkb(jz)
     &          + zdnm(jz) + zdhf(jz)
        dhtot(jz)=zdsum
        dztot(jz)=zdsum
c
       xetot(jz) =   thrme(jz) + thrbe(jz) + thkbe(jz)
     &  + thnme(jz) + thhfe(jz)
     &  + thtie(jz) + thrlwe(jz) - cthery(17)*cthery(68)*zdsum/zetae
       xitot(jz) =   thrmi(jz) + thrbi(jz) + thkbi(jz)
     &  + thnmi(jz) + thhfi(jz)
     &  + thtii(jz) + thrlwi(jz) - cthery(17)*cthery(69)*zdsum/zetai
c
c
c ..............................
c  arrays for diagnostic output
c ..............................
c
        threti(jz) = zetai
        thdinu(jz) = zdiafr / znueff
        thfith(jz) = zfith
        thdte(jz)  = zdte
        thdi(jz)   = zdi
        thbpbc(jz) = zbpbc1
        thlni(jz)  = zln
        thlti(jz)  = zlti
        thdia(jz)  = zdiafr / zwn**2
c
        thlsh(jz)  = zlsh
        thlpr(jz)  = zlpr
        thlarp(jz) = zlarpo
        thrhos(jz) = zrhos
c
        thvthe(jz) = zvthe
        thvthi(jz) = zvthi
        thsoun(jz) = zsound
        thalfv(jz) = zvalfv
c
        thbeta(jz) = zbeta
        thetth(jz) = zetith
        thsrhp(jz) = zsrhp
        thdias(jz) = zfdias
        thtau(jz) = zti / zte
c
        zrsist(jz) = zresis
        zdshat(jz) = zshat
        zdnuhat(jz) = znuhat
        zdbetae(jz) = zbetae
        zdbetah(jz) = zbetah
        zdbetaz(jz) = zbetaz
        zkpar(jz) = zkparl
c
c..gradient scale lengths
c  print smoothed values if lthery(33) .gt. 0
c
      if ( lthery(33) .gt. 0 ) then
        zslne(jz) = zlne
        zslni(jz) = zlni
        zslnh(jz) = zlnh
        zslnz(jz) = zlnz
        zslte(jz) = zlte
        zslti(jz) = zlti
        zslpr(jz) = zlpr
      endif
c
 300  continue
c
        zlastime = time
c
c   end of the main do-loop over the radial index, "jz"----------
c
\end{verbatim}

In order to limit the time rate of change of the theory-based 
diffusivities, the old values are kept in local arrays.
The difference between the old and the new values is computed
\[ \Delta \chi_j = \chi^{N}_j - \chi^{N-1}_j \]
and stored in local arrays.
This difference is then spatially averaged
\[  \bar{ \Delta \chi_j }
    = ( \Delta \chi_{j-1} + 2 \Delta \chi_j +  \Delta \chi_{j+1} ) / 4 \]
Finally, the adjusted diffusivity is computed
\[ \chi^N_j = \chi^{N-1}_j
 + \Delta \chi_j /
 ( 1. + c_{60} |  \Delta \chi_j - \bar{ \Delta \chi_j} | ). \]
Here $ c_{60} = {\tt cthery(60)} $ with default value 0.0.
A recommended value is $ {\tt cthery(60)} = 10.0 $
if the diffusivities show the pattern of a numerical instability.
This adjustment suppresses large local changes in the diffusivities.
Here, this algorithm is applied only to the ITG ($\eta_i$) mode.

\begin{verbatim}
      if ( cthery(60) .gt. zepslon ) then
c
        if ( nstep .lt. 1 ) then
          do jz=1,medge
            zoetai(jz) = thigi(jz)
            zoetae(jz) = thige(jz)
            zoetad(jz) = zddig(jz)
            zoetaz(jz) = zdzig(jz)
          enddo
        endif
c
        do jz=1,medge
          zdleti(jz) = thigi(jz) - zoetai(jz)
          zdlete(jz) = thige(jz) - zoetae(jz)
          zdletd(jz) = zddig(jz) - zoetad(jz)
          zdletz(jz) = zdzig(jz) - zoetaz(jz)
        enddo
c
        do jz=maxis+1,medge-1
          ztemp1(jz)=0.25*(zdleti(jz-1)+2.0*zdleti(jz)+zdleti(jz+1))
          ztemp2(jz)=0.25*(zdlete(jz-1)+2.0*zdlete(jz)+zdlete(jz+1))
          ztemp3(jz)=0.25*(zdletd(jz-1)+2.0*zdletd(jz)+zdletd(jz+1))
          ztemp4(jz)=0.25*(zdletz(jz-1)+2.0*zdletz(jz)+zdletz(jz+1))
        enddo
c
        do jz=maxis+1,medge-1
          thigi(jz) = zoetai(jz) + zdleti(jz)
     &     / ( 1. + cthery(60) * abs ( zdleti(jz) - ztemp1(jz) ) )
          thige(jz) = zoetae(jz) + zdlete(jz)
     &     / ( 1. + cthery(60) * abs ( zdlete(jz) - ztemp2(jz) ) )
          zddig(jz) = zoetad(jz) + zdletd(jz)
     &     / ( 1. + cthery(60) * abs ( zdletd(jz) - ztemp3(jz) ) )
          zdzig(jz) = zoetaz(jz) + zdletz(jz)
     &     / ( 1. + cthery(60) * abs ( zdletz(jz) - ztemp4(jz) ) )
        enddo
c
        if (  nstep .gt. istep ) then
          istep = nstep
          do jz=1,medge
            zoetai(jz) = thigi(jz)
            zoetae(jz) = thige(jz)
            zoetad(jz) = zddig(jz)
            zoetaz(jz) = zdzig(jz)
          enddo
        endif
c
      endif
\end{verbatim}

If {\tt cthery(61)} > 0.0, the same algorithm is applied to the 
trapped electron modes.

\begin{verbatim}
c
      if ( cthery(61) .gt. zepslon ) then
c
        if ( nstep .lt. 1 ) then
          do jz=1,medge
            zotmai(jz) = thdri(jz)
            zotmae(jz) = thdre(jz)
            zotmad(jz) = zddtem(jz)
            zotmaz(jz) = zdztem(jz)
          enddo
        endif
c
        do jz=1,medge
          zdltmi(jz) = thdri(jz) - zotmai(jz)
          zdltme(jz) = thdre(jz) - zotmae(jz)
          zdltmd(jz) = zddtem(jz) - zotmad(jz)
          zdltmz(jz) = zdztem(jz) - zotmaz(jz)
        enddo
c
        do jz=maxis+1,medge-1
          ztemp1(jz)=0.25*(zdltmi(jz-1)+2.0*zdltmi(jz)+zdltmi(jz+1))
          ztemp2(jz)=0.25*(zdltme(jz-1)+2.0*zdltme(jz)+zdltme(jz+1))
          ztemp3(jz)=0.25*(zdltmd(jz-1)+2.0*zdltmd(jz)+zdltmd(jz+1))
          ztemp4(jz)=0.25*(zdltmz(jz-1)+2.0*zdltmz(jz)+zdltmz(jz+1))
        enddo
c
        do jz=maxis+1,medge-1
          thdri(jz) = zotmai(jz) + zdltmi(jz)
     &     / ( 1. + cthery(61) * abs ( zdltmi(jz) - ztemp1(jz) ) )
          thdre(jz) = zotmae(jz) + zdltme(jz)
     &     / ( 1. + cthery(61) * abs ( zdltme(jz) - ztemp2(jz) ) )
          zddtem(jz) = zotmad(jz) + zdltmd(jz)
     &     / ( 1. + cthery(61) * abs ( zdltmd(jz) - ztemp3(jz) ) )
          zdztem(jz) = zotmaz(jz) + zdltmz(jz)
     &     / ( 1. + cthery(61) * abs ( zdltmz(jz) - ztemp4(jz) ) )
        enddo
c
        if (  nstep .gt. istep ) then
          istep = nstep
          do jz=1,medge
            zotmai(jz) = thdri(jz)
            zotmae(jz) = thdre(jz)
            zotmad(jz) = zddtem(jz)
            zotmaz(jz) = zdztem(jz)
          enddo
        endif
c
      endif
c
c..finally, add ITG mode and TEM transport
c  to the total effective thermal diffusivities
c
c  Skip this if lthery(7) > 20  .and. lthery(8) < 21
c    (Full matrix difthi used instead)
c
      if ( lthery(7) .lt. 20 .or. lthery(8) .gt. 20 ) then
c
        do 280 jz=maxis+1,medge
          xitot(jz) = xitot(jz) + thdri(jz) + thigi(jz)
          xetot(jz) = xetot(jz) + thdre(jz) + thige(jz)
c
c..hydrogenic diffusivity
c
            dhtot(jz) = dhtot(jz) + zddtem(jz) + zddig(jz)
c
c..impurity diffusivity
c
            dztot(jz) = dztot(jz) + zddtem(jz) + zdzig(jz)
 280    continue
c
      endif
c
c..smoothing applied to effective diffusivities and matrix
c
      if ( lthery(28) .gt. 0 ) then
c
        ismooth = lthery(28)
        zsmooth = cthery(127)
        ilower  = maxis + 1
        iupper  = medge
c
        call smooth2 ( xetot, 1, ztemp1, ztemp2, 1
     &    , ilower, iupper, ismooth, zsmooth )
        call smooth2 ( xitot, 1, ztemp1, ztemp2, 1
     &    , ilower, iupper, ismooth, zsmooth )
c
        call smooth2 ( dhtot, 1, ztemp1, ztemp2, 1
     &    , ilower, iupper, ismooth, zsmooth )
        call smooth2 ( vhtot, 1, ztemp1, ztemp2, 1
     &    , ilower, iupper, ismooth, zsmooth )
c
        call smooth2 ( dztot, 1, ztemp1, ztemp2, 1
     &    , ilower, iupper, ismooth, zsmooth )
        call smooth2 ( vztot, 1, ztemp1, ztemp2, 1
     &    , ilower, iupper, ismooth, zsmooth )
c
        if ( lthery(7) .gt. 20 ) then
c
          idim = matdim
          idim2 = idim**2
c
          do j1=1,idim2
            call smooth2 ( difthi, idim2, ztemp1, ztemp2, j1
     &        , ilower, iupper, ismooth, zsmooth )
          enddo
c
          do j1=1,idim
            call smooth2 ( velthi, idim, ztemp1, ztemp2, j1
     &        , ilower, iupper, ismooth, zsmooth )
          enddo
c
        endif
c
      endif
c
\end{verbatim}

Boundary conditions at the magnetic axis and at the plasma edge:

\begin{verbatim}
c
c..values at "r=0"
c
        dhtot(maxis) = dhtot(jzmin)
        dztot(maxis) = dztot(jzmin)
c
        vhtot(maxis) = 0.0
        vztot(maxis) = 0.0
c
        xetot(maxis) = xetot(jzmin)
        xitot(maxis) = xitot(jzmin)
c
c..extrapolate density * diffusivity from penultimate to edge grid point
c
      if ( lthery(2) .eq. 1 ) then
c
        dhtot(medge) = dhtot(medge-1) *
     &      densi(medge-1) / densi(medge)
        vhtot(medge) = vhtot(medge-1) *
     &      densi(medge-1) / densi(medge)
c
        dztot(medge) = dztot(medge-1) *
     &      densi(medge-1) / densi(medge)
        vztot(medge) = vztot(medge-1) *
     &      densi(medge-1) / densi(medge)
c
        xetot(medge) = xetot(medge-1) *
     &      dense(medge-1) / dense(medge)
        xitot(medge) = xitot(medge-1) *
     &      densi(medge-1) / densi(medge)
c
      endif
c
c..set up total diffusivities for printout
c
      do jz=maxis,medge
        zchie(jz) = xetot(jz)
        zchii(jz) = xitot(jz)
        zdifh(jz) = dhtot(jz)
        zdifz(jz) = dztot(jz)
      enddo
c
      if ( lthery(7) .ge. 20 .and. lthery(8) .lt. 21 ) then
        do jz=maxis,medge
          zchie(jz) = zchie(jz) + thdre(jz) + thige(jz)
          zchii(jz) = zchii(jz) + thdri(jz) + thigi(jz)
          zdifh(jz) = zdifh(jz) + zddtem(jz) + zddig(jz)
          zdifz(jz) = zdifz(jz) + zddtem(jz) + zdzig(jz)
        enddo
      endif
c
\end{verbatim}

%**********************************************************************c

\subsection{Printout}

\begin{verbatim}
c
c  print theory's output
c
      if ( lprint .lt. 1 ) return
c
cbate      entry prtheory
c
 900  continue
c
c..Profiles as a function of major radius
c
      icntr = maxis
      iedge = medge + 1
c
c..total theory-based diffusivities
c
       write(nprint,10007) nstep, time
       write(nprint,10009)
       write(nprint,10008)
c
c  weithe(jz) used to be eithes(jz)/(useh*uesp)
c
      do jz=jzmin,medge
        write(nprint,101) jz, rminor(jz), xetot(jz)
     &     , xitot(jz), weithe(jz), zdifh(jz), zdifz(jz)
      enddo
c
c..electron thermal diffusivities
c
      write(nprint,104)  nstep, time
      write(nprint,105)
c
      do 920 jz=jzmin,medge
        write(nprint,106) jz, rminor(jz), thdre(jz), thige(jz)
     &  , thrbb(jz), thrbgb(jz), thkbe(jz), thnme(jz)
     &  , zchie(jz)
 920  continue
c
      write(nprint,117)
c
      do 925 jz=jzmin,medge
        write(nprint,118) jz, rminor(jz), thtie(jz)
     &  , thhfe(jz), thcee(jz), thhme(jz), zchie(jz)
 925  continue
c
c..ion thermal diffusivities
c
      write(nprint,107)  nstep, time
      write(nprint,108)
c
      do 930 jz=jzmin,medge
      write(nprint,106) jz, rminor(jz), thdri(jz), thigi(jz)
     &  , thtii(jz), thrbi(jz), thkbi(jz) ,thnmi(jz)
     &  , zchii(jz)
 930  continue
c
      write(nprint,119)
c
      do 935 jz=jzmin,medge
        write(nprint,120) jz, rminor(jz)
     &      , thrmi(jz), thhfi(jz), thcei(jz), zchii(jz)
 935  continue
c
c..hydrogen particle diffusivity
c
      write (nprint,103)  nstep, time
      write (nprint,109)
c
      do jz=jzmin,medge
        write(nprint,106)jz,rminor(jz),zddtem(jz),zddig(jz)
     &    ,zdti(jz),zdrm(jz),zdrb(jz),zdkb(jz)
     &    ,zdnm(jz),zdhf(jz),zdifh(jz)
      enddo
c
c..frequencies and 1/k_perp^2
c
      write (nprint,103)  nstep, time
      write (nprint,170)
c
      do jz=jzmin,medge
        write(nprint,171) rminor(jz), zgmitg(jz), zomitg(jz)
     &      , zgm2nd(jz), zom2nd(jz), zgmtem(jz), zomtem(jz)
     &      , zomegde(jz), zomegse(jz), zkinvsq(jz), wexbs(jz)
      enddo
c
c..Print out diffusivity matrix
c
      if ( lthery(7) .ge. 21  .and.  lthery(7) .le. 23 ) then
c
        write(nprint,161)  nstep, time
c
        write(nprint,135)
c
        do jz=jzmin,medge
          write(nprint,106) jz, rminor(jz), thigi(jz)
     &      , velthi(1,jz), (difthi(1,j2,jz),j2=1,4), zprfmx(jz)
        enddo
c
        write(nprint,162)  nstep, time
c
        write(nprint,135)
c
        do jz=jzmin,medge
          write(nprint,106) jz, rminor(jz), zddig(jz)
     &      , velthi(2,jz), (difthi(2,j2,jz),j2=1,4)
        enddo
c
        write(nprint,163)  nstep, time
c
        write(nprint,135)
c
        do jz=jzmin,medge
          write(nprint,106) jz, rminor(jz), thige(jz)
     &      , velthi(3,jz), (difthi(3,j2,jz),j2=1,4)
        enddo
c
        write(nprint,164)  nstep, time
c
        write(nprint,135)
c
        do jz=jzmin,medge
          write(nprint,106) jz, rminor(jz), zdzig(jz)
     &      , velthi(4,jz), (difthi(4,j2,jz),j2=1,4)
        enddo
c
      endif
c
c..diagnostic arrays
c
      if ( lthery(29) .gt. 2 ) then
c
        write(nprint,110)  nstep, time
        write(nprint,111)
c
        do jz=jzmin,medge
          write(nprint,106)jz,rminor(jz),threti(jz),thetth(jz)
     &      ,thfith(jz),thbpbc(jz),thnust(jz),thsrhp(jz),thdias(jz)
     &      ,zdifh(jz),zdifz(jz)
        enddo
c
      endif
c
      write(nprint,110)  nstep, time
      write(nprint,113)
c
      do jz=jzmin,medge
        write(nprint,106)jz,rminor(jz)
     &        ,zslne(jz),zslnh(jz),zslnz(jz),zslte(jz),zslti(jz)
     &        ,thlpr(jz),thlsh(jz),thrhos(jz)
      enddo
c
      if ( lthery(29) .gt. 2 ) then
c
        write(nprint,110)  nstep, time
        write(nprint,114)
c
        do jz=jzmin,medge
          write(nprint,106)jz,rminor(jz),thvthe(jz),thvthi(jz)
     &        ,thalfv(jz),thsoun(jz)/(rmajor(jz))
     &        ,zgmitg(jz),zgmtem(jz),zrsist(jz)
        enddo
c
c..dimensionless variables
c
        write(nprint,110)  nstep, time
        write(nprint,140)
c
        do jz=jzmin,medge
          write(nprint,126)jz,rminor(jz),thnust(jz)
     &      ,thrstr(jz),q(jz),zdshat(jz), thbeta(jz),thtau(jz)
        enddo
      endif
c
c..diagnostic output for the Hahm-Tang TEM when lthery(5) = 1
c
      if ( lthery(5) .eq. 1 .and. lthery(29) .ge. 5 ) then
c
        write(nprint,115)  nstep, time
        write(nprint,116)
c
      do jz=jzmin,medge
        write(nprint,106) jz, rminor(jz), zddtem(jz)
     &    , zhctem(jz), zhdtem(jz), zd1tem(jz)
     &    , zh1tem(jz), zk1tem(jz), weithe(jz)
      enddo
c
      endif
c
c..diagnostic output for Nordman-Weiland ITG model when lthery(7)=21-30
c
      if ( lthery(7) .gt. 21 .and. lthery(29) .gt. 2 ) then
c
        write(nprint,158) ieq, nstep, time
        write(nprint,159)
c
        do jz=jzmin,medge
          write(nprint,160) jz,rminor(jz),q(jz),zdshat(jz)
     &      ,zdnuhat(jz),zdbetae(jz),zdbetah(jz)
     &      ,zdbetaz(jz),zkpar(jz)
        enddo
c
      endif
c
c..diagnostic printout for the kinetic ballooning mode
c
      if ( lthery(17) .gt. -1 .and. lthery(29) .ge. 5 ) then
c
        write(nprint,110)  nstep, time
        write(nprint,112)
c
      do jz=jzmin,medge

        write(nprint,106) jz, rminor(jz)
     &    , zbprima(jz), zbc1a(jz), thbpbc(jz), zbc2a(jz)
     &    , thlpr(jz), thlarp(jz), zdka(jz)
      enddo
c
      endif
c
c..diagnostic output for the Carreras-Diamond RB model
c
      if ( lthery(15) .le. 2 .and. lthery(29) .ge. 5 ) then
c
        write(nprint,110)  nstep, time
        write(nprint,121)
c
      do jz=jzmin,medge
c
        write(nprint,124) jz, rminor(jz)
     &    , thlamb(jz), zfstarrb(jz), zfdiarb(jz)
c
      enddo
c
      endif
c
c..diagnostic output for the Guzdar-Drake RB model
c
      if ( lthery(13) .gt. 0  .and.  lthery(29) .ge. 5 ) then
c
        write(nprint,110)  nstep, time
        write(nprint,122)
c
      do jz=jzmin,medge
c
        write(nprint,125) jz, rminor(jz)
     &    , zwstrnm(jz), znuiz(jz), zcmiz(jz), zalphz(jz)
c
      enddo
c
      endif
c
c..diagnostic output for the neoclassical MHD model
c
      if ( lthery(15) .eq. 2  .and.  lthery(29) .ge. 5 ) then
c
        write(nprint,110)  nstep, time
        write(nprint,123)
c
      do jz=jzmin,medge
c
        write(nprint,106) jz, rminor(jz)
     &    , zxnmz(jz), zexbnz(jz),zdelez(jz),zlambz(jz)
     &    , zgammz(jz),zflamz(jz),zfgamz(jz),zfdiaz(jz)
c
      enddo
c
      endif
c
c**********************************************************************c
c Format statements
c
10007 format(/,10x,'transport coefficients from theory'
     &  ,' at step ',i5,'  at time ',0pf13.6,' sec',/)
10008 format(4x,'zone',6x,'radius',9x,'chi-elc',8x,'chi-ion',
     #       8x,'intrchg',10x,'zdifh',10x,'zdifz')
10009 format(16x,'m',13x,'m*m/s',10x,'m*m/s',12x,'w',
     #       12x,'m*m/s')
c
 101  format(5x,i2,6x,0pf6.3,8x,5(1pe11.4,4x))
 103  format(/,10x,'hydrogenic ion diffusion coefficients'
     &  ,' at step ',i5,'  at time ',0pf13.6,' sec',/,
     &       10x,37('-'))
 104  format(/,10x,'electron theoretical diffusion coefficients'
     &  ,' at step ',i5,'  at time ',0pf13.6,' sec',/,
     &       10x,43('-'))
 105  format(12x,'m',9x,'m2/s',6(8x,'m2/s'),/
     & 5x,'jz',3x,'radius',6x,'xedr',8x,'xeig',7x
     &         ,'xerb-B',6x,'xerbgB',7x,'xekb',8x,'xenm'
     &         ,8x,'xethe',7x,' ',6x,' ')
 106  format(5x,i2,3x,0pf6.3,4x,9(1pe10.3,2x))
 107  format(/,10x,'ion theoretical diffusion coefficients'
     &  ,' at step ',i5,'  at time ',0pf13.6,' sec',/,
     &       10x,37('-'))
 108  format(12x,'m',9x,'m2/s',6(8x,'m2/s'),/
     &  5x,'jz',3x,'radius',6x,'xidr',8x,'xiig',8x,'xiti',8x
     &         ,'xirb',8x,'xikb',8x,'xinm'
     &         ,8x,'xithe',7x,' ',6x,' ')
 109  format(12x,'m',9x,'m2/s',7(8x,'m2/s'),/
     &  5x,'jz',3x,'radius',6x,'dhdr',8x,'dhig',8x,'dhti',8x
     &         ,'dhrm',8x,'dhrb',8x,'dhkb',8x,'dhnm'
     &         ,8x,'dhhf',8x,'dhthe')
 110  format(/,10x,'diagnostic arrays from sbrtn THEORY'
     &  ,' at step ',i5,'  at time ',0pf13.6,' sec',/)
 111  format(5x,'jz',3x,'radius',7x,'eta_i ',4x,'eta_i^th'
     #       ,5x,'f_ith',5x,'beta_ratio',5x,'nu_e^*',5x,'Reynolds'
     #       ,5x,'zfdias',6x,'difhyd',6x,'difimp'/)
 112  format(5x,'jz',3x,'radius'
     #       ,7x,'zbprim',6x,'zbc1',8x,'bp/zbc1',5x,'zbc2',8x,'L_p '
     #       ,8x,'zlarpo',6x,'zdk',9x,' ')
 113  format(5x,'jz',3x,'radius'
     #       ,7x,'L_ne',8x,'L_nH',8x,'L_nZ',8x,'L_Te',8x,'L_Ti'
     #       ,8x,'L_p',9x,'L_S',9x,'rho_S')
 114  format(5x,'jz',3x,'radius'
     &       ,8x,'v_the',7x,'v_thi',9x,'v_A',5x,'c_s / R'
     &       ,2x,'gamma_etai',3x,'gamma_tem',2x,'resistivity')
c
 115  format (
     & /,10x,'special diagnostic output for the Hahm-Tang TEM'
     &  ,' at step ',i5,'  at time ',0pf13.6,' sec'
     & /,10x,'-----------------------------------------------')
 116  format (5x,'jz',3x,'radius',7x,'zddtem',7x,'zhctem',7x,'zhdtem'
     &   ,7x,'zd1tem',7x,'zh1tem',7x,'zk1tem',7x,'weithe',/)
 117  format(/,12x,'m',9x,'m2/s',4(8x,'m2/s'),/
     & ,5x,'jz',3x,'radius',6x,'xeti',8x,'xehf',8x,'xece',8x
     &   ,'xehm',8x,'xethe')
 118  format(5x,i2,3x,0pf6.3,4x,5(1pe10.3,2x))
 119  format(/,12x,'m',9x,'m2/s',3(8x,'m2/s'),/
     & ,5x,'jz',3x,'radius',6x,'xirm',8x,'xihf',8x,'xice',8x,'xithe')
 120  format(5x,i2,3x,0pf6.3,4x,4(1pe10.3,2x))
 121  format(5x,'jz',3x,'radius'
     &       ,6x,'thlamb',6x,'zfstar',6x,'zfdias')
 122  format(/,5x,'jz',3x,'radius'
     &       ,6x,'zwstrp',6x,'znui',9x,'zcmi',6x,'zalphe')
 123  format(/,5x,'jz',3x,'radius'
     &       ,6x,'zxnmz ',6x,'zexbnz',6x,'zdelez',6x,'zlambn'
     &       ,6x,'zgammh',6x,'zflamb',6x,'zfgamh',6x,'zfdiaz')
 124  format(5x,i2,3x,0pf6.3,4x,4(1pe10.3,2x))
 125  format(5x,i2,3x,0pf6.3,4x,5(1pe10.3,2x))
 126  format(5x,i2,3x,0pf6.3,1x,9(1pe10.3,1x))

c
 130  format (
     & /,10x,'Profiles as a function of major radius'
     & /,t4,'rmajor(m)',t17,'ne(m^-3)',t30,'Te(keV)',t43,'Ti(keV)'
     &  ,t56,'Zeff')
 132  format (5(2x,1pe11.4))
c
 135  format (5x,'jz',3x,'radius',2x,'diffusivity',4x,'velthi(*)'
     &  ,'  difthi(*,1) difthi(*,2) difthi(*,3) difthi(*,4)'
     &  ,t95,'perform')
c
 140  format (5x,'jz',3x,'radius',4x,'nustar',4x,'rhostar'
     &  ,6x,'q',9x,'shear',6x,'beta',6x,'Ti/Te')
c
 158  format (
     &  /,10x,'diagnostic output for Weiland ITG model'
     &  ,' with ',i2,' equations'
     &  ,' at step ',i5,'  at time ',0pf13.6,' sec'
     &  /,10x,'---------------------------------------')
c
 159  format (/,5x,'jz',3x,'radius',5x,'zq',7x,'zshat',6x
     &  ,'znuhat',5x,'zbetae',5x,'zbetah',5x,'zbetaz',5x
     &  ,'zkparl')
 160  format(5x,i2,3x,0pf6.3,1x,9(1pe10.3,1x))
 161  format(/,10x,'Ion thermal channel of diffusivity matrix'
     &  ,' at step ',i5,'  at time ',0pf13.6,' sec',/)
c
 162  format(/,10x,'Hydrogen particle channel of diffusivity matrix'
     &  ,' at step ',i5,'  at time ',0pf13.6,' sec',/)
c
 163  format(/,10x,'Electron thermal channel of diffusivity matrix'
     &  ,' at step ',i5,'  at time ',0pf13.6,' sec',/)
c
 164  format(/,10x,'Impurity particle channel of diffusivity matrix'
     &  ,' at step ',i5,'  at time ',0pf13.6,' sec',/)
c
 170  format(/,3x,'radius',3x,'gammaitg',3x,'omegaitg'
     & ,3x,'gamma2nd',3x,'omega2nd',3x,'gammatem',3x,'omegatem'
     & ,3x,'omega_De',3x,'omega_*e',3x,'1./k_r^2',3x,'wexb')
 171  format(3x,0pf6.3,1x,11(1pe10.3,1x))
c
 990  return
      end
\end{verbatim}

%**********************************************************************c

\section{Recommended Input Values}

After benchmarking the Weiland ITG model with parallel ion motion and
finite beta effects and the Guzdar-Drake resistive ballooning model with Ohmic, 
L-mode, and H-mode shots at Lehigh University, the following values are recommended 
for the best fit to experimental data:

                               \begin{verbatim}

 ! Multi Mode Model in sbrtn THEORY as of Mar 1996
 !
 lthery(3)  = 4  ! use hydrogen and impurity density scale length in etaw14
 lthery(4)  = 1  ! use neoclassical resistivity in sbrtn THEORY
 lthery(5)  = 2  ! skip dissipative trapped electron mode (DTEM)
 lthery(6)  = 1  ! min[1.0,0.1/\nu_e^*] transition to collisionless TEM
 lthery(7)  = 27 ! Weiland ITG model with 10 equations
 lthery(8)  = 21 ! Use effective diffusivities
 lthery(9)  = 2  ! linear ramp form for f_ith
 lthery(13) = 3  ! Guzdar-Drake resistive ballooning mode (1994)
 lthery(14) = 1  ! Single iteration for lambda in RB mode
 lthery(15) = 0  ! Neoclassical MHD model w/ Callen corrections (no longer used)
 lthery(16) = 0  ! Circulating Electron model
 lthery(17) = 1, ! 1995 kinetic ballooning model
 lthery(25) = 0  ! Rebut-Lallia-Watkins model
 lthery(26) = 100, ! time-step for diagnostic output
 lthery(27) = 1, ! replace negative diffusivity with convective velocity
 lthery(29) = 5, ! more printout
 lthery(30) = 1, ! retain sign of gradient scale lengths
 lthery(31) = 1, ! Monotonic gradient scale lengths at axis
 lthery(32) = -2, ! smooth inverse gradient scale lengths over -2 pts
 !
 !  misc. parameters for sub. theory
 !
 cthery(1)  = 0.0      ! Divertor shear off
 cthery(3)  0.5        ! minimum shear
 cthery(7)  = 1.0      ! Width of linear ramp in ITG mode
 cthery(8)  = 3.5      ! for fbeta-th in kinetic ballooning
 cthery(12) =-4.0      ! Elongation exponent for TEM,ITG modes
 cthery(14) =-4.0      ! Elongation exponent for RB mode
 cthery(15) =-4.0      ! Elongation exponent for KB mode
 cthery(17) = 0.0      ! turn off convective correction in sbrtn THEORY
 cthery(20) = 1.0      ! min[c20,c21*0.1/\nu_e^*] in Dominguez-Waltz DTEM
 cthery(21) = 1.0      ! min[c21,c21*0.1/\nu_e^*] in Dominguez-Waltz DTEM
 cthery(22) = 0.25     ! exp[-c34(Ti/Te-1)**2] multiplying TEM mode
 cthery(23) = 0.95     ! Numerical correction in D^{DR} of original TEM model
 cthery(29) = 0.0,     ! No critical gradient in OHE eta_i mode
 cthery(30) = 1.0      ! eta-ith = max[c30, c31*L_n/R]
 cthery(31) = 5.5      ! eta-ith = max[c31, c31*L_n/R]
 cthery(34) = 0.0      ! exp[-c_{34} (Ti/Te-1.)**2] multiplying ITG mode
 cthery(35) = 5.0      ! exp[-c35*L_n, c36*L_Ti] multiplying ITG mode
 cthery(36) = 4.0      ! exp[-c36*L_n, c36*L_Ti] multiplying ITG mode
 cthery(41) = 2.0      ! Toroidal mode # suggested by Carreras in CD RB model
 cthery(42) = 1.00     ! diamagnetic stabilization of Carreras-Diamond (CD) RB mode
 cthery(43) = 0.16667  ! ( 1 + f_\star )**(-cthery(43)) in CD model
 cthery(44) = 4.0      ! E x B multiplier in RB mode in CD model
 cthery(45) = 1.0      ! Numerical correction in RB mode in CD model
 cthery(47) = 8.0      ! q-exponent for Lambda in RB mode in CD model
 cthery(48) = 0.0      ! ln(256/Lambda^3) used in single iteration of Lambda in CD model
 cthery(49) = 1.0      ! q-exponent for D^{RB} in RB mode in CD model
 cthery(50) = 5*100.0, ! limit L_n to c_50 times major radius
 ! cthery(60) = 5.0    ! Limit local change in diffusivity to 20% (not used with Weiland)
 cthery(70) = 1.0      ! Growth rate multiplier in NMHD mode
 cthery(71) =-4.0      ! Elongation exponent in NMHD mode
 cthery(72) = 0.0      ! Coeff of chi_e0^{NM} added to D^{NM} in NMHD mode
 cthery(73) = 1.0      ! Coeff of D_p^{NM} added to chi_e^{NM} in NMHD mode
 cthery(74) = 0.0      ! Coeff of chi_e0^{NM} added to chi_i^{NM} in NMHD mode
 cthery(75) = 1.0      ! (c75*omega_*/gamma)^6 in dia stabil of NMHD mode
 cthery(76) = 0.16667  ! Exponential in dia stabil of NMHD mode
 cthery(77) = 10.0     ! Average toroidal mode number in NMHD mode
 cthery(78) 1.0        ! coeff of beta_prime_1 in kinetic ballooning mode
 cthery(79) 1.0        ! coeff of beta_prime_2 in kinetic ballooning mode
 cthery(80) = 1.0      ! Multiplier in diffusivity for CE mode
 cthery(81) =-4.0      ! Elongation exponent for CE mode
 cthery(82) = 1.0      ! Multiplier in diffusivity for High-m Tearing mode
 cthery(83) =-4.0      ! Elongation exponent for High-m Tearing mode
 cthery(85) = 2.0,     ! Specify diamagnetic stabilization in Guzdar-Drake RB model
 cthery(86) = 0.15     ! Dia stabilization in Guzdar-Drake RB model
 cthery(111) = 0.0,    ! difthi -> velthi for chi_i
 cthery(112) = 0.0,    ! difthi -> velthi for hydrogen
 cthery(113) = 0.0,    ! difthi -> velthi for chi_e
 cthery(114) = 0.0,    ! difthi -> velthi for impurity
 cthery(119) = 1.0,    ! coeff of finite beta in etaw14
 cthery(121) = 1.0,    ! set fast particle fraction for use in etaw14
 cthery(123) = 1.0,    ! coeff of k_ii in etaw14
 cthery(124) = 0.0,    ! coeff of nuhat in etaw14
 !
 ! contributions to fluxes and interchange(for sub. theory)
 !
 !particles elec-energy ion-energy
 fdr=0.00 0.00 0.00
 fig=0.80 0.80 0.80
 frm=0.00 0.00 0.00
 fkb=1.00 0.65 0.65
 frb=1.00 1.00 1.00
 fhf=0.00 0.00 0.00
 fmh=0.00 0.00 0.00
 fec=0.00 0.00 0.00
 
                \end{verbatim}

%**********************************************************************c

\section{Changes to the Default Model}
The default model used in subroutine {\tt theory} was developed by
C.~E. Singer as documented
in ``Theoretical Particle and Energy Flux Formulas for Tokamaks,''
Comments on Plasma Physics and Controlled Fusion, {\bf 11}, 165 (1988)
(\cite{Comments}, hereafter referred to as the
Comments paper).  The flux formulas are identical to
those in the comments paper, with three exceptions.  First, a
correction due to Romanelli to the $\eta_{i}$-mode threshold
is included \cite{Romanelli}.
Second, a minor change in the high-collisionally
cut-off on the $\eta_{e}$ mode is included.  Third,
an option is included for specifying a harmonic divergence
in the shear used in the formulas.This option is included in case the user
does not
want to include enough moments in the equilibrium solver to
get an adequate description of the shear near the separatrix in
H-mode plasmas.
Additional coding which allows a stand-alone calculation of the transport
fluxes is also outlined.

%**********************************************************************c

\section{Numerical Methods}

For simplicity in constructing the coding, we first compute
all of the anomalous transport fluxes (given in Sections 1.2-1.7 of
the Comments paper~\cite{Comments})
from diffusivities of
the form $\Gamma_{a}/(n_{a}/L_{ni})$ and $Q_{j}/(n_{j}T_{j}/L_{Tj})$,
where $\Gamma_{a}$ and $Q_{j}$ are the particle and energy fluxes.
Effective diffusivities associated with ``convective''
fluxes normally controlled by the input variables {\tt cthery(68)} and
{\tt cthery(69)} are then subtracted to get thermal diffusivities.
A related problem is that the ion energy fluxes associated with
the Ware pinch effect appear not to have been coded into BALDUR
\cite{BALDUR}.  Rather than trying to sort out these complications,
one might eventually want to include in subroutine {\tt theory}
coding for the complete simplified
neoclassical energy fluxes defined in the Comments paper.
Input switches could then be defined so that the user who wants to
substitute the more complicated pieces of these neoclassical
energy fluxes with expressions already available in BALDUR
can turn off the parts in subroutine {\tt theory} which
would give duplication.
Care would have to be taken that calling subroutine {\tt theory}
with default input values for parameters which previously
controlled neoclassical energy fluxes will result
in a self-consistent formulation.  In other words, calling
subroutine {\tt theory}
with all default input parameters should produce no additional
particle or energy fluxes.  At present, however,
we retain the existing neoclassical fluxes and add only
a new set of options for anomalous fluxes.

As far as finite differencing goes, we calculate
parameters which include scale heights in a manner analogous to
that used in subroutine {\tt empirc}.  This means that a call to subroutine
{\tt xscale} preceeds the call to  subroutine {\tt theory},
just as for {\tt empirc}.  The more stringent singularity-prevention
limitations described in Section 1.8 of the Comments paper are applied
to the scale heights in subroutine {\tt theory}.

%**********************************************************************c

\section{Calculation Order}

Here we describe, in narrative and coded form,
the subroutine structure and the calculations required
in the order in which they occur in the new subroutine.
For the present, we include only anomalous fluxes due to microinstabilities.
After we have some practice using the new subroutine,
we will decide about coding the
post-sawtooth transport enhancement and neoclassical formulas.
In a Table 1, we list the coding
and algebraic symbols for some preexisting BALDUR
{\tt common} variables which are needed in the new subroutine.

Some notes about interaction with BALDUR and its coding
conventions are in order here.
The array index set equal to {\tt 1} indicates
that parameters defined at both zone centers and zone boundaries
are being used at zone centers.  The index {\tt jz}
indicates a dummy index for the zones.  Care is taken
with the innermost dummy zone associated with
BALDUR's boundary conditions.  Note that the order of these
indices are interchanged in the arrays {\tt rmajor(jz)} and
{\tt rmajor(jz)} compared to other such variables.
This was evidently done deliberately when Glenn Bateman
added these to the older
BALDUR variables.
We have assumed that {\tt rminor(jz)} is
the midplane halfwidth required for computing the inverse
aspect ratio in the Comments paper formulas.
The real variables local to {\tt theory}
(beginning with {\tt z} by OLYMPUS convention)
are all in standard units except tempretures which are
in keV, so the
convention of indicating units with
the last coding letter for these variables is {\it not}
followed.  For parallism  with the
coding for {\tt empirc}, we do
maintain this units convention in the new
{\tt common} variables added, however.
Coding conventions local to subroutine {\tt theory}
are (a) variables beginning with {\tt zc}
depend only on physical constants, (b) variables whose values are
modified later in the subroutine are indented an extra space,
and (c) coding statements giving the final definition
of each variable in the subroutine (as originally written)
are given a statement label which is identical to
the corresponding equation number in the original
version of this documentation document.

%**********************************************************************c

\subsection{Code Verification}

To aid in performing checks the coding, Table~2
is provided here to give a convenient set of numerical results
\cite{Comments}.
Also shown in Table~2 are some other quantities useful
for estimating transport fluxes.  (In constructing this table, we have
assumed $n_{i}\approx n_{e}$, so the more exact expressions given in the
text above
should be used for consistency in detailed multispecies transport code work.)
As one check on the coding, by setting the variables in Table~2 to
unity using the stand-alone driver routine and verifying that the coding
described above reproduces the hand-calculated numerical coefficients
shown in Table~2.

\section{Nmemonics}

The mode abbreviations used here are
\begin{center}
\begin{tabular}{llll}
    &             &                                         &        \\
    & {\tt dr}    & drift (other than $\eta_{i}$-mode)      &        \\
    & {\tt ig}    & $\eta_i$-mode                           &        \\
    & {\tt rm}    & rippling mode                           &        \\
    & {\tt rb}    & resistive ballooning                    &        \\
    & {\tt kb}    & kinetic ballooning                      &        \\
    & {\tt hf}    & high frequency ($\eta_{e}$)             &        \\
    & {\tt mh}    & neoclassical MHD                        &        \\
    & {\tt ec}    & circulating electron                    &        \\
    & {\tt hm}    & high-m tearing                          &        \\
    & {\tt rlw}   & Rebut-Lallia-Watkins                    &        \\
    &             &                                         &
\end{tabular}
\end{center}

Table 3 lists coding symbols for
variables local to subroutine {\tt theory}, statement labels for variables
defined in labelled statements (which correspond to
equation numbers in the present document), the
corresponding algebraic symbol, and the nmemonic
used to generate the coding symbol.
 
%**********************************************************************c

\begin{thebibliography}{99}

\bibitem{BALDUR} C. E. Singer, D. E. Post, D. R. Mikkelsen,
M. H. Redi, A. McKenney, et al., ``BALDUR: A One-dimensional
Plasma Transport Code,'' Comp. Phys. Communications {\bf 48}
(1988, in press).

\bibitem{Bateman} G. Bateman, ``Multi-Mode Simulations of Transport
in TFTR,'' IAEA Technical Committee Meeting on Tokamak Transport,''
(8-10 October, 1990).

\bibitem{Callen} J. Callen, University
of Wisconsin personal communication (dated 4/6/91).

\bibitem{hortoncomm}
M.~Ottoviani, W.~Horton, M.~Erba
``Thermal Transport from a Phenomonological Description of ITG-Driven
Turbulence'', private communication from W.~Horton, March 1996

\bibitem{carr89a}
B.~A. Carreras and P.~H. Diamond,
``Thermal diffusivity induced by resistive pressure-gradient-driven
turbulence,'' Physics of Fluids B {\bf 1} (1989) 1011.

\bibitem{Carreras} B. Carreras, Oak Ridge National Laboratory personal
communication (27 Nov., 1989).

\bibitem{drake93}
P.~N. Guzdar, J.~F. Drake, D. McCarthy, A.~B. Hassam, and C.~S. Liu,
``Three-dimensional Fluid Simulations of the Nonlinear Drift-resistive
Ballooning Modes in Tokamak Edge Plasmas,'' Physics of Fluids B {\bf 5}
(1993) 3712.

\bibitem{guzcomm} P.N. Guzdar, University of Maryland, personal communication
 (11/94).

\bibitem{drakecom} J.F. Drake, University of Maryland, personal
communication (6/94).

\bibitem{drakecom2} J.F. Drake, University of Maryland, personal
communication (3/95).

\bibitem{Comments} C. E. Singer, ``Theoretical Particle and Energy
Flux Formulas for Tokamaks,'' Comments on Plasma Physics and Controlled
Fusion {\bf 11} (1988) 165.

\bibitem{gbcomm} G. Bateman, Princeton Plasma Physics Laboratory, 
personal communication (9/95).

\bibitem{diam85a} P.~H. Diamond, P.~L. Similon, T.~C. Hender, and
B.~A. Carreras, ``Kinetic theory of resistive ballooning modes,''
Phys. Fluids {\bf 28} (1985) 1116--1125.

\bibitem{domn87} R. R. Dominguez and R. E. Waltz,
Nucl. Fusion {\bf 27} (1987) 65.

\bibitem{domn89a} R. R. Dominguez and M. N. Rosenbluth,
Nuclear Fusion {\bf 29} (1989) 844.

\bibitem{Dominguez} R. Dominguez, ``DIII-D Hot Ion Plasma
Simulations,'' preliminary draft of General
Atomics Report GA-A20382 (11 Feb., 1991).

\bibitem{DW} R. R. Dominguez and R. E. Waltz, ``Ion Temperature
Gradient Mode and H-mode Conefinement,'' Nuclear Fusion
{\bf 29} (1989) 885.

\bibitem{nord90a} H. Nordman, J. Weiland, and A. Jarmen, 
``Simulation of toroidal drift mode turbulence driven by 
temperature gradients and electron trapping,'' 
Nucl. Fusion {\bf 30} (1990) 983--996.
 
\bibitem{Ghanem} E-S. Ghanem, C. E. Singer, G. Bateman, and D.
P. Stotler, ``Multiple Mode Model of Tokamak Transport,''
Nuclear Fusion {\bf 30} (1990) 1595.

\bibitem{Ghanphd} E-S. Ghanem, University of Illinois at
Urbana-Champaign Department of Nuclear Engineering
PhD Thesis (Jan, 1991).

\bibitem{Guzdar} P. Guzdar, private communication (18 April, 1988).

\bibitem{hahm87a} T.~S. Hahm, P.~H. Diamond, P.~W. Terry, L. Garcia,
and B.~A. Carreras, Physics of Fluids {\bf 30} (1987) 1452.

\bibitem{hama89a} S. Hamaguchi and W. Horton,
``Fluctuation Spectrum and Transport from Ion Temperature Gradient Driven
Modes in Sheared Magnetic Fields,''
University of Texas, Institute for Fusion Studies report
IFSR \# 383 (August 1989).

\bibitem{hahm89a} T.~S. Hahm and W.~M. Tang,
Physics of Fluids {\bf 1} B (1989) 1185.

\bibitem{hahm90a}
T.~S. Hahm and W.~M. Tang,
``Weak Turbulence Theory of Collisionless Trapped Electron Driven
Drift Instability in Tokamaks,''
Princeton Plasma Physics report PPPL-2721 (Sept, 1990).

\bibitem{hahm90b}
T.~S. Hahm and W.~M. Tang, private communication.

\bibitem{Hamaguchi} S. Hamaguchi aand W. Horton, ``Fluctuation
Spectrum and Transport from Ion Temperature Gradient
Driven Modes in Sheared Magnetic Fields,'' University of Texas
Institute for Fusion Studies Rep. IFSR \#383 (August, 1989).

\bibitem{HD} T. S. Hahm and P. H. Diamond, Phys. Fluids {\bf 30} (1987) 133.

\bibitem{Hirshman} S. P. Hirshman and D. Sigmar, Nucl. Fusion {\bf 21}
(1981) 1079.

\bibitem{IAEA1986} J. D. Callen, W. X. Qu, K. D. Siebert,
B. A. Carreras, K. C. Shaing, and I. A. Spong,
in {\it Plasma Physics and Controlled Nuclear Fusion Research}
(IAEA, Vienna, 1987) Vol II (1986) p. 157.

\bibitem{IAEA} C. E. Singer, W. Choe, D. Cox, T. Djemil,
E. Ghanem, J, Kinsey, G. Miley, J. Park, D. Ruzic,
S. Hu, N. Tiouririne-Ougouag, V. Varadarajan,
R. R. Dominguez, R. E. Waltz, and G. Bateman,
``Predictive Modelling of Tokamak Plasmas,''
Thirteenth Conf. on Plasma Physics and Controlled Fusion
Research (Washington, October, 1991), paper IAEA-CN-53/D-1-1.

\bibitem{JCP} S. P. Hirshman and S. Jardin, Phys. Fluids {\bf 22} (1978) 731.

\bibitem{Kinsey} J. Kinsey, University of Illinois at
Urbana-Champaign Department of Nuclear Engineering
Masters Thesis (May, 1991).

\bibitem{Kwon} O. J. Kwon, P. H. Diamond, and H. Biglari,
Phys. Fluids {\bf B2} (1990) 291.

\bibitem{lee86a} G. S. Lee and P. H. Diamond,
``Theory of ion-temperature-gradient-driven turbulence in tokamaks,''
Physics of Fluids {\bf 29} (1986) 3291.

\bibitem{matt89a} N. Mattor and P.~H. Diamond, 
Physics of Fluids {\bf 1} B (1989) 1980.

\bibitem{Redi} M. Redi and G. Bateman, ``Transport
Simulations of TFTR Experiments to Test Theoretical
Models for $\chi_{e}$ and $\chi_{i}$, Princeton
Plasma Physics Laboratory Rep. PPPL-2694 (August, 1990).

\bibitem{RLW88a}
P.~H. Rebut, P.~P. Lallia, and M.~L. Watkins,
``The Critical Temperature Gradient Model of Plasma Transport:
Applications to JET and Future Tokamaks,''
IAEA Nice Meeting, Vol. II, 191 (1988).

\bibitem{rebu91a}
P.~H. Rebut, M.~L. Watkins, D.~J. Gambier, and D. Boucher,
``A Program toward a fusion reactor,''
Phys. Fluids B {\bf 3} (1991) 2209--2219.

\bibitem{Romanelli} F. Romanelli, Joint European Undertaking
Report JET-IR-16 (1987).

\bibitem{Rosenbluth} R. R. Dominguez and M. N. Rosenbluth,
Nucl. Fusion {\bf 29} (1989) 844.

\bibitem{Ross} D. Ross, P. H. Diamond, J. F. Drake,
F. L. Hinton, F. W. Perkins, W. M. Tang, R. E. Waltz,
and S. J. Zweben, ``Thermal and Particle Transport
for Ignition Studies,'' DOE/ET-53193-7 and
University of Texas Fusion Research Center Report
FRCR \#295 (1987).

\bibitem{Sherwood} R. Dominguez and
R. E. Waltz, Sherwood Theory Meeting Abstracts (1988).

\bibitem{Singer} C.E.Singer, G.Bateman, and D.D.Stotler,
``Boundary Conditions for OH, L, and H-mode Simulations,''
Princeton University Plasma Physics Report PPPL-2527 (1988).

\bibitem{Waltz} R. E. Waltz and R. R. Dominguez, Phys. Fluids
{\bf B1} (1989) 1935.

\end{thebibliography}

%**********************************************************************c

\begin{table}
\begin{center}
\underline{Table 1.  Variables already in {\tt common}}
\end{center}
\begin{tabular}{lll}
                   &                      &           \\
\underline{Symbol} & \underline{Coding}   & Units     \\
                   &                      &           \\
$A_{i}$            & {\tt aimass(jz)}     &           \\
$R$                & {\tt rmajor(jz)}     & cm         \\
$r$                & {\tt rminor(jz)}     & cm         \\
$B_{o}$            & {\tt btor(jz)}       & T         \\
$q$                & {\tt q(jz)}          &           \\
$n_{e}$            & {\tt dense(jz)}      & m$^{-3}$  \\
$L_{ne}$           & {\tt slnes(jz)}      & cm         \\
$L_{ni}$           & {\tt slnis(jz)}      & cm         \\
$L_{p}$            & {\tt slprs(jz)}      & cm         \\
$L_{T_{e}}$        & {\tt sltes(jz)}      & cm         \\
$L_{T_{i}}$        & {\tt sltis(jz)}      & cm         \\
$T_{e}$            & {\tt tekev(jz)}      & keV        \\
$T_{i}$            & {\tt tikev(jz)}      & keV       \\
$Z_{eff}$          & {\tt xzeff(jz)}      &           \\
$\theta_{shear}$   & {\tt slbps(jz)}      &           \\
$C_{vp}^{e}$       & {\tt cthery(68)}     &           \\
$C_{vp}^{i}$       & {\tt cthery(69)}     &           \\
Center zone index        & {\tt maxis}    &       \\
First real zone index    & {\tt jzmin}    &       \\
Number of zones computed & {\tt medge}    &
\end{tabular}
\end{table}


\begin{table}
\begin{center}
\underline{Table 2.  Nominal Diffusivities}

\begin{tabular}{ll} &                                                       \\
$D_{eff}^{W}$       & $\equiv -\Gamma_{a}^{W}/(\partial n_{a}/\partial r)$  \\
                    & $=2E_{o}B_{o}^{-1}L_{ni}\epsilon^{3/2}q^{-1}
                       [1+(\nu_{e}^{*})^{1/2}+\nu_{e}^{*}]^{-1}$            \\
                    & \\
$D_{eff}^{P}$       & $\equiv -\Gamma_{I}^{P}/(\partial n_{I}/\partial r)$  \\
                    & $=4.02T_{i}^{3/2}B_{o}^{-2}R_{o}^{-1}qA_{i}^{1/2}
                       A_{I}^{1/2}Z_{I}^{-1}(1+1.5\eta_{i})\eta_{I}^{-1}$   \\
                    & \\
${\hat D}$          & $\equiv \epsilon ^{1/2}\omega_{e}^{*}/k_{\perp}^{2}$  \\
         & $=10.8T_{e}^{3/2}B_{o}^{-2}L_{ni}^{-1}\epsilon^{1/2}A_{i}^{1/2}$ \\
      & $\omega_{e}^{*}/\nu_{eff}=10.1(\ln \lambda )^{-1}n_{20}^{-1}T_{e}^{2}
         L_{ni}^{-1}\epsilon A_{i}^{-1/2}Z_{eff}^{-1}$                       \\
                    & \\
${\hat D}_{i}$      & $=15.2T_{e}T_{i}^{1/2}B_{o}^{-2}
                         R_{o}^{-1/2}L_{ni} ^{-1/2}
                         A_{i}^{1/2}\eta_{i}^{1/2}$                         \\
                    & \\
$D_{\nabla \eta}$   & $=3.16\times 10^{-4}(\ln \lambda )^{1/3}
               E_{o}^{4/3}n_{20}^{1/3}T_{i}^{-5/6}B_{o}^{-4/3}R_{o}^{2}r^{2/3}
                L_{\sigma}^{-4/3}q^{2}A_{i}^{1/6}Z_{I}^{2/3}{\hat s}^{-2}$  \\
                    & \\
$\chi_{e}^{RB}$     & $\equiv -Q_{e}^{RB}/(n_{e}\partial T_{e}/\partial r)
                       =2.29\times 10^{-7}(\ln \lambda )^{3/2}\Lambda_{S}^{2}
                       (f_{*}^{-1}+1)^{-1/4}$                              \\
  & $\ \ \  n_{20}^{5/2}T_{e}^{-7/4}T_{i}^{-3/4}
   (T_{e}+T_{i})^{2}B_{o}^{-4}R_{o}^{3/2}
   L_{ni}^{1/2}L_{p}^{-3/2}q^{5}A_{i}^{1/4}Z_{eff}^{3/2}{\hat s}^{-3/2}$  \\
 & $\Lambda_{S} \ =9.70+0.98\log _{10}[(\ln \lambda )^{-1}
  n_{20}^{-1}T_{e}^{3/2}(T_{e}+T_{i})^{-1/2}
  B_{o}^{2}R_{o}A_{i}^{-1/2}Z_{eff}^{-1}]$                                 \\
 & $f_{*}^{-1}=3.27\times 10^{-11}(\ln \lambda )^{2}n_{20}^{2}T_{e}^{-3}
    T_{i}^{-3}(T_{e}+T_{i})^{2}L_{ni}^{2}q^{4}A_{i}^{-1}Z_{eff}^{2}$         \\
                    & \\
$D^{KB}$ & $=1.94T_{e}^{1/2}T_{i}B_{o}^{-2}L_{ni}^{-1}A_{i}^{1/2}f_{\beta th}
            (1+\beta '/\beta_{c1}')[1-\beta '/\beta_{c2}',0]_{max}$        \\
                    & \\
$\chi_{e,Md}$ & $=0.530n_{20}^{-1}T_{e}^{1/2}R_{o}^{-1}q^{-1}(1+\eta_{e})
                 \eta_{e}{\hat s}$                                         \\
                    & \\
$\chi_{RR}$ & $=5.97\times 10^{6}a^{2}T_{e}^{1/2}R_{o}^{-1}q^{-2}$        \\
  & $t_{rec}(U_{R}=.5)=1.88\times 10^{-2}(\ln \lambda )^{-1/2}n_{20}^{1/4}
     T_{e}^{3/4}B_{o}^{-1/2}R_{o}^{1/2}aA_{i}^{1/4}Z_{eff}^{-1/2}$
\end{tabular}
\end{center}
\end{table}



\begin{table}
\begin{center}
\underline{Table 3a-d.  Symbols for Variables Local to {\tt theory}}

\begin{tabular}{lllp{3.in}}
 & & & \\
\underline{Symbol} & \underline{Eq.} &    & \underline{Meaning} \\
 & & & \\
{\tt zai   } &    & $A_{i}$
                    & Average {\tt a}tomic mass of {\tt i}ons    \\
{\tt zb    } &    & $B_{0}$
                    & Toroidal {\tt B}-field                     \\
{\tt zbc1  } & 40 & $\beta_{c1}'$
                    & {\tt b}eta {\tt c}ritical gradient-{\tt 1} \\
{\tt zbc2  } & 58 & $\beta_{c2}'$
                    & {\tt b}eta {\tt c}ritical gradient-{\tt 2} \\
{\tt zbeta } &  3 & $\beta$
                    & {\tt beta}                                 \\
{\tt zbetap} &  9 & $\beta_{p}$
                    & {\tt beta} {\tt p}oloidal                  \\
{\tt zbpbc1} & 41 & $\beta '/\beta_{c1}'$
                    & {\tt b}eta {\tt p}rime/{\tt b}eta-{\tt c}rit-1' \\
{\tt zbprim} & 39 & $\beta '$
                    & {\tt b}eta {\tt prim}e                      \\
{\tt zcc   } &    & $c$
                    & speed of light {\tt c}onstant, {\tt c}      \\
{\tt zceps0 } &    & $\epsilon_{0}$
                    & {\tt c}onstant {\tt eps}ilon-{\tt 0}        \\
{\tt zcf   } &    &
                    & {\tt c}onstant for collision {\tt f}requencies \\
{\tt zckb  } &    & $k_{b}$
                    & {\tt c}onstant {\tt k}, {\tt B}oltzmann     \\
{\tt zcme  } &    & $m_{e}$
                    & {\tt c}onstant {\tt m}ass of {\tt e}lectron \\
{\tt zcmp  } &    & $m_{p}$
                    & {\tt c}onstant {\tt m}ass of {\tt p}roton   \\
{\tt zcmu0 } &    & $\mu_{0}$
                    & {\tt c}onstant {\tt mu}-{\tt 0}             \\
{\tt zdd   } & 42 & $D^{DR}$
                    & nominal {\tt d}iffusivity for {\tt d}rift modes \\
{\tt zddtem(jz)} & 43 & $D_{a}^{DR}$
                    & {\tt d}iffusivity for {\tt dr}ift modes     \\
{\tt zdi   } & 37 & ${\hat D}_{i}$
                    & {\tt d}iffusivity for eta-{\tt i} mode      \\
{\tt zdiafr} & 34 & $\omega_{e}^{*}$
                    & {\tt dia}magnetic drift {\tt f}requency    \\
{\tt zdgret} & 47 & $D_{\nabla \eta}$
                    & {\tt d}iffusivity due to {\tt g}radient of {\tt et}a \\
{\tt zdhf  } & 68 & $D_{a}^{HF}$
                    & {\tt d}iffusivity, {\tt h}igh {\tt f}requency mode \\
{\tt zdk   } & 60 & $D^{KB}$
                & nominal {\tt d}iffusivity, {\tt k}inetic {\tt b}allooning \\
{\tt zdkb  } & 61 & $D^{KB}_{a}$
                    & {\tt d}iffusivity, {\tt k}inetic {\tt b}allooning \\
{\tt zdrm  } & 48 & $D^{RM}_{a}$
                    & {\tt d}iffusivity, {\tt r}ippling {\tt m}ode \\
{\tt zdrb  } & 55 & $D^{RB}_{a}$
                    & {\tt d}iffusivity, {\tt r}esitive {\tt b}allooning \\
{\tt zdsum } & 71 & $D_{a}$
                    & {\tt d}iffusivity {\tt s}um                \\
{\tt zdte  } & 38 & ${\hat D}_{te}$
                    & {\tt d}iffusivity, {\tt t}rapped {\tt e}lectron mode
\end{tabular}
\end{center}
\end{table}

\end{document}

\begin{table}
\begin{center}
\underline{Table 4. Control Parameters in {\tt Theory}}

\begin{tabular}{llcp{3.0in}}
& & & \\
\underline{Parameter}&\underline{Coding}&\underline{Default}
&\underline{Meaning}\\
& & & \\

$F_{a}^{DR}$&${\tt fdr(1)}$ & 0.0 & particle contribution to drift wave mode\\
$F_{e}^{DR}$&${\tt fdr(2)}$ & 0.0 & electron contribution to drift wave mode
\\
$F_{i}^{DR}$&${\tt fdr(3)}$ & 0.0 & ion contribution to drift wave mode \\
$F_{\Delta}^{DR}$&${\tt fdrint}  $ & 0.0 & electron-ion energy interchange
coeff. \\
$F_{a}^{RM}$&${\tt frm(1)}  $ & 0.0 & particle contribution to rippling mode\\
$F_{e}^{RM}$&${\tt frm(2)}  $ & 0.0 & electron contribution to rippling mode\\
$F_{i}^{RM}$&${\tt frm(3)}  $ & 0.0 & ion contribution to rippling mode\\
$F_{a}^{RB}$&${\tt frb(1)}  $ & 0.0 & particle contribution to res. ball.
mode\\
$F_{e}^{RB}$&${\tt frb(2)}  $ & 0.0 & electron contribution to res. ball.
mode\\
$F_{i}^{RB}$&${\tt frb(3)}  $ & 0.0 & ion contribution to res. ball. mode\\
$F_{a}^{KB}$&${\tt fkb(1)}  $ & 0.0 & particle contribution to kin. ball.
mode\\
$F_{e}^{KB}$&${\tt fkb(2)}  $ & 0.0 & electron contribution to kin. ball.
mode\\
$F_{i}^{KB}$&${\tt fkb(3)}  $ & 0.0 & ion contribution to kin. ball. mode\\
$F_{a}^{HF}$&${\tt fhf(1)}  $ & 0.0 & particle contribution to $\eta_{e}$
mode\\
$F_{e}^{HF}$&${\tt fhf(2)}  $ & 0.0 & electron contribution to $\eta_{e}$
mode\\
$F_{i}^{HF}$&${\tt fhf(3)}  $ & 0.0 & ion contribution to $\eta_{e}$ mode\\
$c_{1}$&${\tt cthery(1)}$ & 0.0 & shear switch\\
$c_{2}$&${\tt cthery(2)}$ & 1.0 & shear parameter\\
$c_{3}$&${\tt cthery(3)}$ & 0.5 & minimum shear \\
$c_{4}$&${\tt cthery(4)}$ & 0.0 & gradient of Z-effective\\
$c_{5}$&${\tt cthery(5)}$ & 6.0 & impurity charge\\
$c_{6}$&${\tt cthery(6)}$ & 1.0 & switch to turn on or off $f_{ith}$\\
$c_{7}$&${\tt cthery(7)}$ & 6.0 & controls the smoothing of $f_{ith}$ \\
$c_{8}$&${\tt cthery(8)}$ & 6.0 & controls the smoothing of $f_{bth}$ \\
$c_{9}$&${\tt cthery(9)}$ & 6.0 & controls the smoothing of $f_{eth}$ \\
$c_{10}$&${\tt cthery(10)}$& 0.0 & switch for $f_{*}$ in res. ball. model\\
$c_{11}$&${\tt cthery(11)}$& 1.0 & sets the value of $\eta_{i}^{th}$\\
$c_{12-15}$&${\tt cthery(12-15)}$&0's & elongation parameters\\
 & & &
\end{tabular}
\end{center}
\end{table}

%**********************************************************************c

The following abreviations are used in the chronology of changes below:
\begin{center}
\begin{tabular}{lll}
    &               &             \\
\underline{Abbreviation} & \underline{Name} & \underline{Affiliation}\\
rgb & Glenn Bateman & PPPL \\
pis & P{\"a}r Strand & Chalmers \\
jcc & J.~C. Cummings & PPPL \\
emg & E.~S. Ghanem  & U. Illinois \\
jek & Jon Kinsey    & Lehigh Univ. \\
ajr & Aaron J.~Redd & Lehigh Univ. \\
mhr & Martha Redi   & PPPL \\
ces & C.~E. Singer  & U. Illinois \\
dps & D.~P. Stotler & PPPL \\
\end{tabular}
\end{center}

\begin{verbatim}
c--------1---------2---------3---------4---------5---------6---------7-c
c@theory   .../baldur/code/bald/theory.tex
c  pis 16-jul-98 added cthery(129) as multiplier to wexbs
c  pis 16-jul-98 added cthery(130) as multiplier to impurity heat flux
c  pis 07-jul-98 added diagnostics for wexb
c  pis 07-jul-98 corrected def of omegde(jz) and moved it to before etaw17
c  pis 15-may-98 added ExB shearing rate (wexbs) to argument list
c  pis 15-may-98 replaced etaw17 with etaw17diff, order of diagnostic 
c    output have been changed
c  pis 15-may-98 switches pertaining to choice of eigensolvers obsolete 
c    with introduction of etaw17 i.e. iletai(6) and iletai(10)
c  pis 14-may-98 etaw16 replaced by etaw17 using non-proprietary solvers 
c    for generalized eigenvalue equation, allowing for ExB shearing rate 
c    reduction of growthrates
c  rgb 30-mar-98 diagnostic output added for drift Alfven mode
c    controlled by cthery(88)
c  rgb 02-mar-98 call sda01dif for Bruce Scott's Drift Alfven model
c    added zelong to diagnostic output
c  rgb 24-feb-98 etaw14 --> etaw16, cetain(25) = cthery(122), 
c    zelong added to argumentl list of etaw16
c  rgb 25-feb-97 print zgmitg, zomitg, zgm2nd, zom2nd, zgmtem, zomtem
c    zomegde, zomegse, zkinvsq
c  rgb 21-nov-96 lprint .gt. 0 turns on diagnostic printout from etaw..
c  rgb 06-nov-96 cthery(126) > zepslon controls zftrap for etaw14
c    cthery(126) < zepslon multiplies zftrap * abs(cthery(126))
c  ajr 01-oct-96 added Ottoviani-Horton-Erba eta_i model
c  rgb 13-sep-96 cthery(85) --> nint(cther(85)
c  rgb 01-jul-96 force zero gradients to round up
c  rgb 21-feb-96 revised printout for kinetic ballooning mode
c  rgb 14-feb-96 smooth the relative superthermal ion density
c    if lthery(19) > 0
c  rgb 13-feb-96 etaw12 --> etaw14  and changed order of arguments
c    zgnh, zgnz, zgns --> zgne, zgnh, zgnz
c  rgb 12-feb-96 zgrdns = ....  / max ( zfnsne, 1.e-6 )
c    zgrdns may be 0.0 and remove zepsns
c  rgb 05-feb-96 lthery(26) timestep for diagnostic printout for etaw*
c     remove use of cthery(125) for this purpose
c  rgb 21-jan-96 corrected zovfkb = max(zexkb,-abs(zlgeps)) ...
c    fixed zdk = abs( zsound * zrhos**2 * zfbthn / zlpr )
c  jek 18-jan-96 added new kinetic ballooning mode model
c  jek 08-dec-95 use etaw12.f for Weiland ITG/TEM model
c  jek 02-dec-95 added Guzdar-Drake resistive ballooning model
c                and added diagnostic printout
c  rgb 07-may-95 change zlmin from 1.e-6 to 1.e-4 because units for
c    minor radius were changed from cm to m
c  rgb 06-may-95 skip directly to printout if lprint < 0
c  rgb 15-apr-95 replace cpvelc -> cthery(68), cpvion -> cthery(69)
c    diagnostic ouput controlled by lprint
c  rgb 11-apr-95 common blocks -> argument list
c  rgb 07-jan-95 Inserted elongation factor into Weiland and Horton etai
c  rgb 19-oct-94 Smooth reciprocals of gradient scale lengths when 
c    lthery(32) < 0 and gradient scale lengths when lthery(32) > 0
c    added Z_eff to profiles as a function of major radius
c  rgb 07-oct-94 No kinetic ballooning mode it zlpr < 0.0
c  rgb 15-jul-94 revised computation of zepsns
c  rgb 14-jul-94 compute zfnsne directly from fast particle densities
c    had to include '../com/cd3he.m' for rh1fst and rh2fst
c  rgb 29-jun-94 replace negative diffusivities with convective
c    velocity when lthery(27) > 0
c  rgb 25-jun-94 temporary diagnostic output
c  rgb 20-jun-94 include maxis + 1 in printout of profiles as a 
c    function of major radius
c  rgb 04-jun-94 implement cthery(111) to cthery(114) to transfer from
c    diffusivity matrix to convective velocity
c  rgb 25-may-94 implemented sbrtn etawn8 with superthermal ions
c    correctly set zlnz from zslnz
c  rgb 09-may-94 entry prtheory added
c  rgb 25-apr-94 compute zlnz directly and limit zlnh and zlnz
c  rgb 02-apr-94 print out fluxes and sources as well as vftot
c  rgb 25-mar-94 compute effective convective velocities and print out
c  rgb 20-mar-94 print neoclassical and empirical diffusivity columns
c  rgb 06-mar-94 compute zslnh(jz) = n_H / ( d n_H / dr )
c  rgb 25-feb-94 smooth diffusivities if lthery(28) > 0
c    fixed smoothing when cthery(60) or cthery(61) > 0.0
c  rgb 22-feb-94 diagnostic printout if iprint > 0
c  rgb 02-feb-94 when lthery(8) > 20, use effective diffusivities
c    and set the diffusivity matrix to zero
c  rgb 29-jan-94 compute difthi and velthi from sbrtn etawn7
c  rgb 11-jan-94 always print out gradient scale lengths
c  rgb 06-jan-94 Use IMSL rather than NAG14 routine in sbrtn etawn6
c  rgb 02-jan-94 set zimpz = max ( zimpz, cthery(120) )
c    set iprint = lthery(29) - 10 before calling etawn6
c  rgb 29-nov-93 Use lthery(29) to limit diagnostic printout
c    print out hydrogen particle diffusivities
c    print out effective eta_i mode diffusivities next to matrix
c    set zcetai(32) = cthery(128)
c  rgb 23-nov-93 Print out diffusivity matrix from Weiland model
c    set zcetai(32) = cthery(39)  or = 1.e-6 if cthery(39) < zepslon
c  rgb 22-nov-93 protected etae mode from overflow 
c    control etae mode using lthery(20)
c  rgb 21-nov-93 changed zcetai(32) from sqrt(zepslon) to 1.e-6
c    fixed computation of difthi, zgmitg, and zgmtem
c    skip trapped electron mode models if lthery(6) < 0
c  rgb 07-nov-93 set zcetai(32) = sqrt ( zepslon )
c  jek 25-jun-92 added circ-electron and high-m tearing models
c  rgb 04-sep-93 implement sbrtn etawn6 to compute Weiland model
c    with the effect of impurities, trapped electrons and FLR effects
c  jek 04-sep-93 corrections to the Carreras-Diamond resistive
c    ballooning mode based on comments by Dave Ross
c  rgb 20-feb-93 replaced zlne with zln for Nordman-Weiland model
c  rgb 20-feb-93 print out profiles as a function of major radius
c  rgb 18-feb-93 print out effective particle diffisivities
c  rgb 10-feb-93 implemented matrix form of Nordman-Weiland Model
c  rgb 11-nov-92 corrected switch between thdre and thdri since 20-sep-92
c  rgb 06-nov-92 implement control by cthery(38) and cthery(39)
c     for the Kim-Horton eta_i mode model
c  rgb 02-nov-92 compute diffusivity from Kim-Horton eta_i mode model
c  rgb 28-oct-92 Revised Nordman-Weiland model by combining eta_i and
c     TEM modes and normalizing frequencies by omega_{De}
c  rgb 22-sep-92 inserted abs(...) to deal with negative zl** values
c  rgb 21-sep-92 retain sign of gradient scale lengths when lthery(30)=1
c  rgb 20-sep-92 limit local change in TEM modes when cthery(61) .gt. 0.
c    Remove old use of cthery(61) and cthery(62)
c  rgb 17-sep-92 k_y \rho_s = cthery(38) or 0.316
c  rgb 14-sep-92 Weiland-Nordman NF 30 (1990) 983 \eta_i mode model
c  rgb 01-apr-92 neoclassical MHD toroidal mode number  cthery(77)
c  rgb 30-mar-92 use \nu_{ii} rather than \nu_{ei} in \nu_{*i}
c    There have been many revisions to the neoclassical MHD model
c  rgb 26-mar-92 neoclassical MHD proper conversion to MKS units
c  rgb 25-mar-92 print out zalphz(jz),...,zfdiaz(jz)
c  rgb 24-mar-92 lthery(15)=1 for original neoclassical MHD, =2 revised
c  rgb 20-mar-92 corrections to neoclassical MHD
c  rgb 19-mar-92 temporary arrays to debug neoclassical MHD
c  rgb 17-mar-92 revised Rebut-Lallia-Watkins model PF B 3 (1991) 2209
c  rgb 16-mar-92 implemented ExB part of neoclassical MHD for particle
c    and ion thermal diffusivity controlled by fmh(1) and fmh(3)
c  Note:  fmh(j) is used for the coefficients of neoclassical MHD 
c    because fnm(j) is already used in common block adsdat in file comadp.m
c  rgb 16-feb-92 replace nusti(jz) (from Kinsey) by znusti (scalar)
c  rgb 14-feb-92 eta_i mode times q(jz)**cthery(37)
c    cthery(45)=1.0 to correct resistive ballooning mode diffusivity to
c      more closely match the analytic solution
c  rgb 13-feb-92 changed lthery(33) to lthery(15) for neoclassical MHD mode
c  jek 16-nov-91 20.05  added neoclassical MHD mode
c  rgb 17-oct-91 20.17  multiply trapped electron mode contributions
c      by exp[-cthery(22)*(Ti/Te-1)**2]
c  rgb 18-jul-91 20.02  let zetai=zlnj/zlti and use zlnj in etai-mode forms
c      where zlnj=zlne if lthery(3).lt.1, else zlnj=zlni
c      and   zln =zlne if lthery(3).lt.2, else zln =zlni
c      Also, changes made to weithe(jz) 
c  rgb 10-jun-91 19.09  implemented changes from Martha Redi and J. Cummings
c      but decided to keep zetai=zln/zlti with zln determined by lthery(3)
c      that is, zln=zlne if lthery(3).eq.0 and zln=zlni if lthery(3).ne.0
c  jcc 01-mar-91 19.08  replace term removed in version 19.07 and add two
c                       new trapped electron mode theory options.  if
c                       lthery(6)=7, use Hahm-Tang CTEM with Rewoldt
c                       transition to dissipative regime.  if lthery(6)=8,
c                       use Hahm-Tang CTEM and Kadomtsev-Pogutse DTEM.
c  jcc 26-feb-91 19.07  remove {sqrt[zrmaj/zln]-log[sqrt(zrmaj/zln)]-1}
c                       from Hahm-Tang formula for zdte to see if it is the
c                       cause of crashes with high q, low beam power cases.
c  jcc 18-feb-91 19.06  made zetai equal zlni/zlti instead of zln/zlti,
c                       and used zlne explicitly in definitions of 
c                       zetae and zdiafr.
c  jcc 04-feb-91 19.05  add separate printout of Bohm and gyro-Bohm
c                       contributions to electron thermal diffusivity
c                       compile all routines with cft77.
c  jcc 04-feb-91 separate gyro-Bohm and Bohm contributions to thrbe
c                they are labeled thrbgb and thrbb, respectively.
c  jcc 28-jan-91 19.04  change dbeams to give proper balanced injection.
c  jcc 19-sep-90 19.03  made lthery(6)=4 a Hahm-Tang CTEM only option
c  jcc 17-sep-90 19.02  added complementary Hahm-Tang DTEM model
c                       for nu-star > 0.1        
c  jcc 05-aug-90 19.01  fixed errors in separation of particle and heat
c                       fluxes due to trapped electron modes from those
c                       due to eta-i modes, eliminated possible double-
c                       counting of contributions.
c  jcc 04-aug-90 19.00  consolidated MMM, Rewoldt transition, Hahm-Tang
c                       CTEM model and Kadomtsev-Pogutse DTEM model.
c                       Also adding diagnostic for validity of Hahm-Tang.
c  jcc 20-jul-90 18.4601  added Hahm, Tang CTEM model (IAEA 1990).
c  rgb 05-may-91 18.88  add Rebut-Lallia-Watkins model after $\eta_e$ mode
c  rgb 26-dec-90 18.81  go back to cft77 compiler
c  rgb 12-oct-90 18.72  limit \chi^{ITG} with 2nd and 4th differences
c  rgb 05-oct-90 18.71  limit local rate of change only for eta_i mode
c  rgb 05-oct-90 18.70  skip straight to printout when knthe = 3
c  rgb 04-oct-90 18.69  when cthery(60) .gt. zepslon, limit the rate of
c      change of diffusivities relative to their average rate of change
c  rgb 02-oct-90 18.68  lthery(31)=1  r * (gradient scale lengths) 
c      monotonic near the magnetic axis by changing value at jz=maxis+1
c  rgb 20-sep-90 18.66  exp[-min(5 L_n, 4 L_T) / L_s] in Hamaguchi-Horton
c  rgb 14-sep-90 18.64  Revised Hahm-Tang TEM model and
c      always compute zdiafr using the electron density zlne
c  rgb 12-sep-90 18.63  Revised precond and smoothing grad scale lengths
c  rgb 10-sep-90 18.62  Diagnostic output for the Hahm-Tang when lthery(5)=1
c  rgb 05-sep-90 18.61  Hahm-Tang trapped electron mode when lthery(5)=1
c  rgb 03-sep-90 18.60  use (1+\kappa^2)/2 instead of \kappa if lthery(12)=1
c  rgb 20-aug-90 18.54  minor changes to output
c  rgb 30-jul-90 18.51  precondition gradient scale lengts before smoothing
c  rgb 19-jul-90 18.49  Allow smoothing directly on gradient scale lengths
c    changes to argument list in calls to sbrtn smooth
c  rgb 15-jul-90 18.48  Hamaguchi-Horton eta_i mode, first attempt
c  rgb 12-jul-90 18.47  Dominguez-Rosenbluth \eta_i^{th}
c  rgb 05-may-90 18.32  exp[-cthery(34)*(Ti/Te-1)**2] factor
c      multiplying the eta_i mode (IG) in all models
c        Place upper bound on scale lengths relative to major radius
c      controlled by cthery(50...) when these coefficients are positive
c  rgb 16-apr-90 replaced zspres with zresis throughout
c      lthery(4) controls type of resistivity used throughout
c      store zresis in array zrsist(jz) and print out
c  rgb 10-apr-90 removed eta_i contribution from drift wave particle diff
c  rgb 09-apr-90 thrbe(jz)=...+cthery(44)*zxrb*zfdias
c  rgb 20-mar-90 compute and use ion density scale length slnis(jz)
c      if lthery(3)=0 zln = zlne, else zln = zlni
c      zlne replaced by zln almost everywhere
c      collect formatted output from file DIO all together
c  rgb 16-mar-90 Lee-Diamond eta_i mode theory when lthery(7)=1
c  rgb 09-mar-90 input fig(j) for eta_i and fti(j) for trapped ion
c      separate sections and columns of output for eta_i and trapped ion
c  rgb 17-feb-90 linear ramp form for f_ith when lthery(9=2
c  rgb 16-feb-90 add cthery(44)*D^{RB} to \chi_e^{RB}
c  rgb 15-feb-90 implemented diamagnetic stabilization of new resistive
c      ballooning mode using cthery(42) and cthery(43)
c    extrapolate diffusivities to edge grid point when lthery(2)=1
c  rgb 08-feb-90 implemented variables in common /comth3/
c      added new page and rearranged printout
c  rgb 03-feb-90 Mattor-Diamond, Hahm-Tang \eta_i^{th} when lthery(8)=1
c      Hahm-Diamond... estimate for \chi_e^{RM} when lthery(11)=1
c  rgb 01-feb-90 revised form for f_{ith} when lthery(9)=1
c  rgb 30-jan-90 Rewoldt's transition controlled by lthery(6)=1
c  rgb 29-jan-90 Implemented Greg Rewoldt's suggestion for transition from
c      dissipative to collisionless trapped electron mode in \hat{D}_{te}
c  rgb 28-jan-90 Carreras-Diamond resistive ballooning mode
c      PF B1 (1989) 1011-1017
c  rgb 15-jan-90 zlpr corrected and shear(jz) used to compute shear
c      These changes restore the code to what it was the last half of 1989
c  rgb 12-jan-90 zs1,...,zs20 replaced by zs(js), js=1,128
c      zs(js)=ctheory(js)
c  rgb 28-nov-89 moved dimension zx... to common /comth2/ th...
c  rgb 27-oct-89 cthery(19) coefficient of f_ith added to 5/2 in
c      drift wave electron thermal diffusivity (default = -1.5)
c  emg 04-oct-89 fix a bug in the expression of zetith,eq. 34. switch
c                the 1. into zs11
c  emg 29-sep-89 convert zetith,zfith,zdi,zdte into arrays and
c                add printout commands to printthem in BALDUR's
c                long output.
c  rgb 09-oct-89 cthery(17) controls subtraction of convective flux
c     to correct thermal flux (CPP Eq. 73 and 74)
c   cthery(20) (default=1.0) controls transition in CPP Eq. 37
c  rgb 03-oct-89 add diagnostic output of threti(j),thdinu(j),...
c     new page of printout
c    Moved computation of values at r=0 after 300 continue
c  ces 25-jul-89 add numerical overflow protection to zlpr
c  ces 25-jul-89 add variable eta-i-crit
c  rgb 13-jun-89 removed zt and zdt from argument list
c     and defined zt znd zdt just before do 400
c  rgb 09-jun-89 use civic with lang=cft rather than cft77 compiler
c  emg 20-apr-89 change the energy interchange name from 'eithes' to
c               'weithe' to be compatable with other sources names
c                and add the new name into the same common block
c  emg 09-mar-89 add cthery to 'fstar' in the resistive
c                ballooning model with default 0.0
c  emg 03-mar-89 add the print commands for the detailed diffusion
c                coefficients.
c  emg 26-feb-89 correct the expression for the pressure scale
c                length, zlpr.
c  emg 16-feb-89 correct the expression for the shear in
c                equation (23)
c  emg 19-dec-88 15.06 add the overflow protection for
c                the exponential terms
c  emg 07-dec-88 15.06 change the constants in exponents
c                into cthery's
c  dps 20-oct-88 15.06 added to BALDPN S, V.15.06.
c  ces 17-oct-88 correct parentheses in statement 25, zsdiv=...
c  emg 11-oct-88 finish first standard version for shipping to PPPL
c  emg 23-jun-88 subroutine set up
c
c***********************************************************************
\end{verbatim}

\end{document}             % End of document.
